\documentclass[12pt,a4paper]{article}

% Paquetes básicos
\usepackage[utf8]{inputenc}
\usepackage[spanish]{babel}
\usepackage[margin=2.5cm]{geometry}
\usepackage{xcolor}
\usepackage{tcolorbox}
\usepackage{lipsum}
\usepackage{hyperref}

% Paquetes para figuras
\usepackage{graphicx}
\usepackage{float}
\usepackage{caption}
\usepackage{subcaption}
\usepackage{wrapfig}

% Configurar ruta de figuras
\graphicspath{{../figuras/}}

% Colores personalizados
\definecolor{ejemplocolor}{RGB}{41,128,185}

% Configuración de hyperref
\hypersetup{
	colorlinks=true,
	linkcolor=blue,
	urlcolor=blue,
	citecolor=blue,
	pdftitle={Aplicaciones del Entorno Figure en LaTeX},
	pdfauthor={Ejemplos Prácticos}
}

\title{\textbf{\Huge Aplicaciones del Entorno \texttt{figure}}\\[0.5cm]\large Ejemplos Prácticos con Imágenes Reales}
\author{}
\date{\today}

\begin{document}

\maketitle
\thispagestyle{empty}

\begin{tcolorbox}[colback=blue!5,colframe=blue!75!black,title=Introducción]
Este documento contiene ejemplos prácticos de aplicación del entorno \texttt{figure} en \LaTeX{}, mostrando todos los comandos y opciones disponibles con imágenes reales.

\textbf{Contenido:}
\begin{itemize}
\item Opciones de posicionamiento de figuras
\item Diferentes formas de insertar y ajustar imágenes
\item Uso de captions y referencias cruzadas
\item Subfiguras y figuras múltiples
\item Figuras envolventes con texto
\end{itemize}
\end{tcolorbox}

\tableofcontents
\newpage

\section{Opciones de Posicionamiento}

Las opciones de posicionamiento controlan dónde LaTeX coloca la figura en el documento.

\subsection{Opción [h] - Here (Aquí)}

La opción \texttt{[h]} intenta colocar la figura en la posición exacta donde aparece en el código.

\begin{figure}[h]
\centering
\includegraphics[width=0.5\textwidth]{Latex1.png}
\caption{Figura con opción [h] - Colocada aquí en el texto}
\label{fig:ejemplo-h}
\end{figure}

Como puede verse, la Figura \ref{fig:ejemplo-h} aparece aproximadamente donde se definió en el código.

\lipsum[1]

\subsection{Opción [t] - Top (Arriba)}

La opción \texttt{[t]} coloca la figura en la parte superior de la página.

\begin{figure}[t]
\centering
\includegraphics[width=0.4\textwidth]{Latex2.png}
\caption{Figura con opción [t] - En la parte superior de la página}
\label{fig:ejemplo-t}
\end{figure}

\lipsum[2]

\subsection{Opción [b] - Bottom (Abajo)}

La opción \texttt{[b]} coloca la figura en la parte inferior de la página.

\begin{figure}[b]
\centering
\includegraphics[width=0.4\textwidth]{latex1.jpg}
\caption{Figura con opción [b] - En la parte inferior de la página}
\label{fig:ejemplo-b}
\end{figure}

\lipsum[3]

\newpage

\subsection{Opción [H] - HERE (Forzada)}

La opción \texttt{[H]} (mayúscula) fuerza la posición exacta, sin permitir que LaTeX la mueva. Requiere el paquete \texttt{float}.

\lipsum[4][1-3]

\begin{figure}[H]
\centering
\includegraphics[width=0.5\textwidth]{latex2.jpg}
\caption{Figura con opción [H] - Posición exacta forzada}
\label{fig:ejemplo-H}
\end{figure}

La Figura \ref{fig:ejemplo-H} está exactamente donde se colocó en el código fuente, sin excepciones.

\lipsum[5][1-3]

\subsection{Opción [htbp] - Combinación Flexible}

La opción \texttt{[htbp]} es la más flexible, permitiendo a LaTeX elegir la mejor posición entre varias opciones.

\begin{figure}[htbp]
\centering
\includegraphics[width=0.45\textwidth]{latex3.png}
\caption{Figura con opción [htbp] - LaTeX elige la mejor posición}
\label{fig:ejemplo-htbp}
\end{figure}

\lipsum[6]

\section{Ajustes de Tamaño con includegraphics}

\subsection{Ajuste por Ancho (width)}

Podemos controlar el ancho de la imagen usando la opción \texttt{width}.

\begin{figure}[H]
\centering
\includegraphics[width=0.3\textwidth]{Latex4.png}
\caption{Imagen con ancho de 0.3\textbackslash textwidth}
\label{fig:width-30}
\end{figure}

\begin{figure}[H]
\centering
\includegraphics[width=0.6\textwidth]{Latex4.png}
\caption{La misma imagen con ancho de 0.6\textbackslash textwidth}
\label{fig:width-60}
\end{figure}

\begin{figure}[H]
\centering
\includegraphics[width=0.9\textwidth]{Latex4.png}
\caption{La misma imagen con ancho de 0.9\textbackslash textwidth}
\label{fig:width-90}
\end{figure}

Las Figuras \ref{fig:width-30}, \ref{fig:width-60} y \ref{fig:width-90} muestran la misma imagen con diferentes anchos.

\newpage

\subsection{Ajuste por Altura (height)}

También podemos especificar la altura de la imagen.

\begin{figure}[H]
\centering
\includegraphics[height=4cm]{latex4.jpg}
\caption{Imagen con altura fija de 4cm}
\label{fig:height-4cm}
\end{figure}

\begin{figure}[H]
\centering
\includegraphics[height=6cm]{latex4.jpg}
\caption{La misma imagen con altura de 6cm}
\label{fig:height-6cm}
\end{figure}

\subsection{Ajuste por Escala (scale)}

La opción \texttt{scale} permite escalar la imagen proporcionalmente.

\begin{figure}[H]
\centering
\includegraphics[scale=0.3]{Latex31.png}
\caption{Imagen con escala 0.3 (30\% del tamaño original)}
\label{fig:scale-03}
\end{figure}

\begin{figure}[H]
\centering
\includegraphics[scale=0.6]{Latex31.png}
\caption{Imagen con escala 0.6 (60\% del tamaño original)}
\label{fig:scale-06}
\end{figure}

\newpage

\subsection{Rotación con angle}

Podemos rotar imágenes usando la opción \texttt{angle} (en grados, sentido antihorario).

\begin{figure}[H]
\centering
\includegraphics[width=0.4\textwidth]{Copia_de_Latex1.png}
\caption{Imagen sin rotación (0 grados)}
\label{fig:angle-0}
\end{figure}

\begin{figure}[H]
\centering
\includegraphics[width=0.4\textwidth,angle=45]{Copia_de_Latex1.png}
\caption{Imagen rotada 45 grados}
\label{fig:angle-45}
\end{figure}

\begin{figure}[H]
\centering
\includegraphics[width=0.4\textwidth,angle=90]{Copia_de_Latex1.png}
\caption{Imagen rotada 90 grados}
\label{fig:angle-90}
\end{figure}

\newpage

\subsection{Mantener Proporción (keepaspectratio)}

La opción \texttt{keepaspectratio} mantiene las proporciones cuando especificamos tanto ancho como altura.

\begin{figure}[H]
\centering
\includegraphics[width=6cm,height=3cm]{Copia_de_Latex2.png}
\caption{Sin keepaspectratio - Imagen deformada}
\label{fig:no-keep}
\end{figure}

\begin{figure}[H]
\centering
\includegraphics[width=6cm,height=3cm,keepaspectratio]{Copia_de_Latex2.png}
\caption{Con keepaspectratio - Proporciones mantenidas}
\label{fig:keep}
\end{figure}

Compare las Figuras \ref{fig:no-keep} y \ref{fig:keep}: la segunda mantiene las proporciones originales.

\subsection{Opciones Combinadas}

Podemos combinar múltiples opciones para lograr efectos específicos.

\begin{figure}[H]
\centering
\includegraphics[width=0.5\textwidth,angle=10,keepaspectratio]{Copia_de_Latex31.png}
\caption{Imagen con múltiples opciones: ancho 0.5\textbackslash textwidth, rotación 10°, proporciones mantenidas}
\label{fig:combinado}
\end{figure}

\newpage

\section{Tipos de Caption (Pie de Figura)}

\subsection{Caption Básico}

Un caption simple proporciona descripción y numeración automática.

\begin{figure}[H]
\centering
\includegraphics[width=0.5\textwidth]{Copia_de_Latex4.png}
\caption{Este es un caption básico que describe la figura}
\label{fig:caption-basico}
\end{figure}

\subsection{Caption con Versión Corta}

Podemos proporcionar una versión corta para el índice de figuras usando \texttt{\textbackslash caption[corto]\{largo\}}.

\begin{figure}[H]
\centering
\includegraphics[width=0.5\textwidth]{Copia_de_latex1.jpg}
\caption[Versión Corta]{Este es un caption largo y detallado que aparece bajo la figura, pero la versión corta "Versión Corta" aparecerá en el índice de figuras}
\label{fig:caption-corto}
\end{figure}

\subsection{Caption Sin Numeración}

Usando \texttt{\textbackslash caption*} podemos crear captions sin número (requiere paquete \texttt{caption}).

\begin{figure}[H]
\centering
\includegraphics[width=0.4\textwidth]{Copia_de_latex2.jpg}
\caption*{Esta figura tiene un caption sin numeración}
\end{figure}

\newpage

\section{Referencias Cruzadas}

\subsection{Uso de label y ref}

Las etiquetas permiten referenciar figuras en cualquier parte del documento.

\begin{figure}[H]
\centering
\includegraphics[width=0.45\textwidth]{Copia_de_latex3.png}
\caption{Figura para demostrar referencias cruzadas}
\label{fig:referencias}
\end{figure}

Podemos referirnos a la Figura \ref{fig:referencias} desde cualquier lugar del documento. La figura está en la página \pageref{fig:referencias}.

\subsection{Múltiples Referencias}

Podemos hacer referencias a varias figuras en el mismo párrafo:

Como se puede observar en las Figuras \ref{fig:ejemplo-h}, \ref{fig:ejemplo-t} y \ref{fig:ejemplo-b}, cada opción de posicionamiento produce resultados diferentes. La Figura \ref{fig:ejemplo-H} muestra el caso especial de posicionamiento forzado.

\section{Alineación de Figuras}

\subsection{Centrado (centering)}

La alineación centrada es la más común para figuras.

\begin{figure}[H]
\centering
\includegraphics[width=0.4\textwidth]{Copia_de_latex4.jpg}
\caption{Figura centrada (opción más común)}
\label{fig:centrada}
\end{figure}

\newpage

\subsection{Alineación a la Izquierda (raggedright)}

\begin{figure}[H]
\raggedright
\includegraphics[width=0.4\textwidth]{latex1.jpg}
\caption{Figura alineada a la izquierda}
\label{fig:izquierda}
\end{figure}

\subsection{Alineación a la Derecha (raggedleft)}

\begin{figure}[H]
\raggedleft
\includegraphics[width=0.4\textwidth]{latex2.jpg}
\caption{Figura alineada a la derecha}
\label{fig:derecha}
\end{figure}

\section{Subfiguras}

Las subfiguras permiten incluir múltiples imágenes relacionadas en una sola figura, cada una con su propio subcaption.

\subsection{Dos Subfiguras Lado a Lado}

\begin{figure}[H]
\centering
\begin{subfigure}{0.45\textwidth}
	\centering
	\includegraphics[width=\textwidth]{latex3.png}
	\caption{Primera subfigura}
	\label{fig:sub-a}
\end{subfigure}
\hfill
\begin{subfigure}{0.45\textwidth}
	\centering
	\includegraphics[width=\textwidth]{latex4.jpg}
	\caption{Segunda subfigura}
	\label{fig:sub-b}
\end{subfigure}
\caption{Dos subfiguras mostrando diferentes imágenes}
\label{fig:dos-subfiguras}
\end{figure}

Podemos referenciar la figura completa (\ref{fig:dos-subfiguras}) o las subfiguras individuales (\ref{fig:sub-a} y \ref{fig:sub-b}).

\newpage

\subsection{Cuatro Subfiguras en Cuadrícula}

\begin{figure}[H]
\centering
\begin{subfigure}{0.45\textwidth}
	\centering
	\includegraphics[width=\textwidth]{Latex1.png}
	\caption{Subfigura A}
	\label{fig:grid-a}
\end{subfigure}
\hfill
\begin{subfigure}{0.45\textwidth}
	\centering
	\includegraphics[width=\textwidth]{Latex2.png}
	\caption{Subfigura B}
	\label{fig:grid-b}
\end{subfigure}

\vspace{0.5cm}

\begin{subfigure}{0.45\textwidth}
	\centering
	\includegraphics[width=\textwidth]{Latex31.png}
	\caption{Subfigura C}
	\label{fig:grid-c}
\end{subfigure}
\hfill
\begin{subfigure}{0.45\textwidth}
	\centering
	\includegraphics[width=\textwidth]{Latex4.png}
	\caption{Subfigura D}
	\label{fig:grid-d}
\end{subfigure}
\caption{Cuatro subfiguras organizadas en cuadrícula 2x2}
\label{fig:cuadricula}
\end{figure}

La Figura \ref{fig:cuadricula} muestra un arreglo de cuatro subfiguras (\ref{fig:grid-a}, \ref{fig:grid-b}, \ref{fig:grid-c} y \ref{fig:grid-d}).

\newpage

\subsection{Tres Subfiguras con Diferentes Tamaños}

\begin{figure}[H]
\centering
\begin{subfigure}{0.6\textwidth}
	\centering
	\includegraphics[width=\textwidth]{Copia_de_Latex1.png}
	\caption{Subfigura principal (60\% de ancho)}
	\label{fig:tres-principal}
\end{subfigure}

\vspace{0.5cm}

\begin{subfigure}{0.28\textwidth}
	\centering
	\includegraphics[width=\textwidth]{Copia_de_Latex2.png}
	\caption{Subfigura secundaria 1}
	\label{fig:tres-sec1}
\end{subfigure}
\hfill
\begin{subfigure}{0.28\textwidth}
	\centering
	\includegraphics[width=\textwidth]{Copia_de_Latex31.png}
	\caption{Subfigura secundaria 2}
	\label{fig:tres-sec2}
\end{subfigure}
\caption{Tres subfiguras con diferentes tamaños}
\label{fig:tres-subfiguras}
\end{figure}

\section{Figuras Envolventes (wrapfig)}

El paquete \texttt{wrapfig} permite que el texto fluya alrededor de las figuras.

\subsection{Figura Envuelta a la Derecha}

\begin{wrapfigure}{r}{0.4\textwidth}
\centering
\includegraphics[width=0.38\textwidth]{Copia_de_Latex4.png}
\caption{Figura envuelta a la derecha}
\label{fig:wrap-derecha}
\end{wrapfigure}

\lipsum[7-8]

Como puede verse, el texto fluye naturalmente alrededor de la Figura \ref{fig:wrap-derecha}, que está posicionada a la derecha. Este efecto es particularmente útil en documentos tipo artículo o revista donde se busca un diseño más compacto y visualmente atractivo.

\lipsum[9]

\subsection{Figura Envuelta a la Izquierda}

\begin{wrapfigure}{l}{0.35\textwidth}
\centering
\includegraphics[width=0.33\textwidth]{Copia_de_latex1.jpg}
\caption{Figura envuelta a la izquierda}
\label{fig:wrap-izquierda}
\end{wrapfigure}

\lipsum[10-11]

En este caso, la Figura \ref{fig:wrap-izquierda} está posicionada a la izquierda y el texto fluye por su lado derecho. Este tipo de disposición es muy efectiva para mantener la continuidad visual del texto mientras se incorporan elementos gráficos relevantes.

\lipsum[12][1-2]

\newpage

\section{Ejemplos Avanzados}

\subsection{Figura con Múltiples Opciones Combinadas}

Este ejemplo combina varias opciones para lograr un resultado específico.

\begin{figure}[H]
\centering
\fbox{\includegraphics[width=0.6\textwidth,angle=5,keepaspectratio]{Copia_de_latex2.jpg}}
\caption[Ejemplo Avanzado]{Figura con marco (\texttt{\textbackslash fbox}), rotación de 5 grados, ancho de 0.6\textbackslash textwidth y proporciones mantenidas}
\label{fig:avanzado1}
\end{figure}

\subsection{Dos Figuras Independientes Lado a Lado}

Usando \texttt{minipage} podemos colocar dos figuras independientes lado a lado.

\begin{figure}[H]
\centering
\begin{minipage}{0.45\textwidth}
	\centering
	\includegraphics[width=\textwidth]{Copia_de_latex3.png}
	\caption{Primera figura independiente}
	\label{fig:mini-a}
\end{minipage}
\hfill
\begin{minipage}{0.45\textwidth}
	\centering
	\includegraphics[width=\textwidth]{Copia_de_latex4.jpg}
	\caption{Segunda figura independiente}
	\label{fig:mini-b}
\end{minipage}
\end{figure}

A diferencia de las subfiguras, las Figuras \ref{fig:mini-a} y \ref{fig:mini-b} tienen numeración independiente.

\newpage

\subsection{Figura con Texto Descriptivo Adicional}

\begin{figure}[H]
\centering
\includegraphics[width=0.5\textwidth]{latex1.jpg}
\caption{Ejemplo de figura científica}
\label{fig:cientifica}

\vspace{0.3cm}

\small
\textbf{Nota:} Esta figura muestra un ejemplo típico de contenido científico en LaTeX. Las figuras pueden incluir texto adicional debajo del caption para proporcionar contexto extra o notas metodológicas.
\normalsize
\end{figure}

\section{Lista de Figuras}

Al final del documento, podemos incluir una lista automática de todas las figuras.

\newpage
\listoffigures

\section{Mejores Prácticas}

\begin{tcolorbox}[colback=green!10,colframe=green!75!black,title=Recomendaciones]
\begin{enumerate}
\item \textbf{Posicionamiento}: Usa \texttt{[htbp]} para máxima flexibilidad, o \texttt{[H]} solo cuando necesites posición exacta.

\item \textbf{Tamaño}: Prefiere especificar anchos relativos (\texttt{0.5\textbackslash textwidth}) en lugar de absolutos (5cm) para mejor adaptabilidad.

\item \textbf{Referencias}: Siempre coloca \texttt{\textbackslash label} DESPUÉS de \texttt{\textbackslash caption}.

\item \textbf{Alineación}: Usa \texttt{\textbackslash centering} en lugar de \texttt{\textbackslash begin\{center\}} dentro de figuras para evitar espacio vertical extra.

\item \textbf{Captions}: Sé descriptivo en los captions; deben permitir que el lector entienda la figura sin leer el texto principal.

\item \textbf{Formato}: Usa .png para gráficos con áreas de color sólido, .jpg para fotografías.

\item \textbf{Nombrado}: Usa nombres de archivo descriptivos y prefijos \texttt{fig:} en labels.
\end{enumerate}
\end{tcolorbox}

\begin{tcolorbox}[colback=blue!10,colframe=blue!75!black,title=Paquetes Importantes]
\begin{itemize}
\item \texttt{graphicx}: Esencial para incluir imágenes
\item \texttt{float}: Añade la opción [H]
\item \texttt{caption}: Personalización avanzada de captions
\item \texttt{subcaption}: Para crear subfiguras
\item \texttt{wrapfig}: Para figuras envolventes
\end{itemize}
\end{tcolorbox}

\begin{tcolorbox}[colback=yellow!10,colframe=orange!75!black,title=Errores Comunes a Evitar]
\begin{itemize}
\item No usar extensión de archivo en \texttt{\textbackslash includegraphics} (LaTeX la detecta automáticamente)
\item Poner \texttt{\textbackslash label} antes de \texttt{\textbackslash caption}
\item Abusar de la opción [H] (hace el documento menos flexible)
\item No especificar ruta de figuras con \texttt{\textbackslash graphicspath}
\item Usar tamaños absolutos que no escalan bien
\end{itemize}
\end{tcolorbox}

\section{Resumen de Comandos}

\begin{tcolorbox}[colback=gray!10,colframe=gray!75!black,title=Referencia Rápida]
\textbf{Opciones de posición:}
\begin{itemize}
\item \texttt{[h]} - aquí
\item \texttt{[t]} - arriba
\item \texttt{[b]} - abajo
\item \texttt{[p]} - página de flotantes
\item \texttt{[H]} - forzado aquí
\item \texttt{[htbp]} - flexible
\end{itemize}

\textbf{Opciones de includegraphics:}
\begin{itemize}
\item \texttt{width=} - ancho
\item \texttt{height=} - altura
\item \texttt{scale=} - escala
\item \texttt{angle=} - rotación
\item \texttt{keepaspectratio} - mantener proporciones
\item \texttt{trim=, clip} - recortar
\end{itemize}

\textbf{Comandos principales:}
\begin{itemize}
\item \texttt{\textbackslash centering} - centrar
\item \texttt{\textbackslash caption\{\}} - pie de figura
\item \texttt{\textbackslash label\{\}} - etiqueta
\item \texttt{\textbackslash ref\{\}} - referencia
\end{itemize}
\end{tcolorbox}

\vspace{1cm}

\begin{center}
\large\textbf{Fin del Documento}

\vspace{0.5cm}

\normalsize\textit{Ejemplos de aplicación del entorno figure en \LaTeX{}}

\today
\end{center}

\end{document}
