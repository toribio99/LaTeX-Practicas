% !TEX encoding = UTF-8 Unicode
\documentclass[11pt,a4paper]{article}

% Paquetes necesarios
\usepackage[utf8]{inputenc}
\usepackage[spanish]{babel}
\usepackage[margin=2cm]{geometry}
\usepackage{xcolor}
\usepackage{tcolorbox}
\usepackage{enumitem}
\usepackage{fontawesome5}
\usepackage{listings}
\usepackage{paracol}

% Colores personalizados
\definecolor{categorycolor}{RGB}{41,128,185}
\definecolor{commandcolor}{RGB}{39,174,96}
\definecolor{examplecolor}{RGB}{149,165,166}

% Configuración de listings
\lstset{
	basicstyle=\ttfamily\small,
	breaklines=true,
	columns=fullflexible,
	keepspaces=true
}

% Título
\title{\textbf{\Huge Entorno \texttt{paracol} en \LaTeX{}}\\\large Guía Completa de Comandos y Opciones}
\author{}
\date{\today}
 \usepackage[
%colorlinks=true,        % Enlaces con color (en lugar de cajas)
linkcolor=blue,         % Color de enlaces internos
urlcolor=cyan,          % Color de URLs
citecolor=green,        % Color de citas bibliográficas
filecolor=magenta,      % Color de enlaces a archivos
pdfborder={0 0 0},      % Sin bordes en los enlaces
bookmarks=true,         % Crear marcadores en el PDF
bookmarksopen=true,     % Marcadores expandidos al abrir
pdftitle={Mi Título},   % Título del PDF
pdfauthor={Mi Nombre},  % Autor del PDF
pdfsubject={Tema},      % Tema del documento
pdfkeywords={palabra1, palabra2}, % Palabras clave
%hidelinks,              % Ocultar todos los bordes/colores de enlaces
unicode=true,           % Permitir caracteres Unicode en marcadores
breaklinks=true         % Permitir saltos de línea en enlaces
]{hyperref}

\begin{document}
	
	\maketitle
	
	\begin{tcolorbox}[colback=blue!5,colframe=blue!75!black,title=\faInfoCircle\ Introducción]
		El paquete \texttt{paracol} permite crear columnas paralelas con contenido independiente en cada columna. Es ideal para documentos bilingües, comparaciones de texto, o comentarios paralelos. A diferencia de \texttt{multicol}, el contenido NO fluye automáticamente entre columnas.
	\end{tcolorbox}
	
	\tableofcontents
	
	\section{Paquete paracol}
	
	\subsection*{Carga del paquete}
	
	\subsubsection*{\texttt{\textbackslash usepackage\{paracol\}}}
	\begin{tcolorbox}[colback=green!5,colframe=green!50!black]
		\textbf{Descripción:} Carga básica del paquete
		
		\textbf{Ejemplo:}
		\begin{lstlisting}[language=TeX]
			\usepackage{paracol}
		\end{lstlisting}
	\end{tcolorbox}
	
	\subsubsection*{\texttt{\textbackslash usepackage[colaction]\{paracol\}}}
	\begin{tcolorbox}[colback=green!5,colframe=green!50!black]
		\textbf{Descripción:} Habilita acciones sincronizadas entre columnas
		
		\textbf{Ejemplo:}
		\begin{lstlisting}[language=TeX]
			\usepackage[colaction]{paracol}
		\end{lstlisting}
	\end{tcolorbox}
	
	\section{Entorno paracol}
	
	\subsection*{Sintaxis básica}
	\begin{lstlisting}[language=TeX]
		\begin{paracol}{n}
			... contenido columna 1 ...
			\switchcolumn
			... contenido columna 2 ...
		\end{paracol}
	\end{lstlisting}
	
	\subsection{Inicialización}
	
	\subsubsection*{\texttt{\textbackslash begin\{paracol\}\{n\}}}
	\begin{tcolorbox}[colback=green!5,colframe=green!50!black]
		\textbf{Descripción:} Inicia entorno con n columnas (2-10 típicamente)
		
		\textbf{Ejemplo:}
		\begin{lstlisting}[language=TeX]
			\begin{paracol}{2}
				Contenido columna 1
				\switchcolumn
				Contenido columna 2
			\end{paracol}
		\end{lstlisting}
		
		\tcblower
		\faLightbulb\ \textbf{Nota:} n = número de columnas deseadas
	\end{tcolorbox}
	
	\subsubsection*{\texttt{\textbackslash begin\{paracol\}\{2\}} -- Caso común}
	\begin{tcolorbox}[colback=green!5,colframe=green!50!black]
		\textbf{Descripción:} Caso más común: dos columnas
		
		\textbf{Ejemplo:}
		\begin{lstlisting}[language=TeX]
			\begin{paracol}{2}
				Texto izquierda
				\switchcolumn
				Texto derecha
			\end{paracol}
		\end{lstlisting}
	\end{tcolorbox}
	
	\subsubsection*{\texttt{\textbackslash end\{paracol\}}}
	\begin{tcolorbox}[colback=green!5,colframe=green!50!black]
		\textbf{Descripción:} Termina el entorno de columnas paralelas
		
		\textbf{Ejemplo:}
		\begin{lstlisting}[language=TeX]
			\begin{paracol}{2}
				...
			\end{paracol}
		\end{lstlisting}
	\end{tcolorbox}
	
	%\newpage
	
	\section{Comando \textbackslash switchcolumn}
	
	Cambia entre columnas.
	
	\subsection*{\texttt{\textbackslash switchcolumn}}
	\begin{tcolorbox}[colback=green!5,colframe=green!50!black]
		\textbf{Descripción:} Cambia a la siguiente columna
		
		\textbf{Ejemplo:}
		\begin{lstlisting}[language=TeX]
			Columna 1
			\switchcolumn
			Columna 2
			\switchcolumn
			Columna 3
		\end{lstlisting}
	\end{tcolorbox}
	
	\subsection*{\texttt{\textbackslash switchcolumn[n]}}
	\begin{tcolorbox}[colback=green!5,colframe=green!50!black]
		\textbf{Descripción:} Cambia a columna específica n (0-indexed)
		
		\textbf{Ejemplo:}
		\begin{lstlisting}[language=TeX]
			Columna 0
			\switchcolumn[2]
			Salta a columna 2
		\end{lstlisting}
		
		\tcblower
		\faLightbulb\ \textbf{Nota:} Primera columna = 0, segunda = 1, etc.
	\end{tcolorbox}
	
	\subsection*{\texttt{\textbackslash switchcolumn*}}
	\begin{tcolorbox}[colback=green!5,colframe=green!50!black]
		\textbf{Descripción:} Cambia de columna y sincroniza altura
		
		\textbf{Ejemplo:}
		\begin{lstlisting}[language=TeX]
			Texto 1
			\switchcolumn*
			Texto 2 alineado verticalmente
		\end{lstlisting}
		
		\tcblower
		\faLightbulb\ \textbf{Nota:} Útil para alinear contenido horizontalmente
	\end{tcolorbox}
	
	\subsection*{\texttt{\textbackslash switchcolumn*[n]}}
	\begin{tcolorbox}[colback=green!5,colframe=green!50!black]
		\textbf{Descripción:} Cambia a columna n y sincroniza
		
		\textbf{Ejemplo:}
		\begin{lstlisting}[language=TeX]
			\switchcolumn*[1]
			Salta a columna 1 sincronizada
		\end{lstlisting}
	\end{tcolorbox}
	
	\section{Control de Columnas}
	
	\subsection{Ancho de columnas}
	
	\subsubsection*{\texttt{\textbackslash columnratio\{ratio\}}}
	\begin{tcolorbox}[colback=green!5,colframe=green!50!black]
		\textbf{Descripción:} Define proporción de anchos para 2 columnas
		
		\textbf{Ejemplo:}
		\begin{lstlisting}[language=TeX]
			\columnratio{0.6}
			\begin{paracol}{2}
				% Col 1: 60%, Col 2: 40%
			\end{paracol}
		\end{lstlisting}
		
		\tcblower
		\faLightbulb\ \textbf{Nota:} Solo para 2 columnas, valor entre 0 y 1
	\end{tcolorbox}
	
	\subsubsection*{\texttt{\textbackslash columnratio\{r1,r2,...\}}}
	\begin{tcolorbox}[colback=green!5,colframe=green!50!black]
		\textbf{Descripción:} Define proporciones para múltiples columnas
		
		\textbf{Ejemplo:}
		\begin{lstlisting}[language=TeX]
			\columnratio{0.3,0.5,0.2}
			\begin{paracol}{3}
				% 30%, 50%, 20%
			\end{paracol}
		\end{lstlisting}
		
		\tcblower
		\faLightbulb\ \textbf{Nota:} La suma debe ser $\leq$ 1.0
	\end{tcolorbox}
	
	\subsubsection*{\texttt{\textbackslash setcolumnwidth\{w1,w2,...\}}}
	\begin{tcolorbox}[colback=green!5,colframe=green!50!black]
		\textbf{Descripción:} Define anchos absolutos de columnas
		
		\textbf{Ejemplo:}
		\begin{lstlisting}[language=TeX]
			\setcolumnwidth{4cm,6cm,5cm}
			\begin{paracol}{3}
				...
			\end{paracol}
		\end{lstlisting}
		
		\tcblower
		\faLightbulb\ \textbf{Nota:} Usar unidades: cm, pt, in, etc.
	\end{tcolorbox}
	
	\subsection{Separación entre columnas}
	
	\subsubsection*{\texttt{\textbackslash columnsep}}
	\begin{tcolorbox}[colback=green!5,colframe=green!50!black]
		\textbf{Descripción:} Espacio horizontal entre columnas
		
		\textbf{Ejemplo:}
		\begin{lstlisting}[language=TeX]
			\setlength{\columnsep}{2cm}
			\begin{paracol}{2}
				...
			\end{paracol}
		\end{lstlisting}
		
		\tcblower
		\faLightbulb\ \textbf{Nota:} Por defecto: 35pt
	\end{tcolorbox}
	
	\subsubsection*{\texttt{\textbackslash columnseprule}}
	\begin{tcolorbox}[colback=green!5,colframe=green!50!black]
		\textbf{Descripción:} Grosor de línea separadora
		
		\textbf{Ejemplo:}
		\begin{lstlisting}[language=TeX]
			\setlength{\columnseprule}{0.4pt}
			\begin{paracol}{2}
				...
			\end{paracol}
		\end{lstlisting}
		
		\tcblower
		\faLightbulb\ \textbf{Nota:} 0pt = sin línea (por defecto)
	\end{tcolorbox}
	
	\subsubsection*{\texttt{\textbackslash columnseprulecolor}}
	\begin{tcolorbox}[colback=green!5,colframe=green!50!black]
		\textbf{Descripción:} Color de la línea separadora
		
		\textbf{Ejemplo:}
		\begin{lstlisting}[language=TeX]
			\renewcommand{\columnseprulecolor}{\color{blue}}
			\setlength{\columnseprule}{1pt}
		\end{lstlisting}
		
		\tcblower
		\faLightbulb\ \textbf{Nota:} Requiere xcolor
	\end{tcolorbox}
	
	\subsection{Numeración de columnas}
	
	\subsubsection*{\texttt{\textbackslash globalcounter\{counter\}}}
	\begin{tcolorbox}[colback=green!5,colframe=green!50!black]
		\textbf{Descripción:} Contador global (compartido entre columnas)
		
		\textbf{Ejemplo:}
		\begin{lstlisting}[language=TeX]
			\globalcounter{equation}
			\begin{paracol}{2}
				...
			\end{paracol}
		\end{lstlisting}
		
		\tcblower
		\faLightbulb\ \textbf{Nota:} Ecuaciones numeradas consecutivamente
	\end{tcolorbox}
	
	\subsubsection*{\texttt{\textbackslash globalcounter*}}
	\begin{tcolorbox}[colback=green!5,colframe=green!50!black]
		\textbf{Descripción:} Todos los contadores como globales
		
		\textbf{Ejemplo:}
		\begin{lstlisting}[language=TeX]
			\globalcounter*
			\begin{paracol}{2}
				...
			\end{paracol}
		\end{lstlisting}
	\end{tcolorbox}
	
	\subsubsection*{\texttt{\textbackslash localcounter\{counter\}}}
	\begin{tcolorbox}[colback=green!5,colframe=green!50!black]
		\textbf{Descripción:} Contador local (independiente por columna)
		
		\textbf{Ejemplo:}
		\begin{lstlisting}[language=TeX]
			\localcounter{footnote}
			\begin{paracol}{2}
				...
			\end{paracol}
		\end{lstlisting}
		
		\tcblower
		\faLightbulb\ \textbf{Nota:} Notas al pie independientes
	\end{tcolorbox}
	
	%\newpage
	
	\section{Características Avanzadas}
	
	\subsection{Fondo de columnas}
	
	\subsubsection*{\texttt{\textbackslash backgroundcolor\{c\}[color]}}
	\begin{tcolorbox}[colback=green!5,colframe=green!50!black]
		\textbf{Descripción:} Define color de fondo para columna c
		
		\textbf{Ejemplo:}
		\begin{lstlisting}[language=TeX]
			\usepackage{xcolor}
			\backgroundcolor{c}[gray]{0.9}
			\begin{paracol}{2}
				...
			\end{paracol}
		\end{lstlisting}
		
		\tcblower
		\faLightbulb\ \textbf{Nota:} c = 0,1,2... (columna), requiere xcolor
	\end{tcolorbox}
	
	\subsubsection*{\texttt{\textbackslash backgroundcolor\{c\}(color)}}
	\begin{tcolorbox}[colback=green!5,colframe=green!50!black]
		\textbf{Descripción:} Color de fondo con modelo de color
		
		\textbf{Ejemplo:}
		\begin{lstlisting}[language=TeX]
			\backgroundcolor{0}(rgb){1,0.9,0.9}
			\begin{paracol}{2}
				...
			\end{paracol}
		\end{lstlisting}
	\end{tcolorbox}
	
	\subsubsection*{\texttt{\textbackslash nobackgroundcolor\{c\}}}
	\begin{tcolorbox}[colback=green!5,colframe=green!50!black]
		\textbf{Descripción:} Elimina color de fondo de columna c
		
		\textbf{Ejemplo:}
		\begin{lstlisting}[language=TeX]
			\nobackgroundcolor{1}
			\begin{paracol}{2}
				...
			\end{paracol}
		\end{lstlisting}
	\end{tcolorbox}
	
	\subsection{Notas al pie}
	
	\subsubsection*{\texttt{\textbackslash footnotelayout\{mode\}}}
	\begin{tcolorbox}[colback=green!5,colframe=green!50!black]
		\textbf{Descripción:} Define disposición de notas al pie
		
		\textbf{Ejemplo:}
		\begin{lstlisting}[language=TeX]
			\footnotelayout{m}
			\begin{paracol}{2}
				Texto\footnote{Nota}
			\end{paracol}
		\end{lstlisting}
		
		\tcblower
		\faLightbulb\ \textbf{Nota:} m=merged, c=columned, p=page
	\end{tcolorbox}
	
	\subsubsection*{\texttt{\textbackslash footnote\{text\}}}
	\begin{tcolorbox}[colback=green!5,colframe=green!50!black]
		\textbf{Descripción:} Nota al pie normal en paracol
		
		\textbf{Ejemplo:}
		\begin{lstlisting}[language=TeX]
			Texto\footnote{Esta es una nota}
		\end{lstlisting}
	\end{tcolorbox}
	
	\subsubsection*{\texttt{\textbackslash footnotemark, \textbackslash footnotetext}}
	\begin{tcolorbox}[colback=green!5,colframe=green!50!black]
		\textbf{Descripción:} Marca y texto separados
		
		\textbf{Ejemplo:}
		\begin{lstlisting}[language=TeX]
			Texto\footnotemark
			\switchcolumn
			\footnotetext{Nota en otra columna}
		\end{lstlisting}
	\end{tcolorbox}
	
	\subsection{Saltos de página}
	
	\subsubsection*{\texttt{\textbackslash flushpage}}
	\begin{tcolorbox}[colback=green!5,colframe=green!50!black]
		\textbf{Descripción:} Fuerza que todas las columnas lleguen al final de página
		
		\textbf{Ejemplo:}
		\begin{lstlisting}[language=TeX]
			\begin{paracol}{2}
				Texto columna 1
				\switchcolumn
				Texto columna 2
				\flushpage
			\end{paracol}
		\end{lstlisting}
		
		\tcblower
		\faLightbulb\ \textbf{Nota:} Útil antes de cambiar configuración
	\end{tcolorbox}
	
	\subsubsection*{\texttt{\textbackslash clearpage}}
	\begin{tcolorbox}[colback=green!5,colframe=green!50!black]
		\textbf{Descripción:} Termina página actual en modo paracol
		
		\textbf{Ejemplo:}
		\begin{lstlisting}[language=TeX]
			\begin{paracol}{2}
				...
				\clearpage
				Nueva página
			\end{paracol}
		\end{lstlisting}
	\end{tcolorbox}
	
	\subsection{Sincronización}
	
	\subsubsection*{\texttt{\textbackslash synccounter\{counter\}}}
	\begin{tcolorbox}[colback=green!5,colframe=green!50!black]
		\textbf{Descripción:} Sincroniza contador entre columnas
		
		\textbf{Ejemplo:}
		\begin{lstlisting}[language=TeX]
			\synccounter{equation}
			\begin{paracol}{2}
				...
			\end{paracol}
		\end{lstlisting}
	\end{tcolorbox}
	
	\subsubsection*{\texttt{\textbackslash syncallcounters}}
	\begin{tcolorbox}[colback=green!5,colframe=green!50!black]
		\textbf{Descripción:} Sincroniza todos los contadores
		
		\textbf{Ejemplo:}
		\begin{lstlisting}[language=TeX]
			\syncallcounters
			\begin{paracol}{2}
				...
			\end{paracol}
		\end{lstlisting}
	\end{tcolorbox}
	
	%\newpage
	
	\section{Contenido que Abarca Columnas}
	
	\subsection{Entornos especiales}
	
	\subsubsection*{\texttt{\textbackslash begin\{leftcolumn*\}}}
	\begin{tcolorbox}[colback=green!5,colframe=green!50!black]
		\textbf{Descripción:} Solo en columna izquierda, ancho completo
		
		\textbf{Ejemplo:}
		\begin{lstlisting}[language=TeX]
			\begin{paracol}{2}
				\begin{leftcolumn*}
					Título ancho completo
				\end{leftcolumn*}
				Columna 1
				\switchcolumn
				Columna 2
			\end{paracol}
		\end{lstlisting}
	\end{tcolorbox}
	
	\subsubsection*{\texttt{\textbackslash begin\{rightcolumn*\}}}
	\begin{tcolorbox}[colback=green!5,colframe=green!50!black]
		\textbf{Descripción:} Solo en columna derecha, ancho completo
		
		\textbf{Ejemplo:}
		\begin{lstlisting}[language=TeX]
			\begin{rightcolumn*}
				Contenido
			\end{rightcolumn*}
		\end{lstlisting}
	\end{tcolorbox}
	
	\subsubsection*{\texttt{\textbackslash begin\{column*\}}}
	\begin{tcolorbox}[colback=green!5,colframe=green!50!black]
		\textbf{Descripción:} Contenido que abarca todas las columnas
		
		\textbf{Ejemplo:}
		\begin{lstlisting}[language=TeX]
			\begin{paracol}{3}
				Col1
				\switchcolumn
				Col2
				\switchcolumn
				Col3
				\switchcolumn*
				\begin{column*}
					Título completo
				\end{column*}
			\end{paracol}
		\end{lstlisting}
		
		\tcblower
		\faLightbulb\ \textbf{Nota:} Muy útil para títulos y secciones
	\end{tcolorbox}
	
	%\newpage
	
	\section*{\faCheckCircle\ Ejemplos Completos}
	
	\subsection*{Ejemplo 1: Básico (dos columnas)}
	
	\begin{tcolorbox}[colback=green!10,colframe=green!75!black,title=\faCode\ Dos columnas simples]
		\begin{lstlisting}[language=TeX]
			\usepackage{paracol}
			
			\begin{paracol}{2}
				Este texto aparece en la columna izquierda.
				Puedo escribir varios párrafos aquí.
				
				\switchcolumn
				
				Este texto aparece en la columna derecha.
				Completamente independiente del izquierdo.
			\end{paracol}
		\end{lstlisting}
	\end{tcolorbox}
	
	\subsection*{Ejemplo 2: Avanzado (3 columnas con colores)}
	
	\begin{tcolorbox}[colback=purple!10,colframe=purple!75!black,title=\faCode\ Tres columnas con fondos de color]
		\begin{lstlisting}[language=TeX]
			\usepackage{paracol}
			\usepackage{xcolor}
			
			\columnratio{0.3,0.4,0.3}
			\setlength{\columnseprule}{0.4pt}
			\backgroundcolor{0}[gray]{0.95}
			\backgroundcolor{2}[gray]{0.95}
			
			\begin{paracol}{3}
				Columna 1 (30\%)
				Con fondo gris
				
				\switchcolumn
				
				Columna 2 (40\%)
				Sin fondo
				
				\switchcolumn
				
				Columna 3 (30\%)
				Con fondo gris
			\end{paracol}
		\end{lstlisting}
	\end{tcolorbox}
	
	\subsection*{Ejemplo 3: Práctico (documento bilingüe)}
	
	\begin{tcolorbox}[colback=orange!10,colframe=orange!75!black,title=\faCode\ Documento bilingüe]
		\begin{lstlisting}[language=TeX]
			\usepackage{paracol}
			\columnratio{0.5}
			
			\begin{paracol}{2}
				\section{Introduction}
				This is the English version of the text.
				We can have complete paragraphs here.
				
				\switchcolumn
				
				\section{Introducción}
				Esta es la versión en español del texto.
				Podemos tener párrafos completos aquí.
				
				\switchcolumn*
				
				\begin{column*}
					\section{Common Section}
					This section spans both columns.
				\end{column*}
			\end{paracol}
		\end{lstlisting}
	\end{tcolorbox}
	
	%\newpage
	
	\section*{\faLightbulb\ Tips Importantes}
	
	\begin{tcolorbox}[colback=blue!10,colframe=blue!75!black]
		\begin{itemize}[leftmargin=*]
			\item Usa \texttt{\textbackslash switchcolumn*} para sincronizar altura entre columnas
			\item \texttt{\textbackslash flushpage} antes de cambiar configuración de columnas
			\item \texttt{\textbackslash globalcounter} para ecuaciones numeradas consecutivamente
			\item Primera columna es índice 0, no 1
			\item Para documentos bilingües, paracol es ideal
			\item Los anchos con \texttt{\textbackslash columnratio} deben sumar $\leq$ 1.0
			\item Puedes anidar entornos normales (figuras, ecuaciones) dentro de paracol
		\end{itemize}
	\end{tcolorbox}
	
	\section*{\faExclamationTriangle\ Diferencias paracol vs multicol}
	
	\begin{tcolorbox}[colback=yellow!10,colframe=orange!75!black,title=paracol vs multicol]
		\begin{description}[leftmargin=*]
			\item[\textbf{multicol}:] Contenido fluye automáticamente entre columnas (como periódico). El texto se distribuye de forma balanceada.
			\item[\textbf{paracol}:] Contenido independiente en cada columna. Tú controlas qué va en cada columna con \texttt{\textbackslash switchcolumn}.
		\end{description}
		
		\textbf{Cuándo usar paracol:}
		\begin{itemize}[leftmargin=*]
			\item Documentos bilingües (texto en dos idiomas lado a lado)
			\item Comparación de versiones (código antiguo vs nuevo)
			\item Comentarios paralelos al texto principal
			\item CVs con columnas independientes
			\item Notas marginales extensas
		\end{itemize}
		
		\textbf{Cuándo usar multicol:}
		\begin{itemize}[leftmargin=*]
			\item Artículos estilo revista/periódico
			\item Listas largas en múltiples columnas
			\item Referencias bibliográficas
			\item Cuando quieres balanceo automático
		\end{itemize}
	\end{tcolorbox}
	
	\vspace{1cm}
	
	\begin{center}
		\textit{Documento generado con \LaTeX{} -- \today}
	\end{center}
	
\end{document}