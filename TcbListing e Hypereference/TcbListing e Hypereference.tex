\documentclass[11pt,a4paper]{article}

% Paquetes necesarios
\usepackage{ifxetex,ifluatex}
\ifxetex
  % XeLaTeX no necesita inputenc
\else
  \ifluatex
    % LuaLaTeX no necesita inputenc
  \else
    % pdfLaTeX necesita inputenc
    \usepackage[utf8]{inputenc}
  \fi
\fi
\usepackage[T1]{fontenc}
\usepackage[spanish]{babel}
\usepackage[margin=2.5cm]{geometry}
\usepackage{amsmath}
\usepackage{xcolor}
\usepackage{tcolorbox}
\tcbuselibrary{listings,skins,breakable}
\usepackage{listings}

% Definición de colores
\definecolor{azuloscuro}{RGB}{0,51,102}
\definecolor{azulclaro}{RGB}{230,240,250}
\definecolor{verdeoscuro}{RGB}{0,100,0}
\definecolor{rojoclaro}{RGB}{255,230,230}
\definecolor{grisclaro}{RGB}{240,240,240}

% Configuración de listings para código
\lstset{
    basicstyle=\ttfamily\small,
    breaklines=true,
    columns=flexible,
    showstringspaces=false
}

% IMPORTANTE: hyperref debe ir casi al final
\usepackage[
    colorlinks=true,
    linkcolor=azuloscuro,
    urlcolor=blue,
    citecolor=verdeoscuro,
    pdftitle={Guía de tcblisting e hyperref},
    pdfauthor={Documentación LaTeX},
    pdfsubject={Paquetes LaTeX},
    bookmarks=true,
    bookmarksopen=true,
    unicode=true
]{hyperref}

% Título
\title{\textbf{Guía Práctica:} \\
\large Entorno \texttt{tcblisting} y Paquete \texttt{hyperref}}
\author{Documentación de LaTeX}
\date{\today}

\begin{document}

\maketitle

\tableofcontents
\newpage

% =====================================
% PARTE 1: TCBLISTING
% =====================================

\section{El Entorno \texttt{tcblisting}}

\subsection{?`Qué es \texttt{tcblisting}?}

El entorno \texttt{tcblisting} es parte del paquete \texttt{tcolorbox} y se utiliza para \textbf{mostrar código fuente con formato dentro de una caja coloreada y estilizada}.

\subsection{Características Principales}

\begin{itemize}
    \item Combina cajas estilizadas (\texttt{tcolorbox}) con listados de código (\texttt{listings})
    \item Permite resaltado de sintaxis
    \item Altamente personalizable en colores, bordes y estilos
    \item Soporta múltiples lenguajes de programación
    \item Puede mostrar código y su salida compilada simultáneamente
\end{itemize}

\subsection{Modos de Uso}

\subsubsection{Modo 1: Solo Código (\texttt{listing only})}

Muestra únicamente el código fuente sin compilarlo.

\begin{tcblisting}{
    listing only,
    title=Ejemplo: Código Python,
    colback=grisclaro,
    colframe=azuloscuro,
    fonttitle=\bfseries,
    listing options={language=Python}
}
def factorial(n):
    if n == 0:
        return 1
    return n * factorial(n-1)

print(factorial(5))
\end{tcblisting}

\subsubsection{Modo 2: Código LaTeX con Salida (\texttt{text side listing})}

Muestra el código LaTeX a la izquierda y el resultado compilado a la derecha.

\begin{tcblisting}{
    text side listing,
    title=Ejemplo: Ecuación,
    colback=azulclaro,
    colframe=azuloscuro,
    fonttitle=\bfseries
}
La fórmula de la energía cinética es:
\begin{equation*}
    E_k = \frac{1}{2}mv^2
\end{equation*}
\end{tcblisting}

\subsubsection{Modo 3: Solo Salida (\texttt{text only})}

Muestra únicamente el resultado compilado, sin el código.

\begin{tcblisting}{
    text only,
    title=Ejemplo: Solo Resultado,
    colback=rojoclaro,
    colframe=red!70!black,
    fonttitle=\bfseries
}
\textbf{Teorema de Pitágoras:}
\[
    a^2 + b^2 = c^2
\]
\end{tcblisting}

\subsection{Opciones Principales de \texttt{tcblisting}}

\subsubsection{Opción 1: \texttt{listing only}}

\textbf{Descripción:} Muestra únicamente el código fuente sin compilarlo ni mostrar su salida.

\textbf{Uso:} Para mostrar código de cualquier lenguaje de programación o LaTeX sin ejecutarlo.

\textbf{Ejemplo de uso:}
\begin{verbatim}
\begin{tcblisting}{listing only}
    código aquí
\end{tcblisting}
\end{verbatim}

\subsubsection{Opción 2: \texttt{text side listing}}

\textbf{Descripción:} Muestra el código LaTeX a la izquierda y el resultado compilado a la derecha, en disposición horizontal lado a lado.

\textbf{Uso:} Ideal para tutoriales de LaTeX donde se quiere mostrar el código y su resultado simultáneamente.

\textbf{Ejemplo de uso:}
\begin{verbatim}
\begin{tcblisting}{text side listing}
    código LaTeX aquí
\end{tcblisting}
\end{verbatim}

\subsubsection{Opción 3: \texttt{listing and text}}

\textbf{Descripción:} Muestra el código LaTeX arriba y el resultado compilado abajo, en disposición vertical.

\textbf{Uso:} Similar a \texttt{text side listing} pero con disposición vertical.

\textbf{Ejemplo de uso:}
\begin{verbatim}
\begin{tcblisting}{listing and text}
    código LaTeX aquí
\end{tcblisting}
\end{verbatim}

\subsubsection{Opción 4: \texttt{text only}}

\textbf{Descripción:} Muestra únicamente el resultado compilado del código LaTeX, sin mostrar el código fuente.

\textbf{Uso:} Para mostrar solo la salida de código LaTeX en una caja estilizada.

\textbf{Ejemplo de uso:}
\begin{verbatim}
\begin{tcblisting}{text only}
    código LaTeX que se compilará
\end{tcblisting}
\end{verbatim}

\subsubsection{Opción 5: \texttt{title}}

\textbf{Descripción:} Define el título que aparece en la parte superior de la caja.

\textbf{Valores:} Cualquier texto

\textbf{Ejemplo de uso:}
\begin{verbatim}
\begin{tcblisting}{title=Mi Código Python}
    código aquí
\end{tcblisting}
\end{verbatim}

\subsubsection{Opción 6: \texttt{colback}}

\textbf{Descripción:} Define el color de fondo de la caja.

\textbf{Valores:} Cualquier color definido o mezclas (ej: \texttt{yellow!10}, \texttt{blue!20})

\textbf{Ejemplo de uso:}
\begin{verbatim}
\begin{tcblisting}{colback=yellow!10}
    código aquí
\end{tcblisting}
\end{verbatim}

\subsubsection{Opción 7: \texttt{colframe}}

\textbf{Descripción:} Define el color del borde/marco de la caja.

\textbf{Valores:} Cualquier color definido o mezclas

\textbf{Ejemplo de uso:}
\begin{verbatim}
\begin{tcblisting}{colframe=red!75!black}
    código aquí
\end{tcblisting}
\end{verbatim}

\subsubsection{Opción 8: \texttt{fonttitle}}

\textbf{Descripción:} Define el estilo de la fuente del título.

\textbf{Valores:} Comandos de fuente como \texttt{\textbackslash bfseries}, \texttt{\textbackslash itshape}, etc.

\textbf{Ejemplo de uso:}
\begin{verbatim}
\begin{tcblisting}{fonttitle=\bfseries}
    código aquí
\end{tcblisting}
\end{verbatim}

\subsubsection{Opción 9: \texttt{listing options}}

\textbf{Descripción:} Permite pasar opciones al paquete \texttt{listings} para controlar el resaltado de sintaxis.

\textbf{Valores:} Opciones de \texttt{listings} como \texttt{language}, \texttt{numbers}, \texttt{numberstyle}, etc.

\textbf{Ejemplo de uso:}
\begin{verbatim}
\begin{tcblisting}{listing options={language=Python, numbers=left}}
    código aquí
\end{tcblisting}
\end{verbatim}

\textbf{Sub-opciones comunes:}
\begin{itemize}
    \item \texttt{language=Python/C/Java/...} - Define el lenguaje de programación
    \item \texttt{numbers=left/right/none} - Posición de números de línea
    \item \texttt{numberstyle=\textbackslash tiny} - Estilo de los números de línea
    \item \texttt{stepnumber=1} - Incremento de numeración
    \item \texttt{basicstyle=\textbackslash ttfamily} - Estilo básico del código
\end{itemize}

\subsubsection{Opción 10: \texttt{label}}

\textbf{Descripción:} Asigna una etiqueta para referencias cruzadas.

\textbf{Valores:} Cualquier identificador único

\textbf{Ejemplo de uso:}
\begin{verbatim}
\begin{tcblisting}{label=code:ejemplo}
    código aquí
\end{tcblisting}
\end{verbatim}

Luego se puede referenciar con \texttt{\textbackslash ref\{code:ejemplo\}}.

\subsubsection{Opción 11: \texttt{boxrule}}

\textbf{Descripción:} Define el grosor del borde de la caja.

\textbf{Valores:} Dimensión en pt, mm, cm, etc.

\textbf{Ejemplo de uso:}
\begin{verbatim}
\begin{tcblisting}{boxrule=2pt}
    código aquí
\end{tcblisting}
\end{verbatim}

\subsubsection{Opción 12: \texttt{sharp corners}}

\textbf{Descripción:} Hace que las esquinas de la caja sean rectangulares en lugar de redondeadas.

\textbf{Valores:} No requiere valor, es una opción booleana

\textbf{Ejemplo de uso:}
\begin{verbatim}
\begin{tcblisting}{sharp corners}
    código aquí
\end{tcblisting}
\end{verbatim}

\subsubsection{Opción 13: \texttt{arc}}

\textbf{Descripción:} Define el radio de curvatura de las esquinas redondeadas.

\textbf{Valores:} Dimensión (por defecto 4mm)

\textbf{Ejemplo de uso:}
\begin{verbatim}
\begin{tcblisting}{arc=1mm}
    código aquí
\end{tcblisting}
\end{verbatim}

\subsubsection{Opción 14: \texttt{breakable}}

\textbf{Descripción:} Permite que la caja se divida entre páginas si es muy larga.

\textbf{Valores:} No requiere valor

\textbf{Ejemplo de uso:}
\begin{verbatim}
\begin{tcblisting}{breakable}
    código muy largo aquí
\end{tcblisting}
\end{verbatim}

\subsection{Ejemplos de Personalización}

\subsubsection{Colores de Fondo y Marco}

\begin{tcblisting}{
    listing only,
    title=Caja con Colores Personalizados,
    colback=yellow!10,
    colframe=red!75!black,
    listing options={language=C}
}
#include <stdio.h>
int main() {
    printf("Hola Mundo\n");
    return 0;
}
\end{tcblisting}

\subsubsection{Sin Título}

\begin{tcblisting}{
    listing only,
    colback=green!5,
    colframe=verdeoscuro,
    listing options={language=Java}
}
public class Saludo {
    public static void main(String[] args) {
        System.out.println("Hola");
    }
}
\end{tcblisting}

\subsubsection{Con Líneas Numeradas}

\begin{tcblisting}{
    listing only,
    title=Código con Números de Línea,
    colback=grisclaro,
    colframe=azuloscuro,
    fonttitle=\bfseries,
    listing options={
        language=Python,
        numbers=left,
        numberstyle=\tiny,
        stepnumber=1
    }
}
def suma(a, b):
    resultado = a + b
    return resultado

x = suma(5, 3)
print(f"La suma es: {x}")
\end{tcblisting}

\subsubsection{Disposición Vertical}

\begin{tcblisting}{
    listing and text,
    title={Código Arriba - Resultado Abajo},
    colback=azulclaro,
    colframe=azuloscuro,
    fonttitle=\bfseries
}
Vector de aceleración:
\[
    \vec{a} = \frac{d\vec{v}}{dt}
\]
\end{tcblisting}

\newpage

% =====================================
% PARTE 2: HYPERREF
% =====================================

\section{El Paquete \texttt{hyperref}}

\subsection{?`Qué es \texttt{hyperref}?}

El paquete \texttt{hyperref} permite crear \textbf{hipervínculos interactivos en documentos PDF}, haciendo que las referencias, URLs y el índice sean clickeables.

\subsection{Funcionalidades Principales}

\begin{enumerate}
    \item \textbf{Enlaces internos:} Referencias cruzadas clickeables (ecuaciones, figuras, secciones)
    \item \textbf{Enlaces externos:} URLs clickeables hacia sitios web
    \item \textbf{Tabla de contenidos interactiva:} Índice navegable
    \item \textbf{Metadatos del PDF:} Información del documento
    \item \textbf{Marcadores (bookmarks):} Panel lateral de navegación
\end{enumerate}

\subsection{Opciones Principales de \texttt{hyperref}}

\subsubsection{Opción 1: \texttt{colorlinks}}

\textbf{Descripción:} Muestra los enlaces con colores en lugar de cajas rectangulares.

\textbf{Valores:} \texttt{true} o \texttt{false}

\textbf{Ejemplo de uso:}
\begin{verbatim}
\usepackage[colorlinks=true]{hyperref}
\end{verbatim}

\textbf{Resultado:} Los enlaces en este documento tienen colores (azul, verde) en lugar de bordes.

\subsubsection{Opción 2: \texttt{linkcolor}}

\textbf{Descripción:} Define el color de los enlaces internos (referencias a secciones, ecuaciones, figuras).

\textbf{Valores:} Cualquier color definido (red, blue, black, o colores personalizados)

\textbf{Ejemplo:} Haz clic para ir a la \hyperref[sec:tcb]{Sección 1 (tcblisting)}.

\begin{verbatim}
\usepackage[linkcolor=azuloscuro]{hyperref}
\end{verbatim}

\subsubsection{Opción 3: \texttt{urlcolor}}

\textbf{Descripción:} Define el color de las URLs externas.

\textbf{Ejemplo:} Visita \url{https://www.latex-project.org}

\begin{verbatim}
\usepackage[urlcolor=blue]{hyperref}
\end{verbatim}

\subsubsection{Opción 4: \texttt{citecolor}}

\textbf{Descripción:} Define el color de las citas bibliográficas.

\textbf{Ejemplo de código:}
\begin{verbatim}
\usepackage[citecolor=green]{hyperref}
\end{verbatim}

\subsubsection{Opción 5: \texttt{pdftitle}}

\textbf{Descripción:} Define el título del documento PDF (visible en propiedades del archivo).

\textbf{Ejemplo:}
\begin{verbatim}
\usepackage[pdftitle={Mi Documento}]{hyperref}
\end{verbatim}

\textbf{Resultado:} El PDF tendrá ``Mi Documento'' como título en sus metadatos.

\subsubsection{Opción 6: \texttt{pdfauthor}}

\textbf{Descripción:} Define el autor del documento PDF.

\textbf{Ejemplo:}
\begin{verbatim}
\usepackage[pdfauthor={Juan Pérez}]{hyperref}
\end{verbatim}

\subsubsection{Opción 7: \texttt{pdfsubject}}

\textbf{Descripción:} Define el tema o asunto del documento.

\textbf{Ejemplo:}
\begin{verbatim}
\usepackage[pdfsubject={Física Cuántica}]{hyperref}
\end{verbatim}

\subsubsection{Opción 8: \texttt{pdfkeywords}}

\textbf{Descripción:} Define palabras clave para el documento (útil para búsquedas).

\textbf{Ejemplo:}
\begin{verbatim}
\usepackage[pdfkeywords={LaTeX, física, mecánica}]{hyperref}
\end{verbatim}

\subsubsection{Opción 9: \texttt{bookmarks}}

\textbf{Descripción:} Habilita o deshabilita los marcadores laterales en el PDF.

\textbf{Valores:} \texttt{true} o \texttt{false}

\textbf{Ejemplo:}
\begin{verbatim}
\usepackage[bookmarks=true]{hyperref}
\end{verbatim}

\textbf{Resultado:} Este PDF tiene marcadores en el panel lateral izquierdo.

\subsubsection{Opción 10: \texttt{bookmarksopen}}

\textbf{Descripción:} Los marcadores aparecen expandidos al abrir el PDF.

\textbf{Ejemplo:}
\begin{verbatim}
\usepackage[bookmarksopen=true]{hyperref}
\end{verbatim}

\subsubsection{Opción 11: \texttt{hidelinks}}

\textbf{Descripción:} Oculta todos los colores y bordes de los enlaces (enlaces invisibles pero funcionales).

\textbf{Ejemplo:}
\begin{verbatim}
\usepackage[hidelinks]{hyperref}
\end{verbatim}

\textbf{Uso recomendado:} Para documentos impresos donde no se desean colores en los enlaces.

\subsubsection{Opción 12: \texttt{unicode}}

\textbf{Descripción:} Permite caracteres Unicode en los marcadores del PDF.

\textbf{Ejemplo:}
\begin{verbatim}
\usepackage[unicode=true]{hyperref}
\end{verbatim}

\textbf{Uso:} Necesario para caracteres especiales como ñ, á, é, ü en marcadores.

\subsubsection{Opción 13: \texttt{breaklinks}}

\textbf{Descripción:} Permite que los enlaces largos se dividan en varias líneas.

\textbf{Ejemplo:}
\begin{verbatim}
\usepackage[breaklinks=true]{hyperref}
\end{verbatim}

\subsubsection{Opción 14: \texttt{pdfborder}}

\textbf{Descripción:} Define el estilo del borde de los enlaces.

\textbf{Valores:} \texttt{\{ancho alto profundidad\}}

\textbf{Ejemplo sin bordes:}
\begin{verbatim}
\usepackage[pdfborder={0 0 0}]{hyperref}
\end{verbatim}

\subsection{Comandos Útiles de \texttt{hyperref}}

\subsubsection{Comando \texttt{\textbackslash url}}

Crea un enlace a una URL mostrando la URL completa.

\textbf{Sintaxis:}
\begin{verbatim}
\url{https://www.ejemplo.com}
\end{verbatim}

\textbf{Ejemplo:} \url{https://www.overleaf.com}

\subsubsection{Comando \texttt{\textbackslash href}}

Crea un enlace con texto personalizado.

\textbf{Sintaxis:}
\begin{verbatim}
\href{URL}{texto a mostrar}
\end{verbatim}

\textbf{Ejemplo:} Visita \href{https://www.latex-project.org}{el sitio oficial de LaTeX}.

\subsubsection{Comando \texttt{\textbackslash hyperref}}

Crea un enlace interno personalizado.

\textbf{Sintaxis:}
\begin{verbatim}
\hyperref[etiqueta]{texto}
\end{verbatim}

\textbf{Ejemplo:} Ve a la \hyperref[sec:tcb]{sección de tcblisting}.

\subsubsection{Comando \texttt{\textbackslash autoref}}

Crea una referencia automática con el tipo de objeto.

\textbf{Sintaxis:}
\begin{verbatim}
\autoref{etiqueta}
\end{verbatim}

\textbf{Ejemplo con ecuación:}

\begin{equation}
    E = mc^2
    \label{eq:einstein}
\end{equation}

Referencia: Como se ve en \autoref{eq:einstein}, la energía y masa están relacionadas.

\subsection{Configuración Completa Recomendada}

Para documentos académicos de física (como tus ejercicios):

\begin{verbatim}
\usepackage[
    colorlinks=true,
    linkcolor=azuloscuro,
    urlcolor=blue,
    citecolor=verdeoscuro,
    pdftitle={Solución Ejercicio 3.35},
    pdfauthor={Física Universitaria},
    pdfsubject={Movimiento Circular},
    pdfkeywords={física, cinemática, LaTeX},
    bookmarks=true,
    bookmarksopen=true,
    unicode=true,
    breaklinks=true
]{hyperref}
\end{verbatim}

\subsection{IMPORTANTE: Orden de Carga}

El paquete \texttt{hyperref} debe cargarse \textbf{casi al final del preámbulo}, después de la mayoría de los paquetes.

\textbf{Orden correcto:}

\begin{verbatim}
% Primero: paquetes básicos
\usepackage{amsmath}
\usepackage{graphicx}
\usepackage{tikz}
\usepackage{tcolorbox}

% Casi al final: hyperref
\usepackage[opciones]{hyperref}

% Después de hyperref (solo estos)
\usepackage{cleveref}  % si se usa
\end{verbatim}

\section{Ejemplos Combinados}

\subsection{Ejemplo 1: Código con Enlace}

\begin{tcblisting}{
    text side listing,
    title=Aceleración Centrípeta,
    colback=azulclaro,
    colframe=azuloscuro,
    fonttitle=\bfseries
}
\label{sec:tcb}
La aceleración centrípeta es:
\begin{equation*}
    a_c = \frac{v^2}{R}
\end{equation*}
Para más información, visita
\href{https://es.wikipedia.org/wiki/Aceleraci\%C3\%B3n_centr\%C3\%ADpeta}{Wikipedia}.
\end{tcblisting}

\subsection{Ejemplo 2: Código Python con Referencia}

\begin{tcblisting}{
    listing only,
    title=Programa de Física,
    colback=green!5,
    colframe=verdeoscuro,
    fonttitle=\bfseries,
    label={code:python},
    listing options={language=Python}
}
import numpy as np

# Calcular velocidad
R = 14.0  # metros
omega = 0.5  # rad/s
v = omega * R
print(f"Velocidad: {v} m/s")
\end{tcblisting}

Puedes ver el código Python en el cuadro anterior (referencia interna funcional si se configura).

\section{Resumen Comparativo}

\begin{center}
\begin{tabular}{|p{4cm}|p{10cm}|}
\hline
\textbf{Característica} & \textbf{Descripción} \\
\hline
\texttt{tcblisting} & Entorno para mostrar código con cajas estilizadas. Parte de \texttt{tcolorbox}. \\
\hline
\texttt{hyperref} & Paquete para crear hipervínculos y hacer PDFs interactivos. \\
\hline
Uso conjunto & Se pueden combinar: código en cajas con enlaces a documentación externa. \\
\hline
Orden de carga & \texttt{tcolorbox} antes, \texttt{hyperref} casi al final del preámbulo. \\
\hline
\end{tabular}
\end{center}

\section{Conclusiones}

\begin{itemize}
    \item \texttt{tcblisting} es ideal para tutoriales, documentación técnica y presentaciones de código.
    \item \texttt{hyperref} mejora significativamente la navegación en documentos PDF largos.
    \item Ambos son altamente personalizables y se complementan bien.
    \item Siempre cargar \texttt{hyperref} casi al final del preámbulo.
\end{itemize}

\end{document}
