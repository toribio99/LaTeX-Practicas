% !TEX encoding = UTF-8 Unicode
\documentclass[10pt,a4paper]{article}

% Paquetes necesarios
\usepackage[utf8]{inputenc}
\usepackage[spanish]{babel}
\usepackage[margin=1.5cm]{geometry}
\usepackage{xcolor}
\usepackage{tcolorbox}
\usepackage{enumitem}
\usepackage{fontawesome5}
\usepackage{listings}
\usepackage{tikz}
\usepackage{multicol}

% Librerías TikZ comunes
\usetikzlibrary{arrows.meta}
\usetikzlibrary{shapes.geometric}

% Colores personalizados
\definecolor{commandcolor}{RGB}{39,174,96}

% Configuración de listings
\lstset{
	basicstyle=\ttfamily\footnotesize,
	breaklines=true,
	columns=fullflexible,
	keepspaces=true
}

% Título
\title{\textbf{\Huge TikZ en \LaTeX{}}\\\large Guía Completa de Comandos, Opciones y Librerías}
\author{}
\date{\today}

 \usepackage[
%colorlinks=true,        % Enlaces con color (en lugar de cajas)
linkcolor=blue,         % Color de enlaces internos
urlcolor=cyan,          % Color de URLs
citecolor=green,        % Color de citas bibliográficas
filecolor=magenta,      % Color de enlaces a archivos
pdfborder={0 0 0},      % Sin bordes en los enlaces
bookmarks=true,         % Crear marcadores en el PDF
bookmarksopen=true,     % Marcadores expandidos al abrir
pdftitle={Mi Título},   % Título del PDF
pdfauthor={Mi Nombre},  % Autor del PDF
pdfsubject={Tema},      % Tema del documento
pdfkeywords={palabra1, palabra2}, % Palabras clave
%hidelinks,              % Ocultar todos los bordes/colores de enlaces
unicode=true,           % Permitir caracteres Unicode en marcadores
breaklinks=true         % Permitir saltos de línea en enlaces
]{hyperref}

\begin{document}
	
	\maketitle
	
	\begin{tcolorbox}[colback=blue!5,colframe=blue!75!black,title=\faInfoCircle\ Introducción]
		TikZ (TikZ ist kein Zeichenprogramm) es un poderoso paquete para crear gráficos vectoriales programáticamente en \LaTeX{}. Permite dibujar desde figuras simples hasta diagramas complejos, gráficas matemáticas, diagramas de flujo, y más.
	\end{tcolorbox}
	
	\tableofcontents
	\newpage
	
	\section{Carga del Paquete}
	
	\subsection*{\texttt{\textbackslash usepackage\{tikz\}}}
	\begin{tcolorbox}[colback=green!5,colframe=green!50!black]
		\textbf{Descripción:} Carga básica del paquete TikZ
		
		\textbf{Ejemplo:}
		\begin{lstlisting}[language=TeX]
			\usepackage{tikz}
		\end{lstlisting}
	\end{tcolorbox}
	
	\subsection*{\texttt{\textbackslash usetikzlibrary\{library\}}}
	\begin{tcolorbox}[colback=green!5,colframe=green!50!black]
		\textbf{Descripción:} Carga librerías adicionales de TikZ
		
		\textbf{Ejemplo:}
		\begin{lstlisting}[language=TeX]
			\usetikzlibrary{arrows.meta, shapes.geometric, positioning}
		\end{lstlisting}
	\end{tcolorbox}
	
	\section{Entorno tikzpicture}
	
	\subsection*{Sintaxis básica}
	\begin{lstlisting}[language=TeX]
		\begin{tikzpicture}[opciones]
			... comandos de dibujo ...
		\end{tikzpicture}
	\end{lstlisting}
	
	\subsection{Opciones Globales del Entorno}
	
	\subsubsection*{\texttt{scale=factor}}
	\begin{tcolorbox}[colback=green!5,colframe=green!50!black]
		\textbf{Descripción:} Escala todo el dibujo
		
		\textbf{Ejemplo:}
		\begin{lstlisting}[language=TeX]
			\begin{tikzpicture}[scale=2]
				\draw (0,0) -- (1,1);
			\end{tikzpicture}
		\end{lstlisting}
	\end{tcolorbox}
	
	\subsubsection*{\texttt{xscale=factor, yscale=factor}}
	\begin{tcolorbox}[colback=green!5,colframe=green!50!black]
		\textbf{Descripción:} Escala solo en eje X o Y
		
		\textbf{Ejemplo:}
		\begin{lstlisting}[language=TeX]
			\begin{tikzpicture}[xscale=2, yscale=0.5]
				\draw (0,0) rectangle (1,1);
			\end{tikzpicture}
		\end{lstlisting}
	\end{tcolorbox}
	
	\subsubsection*{\texttt{rotate=angle}}
	\begin{tcolorbox}[colback=green!5,colframe=green!50!black]
		\textbf{Descripción:} Rota todo el dibujo en grados
		
		\textbf{Ejemplo:}
		\begin{lstlisting}[language=TeX]
			\begin{tikzpicture}[rotate=45]
				\draw (0,0) rectangle (1,1);
			\end{tikzpicture}
		\end{lstlisting}
	\end{tcolorbox}
	
	\subsubsection*{\texttt{xshift=dimension, yshift=dimension}}
	\begin{tcolorbox}[colback=green!5,colframe=green!50!black]
		\textbf{Descripción:} Desplaza todo el dibujo
		
		\textbf{Ejemplo:}
		\begin{lstlisting}[language=TeX]
			\begin{tikzpicture}[xshift=2cm, yshift=1cm]
				\draw (0,0) circle (1);
			\end{tikzpicture}
		\end{lstlisting}
	\end{tcolorbox}
	
	\subsubsection*{\texttt{baseline=dimension}}
	\begin{tcolorbox}[colback=green!5,colframe=green!50!black]
		\textbf{Descripción:} Ajusta la línea base del dibujo
		
		\textbf{Ejemplo:}
		\begin{lstlisting}[language=TeX]
			Texto \begin{tikzpicture}[baseline=-0.5ex]
				\draw (0,0) circle (0.5ex);
			\end{tikzpicture} texto
		\end{lstlisting}
	\end{tcolorbox}
	
	\section{Comandos Básicos de Dibujo}
	
	\subsection{Comando \textbackslash draw}
	
	\subsubsection*{\texttt{\textbackslash draw (x,y) -- (x,y);}}
	\begin{tcolorbox}[colback=green!5,colframe=green!50!black]
		\textbf{Descripción:} Dibuja una línea recta
		
		\textbf{Ejemplo:}
		\begin{lstlisting}[language=TeX]
			\draw (0,0) -- (2,1) -- (3,0);
		\end{lstlisting}
	\end{tcolorbox}
	
	\subsubsection*{\texttt{\textbackslash draw (x,y) -| (x,y);}}
	\begin{tcolorbox}[colback=green!5,colframe=green!50!black]
		\textbf{Descripción:} Línea horizontal luego vertical
		
		\textbf{Ejemplo:}
		\begin{lstlisting}[language=TeX]
			\draw (0,0) -| (2,1);
		\end{lstlisting}
	\end{tcolorbox}
	
	\subsubsection*{\texttt{\textbackslash draw (x,y) |- (x,y);}}
	\begin{tcolorbox}[colback=green!5,colframe=green!50!black]
		\textbf{Descripción:} Línea vertical luego horizontal
		
		\textbf{Ejemplo:}
		\begin{lstlisting}[language=TeX]
			\draw (0,0) |- (2,1);
		\end{lstlisting}
	\end{tcolorbox}
	
	\subsubsection*{\texttt{\textbackslash draw (x,y) .. controls (cx,cy) .. (x,y);}}
	\begin{tcolorbox}[colback=green!5,colframe=green!50!black]
		\textbf{Descripción:} Curva de Bézier con punto de control
		
		\textbf{Ejemplo:}
		\begin{lstlisting}[language=TeX]
			\draw (0,0) .. controls (1,1) .. (2,0);
		\end{lstlisting}
	\end{tcolorbox}
	
	\subsubsection*{\texttt{\textbackslash draw (x,y) to[bend left] (x,y);}}
	\begin{tcolorbox}[colback=green!5,colframe=green!50!black]
		\textbf{Descripción:} Curva con doblez a la izquierda
		
		\textbf{Ejemplo:}
		\begin{lstlisting}[language=TeX]
			\draw (0,0) to[bend left] (2,0);
			\draw (0,-1) to[bend right=60] (2,-1);
		\end{lstlisting}
	\end{tcolorbox}
	
	\subsubsection*{\texttt{\textbackslash draw (x,y) arc (start:end:radius);}}
	\begin{tcolorbox}[colback=green!5,colframe=green!50!black]
		\textbf{Descripción:} Dibuja un arco
		
		\textbf{Ejemplo:}
		\begin{lstlisting}[language=TeX]
			\draw (0,0) arc (0:90:1cm);
			\draw (2,0) arc (0:180:1cm and 0.5cm);
		\end{lstlisting}
	\end{tcolorbox}
	
	\subsection{Formas Básicas}
	
	\subsubsection*{\texttt{\textbackslash draw (x,y) rectangle (x,y);}}
	\begin{tcolorbox}[colback=green!5,colframe=green!50!black]
		\textbf{Descripción:} Dibuja un rectángulo
		
		\textbf{Ejemplo:}
		\begin{lstlisting}[language=TeX]
			\draw (0,0) rectangle (2,1);
		\end{lstlisting}
	\end{tcolorbox}
	
	\subsubsection*{\texttt{\textbackslash draw (x,y) circle (radius);}}
	\begin{tcolorbox}[colback=green!5,colframe=green!50!black]
		\textbf{Descripción:} Dibuja un círculo
		
		\textbf{Ejemplo:}
		\begin{lstlisting}[language=TeX]
			\draw (0,0) circle (1cm);
			\draw (2,0) circle [radius=0.5];
		\end{lstlisting}
	\end{tcolorbox}
	
	\subsubsection*{\texttt{\textbackslash draw (x,y) ellipse (xradius and yradius);}}
	\begin{tcolorbox}[colback=green!5,colframe=green!50!black]
		\textbf{Descripción:} Dibuja una elipse
		
		\textbf{Ejemplo:}
		\begin{lstlisting}[language=TeX]
			\draw (0,0) ellipse (2cm and 1cm);
			\draw (3,0) ellipse [x radius=1, y radius=0.5];
		\end{lstlisting}
	\end{tcolorbox}
	
	\subsubsection*{\texttt{\textbackslash draw (x,y) grid (x,y);}}
	\begin{tcolorbox}[colback=green!5,colframe=green!50!black]
		\textbf{Descripción:} Dibuja una cuadrícula
		
		\textbf{Ejemplo:}
		\begin{lstlisting}[language=TeX]
			\draw (0,0) grid (3,2);
			\draw[step=0.5] (0,0) grid (3,2);
		\end{lstlisting}
	\end{tcolorbox}
	
	\subsection{Comando \textbackslash fill}
	
	\subsubsection*{\texttt{\textbackslash fill (x,y) circle (radius);}}
	\begin{tcolorbox}[colback=green!5,colframe=green!50!black]
		\textbf{Descripción:} Rellena una forma (sin borde)
		
		\textbf{Ejemplo:}
		\begin{lstlisting}[language=TeX]
			\fill (0,0) circle (1);
			\fill[red] (2,0) rectangle (3,1);
		\end{lstlisting}
	\end{tcolorbox}
	
	\subsection{Comando \textbackslash filldraw}
	
	\subsubsection*{\texttt{\textbackslash filldraw (x,y) circle (radius);}}
	\begin{tcolorbox}[colback=green!5,colframe=green!50!black]
		\textbf{Descripción:} Rellena y dibuja el borde
		
		\textbf{Ejemplo:}
		\begin{lstlisting}[language=TeX]
			\filldraw[fill=blue!20, draw=blue] (0,0) circle (1);
		\end{lstlisting}
	\end{tcolorbox}
	
	\subsection{Comando \textbackslash shade}
	
	\subsubsection*{\texttt{\textbackslash shade (x,y) rectangle (x,y);}}
	\begin{tcolorbox}[colback=green!5,colframe=green!50!black]
		\textbf{Descripción:} Dibuja con degradado
		
		\textbf{Ejemplo:}
		\begin{lstlisting}[language=TeX]
			\shade (0,0) rectangle (2,1);
			\shade[left color=red, right color=blue] (0,0) rectangle (2,1);
		\end{lstlisting}
	\end{tcolorbox}
	
	\section{Opciones de Estilo para \textbackslash draw}
	
	\subsection{Colores}
	
	\subsubsection*{\texttt{color=color, draw=color}}
	\begin{tcolorbox}[colback=green!5,colframe=green!50!black]
		\textbf{Descripción:} Color del trazo
		
		\textbf{Ejemplo:}
		\begin{lstlisting}[language=TeX]
			\draw[red] (0,0) -- (1,0);
			\draw[draw=blue] (0,-0.5) -- (1,-0.5);
			\draw[color=green!50!black] (0,-1) -- (1,-1);
		\end{lstlisting}
	\end{tcolorbox}
	
	\subsubsection*{\texttt{fill=color}}
	\begin{tcolorbox}[colback=green!5,colframe=green!50!black]
		\textbf{Descripción:} Color de relleno
		
		\textbf{Ejemplo:}
		\begin{lstlisting}[language=TeX]
			\draw[fill=yellow] (0,0) circle (0.5);
			\draw[fill=red!30] (1,0) rectangle (2,1);
		\end{lstlisting}
	\end{tcolorbox}
	
	\subsubsection*{\texttt{opacity=value}}
	\begin{tcolorbox}[colback=green!5,colframe=green!50!black]
		\textbf{Descripción:} Opacidad (0=transparente, 1=opaco)
		
		\textbf{Ejemplo:}
		\begin{lstlisting}[language=TeX]
			\draw[fill=blue, opacity=0.5] (0,0) circle (1);
		\end{lstlisting}
	\end{tcolorbox}
	
	\subsubsection*{\texttt{fill opacity=value, draw opacity=value}}
	\begin{tcolorbox}[colback=green!5,colframe=green!50!black]
		\textbf{Descripción:} Opacidad separada para relleno y trazo
		
		\textbf{Ejemplo:}
		\begin{lstlisting}[language=TeX]
			\draw[fill=red, fill opacity=0.3, draw opacity=1] (0,0) circle (1);
		\end{lstlisting}
	\end{tcolorbox}
	
	\subsection{Grosor de Línea}
	
	\subsubsection*{\texttt{line width=dimension}}
	\begin{tcolorbox}[colback=green!5,colframe=green!50!black]
		\textbf{Descripción:} Grosor específico de línea
		
		\textbf{Ejemplo:}
		\begin{lstlisting}[language=TeX]
			\draw[line width=2pt] (0,0) -- (2,0);
			\draw[line width=0.5mm] (0,-0.5) -- (2,-0.5);
		\end{lstlisting}
	\end{tcolorbox}
	
	\subsubsection*{\texttt{ultra thin, very thin, thin, semithick, thick, very thick, ultra thick}}
	\begin{tcolorbox}[colback=green!5,colframe=green!50!black]
		\textbf{Descripción:} Grosores predefinidos
		
		\textbf{Ejemplo:}
		\begin{lstlisting}[language=TeX]
			\draw[ultra thin] (0,3) -- (1,3);
			\draw[very thin] (0,2.5) -- (1,2.5);
			\draw[thin] (0,2) -- (1,2);
			\draw[semithick] (0,1.5) -- (1,1.5);
			\draw[thick] (0,1) -- (1,1);
			\draw[very thick] (0,0.5) -- (1,0.5);
			\draw[ultra thick] (0,0) -- (1,0);
		\end{lstlisting}
	\end{tcolorbox}
	
	\subsection{Estilos de Línea}
	
	\subsubsection*{\texttt{solid, dashed, dotted, dashdotted, densely dashed, loosely dashed}}
	\begin{tcolorbox}[colback=green!5,colframe=green!50!black]
		\textbf{Descripción:} Patrones de línea
		
		\textbf{Ejemplo:}
		\begin{lstlisting}[language=TeX]
			\draw[solid] (0,3) -- (2,3);
			\draw[dashed] (0,2.5) -- (2,2.5);
			\draw[dotted] (0,2) -- (2,2);
			\draw[dashdotted] (0,1.5) -- (2,1.5);
			\draw[densely dashed] (0,1) -- (2,1);
			\draw[loosely dashed] (0,0.5) -- (2,0.5);
		\end{lstlisting}
	\end{tcolorbox}
	
	\subsubsection*{\texttt{dash pattern=on length off length}}
	\begin{tcolorbox}[colback=green!5,colframe=green!50!black]
		\textbf{Descripción:} Patrón de guiones personalizado
		
		\textbf{Ejemplo:}
		\begin{lstlisting}[language=TeX]
			\draw[dash pattern=on 2pt off 3pt on 4pt off 4pt] (0,0) -- (3,0);
		\end{lstlisting}
	\end{tcolorbox}
	
	\subsection{Terminaciones de Línea}
	
	\subsubsection*{\texttt{line cap=type}}
	\begin{tcolorbox}[colback=green!5,colframe=green!50!black]
		\textbf{Descripción:} Estilo de terminación (rect, round, butt)
		
		\textbf{Ejemplo:}
		\begin{lstlisting}[language=TeX]
			\draw[line width=5pt, line cap=rect] (0,1) -- (2,1);
			\draw[line width=5pt, line cap=round] (0,0.5) -- (2,0.5);
			\draw[line width=5pt, line cap=butt] (0,0) -- (2,0);
		\end{lstlisting}
	\end{tcolorbox}
	
	\subsubsection*{\texttt{line join=type}}
	\begin{tcolorbox}[colback=green!5,colframe=green!50!black]
		\textbf{Descripción:} Estilo de unión (miter, round, bevel)
		
		\textbf{Ejemplo:}
		\begin{lstlisting}[language=TeX]
			\draw[line width=5pt, line join=miter] (0,0) -- (1,1) -- (2,0);
			\draw[line width=5pt, line join=round] (0,0) -- (1,1) -- (2,0);
			\draw[line width=5pt, line join=bevel] (0,0) -- (1,1) -- (2,0);
		\end{lstlisting}
	\end{tcolorbox}
	
	\subsection{Flechas}
	
	\subsubsection*{\texttt{->, <-, <->}}
	\begin{tcolorbox}[colback=green!5,colframe=green!50!black]
		\textbf{Descripción:} Flechas simples
		
		\textbf{Ejemplo:}
		\begin{lstlisting}[language=TeX]
			\draw[->] (0,0) -- (2,0);
			\draw[<-] (0,-0.5) -- (2,-0.5);
			\draw[<->] (0,-1) -- (2,-1);
		\end{lstlisting}
	\end{tcolorbox}
	
	\subsubsection*{\texttt{-latex, -stealth, -triangle}}
	\begin{tcolorbox}[colback=green!5,colframe=green!50!black]
		\textbf{Descripción:} Tipos de flechas específicas (requiere arrows.meta)
		
		\textbf{Ejemplo:}
		\begin{lstlisting}[language=TeX]
			\usetikzlibrary{arrows.meta}
			\draw[-latex] (0,0) -- (2,0);
			\draw[-stealth] (0,-0.5) -- (2,-0.5);
			\draw[-Triangle] (0,-1) -- (2,-1);
		\end{lstlisting}
	\end{tcolorbox}
	
	\subsubsection*{\texttt{-\{Latex[options]\}}}
	\begin{tcolorbox}[colback=green!5,colframe=green!50!black]
		\textbf{Descripción:} Flecha con opciones personalizadas
		
		\textbf{Ejemplo:}
		\begin{lstlisting}[language=TeX]
			\draw[-{Latex[length=3mm, width=2mm]}] (0,0) -- (2,0);
			\draw[-{Stealth[red, scale=2]}] (0,-0.5) -- (2,-0.5);
		\end{lstlisting}
	\end{tcolorbox}
	
	\section{Nodos (Nodes)}
	
	\subsection{Comando \textbackslash node}
	
	\subsubsection*{\texttt{\textbackslash node at (x,y) \{text\};}}
	\begin{tcolorbox}[colback=green!5,colframe=green!50!black]
		\textbf{Descripción:} Coloca texto en una posición
		
		\textbf{Ejemplo:}
		\begin{lstlisting}[language=TeX]
			\node at (0,0) {Texto};
			\node[red] at (2,0) {Rojo};
		\end{lstlisting}
	\end{tcolorbox}
	
	\subsubsection*{\texttt{\textbackslash node (name) at (x,y) \{text\};}}
	\begin{tcolorbox}[colback=green!5,colframe=green!50!black]
		\textbf{Descripción:} Nodo con nombre para referencias
		
		\textbf{Ejemplo:}
		\begin{lstlisting}[language=TeX]
			\node (A) at (0,0) {A};
			\node (B) at (2,0) {B};
			\draw[->] (A) -- (B);
		\end{lstlisting}
	\end{tcolorbox}
	
	\subsection{Opciones de Nodos}
	
	\subsubsection*{\texttt{anchor=position}}
	\begin{tcolorbox}[colback=green!5,colframe=green!50!black]
		\textbf{Descripción:} Punto de anclaje (north, south, east, west, center, etc.)
		
		\textbf{Ejemplo:}
		\begin{lstlisting}[language=TeX]
			\node[anchor=north] at (0,0) {Norte};
			\node[anchor=south east] at (2,0) {SE};
		\end{lstlisting}
	\end{tcolorbox}
	
	\subsubsection*{\texttt{above, below, left, right}}
	\begin{tcolorbox}[colback=green!5,colframe=green!50!black]
		\textbf{Descripción:} Posicionamiento relativo
		
		\textbf{Ejemplo:}
		\begin{lstlisting}[language=TeX]
			\node (A) at (0,0) {A};
			\node[above] at (A) {Arriba};
			\node[right=1cm] at (A) {Derecha};
		\end{lstlisting}
	\end{tcolorbox}
	
	\subsubsection*{\texttt{draw, fill}}
	\begin{tcolorbox}[colback=green!5,colframe=green!50!black]
		\textbf{Descripción:} Dibuja borde y rellena nodo
		
		\textbf{Ejemplo:}
		\begin{lstlisting}[language=TeX]
			\node[draw] at (0,0) {Con borde};
			\node[draw, fill=yellow] at (2,0) {Relleno};
		\end{lstlisting}
	\end{tcolorbox}
	
	\subsubsection*{\texttt{shape=type}}
	\begin{tcolorbox}[colback=green!5,colframe=green!50!black]
		\textbf{Descripción:} Forma del nodo (rectangle, circle, ellipse, etc.)
		
		\textbf{Ejemplo:}
		\begin{lstlisting}[language=TeX]
			\node[draw, circle] at (0,0) {O};
			\node[draw, rectangle] at (2,0) {R};
		\end{lstlisting}
	\end{tcolorbox}
	
	\subsubsection*{\texttt{minimum width=dimension, minimum height=dimension}}
	\begin{tcolorbox}[colback=green!5,colframe=green!50!black]
		\textbf{Descripción:} Tamaño mínimo del nodo
		
		\textbf{Ejemplo:}
		\begin{lstlisting}[language=TeX]
			\node[draw, minimum width=2cm, minimum height=1cm] at (0,0) {Nodo};
		\end{lstlisting}
	\end{tcolorbox}
	
	\subsubsection*{\texttt{minimum size=dimension}}
	\begin{tcolorbox}[colback=green!5,colframe=green!50!black]
		\textbf{Descripción:} Tamaño mínimo igual en ambas dimensiones
		
		\textbf{Ejemplo:}
		\begin{lstlisting}[language=TeX]
			\node[draw, circle, minimum size=1.5cm] at (0,0) {A};
		\end{lstlisting}
	\end{tcolorbox}
	
	\subsubsection*{\texttt{inner sep=dimension, outer sep=dimension}}
	\begin{tcolorbox}[colback=green!5,colframe=green!50!black]
		\textbf{Descripción:} Separación interna y externa
		
		\textbf{Ejemplo:}
		\begin{lstlisting}[language=TeX]
			\node[draw, inner sep=5pt] at (0,0) {A};
			\node[draw, outer sep=10pt] at (2,0) {B};
		\end{lstlisting}
	\end{tcolorbox}
	
	\subsubsection*{\texttt{text width=dimension}}
	\begin{tcolorbox}[colback=green!5,colframe=green!50!black]
		\textbf{Descripción:} Ancho de texto con ajuste automático
		
		\textbf{Ejemplo:}
		\begin{lstlisting}[language=TeX]
			\node[draw, text width=3cm] at (0,0) {Texto largo que se ajusta automáticamente};
		\end{lstlisting}
	\end{tcolorbox}
	
	\subsubsection*{\texttt{align=type}}
	\begin{tcolorbox}[colback=green!5,colframe=green!50!black]
		\textbf{Descripción:} Alineación de texto (left, center, right, justify)
		
		\textbf{Ejemplo:}
		\begin{lstlisting}[language=TeX]
			\node[draw, text width=2cm, align=center] at (0,0) {Texto\\centrado};
		\end{lstlisting}
	\end{tcolorbox}
	
	\subsubsection*{\texttt{rotate=angle}}
	\begin{tcolorbox}[colback=green!5,colframe=green!50!black]
		\textbf{Descripción:} Rota el nodo
		
		\textbf{Ejemplo:}
		\begin{lstlisting}[language=TeX]
			\node[rotate=45] at (0,0) {Rotado};
		\end{lstlisting}
	\end{tcolorbox}
	
	\subsubsection*{\texttt{font=command}}
	\begin{tcolorbox}[colback=green!5,colframe=green!50!black]
		\textbf{Descripción:} Cambia la fuente del texto
		
		\textbf{Ejemplo:}
		\begin{lstlisting}[language=TeX]
			\node[font=\Large\bfseries] at (0,0) {Grande y negrita};
			\node[font=\tiny] at (2,0) {Pequeño};
		\end{lstlisting}
	\end{tcolorbox}
	
	\section{Coordenadas y Sistemas}
	
	\subsection{Tipos de Coordenadas}
	
	\subsubsection*{\texttt{(x,y)} -- Cartesianas}
	\begin{tcolorbox}[colback=green!5,colframe=green!50!black]
		\textbf{Descripción:} Coordenadas cartesianas estándar
		
		\textbf{Ejemplo:}
		\begin{lstlisting}[language=TeX]
			\draw (0,0) -- (2,1);
			\draw (1cm,2cm) -- (3cm,1cm);
		\end{lstlisting}
	\end{tcolorbox}
	
	\subsubsection*{\texttt{(angle:radius)} -- Polares}
	\begin{tcolorbox}[colback=green!5,colframe=green!50!black]
		\textbf{Descripción:} Coordenadas polares
		
		\textbf{Ejemplo:}
		\begin{lstlisting}[language=TeX]
			\draw (0:1) -- (45:1) -- (90:1);
			\draw (0,0) -- (30:2cm);
		\end{lstlisting}
	\end{tcolorbox}
	
	\subsubsection*{\texttt{++(x,y)} -- Relativas}
	\begin{tcolorbox}[colback=green!5,colframe=green!50!black]
		\textbf{Descripción:} Coordenadas relativas (mueve el punto actual)
		
		\textbf{Ejemplo:}
		\begin{lstlisting}[language=TeX]
			\draw (0,0) -- ++(1,0) -- ++(0,1) -- ++(-1,0);
		\end{lstlisting}
	\end{tcolorbox}
	
	\subsubsection*{\texttt{+(x,y)} -- Relativas sin mover}
	\begin{tcolorbox}[colback=green!5,colframe=green!50!black]
		\textbf{Descripción:} Coordenadas relativas sin actualizar posición
		
		\textbf{Ejemplo:}
		\begin{lstlisting}[language=TeX]
			\draw (0,0) -- +(1,0) -- +(0,1);
		\end{lstlisting}
	\end{tcolorbox}
	
	\subsubsection*{\texttt{(nodename)} -- Referencias a nodos}
	\begin{tcolorbox}[colback=green!5,colframe=green!50!black]
		\textbf{Descripción:} Usa la posición de un nodo
		
		\textbf{Ejemplo:}
		\begin{lstlisting}[language=TeX]
			\node (A) at (0,0) {A};
			\node (B) at (2,1) {B};
			\draw (A) -- (B);
		\end{lstlisting}
	\end{tcolorbox}
	
	\subsubsection*{\texttt{(nodename.anchor)} -- Anclas de nodos}
	\begin{tcolorbox}[colback=green!5,colframe=green!50!black]
		\textbf{Descripción:} Punto específico del nodo
		
		\textbf{Ejemplo:}
		\begin{lstlisting}[language=TeX]
			\node[draw] (A) at (0,0) {A};
			\draw (A.north) -- +(0,1);
			\draw (A.east) -- +(1,0);
		\end{lstlisting}
	\end{tcolorbox}
	
	\subsubsection*{\texttt{(\$(A)!factor!\$(B))} -- Interpolación (calc)}
	\begin{tcolorbox}[colback=green!5,colframe=green!50!black]
		\textbf{Descripción:} Punto entre A y B (requiere calc)
		
		\textbf{Ejemplo:}
		\begin{lstlisting}[language=TeX]
			\usetikzlibrary{calc}
			\coordinate (A) at (0,0);
			\coordinate (B) at (2,2);
			\fill ($0.5*(A) + 0.5*(B)$) circle (2pt);
			\fill ($(A)!0.3!(B)$) circle (2pt);
		\end{lstlisting}
	\end{tcolorbox}
	
	\section{Paths (Caminos)}
	
	\subsection{Operaciones de Path}
	
	\subsubsection*{\texttt{-- cycle}}
	\begin{tcolorbox}[colback=green!5,colframe=green!50!black]
		\textbf{Descripción:} Cierra el camino volviendo al inicio
		
		\textbf{Ejemplo:}
		\begin{lstlisting}[language=TeX]
			\draw (0,0) -- (1,0) -- (1,1) -- (0,1) -- cycle;
		\end{lstlisting}
	\end{tcolorbox}
	
	\subsubsection*{\texttt{[rounded corners]}}
	\begin{tcolorbox}[colback=green!5,colframe=green!50!black]
		\textbf{Descripción:} Esquinas redondeadas
		
		\textbf{Ejemplo:}
		\begin{lstlisting}[language=TeX]
			\draw[rounded corners] (0,0) -- (1,0) -- (1,1) -- (0,1) -- cycle;
			\draw[rounded corners=10pt] (2,0) rectangle (4,1);
		\end{lstlisting}
	\end{tcolorbox}
	
	\subsubsection*{\texttt{plot}}
	\begin{tcolorbox}[colback=green!5,colframe=green!50!black]
		\textbf{Descripción:} Grafica coordenadas o funciones
		
		\textbf{Ejemplo:}
		\begin{lstlisting}[language=TeX]
			\draw plot coordinates {(0,0) (1,1) (2,0.5) (3,2)};
			\draw[domain=0:3] plot (\x, {sin(\x r)});
		\end{lstlisting}
	\end{tcolorbox}
	
	\section{Transformaciones}
	
	\subsection{Comandos de Transformación}
	
	\subsubsection*{\texttt{shift=\{(x,y)\}}}
	\begin{tcolorbox}[colback=green!5,colframe=green!50!black]
		\textbf{Descripción:} Desplaza el dibujo
		
		\textbf{Ejemplo:}
		\begin{lstlisting}[language=TeX]
			\draw[shift={(2,1)}] (0,0) circle (0.5);
		\end{lstlisting}
	\end{tcolorbox}
	
	\subsubsection*{\texttt{rotate around=\{angle:(x,y)\}}}
	\begin{tcolorbox}[colback=green!5,colframe=green!50!black]
		\textbf{Descripción:} Rota alrededor de un punto
		
		\textbf{Ejemplo:}
		\begin{lstlisting}[language=TeX]
			\draw[rotate around={45:(1,1)}] (0,0) rectangle (2,2);
		\end{lstlisting}
	\end{tcolorbox}
	
	\subsubsection*{\texttt{cm=\{a,b,c,d,(x,y)\}}}
	\begin{tcolorbox}[colback=green!5,colframe=green!50!black]
		\textbf{Descripción:} Transformación de matriz personalizada
		
		\textbf{Ejemplo:}
		\begin{lstlisting}[language=TeX]
			\draw[cm={1,0.5,0,1,(0,0)}] (0,0) rectangle (1,1);
		\end{lstlisting}
	\end{tcolorbox}
	
	\section{Scopes (Alcances)}
	
	\subsubsection*{\texttt{\textbackslash begin\{scope\}[options]}}
	\begin{tcolorbox}[colback=green!5,colframe=green!50!black]
		\textbf{Descripción:} Agrupa elementos con opciones locales
		
		\textbf{Ejemplo:}
		\begin{lstlisting}[language=TeX]
			\begin{scope}[red, thick]
				\draw (0,0) -- (1,1);
				\draw (1,0) -- (0,1);
			\end{scope}
			\begin{scope}[xshift=2cm, blue]
				\draw (0,0) rectangle (1,1);
			\end{scope}
		\end{lstlisting}
	\end{tcolorbox}
	
	\section{Clips y Límites}
	
	\subsubsection*{\texttt{\textbackslash clip}}
	\begin{tcolorbox}[colback=green!5,colframe=green!50!black]
		\textbf{Descripción:} Recorta el dibujo a una región
		
		\textbf{Ejemplo:}
		\begin{lstlisting}[language=TeX]
			\begin{scope}
				\clip (0,0) rectangle (2,2);
				\draw (1,1) circle (1.5);
			\end{scope}
		\end{lstlisting}
	\end{tcolorbox}
	
	\subsubsection*{\texttt{\textbackslash useasboundingbox}}
	\begin{tcolorbox}[colback=green!5,colframe=green!50!black]
		\textbf{Descripción:} Define el bounding box del dibujo
		
		\textbf{Ejemplo:}
		\begin{lstlisting}[language=TeX]
			\useasboundingbox (0,0) rectangle (3,2);
			\draw (-1,-1) -- (4,3);
		\end{lstlisting}
	\end{tcolorbox}
	
	\section{Foreach (Bucles)}
	
	\subsubsection*{\texttt{\textbackslash foreach \textbackslash x in \{list\}}}
	\begin{tcolorbox}[colback=green!5,colframe=green!50!black]
		\textbf{Descripción:} Bucle para repetir comandos
		
		\textbf{Ejemplo:}
		\begin{lstlisting}[language=TeX]
			\foreach \x in {0,1,2,3}
			\draw (\x,0) circle (0.2);
			
			\foreach \x in {0,...,5}
			\draw (\x,0) -- (\x,1);
			
			\foreach \x/\y in {0/0, 1/1, 2/0.5}
			\fill (\x,\y) circle (2pt);
		\end{lstlisting}
	\end{tcolorbox}
	
	\newpage
	
	\section{Librerías de TikZ}
	
	\subsection{arrows.meta - Flechas Avanzadas}
	
	\subsubsection*{\texttt{\textbackslash usetikzlibrary\{arrows.meta\}}}
	\begin{tcolorbox}[colback=green!5,colframe=green!50!black]
		\textbf{Descripción:} Tipos de flechas personalizables
		
		\textbf{Tipos disponibles:}
		\begin{itemize}[nosep]
			\item Latex, latex', Stealth, Triangle, To, Bar, Hooks, Classical TikZ Rightarrow
		\end{itemize}
		
		\textbf{Ejemplo:}
		\begin{lstlisting}[language=TeX]
			\draw[-Latex] (0,0) -- (2,0);
			\draw[-{Stealth[scale=2]}] (0,-0.5) -- (2,-0.5);
			\draw[-{Triangle[fill=white]}] (0,-1) -- (2,-1);
		\end{lstlisting}
	\end{tcolorbox}
	
	\subsection{shapes.geometric - Formas Geométricas}
	
	\subsubsection*{\texttt{\textbackslash usetikzlibrary\{shapes.geometric\}}}
	\begin{tcolorbox}[colback=green!5,colframe=green!50!black]
		\textbf{Descripción:} Formas adicionales para nodos
		
		\textbf{Formas disponibles:}
		\begin{itemize}[nosep]
			\item diamond, ellipse, trapezium, semicircle, regular polygon, star, isosceles triangle, kite, dart, circular sector, cylinder
		\end{itemize}
		
		\textbf{Ejemplo:}
		\begin{lstlisting}[language=TeX]
			\node[draw, diamond] at (0,0) {D};
			\node[draw, star, star points=5] at (2,0) {S};
			\node[draw, regular polygon, regular polygon sides=6] at (4,0) {H};
		\end{lstlisting}
	\end{tcolorbox}
	
	\subsection{shapes.symbols - Símbolos}
	
	\subsubsection*{\texttt{\textbackslash usetikzlibrary\{shapes.symbols\}}}
	\begin{tcolorbox}[colback=green!5,colframe=green!50!black]
		\textbf{Descripción:} Formas simbólicas
		
		\textbf{Formas disponibles:}
		\begin{itemize}[nosep]
			\item cloud, starburst, signal, tape, magnifying glass, key
		\end{itemize}
		
		\textbf{Ejemplo:}
		\begin{lstlisting}[language=TeX]
			\node[draw, cloud, cloud puffs=10] at (0,0) {Nube};
			\node[draw, starburst] at (3,0) {Boom};
		\end{lstlisting}
	\end{tcolorbox}
	
	\subsection{shapes.arrows - Flechas como Nodos}
	
	\subsubsection*{\texttt{\textbackslash usetikzlibrary\{shapes.arrows\}}}
	\begin{tcolorbox}[colback=green!5,colframe=green!50!black]
		\textbf{Descripción:} Nodos en forma de flecha
		
		\textbf{Formas disponibles:}
		\begin{itemize}[nosep]
			\item single arrow, double arrow, arrow box
		\end{itemize}
		
		\textbf{Ejemplo:}
		\begin{lstlisting}[language=TeX]
			\node[draw, single arrow] at (0,0) {Flecha};
			\node[draw, double arrow, fill=red!20] at (0,-1) {Doble};
		\end{lstlisting}
	\end{tcolorbox}
	
	\subsection{shapes.callouts - Globos de Diálogo}
	
	\subsubsection*{\texttt{\textbackslash usetikzlibrary\{shapes.callouts\}}}
	\begin{tcolorbox}[colback=green!5,colframe=green!50!black]
		\textbf{Descripción:} Globos de texto
		
		\textbf{Formas disponibles:}
		\begin{itemize}[nosep]
			\item rectangle callout, ellipse callout, cloud callout
		\end{itemize}
		
		\textbf{Ejemplo:}
		\begin{lstlisting}[language=TeX]
			\node[draw, rectangle callout, callout absolute pointer={(1,0)}] 
			at (0,0) {Mensaje};
		\end{lstlisting}
	\end{tcolorbox}
	
	\subsection{shapes.multipart - Nodos Multiparte}
	
	\subsubsection*{\texttt{\textbackslash usetikzlibrary\{shapes.multipart\}}}
	\begin{tcolorbox}[colback=green!5,colframe=green!50!black]
		\textbf{Descripción:} Nodos con múltiples secciones
		
		\textbf{Formas disponibles:}
		\begin{itemize}[nosep]
			\item rectangle split, circle split, ellipse split
		\end{itemize}
		
		\textbf{Ejemplo:}
		\begin{lstlisting}[language=TeX]
			\node[draw, rectangle split, rectangle split parts=3] at (0,0) {
				Parte 1
				\nodepart{two}
				Parte 2
				\nodepart{three}
				Parte 3
			};
		\end{lstlisting}
	\end{tcolorbox}
	
	\subsection{positioning - Posicionamiento Avanzado}
	
	\subsubsection*{\texttt{\textbackslash usetikzlibrary\{positioning\}}}
	\begin{tcolorbox}[colback=green!5,colframe=green!50!black]
		\textbf{Descripción:} Posicionamiento relativo mejorado
		
		\textbf{Opciones disponibles:}
		\begin{itemize}[nosep]
			\item above=of node, below=of node, left=of node, right=of node
			\item above left=of node, above right=of node, etc.
			\item node distance=dimension
		\end{itemize}
		
		\textbf{Ejemplo:}
		\begin{lstlisting}[language=TeX]
			\node[draw] (A) at (0,0) {A};
			\node[draw, above=of A] (B) {B};
			\node[draw, right=2cm of A] (C) {C};
		\end{lstlisting}
	\end{tcolorbox}
	
	\subsection{calc - Cálculos con Coordenadas}
	
	\subsubsection*{\texttt{\textbackslash usetikzlibrary\{calc\}}}
	\begin{tcolorbox}[colback=green!5,colframe=green!50!black]
		\textbf{Descripción:} Operaciones matemáticas con coordenadas
		
		\textbf{Operaciones disponibles:}
		\begin{itemize}[nosep]
			\item \texttt{(\$(A) + (B)\$)} -- Suma
			\item \texttt{(\$(A) - (B)\$)} -- Resta
			\item \texttt{(\$factor*(A)\$)} -- Multiplicación
			\item \texttt{(\$(A)!factor!(B)\$)} -- Interpolación lineal
			\item \texttt{(\$(A)!(x,y)!(B)\$)} -- Proyección
		\end{itemize}
		
		\textbf{Ejemplo:}
		\begin{lstlisting}[language=TeX]
			\coordinate (A) at (0,0);
			\coordinate (B) at (2,2);
			\fill ($(A) + (B)$) circle (2pt);
			\fill ($(A)!0.5!(B)$) circle (2pt);
			\fill ($0.5*(A) + 0.5*(B)$) circle (2pt);
		\end{lstlisting}
	\end{tcolorbox}
	
	\subsection{decorations - Decoraciones}
	
	\subsubsection*{\texttt{\textbackslash usetikzlibrary\{decorations.pathmorphing\}}}
	\begin{tcolorbox}[colback=green!5,colframe=green!50!black]
		\textbf{Descripción:} Deformaciones de paths
		
		\textbf{Decoraciones disponibles:}
		\begin{itemize}[nosep]
			\item zigzag, saw, random steps, bent, coil, bumps, snake
		\end{itemize}
		
		\textbf{Ejemplo:}
		\begin{lstlisting}[language=TeX]
			\draw[decorate, decoration=zigzag] (0,0) -- (3,0);
			\draw[decorate, decoration={coil, amplitude=2mm}] (0,-0.5) -- (3,-0.5);
		\end{lstlisting}
	\end{tcolorbox}
	
	\subsubsection*{\texttt{\textbackslash usetikzlibrary\{decorations.markings\}}}
	\begin{tcolorbox}[colback=green!5,colframe=green!50!black]
		\textbf{Descripción:} Marcas en paths
		
		\textbf{Ejemplo:}
		\begin{lstlisting}[language=TeX]
			\draw[decoration={markings, mark=at position 0.5 with {\arrow{>}}},
			postaction={decorate}] (0,0) -- (3,0);
		\end{lstlisting}
	\end{tcolorbox}
	
	\subsubsection*{\texttt{\textbackslash usetikzlibrary\{decorations.pathreplacing\}}}
	\begin{tcolorbox}[colback=green!5,colframe=green!50!black]
		\textbf{Descripción:} Reemplazo de paths (llaves, corchetes)
		
		\textbf{Decoraciones disponibles:}
		\begin{itemize}[nosep]
			\item brace, border, expanding waves, moveto, lineto, curveto, zigzag, saw, random steps, straight zigzag, ticks, waves, coil, bumps, bent, aspect, triangles, crosses
		\end{itemize}
		
		\textbf{Ejemplo:}
		\begin{lstlisting}[language=TeX]
			\draw[decorate, decoration={brace, amplitude=5pt}] (0,0) -- (3,0);
			\draw[decorate, decoration={border, angle=45, amplitude=5pt}] (0,-0.5) -- (3,-0.5);
		\end{lstlisting}
	\end{tcolorbox}
	
	\subsection{patterns - Patrones de Relleno}
	
	\subsubsection*{\texttt{\textbackslash usetikzlibrary\{patterns\}}}
	\begin{tcolorbox}[colback=green!5,colframe=green!50!black]
		\textbf{Descripción:} Patrones para rellenar áreas
		
		\textbf{Patrones disponibles:}
		\begin{itemize}[nosep]
			\item horizontal lines, vertical lines, north east lines, north west lines, grid, crosshatch, dots, crosshatch dots, fivepointed stars, sixpointed stars, bricks, checkerboard
		\end{itemize}
		
		\textbf{Ejemplo:}
		\begin{lstlisting}[language=TeX]
			\fill[pattern=horizontal lines] (0,0) rectangle (1,1);
			\fill[pattern=dots, pattern color=red] (2,0) rectangle (3,1);
		\end{lstlisting}
	\end{tcolorbox}
	
	\subsection{shadings - Degradados}
	
	\subsubsection*{\texttt{\textbackslash usetikzlibrary\{shadings\}}}
	\begin{tcolorbox}[colback=green!5,colframe=green!50!black]
		\textbf{Descripción:} Degradados predefinidos
		
		\textbf{Shadings disponibles:}
		\begin{itemize}[nosep]
			\item axis, radial, ball, color wheel, bilinear interpolation
		\end{itemize}
		
		\textbf{Ejemplo:}
		\begin{lstlisting}[language=TeX]
			\shade[ball color=blue] (0,0) circle (0.5);
			\shade[left color=red, right color=blue] (2,0) rectangle (4,1);
			\shade[inner color=yellow, outer color=red] (5,0) circle (0.5);
		\end{lstlisting}
	\end{tcolorbox}
	
	\subsection{shadows - Sombras}
	
	\subsubsection*{\texttt{\textbackslash usetikzlibrary\{shadows\}}}
	\begin{tcolorbox}[colback=green!5,colframe=green!50!black]
		\textbf{Descripción:} Efectos de sombra
		
		\textbf{Opciones disponibles:}
		\begin{itemize}[nosep]
			\item drop shadow, copy shadow, circular drop shadow, circular glow
		\end{itemize}
		
		\textbf{Ejemplo:}
		\begin{lstlisting}[language=TeX]
			\node[draw, drop shadow] at (0,0) {Sombra};
			\node[draw, circular drop shadow] at (3,0) {Circular};
		\end{lstlisting}
	\end{tcolorbox}
	
	\subsection{backgrounds - Fondos}
	
	\subsubsection*{\texttt{\textbackslash usetikzlibrary\{backgrounds\}}}
	\begin{tcolorbox}[colback=green!5,colframe=green!50!black]
		\textbf{Descripción:} Dibuja fondos detrás de otros elementos
		
		\textbf{Ejemplo:}
		\begin{lstlisting}[language=TeX]
			\begin{tikzpicture}
				\draw (0,0) -- (2,2);
				\begin{scope}[on background layer]
					\fill[yellow] (-0.5,-0.5) rectangle (2.5,2.5);
				\end{scope}
			\end{tikzpicture}
		\end{lstlisting}
	\end{tcolorbox}
	
	\subsection{fit - Ajuste Automático}
	
	\subsubsection*{\texttt{\textbackslash usetikzlibrary\{fit\}}}
	\begin{tcolorbox}[colback=green!5,colframe=green!50!black]
		\textbf{Descripción:} Nodo que se ajusta a otros nodos
		
		\textbf{Ejemplo:}
		\begin{lstlisting}[language=TeX]
			\node (A) at (0,0) {A};
			\node (B) at (2,1) {B};
			\node[draw, fit=(A) (B), inner sep=5pt] {};
		\end{lstlisting}
	\end{tcolorbox}
	
	\subsection{through - A Través de Puntos}
	
	\subsubsection*{\texttt{\textbackslash usetikzlibrary\{through\}}}
	\begin{tcolorbox}[colback=green!5,colframe=green!50!black]
		\textbf{Descripción:} Círculos que pasan por puntos
		
		\textbf{Ejemplo:}
		\begin{lstlisting}[language=TeX]
			\node[draw, circle through={(1,1)}] at (0,0) {};
		\end{lstlisting}
	\end{tcolorbox}
	
	\subsection{intersections - Intersecciones}
	
	\subsubsection*{\texttt{\textbackslash usetikzlibrary\{intersections\}}}
	\begin{tcolorbox}[colback=green!5,colframe=green!50!black]
		\textbf{Descripción:} Encuentra intersecciones entre paths
		
		\textbf{Ejemplo:}
		\begin{lstlisting}[language=TeX]
			\draw[name path=A] (0,0) -- (2,2);
			\draw[name path=B] (0,2) -- (2,0);
			\fill[red, name intersections={of=A and B}] 
			(intersection-1) circle (2pt);
		\end{lstlisting}
	\end{tcolorbox}
	
	\subsection{graphs - Grafos}
	
	\subsubsection*{\texttt{\textbackslash usetikzlibrary\{graphs\}}}
	\begin{tcolorbox}[colback=green!5,colframe=green!50!black]
		\textbf{Descripción:} Sintaxis simplificada para grafos
		
		\textbf{Ejemplo:}
		\begin{lstlisting}[language=TeX]
			\graph { a -> {b, c} -> d };
		\end{lstlisting}
	\end{tcolorbox}
	
	\subsection{trees - Árboles}
	
	\subsubsection*{\texttt{\textbackslash usetikzlibrary\{trees\}}}
	\begin{tcolorbox}[colback=green!5,colframe=green!50!black]
		\textbf{Descripción:} Estilos y opciones para árboles
		
		\textbf{Opciones:}
		\begin{itemize}[nosep]
			\item level distance, sibling distance, child, edge from parent
		\end{itemize}
		
		\textbf{Ejemplo:}
		\begin{lstlisting}[language=TeX]
			\node {Raíz}
			child {node {A}}
			child {node {B}
				child {node {B1}}
				child {node {B2}}
			};
		\end{lstlisting}
	\end{tcolorbox}
	
	\subsection{mindmap - Mapas Mentales}
	
	\subsubsection*{\texttt{\textbackslash usetikzlibrary\{mindmap\}}}
	\begin{tcolorbox}[colback=green!5,colframe=green!50!black]
		\textbf{Descripción:} Estilo para mapas mentales
		
		\textbf{Ejemplo:}
		\begin{lstlisting}[language=TeX]
			\begin{tikzpicture}[mindmap]
				\node[concept] {Concepto Central}
				child[concept color=blue] { node[concept] {Idea 1} }
				child[concept color=red] { node[concept] {Idea 2} };
			\end{tikzpicture}
		\end{lstlisting}
	\end{tcolorbox}
	
	\subsection{matrix - Matrices de Nodos}
	
	\subsubsection*{\texttt{\textbackslash usetikzlibrary\{matrix\}}}
	\begin{tcolorbox}[colback=green!5,colframe=green!50!black]
		\textbf{Descripción:} Organiza nodos en matriz
		
		\textbf{Ejemplo:}
		\begin{lstlisting}[language=TeX]
			\matrix[matrix of nodes, nodes={draw}] {
				A & B & C \\
				D & E & F \\
			};
		\end{lstlisting}
	\end{tcolorbox}
	
	\subsection{chains - Cadenas}
	
	\subsubsection*{\texttt{\textbackslash usetikzlibrary\{chains\}}}
	\begin{tcolorbox}[colback=green!5,colframe=green!50!black]
		\textbf{Descripción:} Conecta nodos secuencialmente
		
		\textbf{Ejemplo:}
		\begin{lstlisting}[language=TeX]
			\begin{scope}[start chain=going right]
				\node[on chain] {A};
				\node[on chain] {B};
				\node[on chain] {C};
			\end{scope}
		\end{lstlisting}
	\end{tcolorbox}
	
	\subsection{circuits - Circuitos Eléctricos}
	
	\subsubsection*{\texttt{\textbackslash usetikzlibrary\{circuits.ee.IEC\}}}
	\begin{tcolorbox}[colback=green!5,colframe=green!50!black]
		\textbf{Descripción:} Símbolos de circuitos eléctricos (IEC)
		
		\textbf{Símbolos disponibles:}
		\begin{itemize}[nosep]
			\item resistor, capacitor, inductor, voltage source, current source, diode, battery, ground, make contact
		\end{itemize}
		
		\textbf{Ejemplo:}
		\begin{lstlisting}[language=TeX]
			\draw (0,0) to[resistor] (2,0) to[battery] (2,2);
		\end{lstlisting}
	\end{tcolorbox}
	
	\subsection{automata - Autómatas}
	
	\subsubsection*{\texttt{\textbackslash usetikzlibrary\{automata\}}}
	\begin{tcolorbox}[colback=green!5,colframe=green!50!black]
		\textbf{Descripción:} Dibuja autómatas finitos
		
		\textbf{Opciones:}
		\begin{itemize}[nosep]
			\item state, accepting, initial, initial text
		\end{itemize}
		
		\textbf{Ejemplo:}
		\begin{lstlisting}[language=TeX]
			\node[state, initial] (q0) at (0,0) {$q_0$};
			\node[state, accepting] (q1) at (3,0) {$q_1$};
			\draw (q0) edge[loop above] node{0} (q0)
			(q0) edge[bend left] node{1} (q1);
		\end{lstlisting}
	\end{tcolorbox}
	
	\subsection{petri - Redes de Petri}
	
	\subsubsection*{\texttt{\textbackslash usetikzlibrary\{petri\}}}
	\begin{tcolorbox}[colback=green!5,colframe=green!50!black]
		\textbf{Descripción:} Elementos para redes de Petri
		
		\textbf{Opciones:}
		\begin{itemize}[nosep]
			\item place, transition, pre, post, tokens
		\end{itemize}
		
		\textbf{Ejemplo:}
		\begin{lstlisting}[language=TeX]
			\node[place, tokens=2] (p1) at (0,0) {};
			\node[transition] (t1) at (2,0) {};
			\draw[->] (p1) -- (t1);
		\end{lstlisting}
	\end{tcolorbox}
	
	\subsection{calendar - Calendarios}
	
	\subsubsection*{\texttt{\textbackslash usetikzlibrary\{calendar\}}}
	\begin{tcolorbox}[colback=green!5,colframe=green!50!black]
		\textbf{Descripción:} Crea calendarios
		
		\textbf{Ejemplo:}
		\begin{lstlisting}[language=TeX]
			\calendar[dates=2024-01-01 to 2024-01-31];
		\end{lstlisting}
	\end{tcolorbox}
	
	\subsection{folding - Plegado}
	
	\subsubsection*{\texttt{\textbackslash usetikzlibrary\{folding\}}}
	\begin{tcolorbox}[colback=green!5,colframe=green!50!black]
		\textbf{Descripción:} Crea plantillas de plegado 3D
		
		\textbf{Ejemplo:}
		\begin{lstlisting}[language=TeX]
			\pic{cube folding};
		\end{lstlisting}
	\end{tcolorbox}
	
	\subsection{er - Diagramas ER}
	
	\subsubsection*{\texttt{\textbackslash usetikzlibrary\{er\}}}
	\begin{tcolorbox}[colback=green!5,colframe=green!50!black]
		\textbf{Descripción:} Diagramas Entidad-Relación
		
		\textbf{Formas:}
		\begin{itemize}[nosep]
			\item entity, relationship, attribute
		\end{itemize}
		
		\textbf{Ejemplo:}
		\begin{lstlisting}[language=TeX]
			\node[entity] (E) {Entidad};
			\node[relationship] (R) [right=of E] {Relación};
			\draw (E) -- (R);
		\end{lstlisting}
	\end{tcolorbox}
	
	\subsection{datavisualization - Visualización de Datos}
	
	\subsubsection*{\texttt{\textbackslash usetikzlibrary\{datavisualization\}}}
	\begin{tcolorbox}[colback=green!5,colframe=green!50!black]
		\textbf{Descripción:} Sistema avanzado de visualización
		
		\textbf{Ejemplo:}
		\begin{lstlisting}[language=TeX]
			\datavisualization[scientific axes, visualize as smooth line]
			data {x, y
				0, 0
				1, 1
				2, 0.5};
		\end{lstlisting}
	\end{tcolorbox}
	
	\subsection{tikzmark - Marcas en Texto}
	
	\subsubsection*{\texttt{\textbackslash usetikzlibrary\{tikzmark\}}}
	\begin{tcolorbox}[colback=green!5,colframe=green!50!black]
		\textbf{Descripción:} Marca posiciones en el texto para conectar
		
		\textbf{Ejemplo:}
		\begin{lstlisting}[language=TeX]
			Texto \tikzmark{a} más texto \tikzmark{b}.
			\begin{tikzpicture}[overlay, remember picture]
				\draw[->] (pic cs:a) to[bend left] (pic cs:b);
			\end{tikzpicture}
		\end{lstlisting}
	\end{tcolorbox}
	
	\subsection{spy - Ampliación}
	
	\subsubsection*{\texttt{\textbackslash usetikzlibrary\{spy\}}}
	\begin{tcolorbox}[colback=green!5,colframe=green!50!black]
		\textbf{Descripción:} Amplía una región del dibujo
		
		\textbf{Ejemplo:}
		\begin{lstlisting}[language=TeX]
			\begin{tikzpicture}[spy using outlines={circle, magnification=3, size=1cm}]
				\draw (0,0) grid (4,4);
				\spy on (1,1) in node at (3,3);
			\end{tikzpicture}
		\end{lstlisting}
	\end{tcolorbox}
	
	\subsection{perspective - Perspectiva 3D}
	
	\subsubsection*{\texttt{\textbackslash usetikzlibrary\{perspective\}}}
	\begin{tcolorbox}[colback=green!5,colframe=green!50!black]
		\textbf{Descripción:} Proyección en perspectiva
		
		\textbf{Ejemplo:}
		\begin{lstlisting}[language=TeX]
			\begin{tikzpicture}[3d view]
				\draw (0,0,0) -- (1,0,0) -- (1,1,0) -- (0,1,0) -- cycle;
			\end{tikzpicture}
		\end{lstlisting}
	\end{tcolorbox}
	
	\subsection{angles - Ángulos}
	
	\subsubsection*{\texttt{\textbackslash usetikzlibrary\{angles\}}}
	\begin{tcolorbox}[colback=green!5,colframe=green!50!black]
		\textbf{Descripción:} Marca y etiqueta ángulos
		
		\textbf{Ejemplo:}
		\begin{lstlisting}[language=TeX]
			\coordinate (A) at (0,0);
			\coordinate (B) at (2,0);
			\coordinate (C) at (1,1);
			\draw (A) -- (B) -- (C) -- cycle;
			\pic[draw, angle radius=0.5cm] {angle = B--A--C};
		\end{lstlisting}
	\end{tcolorbox}
	
	\subsection{quotes - Comillas para Etiquetas}
	
	\subsubsection*{\texttt{\textbackslash usetikzlibrary\{quotes\}}}
	\begin{tcolorbox}[colback=green!5,colframe=green!50!black]
		\textbf{Descripción:} Sintaxis simplificada para etiquetas
		
		\textbf{Ejemplo:}
		\begin{lstlisting}[language=TeX]
			\draw (0,0) -- node["texto"] (2,0);
			\draw (0,0) edge["etiqueta", ->] (2,2);
		\end{lstlisting}
	\end{tcolorbox}
	
	\subsection{babel - Compatibilidad con Babel}
	
	\subsubsection*{\texttt{\textbackslash usetikzlibrary\{babel\}}}
	\begin{tcolorbox}[colback=green!5,colframe=green!50!black]
		\textbf{Descripción:} Resuelve conflictos con paquete babel
		
		\textbf{Ejemplo:}
		\begin{lstlisting}[language=TeX]
			\usepackage[spanish]{babel}
			\usetikzlibrary{babel}
		\end{lstlisting}
	\end{tcolorbox}
	
	\subsection{bending - Curvatura de Flechas}
	
	\subsubsection*{\texttt{\textbackslash usetikzlibrary\{bending\}}}
	\begin{tcolorbox}[colback=green!5,colframe=green!50!black]
		\textbf{Descripción:} Flechas que se curvan con el path
		
		\textbf{Ejemplo:}
		\begin{lstlisting}[language=TeX]
			\draw[->, bend left] (0,0) to (2,2);
		\end{lstlisting}
	\end{tcolorbox}
	
	\subsection{fadings - Desvanecimientos}
	
	\subsubsection*{\texttt{\textbackslash usetikzlibrary\{fadings\}}}
	\begin{tcolorbox}[colback=green!5,colframe=green!50!black]
		\textbf{Descripción:} Efectos de desvanecimiento
		
		\textbf{Fadings predefinidos:}
		\begin{itemize}[nosep]
			\item fade out, fade inside, fade right, fade left, fade up, fade down
		\end{itemize}
		
		\textbf{Ejemplo:}
		\begin{lstlisting}[language=TeX]
			\fill[red, path fading=east] (0,0) rectangle (3,1);
		\end{lstlisting}
	\end{tcolorbox}
	
	\subsection{external - Externalización}
	
	\subsubsection*{\texttt{\textbackslash usetikzlibrary\{external\}}}
	\begin{tcolorbox}[colback=green!5,colframe=green!50!black]
		\textbf{Descripción:} Guarda dibujos como archivos externos (acelera compilación)
		
		\textbf{Ejemplo:}
		\begin{lstlisting}[language=TeX]
			\usetikzlibrary{external}
			\tikzexternalize
		\end{lstlisting}
	\end{tcolorbox}
	
	\newpage
	
	\section{Opciones Adicionales Importantes}
	
	\subsection{Texto en Paths}
	
	\subsubsection*{\texttt{node[options] \{text\}}}
	\begin{tcolorbox}[colback=green!5,colframe=green!50!black]
		\textbf{Descripción:} Coloca texto a lo largo de un path
		
		\textbf{Ejemplo:}
		\begin{lstlisting}[language=TeX]
			\draw (0,0) -- node[above] {etiqueta} (2,0);
			\draw (0,0) -- node[midway, below] {medio} (2,0);
			\draw (0,0) -- node[pos=0.3] {30\%} (2,0);
		\end{lstlisting}
	\end{tcolorbox}
	
	\subsubsection*{\texttt{sloped}}
	\begin{tcolorbox}[colback=green!5,colframe=green!50!black]
		\textbf{Descripción:} Rota el texto siguiendo el path
		
		\textbf{Ejemplo:}
		\begin{lstlisting}[language=TeX]
			\draw (0,0) -- node[sloped, above] {texto inclinado} (2,1);
		\end{lstlisting}
	\end{tcolorbox}
	
	\subsection{Pics - Dibujos Reutilizables}
	
	\subsubsection*{\texttt{\textbackslash tikzset\{pics/name/.style=\{...\}\}}}
	\begin{tcolorbox}[colback=green!5,colframe=green!50!black]
		\textbf{Descripción:} Define un dibujo reutilizable
		
		\textbf{Ejemplo:}
		\begin{lstlisting}[language=TeX]
			\tikzset{
				pics/my shape/.style={
					code={
						\draw (0,0) circle (0.3);
						\fill (0,0) circle (0.1);
					}
				}
			}
			\pic at (0,0) {my shape};
			\pic at (2,0) {my shape};
		\end{lstlisting}
	\end{tcolorbox}
	
	\subsection{Estilos Personalizados}
	
	\subsubsection*{\texttt{\textbackslash tikzset\{style name/.style=\{...\}\}}}
	\begin{tcolorbox}[colback=green!5,colframe=green!50!black]
		\textbf{Descripción:} Define estilos reutilizables
		
		\textbf{Ejemplo:}
		\begin{lstlisting}[language=TeX]
			\tikzset{
				my arrow/.style={->, thick, blue},
				my node/.style={draw, circle, fill=red!20}
			}
			\draw[my arrow] (0,0) -- (2,0);
			\node[my node] at (1,1) {A};
		\end{lstlisting}
	\end{tcolorbox}
	
	\subsection{Capas (Layers)}
	
	\subsubsection*{\texttt{pgfdeclarelayer}}
	\begin{tcolorbox}[colback=green!5,colframe=green!50!black]
		\textbf{Descripción:} Define capas de dibujo
		
		\textbf{Ejemplo:}
		\begin{lstlisting}[language=TeX]
			\pgfdeclarelayer{background}
			\pgfdeclarelayer{foreground}
			\pgfsetlayers{background,main,foreground}
			
			\begin{pgfonlayer}{background}
				\fill[yellow] (0,0) rectangle (3,2);
			\end{pgfonlayer}
		\end{lstlisting}
	\end{tcolorbox}
	
	\section{Dibujo 3D Básico}
	
	\subsection{Coordenadas 3D}
	
	\subsubsection*{\texttt{(x,y,z)} -- Coordenadas 3D}
	\begin{tcolorbox}[colback=green!5,colframe=green!50!black]
		\textbf{Descripción:} Coordenadas tridimensionales
		
		\textbf{Ejemplo:}
		\begin{lstlisting}[language=TeX]
			\draw (0,0,0) -- (1,0,0) -- (1,1,0) -- (0,1,0) -- cycle;
			\draw (0,0,0) -- (0,0,1);
		\end{lstlisting}
	\end{tcolorbox}
	
	\subsubsection*{\texttt{canvas is xy plane at z=value}}
	\begin{tcolorbox}[colback=green!5,colframe=green!50!black]
		\textbf{Descripción:} Define plano de trabajo
		
		\textbf{Ejemplo:}
		\begin{lstlisting}[language=TeX]
			\begin{scope}[canvas is xy plane at z=0]
				\draw (0,0) rectangle (2,2);
			\end{scope}
			\begin{scope}[canvas is xz plane at y=0]
				\draw (0,0) rectangle (2,2);
			\end{scope}
		\end{lstlisting}
	\end{tcolorbox}
	
	\newpage
	
	\section*{\faCheckCircle\ Ejemplos Completos}
	
	\subsection*{Ejemplo 1: Diagrama de Flujo}
	
	\begin{tcolorbox}[colback=green!10,colframe=green!75!black,title=\faCode\ Diagrama de flujo]
		\begin{lstlisting}[language=TeX]
			\usetikzlibrary{shapes.geometric, arrows.meta, positioning}
			
			\begin{tikzpicture}[
				node distance=1.5cm,
				startstop/.style={rectangle, rounded corners, draw, fill=red!20},
				process/.style={rectangle, draw, fill=blue!20},
				decision/.style={diamond, draw, fill=green!20, aspect=2}
				]
				\node[startstop] (start) {Inicio};
				\node[process, below=of start] (proc1) {Proceso 1};
				\node[decision, below=of proc1] (dec1) {¿Condición?};
				\node[process, below=of dec1] (proc2) {Proceso 2};
				\node[startstop, below=of proc2] (stop) {Fin};
				
				\draw[-Stealth] (start) -- (proc1);
				\draw[-Stealth] (proc1) -- (dec1);
				\draw[-Stealth] (dec1) -- node[right] {Sí} (proc2);
				\draw[-Stealth] (dec1) -| node[above] {No} ++(3,0) |- (proc1);
				\draw[-Stealth] (proc2) -- (stop);
			\end{tikzpicture}
		\end{lstlisting}
	\end{tcolorbox}
	
	\subsection*{Ejemplo 2: Gráfica de Función}
	
	\begin{tcolorbox}[colback=purple!10,colframe=purple!75!black,title=\faCode\ Gráfica matemática]
		\begin{lstlisting}[language=TeX]
			\begin{tikzpicture}
				% Ejes
				\draw[->] (-0.5,0) -- (4,0) node[right] {$x$};
				\draw[->] (0,-0.5) -- (0,3) node[above] {$y$};
				
				% Cuadrícula
				\draw[gray!30, very thin] (0,0) grid (3.5,2.5);
				
				% Función seno
				\draw[blue, thick, domain=0:3.5, samples=100] 
				plot (\x, {1.5 + sin(\x r)});
				
				% Puntos
				\foreach \x in {0,1,2,3}
				\fill (\x, {1.5 + sin(\x r)}) circle (2pt);
				
				% Etiquetas
				\node at (3.5,2.5) {$f(x) = \sin(x) + 1.5$};
			\end{tikzpicture}
		\end{lstlisting}
	\end{tcolorbox}
	
	\subsection*{Ejemplo 3: Diagrama de Circuito}
	
	\begin{tcolorbox}[colback=orange!10,colframe=orange!75!black,title=\faCode\ Circuito simple]
		\begin{lstlisting}[language=TeX]
			\usetikzlibrary{circuits.ee.IEC}
			
			\begin{tikzpicture}[circuit ee IEC]
				\draw (0,0) to[resistor={info=$R_1$}] (2,0)
				to[capacitor={info=$C$}] (2,2)
				to[battery={info=$V$}] (0,2)
				to (0,0);
				\draw (2,0) to[resistor={info=$R_2$}] (4,0);
			\end{tikzpicture}
		\end{lstlisting}
	\end{tcolorbox}
	
	\subsection*{Ejemplo 4: Árbol}
	
	\begin{tcolorbox}[colback=cyan!10,colframe=cyan!75!black,title=\faCode\ Árbol binario]
		\begin{lstlisting}[language=TeX]
			\usetikzlibrary{trees}
			
			\begin{tikzpicture}[
				level 1/.style={sibling distance=3cm},
				level 2/.style={sibling distance=1.5cm},
				every node/.style={circle, draw}
				]
				\node {1}
				child {node {2}
					child {node {4}}
					child {node {5}}
				}
				child {node {3}
					child {node {6}}
					child {node {7}}
				};
			\end{tikzpicture}
		\end{lstlisting}
	\end{tcolorbox}
	
	\subsection*{Ejemplo 5: Formas Geométricas con Patrones}
	
	\begin{tcolorbox}[colback=yellow!10,colframe=yellow!75!black,title=\faCode\ Formas con patrones]
		\begin{lstlisting}[language=TeX]
			\usetikzlibrary{patterns, shapes.geometric}
			
			\begin{tikzpicture}
				\node[draw, star, star points=5, minimum size=2cm,
				fill=yellow, pattern=dots] at (0,0) {};
				\node[draw, regular polygon, regular polygon sides=6,
				minimum size=2cm, fill=blue!20, 
				pattern=horizontal lines] at (3,0) {};
				\node[draw, diamond, minimum size=2cm,
				fill=red!20, pattern=crosshatch] at (6,0) {};
			\end{tikzpicture}
		\end{lstlisting}
	\end{tcolorbox}
	
	\subsection*{Ejemplo 6: Cubo 3D}
	
	\begin{tcolorbox}[colback=red!10,colframe=red!75!black,title=\faCode\ Cubo tridimensional]
		\begin{lstlisting}[language=TeX]
			\begin{tikzpicture}
				% Cara frontal
				\draw[fill=blue!20] (0,0,0) -- (2,0,0) -- (2,2,0) -- (0,2,0) -- cycle;
				% Cara superior
				\draw[fill=red!20] (0,2,0) -- (2,2,0) -- (2,2,2) -- (0,2,2) -- cycle;
				% Cara derecha
				\draw[fill=green!20] (2,0,0) -- (2,2,0) -- (2,2,2) -- (2,0,2) -- cycle;
				% Aristas ocultas
				\draw[dashed] (0,0,0) -- (0,0,2) -- (0,2,2);
				\draw[dashed] (0,0,2) -- (2,0,2);
			\end{tikzpicture}
		\end{lstlisting}
	\end{tcolorbox}
	
	\newpage
	
	\section*{\faLightbulb\ Tips Importantes}
	
	\begin{tcolorbox}[colback=blue!10,colframe=blue!75!black]
		\begin{itemize}[leftmargin=*]
			\item TikZ usa unidades por defecto en cm, pero puedes especificar otras: pt, mm, in
			\item Siempre termina los comandos con punto y coma (;)
			\item Usa \texttt{\textbackslash usetikzlibrary\{babel\}} si usas babel con idiomas que modifican caracteres activos
			\item Para dibujos complejos, considera usar \texttt{external} para acelerar compilación
			\item Los nodos tienen anclas: north, south, east, west, north east, north west, south east, south west, center
			\item Puedes combinar opciones: \texttt{[red, thick, dashed, ->]}
			\item Usa \texttt{remember picture, overlay} para conectar elementos entre diferentes tikzpictures
			\item Para debugging, usa \texttt{[show background rectangle]} con backgrounds library
			\item La librería \texttt{calc} es muy útil para cálculos de coordenadas
			\item Siempre nombra tus nodos importantes para referencias futuras
		\end{itemize}
	\end{tcolorbox}
	
	\section*{\faExclamationTriangle\ Errores Comunes}
	
	\begin{tcolorbox}[colback=red!10,colframe=red!75!black]
		\begin{itemize}[leftmargin=*]
			\item \textbf{Olvidar el punto y coma}: Todos los comandos de dibujo deben terminar con ;
			\item \textbf{Paréntesis vs llaves}: Coordenadas usan (), opciones usan []
			\item \textbf{Unidades inconsistentes}: Mezclar coordenadas con y sin unidades
			\item \textbf{Nombres de nodos duplicados}: Cada nodo debe tener nombre único
			\item \textbf{Olvidar cargar librerías}: Algunas funciones requieren \texttt{\textbackslash usetikzlibrary}
			\item \textbf{Usar comillas incorrectas}: Para texto usa \{texto\}, no "texto"
			\item \textbf{Coordenadas relativas mal usadas}: ++ mueve el cursor, + no lo mueve
		\end{itemize}
	\end{tcolorbox}
	
	\vspace{1cm}
	
	\begin{center}
		\textit{Documento generado con \LaTeX{} y TikZ -- \today}
		
		\textit{Este documento cubre las funcionalidades principales de TikZ.}
		
		\textit{Para más información, consulta el manual oficial de TikZ/PGF.}
	\end{center}
	
\end{document}