\documentclass[12pt,a4paper]{book}

% Paquetes básicos
\usepackage[utf8]{inputenc}
\usepackage[spanish,es-tabla]{babel}
\usepackage[margin=2.5cm]{geometry}
\usepackage{xcolor}
\usepackage{tcolorbox}
\usepackage{enumitem}
\usepackage{fontawesome5}
\usepackage{lipsum}
\usepackage{graphicx}
\usepackage{hyperref}

% Paquetes para personalización de estructura
\usepackage{titlesec}
\usepackage{titletoc}
\usepackage{tocloft}
\usepackage{sectsty}

% Paquetes para ejercicios
\usepackage{amsmath}
\usepackage{amsthm}
\usepackage{xsim}

% Paquetes para portadas
\usepackage{eso-pic}

% Colores personalizados
\definecolor{chaptercolor}{RGB}{70,130,180}
\definecolor{sectioncolor}{RGB}{39,174,96}
\definecolor{fondoejemplo}{RGB}{245,245,245}

% Configuración de hyperref
\hypersetup{
	colorlinks=true,
	linkcolor=blue,
	urlcolor=blue,
	citecolor=blue,
	pdftitle={Aplicación de Personalización de Estructura en LaTeX},
	pdfauthor={Ejemplos Prácticos}
}

% Título
\title{\textbf{\Huge Aplicación de Personalización de Estructura en \LaTeX{}}\\[0.5cm]\large Ejemplos Prácticos de Todos los Comandos y Entornos}
\author{}
\date{\today}

\begin{document}

\maketitle
\thispagestyle{empty}

\begin{tcolorbox}[colback=blue!5,colframe=blue!75!black,title=\faInfoCircle\ Introducción]
Este documento contiene ejemplos prácticos de aplicación de todos los paquetes, comandos y entornos presentados en la guía de Personalización de Estructura en \LaTeX{}. Cada sección incluye ejemplos funcionales con texto de demostración usando \texttt{lipsum}.

\textbf{Contenido:}
\begin{itemize}
\item Personalización de capítulos con \texttt{titlesec}
\item Personalización de secciones con \texttt{sectsty}
\item Tablas de contenido con \texttt{tocloft} y \texttt{titletoc}
\item Creación de portadas personalizadas
\item Sistemas de ejercicios y respuestas
\item Ejemplos completos integrados
\end{itemize}
\end{tcolorbox}

\tableofcontents
\newpage

\part{Personalización de Capítulos y Secciones}

\chapter{Ejemplos con titlesec}

\section{Introducción a los Ejemplos}

\begin{tcolorbox}[colback=green!10,colframe=green!60!black,title=Objetivo de este Capítulo]
En este capítulo veremos ejemplos prácticos de cómo personalizar capítulos y secciones usando el paquete \texttt{titlesec}. Cada ejemplo muestra diferentes estilos y configuraciones.
\end{tcolorbox}

\subsection{Ejemplo 1: Formato Básico de Capítulo}

Este ejemplo muestra un formato simple de capítulo. El código utilizado es:

\begin{verbatim}
\titleformat{\chapter}[display]
  {\normalfont\huge\bfseries}
  {\chaptertitlename\ \thechapter}
  {20pt}
  {\Huge}
\end{verbatim}

\textbf{Texto de demostración:}

\lipsum[1-2]

\subsection{Ejemplo 2: Capítulo con Línea Decorativa}

En este caso, añadimos líneas decorativas antes y después del título del capítulo:

\begin{verbatim}
\titleformat{\chapter}[display]
  {\normalfont\huge\bfseries\color{blue}}
  {\filleft\MakeUppercase{\chaptertitlename} \Huge\thechapter}
  {4ex}
  {\titlerule\vspace{2ex}\filleft}
  [\vspace{2ex}\titlerule]
\end{verbatim}

\textbf{Demostración con texto:}

\lipsum[3]

\begin{tcolorbox}[colback=yellow!10,colframe=orange!75!black,title=\faLightbulb\ Nota]
Las líneas decorativas se crean con \texttt{\textbackslash titlerule} y el espaciado se controla con \texttt{\textbackslash vspace}.
\end{tcolorbox}

\lipsum[4]

\subsection{Ejemplo 3: Formato de Sección con Caja de Color}

Las secciones también pueden personalizarse. Este ejemplo usa cajas de color:

\textbf{Código aplicado:}
\begin{verbatim}
\titleformat{\section}
  {\normalfont\Large\bfseries}
  {\colorbox{blue!80}{\color{white}\thesection}}
  {1em}
  {\colorbox{blue!20}{#1}}
\end{verbatim}

\lipsum[5-6]

\subsection{Ejemplo 4: Ajuste de Espaciado con titlespacing}

El comando \texttt{\textbackslash titlespacing} permite controlar los espacios antes y después de los títulos:

\begin{verbatim}
\titlespacing*{\chapter}{0pt}{50pt}{40pt}
\titlespacing*{\section}{0pt}{20pt}{10pt}
\end{verbatim}

\lipsum[7]

\section{Personalización Avanzada}

\subsection{Capítulo con Número Grande}

Este diseño coloca un número de capítulo muy grande y prominente.

\lipsum[8-9]

\begin{tcolorbox}[colback=cyan!10,colframe=cyan!75!black,title=\faCode\ Ventajas del Diseño]
\begin{itemize}
\item Visual impactante
\item Fácil navegación
\item Estilo moderno
\item Personalizable con colores
\end{itemize}
\end{tcolorbox}

\lipsum[10]

\subsection{Sección con Línea Inferior}

Las secciones pueden tener una línea decorativa debajo del título.

\lipsum[11-12]

\subsubsection{Subsubsección de Ejemplo}

Las subsubsecciones mantienen el formato estándar pero pueden personalizarse también.

\lipsum[13]

\section{Aplicaciones Prácticas}

\subsection{Documentos Académicos}

En documentos académicos, la claridad y jerarquía visual son fundamentales.

\lipsum[14-15]

\subsection{Reportes Técnicos}

Los reportes técnicos se benefician de un formato profesional y estructurado.

\lipsum[16]

\begin{tcolorbox}[colback=red!10,colframe=red!75!black,title=\faExclamationTriangle\ Advertencia]
Recuerda que \texttt{titlesec} no es compatible con las clases KOMA-Script. Si usas \texttt{scrbook} o \texttt{scrartcl}, deberás usar los comandos nativos de KOMA.
\end{tcolorbox}

\lipsum[17]

\chapter{Ejemplos con sectsty}

\section{Introducción al Paquete sectsty}

\begin{tcolorbox}[colback=green!10,colframe=green!60!black,title=¿Qué es sectsty?]
El paquete \texttt{sectsty} proporciona una forma más simple de cambiar los estilos de secciones sin la complejidad de \texttt{titlesec}. Es ideal para cambios de fuente y color rápidos.
\end{tcolorbox}

\lipsum[18-19]

\subsection{Ejemplo 1: Cambiar Fuente de Todas las Secciones}

Con el comando \texttt{\textbackslash allsectionsfont}, podemos cambiar todas las secciones a la vez:

\begin{verbatim}
\allsectionsfont{\sffamily}
\end{verbatim}

\lipsum[20]

\subsection{Ejemplo 2: Colores para Diferentes Niveles}

Podemos asignar colores diferentes a cada nivel:

\begin{verbatim}
\chapterfont{\color{blue}\Huge}
\sectionfont{\color{red}\Large}
\subsectionfont{\color{green}\large}
\end{verbatim}

\lipsum[21-22]

\subsection{Ejemplo 3: Combinación de Estilos}

Es posible combinar diferentes características:

\lipsum[23]

\subsubsection{Subsubsección con Estilo}

Las subsubsecciones también pueden ser estilizadas.

\lipsum[24]

\section{Comparación: titlesec vs sectsty}

\subsection{Ventajas de titlesec}

\begin{itemize}
\item Control total sobre el formato
\item Puede cambiar la estructura completamente
\item Soporta formas complejas (display, hang, etc.)
\item Ideal para diseños personalizados
\end{itemize}

\lipsum[25]

\subsection{Ventajas de sectsty}

\begin{itemize}
\item Sintaxis más simple
\item Rápido para cambios básicos
\item Menos propenso a errores
\item Ideal para ajustes de fuente y color
\end{itemize}

\lipsum[26]

\section{Aplicaciones Combinadas}

\subsection{Uso Conjunto (con Precaución)}

Aunque es posible usar ambos paquetes, se recomienda elegir uno u otro para evitar conflictos.

\lipsum[27-28]

\subsection{Recomendaciones}

\begin{tcolorbox}[colback=blue!10,colframe=blue!75!black,title=\faLightbulb\ Mejores Prácticas]
\begin{enumerate}
\item Para cambios simples: usa \texttt{sectsty}
\item Para personalización completa: usa \texttt{titlesec}
\item No mezcles ambos paquetes en el mismo documento
\item Prueba tu diseño antes de aplicarlo a todo el documento
\end{enumerate}
\end{tcolorbox}

\lipsum[29]

\chapter{Personalización de Tablas de Contenido}

\section{Ejemplos con tocloft}

\subsection{Introducción a las Personalizaciones}

El paquete \texttt{tocloft} permite personalizar completamente el aspecto de las tablas de contenido, listas de figuras y listas de tablas.

\lipsum[30-31]

\subsection{Ejemplo 1: Cambiar el Título del TOC}

Podemos cambiar los nombres de los índices:

\begin{verbatim}
\renewcommand{\contentsname}{Índice General}
\renewcommand{\listfigurename}{Índice de Figuras}
\renewcommand{\listtablename}{Índice de Tablas}
\end{verbatim}

\lipsum[32]

\subsection{Ejemplo 2: Formato de Capítulos en TOC}

Podemos hacer que los capítulos aparezcan en negrita y con texto adicional:

\begin{verbatim}
\renewcommand{\cftchapfont}{\bfseries\large}
\renewcommand{\cftchappagefont}{\bfseries\large}
\renewcommand{\cftchappresnum}{Capítulo }
\renewcommand{\cftchapaftersnum}{:}
\setlength{\cftchapnumwidth}{5em}
\end{verbatim}

\lipsum[33-34]

\subsection{Ejemplo 3: Formato de Secciones en TOC}

Las secciones pueden tener formato diferente al de los capítulos:

\begin{verbatim}
\renewcommand{\cftsecfont}{\normalfont}
\renewcommand{\cftsecpagefont}{\normalfont}
\setlength{\cftsecindent}{2em}
\setlength{\cftsecnumwidth}{3em}
\end{verbatim}

\lipsum[35]

\subsection{Ejemplo 4: Puntos Guía Personalizados}

Los puntos que conectan el título con el número de página pueden personalizarse:

\lipsum[36]

\begin{tcolorbox}[colback=yellow!10,colframe=orange!75!black,title=\faInfoCircle\ Opciones de Puntos]
\begin{itemize}
\item \texttt{\textbackslash cftnodots}: Elimina los puntos
\item Valores bajos (1-2): Puntos más densos
\item Valores altos (5-6): Puntos más espaciados
\end{itemize}
\end{tcolorbox}

\lipsum[37]

\section{Ejemplos con titletoc}

\subsection{Introducción a titletoc}

El paquete \texttt{titletoc} trabaja junto con \texttt{titlesec} y ofrece control más avanzado.

\lipsum[38-39]

\subsection{Ejemplo 1: Formato Personalizado con titlecontents}

El comando \texttt{\textbackslash titlecontents} permite definir el formato completo:

\begin{verbatim}
\titlecontents{chapter}[0em]
  {\vspace{1em}\bfseries\large}
  {\contentslabel{2em}}
  {}
  {\hfill\contentspage}
\end{verbatim}

\lipsum[40]

\subsection{Ejemplo 2: TOC con Diseño Moderno}

Podemos crear diseños más creativos para el TOC:

\lipsum[41-42]

\section{Control de Profundidad del TOC}

\subsection{Ajustar Niveles Visibles}

El contador \texttt{tocdepth} controla qué niveles aparecen en el TOC:

\begin{verbatim}
\setcounter{tocdepth}{3}
% 0 = solo capítulos
% 1 = capítulos y secciones
% 2 = hasta subsecciones
% 3 = hasta subsubsecciones
\end{verbatim}

\lipsum[43]

\subsection{Ejemplo Práctico de Profundidad}

Dependiendo del tipo de documento, necesitaremos diferentes profundidades:

\lipsum[44-45]

\subsubsection{Documentos Largos}

Para libros extensos, es mejor limitar la profundidad a 1 o 2.

\lipsum[46]

\subsubsection{Documentos Cortos}

Para artículos o reportes breves, podemos mostrar todos los niveles.

\lipsum[47]

\section{Mini Tablas de Contenido}

\subsection{Concepto de Mini-TOC}

Las mini tablas de contenido muestran el contenido de un capítulo al inicio del mismo.

\lipsum[48]

\begin{tcolorbox}[colback=cyan!10,colframe=cyan!75!black,title=\faCode\ Paquete minitoc]
El paquete \texttt{minitoc} permite crear mini-TOC automáticamente al inicio de cada capítulo. Esto es muy útil en libros largos donde cada capítulo tiene muchas secciones.
\end{tcolorbox}

\lipsum[49]

\subsection{Aplicaciones de Mini-TOC}

\lipsum[50-51]

\chapter{Creación de Portadas Personalizadas}

\section{Uso del Entorno titlepage}

\subsection{Portadas Simples}

El entorno \texttt{titlepage} crea una página dedicada para el título.

\lipsum[52]

\subsection{Ejemplo 1: Portada Básica Centrada}

Una portada simple con elementos centrados verticalmente:

\begin{verbatim}
\begin{titlepage}
  \centering
  \vspace*{2cm}
  {\Huge\bfseries Título del Documento\par}
  \vspace{1cm}
  {\Large Subtítulo\par}
  \vspace{2cm}
  {\large Autor: Juan Pérez\par}
  \vfill
  {\large Universidad XYZ\par}
  {\large \today\par}
\end{titlepage}
\end{verbatim}

\lipsum[53-54]

\subsection{Ejemplo 2: Portada con Logo}

Las portadas académicas suelen incluir el logo de la institución:

\lipsum[55]

\begin{tcolorbox}[colback=yellow!10,colframe=orange!75!black,title=\faLightbulb\ Consejos]
\begin{itemize}
\item Usa \texttt{\textbackslash vfill} para distribuir el espacio verticalmente
\item \texttt{\textbackslash vspace*} mantiene el espacio al inicio de página
\item Combina tamaños de fuente para crear jerarquía visual
\end{itemize}
\end{tcolorbox}

\lipsum[56]

\subsection{Ejemplo 3: Portada con Color de Fondo}

Podemos añadir colores para hacer la portada más atractiva:

\lipsum[57-58]

\section{Paquete pdfpages}

\subsection{Inserción de PDFs Externos}

El paquete \texttt{pdfpages} permite insertar páginas de documentos PDF externos como portada.

\lipsum[59]

\subsection{Ejemplo 1: Incluir una Portada Pre-diseñada}

Si tienes una portada diseñada en otro programa:

\begin{verbatim}
\includepdf{portada.pdf}
\end{verbatim}

\lipsum[60]

\subsection{Ejemplo 2: Múltiples Páginas}

También podemos incluir varias páginas:

\begin{verbatim}
\includepdf[pages={1-3}]{documento.pdf}
\includepdf[pages=-]{documento.pdf}  % todas
\end{verbatim}

\lipsum[61]

\subsection{Ejemplo 3: Opciones Avanzadas}

El paquete ofrece muchas opciones:

\lipsum[62-63]

\section{Fondos y Marcas de Agua}

\subsection{Paquete background}

Podemos añadir imágenes de fondo a las páginas:

\lipsum[64]

\subsection{Paquete eso-pic}

Este paquete permite posicionar elementos en ubicaciones absolutas:

\lipsum[65-66]

\subsection{Ejemplo: Marca de Agua BORRADOR}

Para documentos en revisión, es útil añadir marcas de agua:

\begin{verbatim}
\AddToShipoutPictureBG{%
  \AtPageCenter{%
    \makebox(0,0){\rotatebox{45}{\textcolor{gray!30}{%
      \fontsize{5cm}{5cm}\selectfont BORRADOR}}}}}
\end{verbatim}

\lipsum[67]

\section{Diseños de Portadas Avanzados}

\subsection{Portada Estilo Corporativo}

Las portadas corporativas suelen tener un diseño más sobrio y profesional.

\lipsum[68-69]

\subsection{Portada Estilo Académico}

Las tesis y trabajos académicos requieren elementos específicos:

\lipsum[70]

\begin{tcolorbox}[colback=blue!10,colframe=blue!75!black,title=\faBook\ Elementos Típicos]
\begin{itemize}
\item Logo de la universidad
\item Nombre completo de la institución
\item Título de la tesis/trabajo
\item Nombre del autor
\item Nombre del director/tutor
\item Requisito que cumple (grado, maestría, etc.)
\item Fecha y lugar
\end{itemize}
\end{tcolorbox}

\lipsum[71]

\chapter{Sistemas de Ejercicios y Respuestas}

\section{Entornos Básicos con amsthm}

\subsection{Introducción a amsthm}

El paquete \texttt{amsthm} proporciona entornos básicos para teoremas, definiciones y ejercicios.

\lipsum[72]

\subsection{Ejemplo 1: Definición de Entornos}

Primero debemos definir los entornos en el preámbulo:

\begin{verbatim}
\theoremstyle{definition}
\newtheorem{ejercicio}{Ejercicio}[chapter]
\theoremstyle{remark}
\newtheorem*{solucion}{Solución}
\end{verbatim}

\lipsum[73]

\subsection{Ejemplo 2: Uso de Ejercicios}

Ahora veamos ejemplos prácticos:

\theoremstyle{definition}
\newtheorem{ejercicio}{Ejercicio}[section]
\theoremstyle{remark}
\newtheorem*{solucion}{Solución}

\begin{ejercicio}
Calcular la derivada de la función $f(x) = x^3 + 2x^2 - 5x + 7$.
\end{ejercicio}

\lipsum[74]

\begin{solucion}
Aplicando las reglas de derivación:
$$f'(x) = 3x^2 + 4x - 5$$
\end{solucion}

\lipsum[75]

\begin{ejercicio}
Resolver la integral $\int (3x^2 + 2x) \, dx$.
\end{ejercicio}

\lipsum[76]

\begin{solucion}
Integrando término a término:
$$\int (3x^2 + 2x) \, dx = x^3 + x^2 + C$$
\end{solucion}

\subsection{Ventajas de amsthm}

\begin{itemize}
\item Simple y directo
\item Integrado con AMS-LaTeX
\item Suficiente para necesidades básicas
\item Numeración automática
\end{itemize}

\lipsum[77]

\section{Sistema Avanzado con xsim}

\subsection{Introducción a xsim}

El paquete \texttt{xsim} es un sistema moderno y flexible para ejercicios y soluciones.

\lipsum[78]

\subsection{Ejemplo 1: Configuración Básica}

La configuración en el preámbulo:

\begin{verbatim}
\usepackage{xsim}
\xsimsetup{
  exercise/name = Ejercicio,
  solution/name = Solución
}
\end{verbatim}

\lipsum[79]

\subsection{Ejemplo 2: Ejercicios con xsim}

Ahora creemos algunos ejercicios:

\begin{exercise}[subtitle={Álgebra Lineal}]
Dadas dos matrices $A$ y $B$ de dimensión $2 \times 2$, donde $A$ tiene elementos 1, 2, 3, 4 y $B$ tiene elementos 5, 6, 7, 8, calcular la suma $A + B$.
\end{exercise}

\lipsum[80]

\begin{solution}
Sumando elemento a elemento, obtenemos que la matriz resultante tiene elementos 6, 8, 10, 12 en ese orden.
\end{solution}

\lipsum[81]

\begin{exercise}[subtitle={Trigonometría}]
Demostrar que $\sin^2(x) + \cos^2(x) = 1$.
\end{exercise}

\begin{solution}
Esta es la identidad pitagórica fundamental. Se deriva directamente del teorema de Pitágoras aplicado al círculo unitario.
\end{solution}

\lipsum[82]

\subsection{Ejemplo 3: Ejercicios con Puntos}

Los ejercicios pueden tener puntuación:

\begin{exercise}[subtitle={Geometría}, points={10}]
Calcular el área de un círculo de radio $r = 5$ cm.
\end{exercise}

\begin{solution}
Usando la fórmula del área del círculo:
$$A = \pi r^2 = \pi (5)^2 = 25\pi \approx 78.54 \text{ cm}^2$$
\end{solution}

\lipsum[83]

\subsection{Ventajas de xsim}

\begin{tcolorbox}[colback=green!10,colframe=green!60!black,title=\faCheckCircle\ Características]
\begin{itemize}
\item Sistema moderno y mantenido activamente
\item Soluciones se pueden imprimir al final
\item Soporte para puntuación y dificultad
\item Plantillas personalizables
\item Filtrado por tema o capítulo
\item Estadísticas automáticas
\end{itemize}
\end{tcolorbox}

\lipsum[84]

\section{Comparación de Sistemas}

\subsection{amsthm vs xsim}

\lipsum[85]

\begin{center}
\begin{tabular}{|p{0.25\textwidth}|p{0.3\textwidth}|p{0.3\textwidth}|}
\hline
\textbf{Característica} & \textbf{amsthm} & \textbf{xsim} \\
\hline
Facilidad de uso & Muy simple & Requiere configuración \\
\hline
Gestión de soluciones & Manual & Automática \\
\hline
Numeración & Automática básica & Avanzada y flexible \\
\hline
Puntuación & No & Sí \\
\hline
Impresión selectiva & No & Sí \\
\hline
Ideal para & Documentos simples & Libros de ejercicios \\
\hline
\end{tabular}
\end{center}

\lipsum[86]

\subsection{Recomendaciones de Uso}

\begin{itemize}
\item \textbf{Para apuntes de clase}: usa \texttt{amsthm}
\item \textbf{Para libros de texto}: usa \texttt{xsim}
\item \textbf{Para exámenes}: usa \texttt{xsim} con ocultación de soluciones
\item \textbf{Para guías de estudio}: usa \texttt{xsim} con soluciones al final
\end{itemize}

\lipsum[87]

\section{Personalización Avanzada}

\subsection{Entornos Personalizados con tcolorbox}

Podemos combinar sistemas de ejercicios con cajas de color:

\lipsum[88]

\begin{tcolorbox}[colback=blue!5,colframe=blue!75!black,title=\faEdit\ Ejercicio 1]
Resolver la ecuación cuadrática:
$$x^2 - 5x + 6 = 0$$
\end{tcolorbox}

\lipsum[89]

\begin{tcolorbox}[colback=green!5,colframe=green!75!black,title=\faCheck\ Solución]
Factorizando:
$$x^2 - 5x + 6 = (x-2)(x-3) = 0$$
Por lo tanto: $x = 2$ o $x = 3$
\end{tcolorbox}

\lipsum[90]

\subsection{Plantillas Visuales}

Las plantillas visuales mejoran la experiencia de lectura:

\lipsum[91-92]

\chapter{Ejemplos Integrados Completos}

\section{Documento Académico Completo}

\subsection{Características del Diseño}

Este tipo de documento combina todos los elementos vistos:

\begin{itemize}
\item Portada institucional
\item Tabla de contenido personalizada
\item Capítulos con diseño moderno
\item Sistema de ejercicios integrado
\end{itemize}

\lipsum[93-94]

\subsection{Aplicación en Tesis}

Las tesis de grado requieren un formato específico:

\lipsum[95]

\begin{tcolorbox}[colback=yellow!10,colframe=orange!75!black,title=\faExclamationTriangle\ Importante]
Siempre verifica los requisitos de formato de tu institución antes de personalizar. Algunas universidades tienen plantillas obligatorias.
\end{tcolorbox}

\lipsum[96]

\subsection{Aplicación en Libros de Texto}

Los libros de texto educativos se benefician enormemente de estos elementos:

\lipsum[97-98]

\section{Reporte Técnico Profesional}

\subsection{Diseño Corporativo}

Los reportes técnicos requieren un diseño sobrio y profesional:

\lipsum[99]

\subsection{Elementos Esenciales}

\begin{itemize}
\item Portada con logo corporativo
\item Índice ejecutivo
\item Tabla de contenidos detallada
\item Capítulos bien estructurados
\item Referencias y bibliografía
\end{itemize}

\lipsum[100]

\section{Material Didáctico}

\subsection{Guías de Estudio}

Las guías de estudio combinan teoría y práctica:

\lipsum[101-102]

\subsection{Cuadernos de Ejercicios}

Los cuadernos de ejercicios se enfocan en la práctica:

\lipsum[103]

\begin{exercise}[subtitle={Práctica Final}]
Integra todos los conceptos aprendidos para resolver:

Dado un sistema de ecuaciones lineales: $2x + 3y = 7$ y $x - y = 1$.

Encuentra los valores de $x$ e $y$.
\end{exercise}

\lipsum[104]

\begin{solution}
De la segunda ecuación: $x = 1 + y$

Sustituyendo en la primera: $2(1+y) + 3y = 7$

Simplificando: $2 + 2y + 3y = 7$, entonces $5y = 5$, por lo tanto $y = 1$

Y entonces: $x = 1 + 1 = 2$

Solución: $(x, y) = (2, 1)$
\end{solution}

\lipsum[105]

\chapter{Mejores Prácticas y Recomendaciones}

\section{Principios de Diseño}

\subsection{Consistencia Visual}

La consistencia es fundamental en cualquier documento largo:

\lipsum[106]

\begin{tcolorbox}[colback=blue!10,colframe=blue!75!black,title=\faLightbulb\ Reglas de Oro]
\begin{enumerate}
\item Mantén el mismo estilo en todo el documento
\item Usa colores de forma coherente
\item Define una jerarquía visual clara
\item No mezcles demasiados estilos diferentes
\end{enumerate}
\end{tcolorbox}

\lipsum[107]

\subsection{Legibilidad}

El diseño debe mejorar, no dificultar, la lectura:

\lipsum[108-109]

\subsection{Espaciado Apropiado}

El espacio en blanco es tan importante como el contenido:

\lipsum[110]

\section{Errores Comunes}

\subsection{Incompatibilidades de Paquetes}

Algunos paquetes no funcionan bien juntos:

\lipsum[111]

\begin{tcolorbox}[colback=red!10,colframe=red!75!black,title=\faExclamationTriangle\ Precaución]
\begin{itemize}
\item \texttt{titlesec} y KOMA-Script son incompatibles
\item \texttt{tocloft} y clase \texttt{memoir} pueden entrar en conflicto
\item Carga \texttt{hyperref} al final del preámbulo
\end{itemize}
\end{tcolorbox}

\lipsum[112]

\subsection{Sobrecomplicación}

No todo documento necesita personalización extrema:

\lipsum[113]

\subsection{Falta de Pruebas}

Siempre prueba los cambios:

\lipsum[114]

\section{Flujo de Trabajo Recomendado}

\subsection{Planificación}

Antes de comenzar a escribir:

\begin{enumerate}
\item Define el tipo de documento
\item Identifica los elementos necesarios
\item Elige los paquetes apropiados
\item Crea una plantilla básica
\item Prueba con contenido de ejemplo
\end{enumerate}

\lipsum[115]

\subsection{Implementación}

Durante la escritura:

\lipsum[116]

\subsection{Revisión Final}

Al terminar:

\lipsum[117]

\section{Recursos y Referencias}

\subsection{Documentación Oficial}

\begin{itemize}
\item \texttt{texdoc titlesec} -- Manual completo de titlesec
\item \texttt{texdoc tocloft} -- Guía de tocloft
\item \texttt{texdoc xsim} -- Documentación de xsim
\item \texttt{texdoc amsthm} -- Referencia de amsthm
\end{itemize}

\lipsum[118]

\subsection{Comunidades y Foros}

\begin{itemize}
\item TeX Stack Exchange (\texttt{tex.stackexchange.com})
\item LaTeX Community (\texttt{latex.org/forum})
\item Overleaf Guides (\texttt{overleaf.com/learn})
\end{itemize}

\lipsum[119]

\subsection{Plantillas Útiles}

Muchas instituciones y editoriales proporcionan plantillas:

\lipsum[120]

\chapter{Conclusiones}

\section{Resumen de Paquetes}

\subsection{Para Capítulos y Secciones}

\begin{itemize}
\item \textbf{titlesec}: Control total y diseños complejos
\item \textbf{sectsty}: Cambios simples y rápidos
\end{itemize}

\lipsum[121]

\subsection{Para Tablas de Contenido}

\begin{itemize}
\item \textbf{tocloft}: Simple y efectivo
\item \textbf{titletoc}: Más potente, trabaja con titlesec
\item \textbf{minitoc}: Mini-TOC por capítulo
\end{itemize}

\lipsum[122]

\subsection{Para Ejercicios}

\begin{itemize}
\item \textbf{amsthm}: Básico y suficiente
\item \textbf{xsim}: Sistema completo y moderno
\item \textbf{exsheets}: Alternativa a xsim (menos mantenido)
\end{itemize}

\lipsum[123]

\section{Palabras Finales}

\subsection{La Importancia del Diseño}

Un buen diseño mejora la comprensión y retención:

\lipsum[124-125]

\subsection{Experimentación}

No tengas miedo de experimentar con diferentes estilos:

\lipsum[126]

\begin{tcolorbox}[colback=green!10,colframe=green!60!black,title=\faCheckCircle\ Mensaje Final]
La personalización de la estructura en \LaTeX{} es una herramienta poderosa para crear documentos profesionales y atractivos. Con los paquetes y técnicas presentados en este documento, tienes todo lo necesario para crear documentos de alta calidad.

¡Sigue experimentando y aprendiendo!
\end{tcolorbox}

\lipsum[127]

\section*{Agradecimientos}

Este documento fue creado como material de apoyo para el aprendizaje de \LaTeX{} y la personalización de documentos. Todos los ejemplos son funcionales y pueden ser adaptados según las necesidades específicas de cada proyecto.

\vspace{1cm}

\begin{center}
\large\textit{Fin del documento}

\vspace{0.5cm}

\normalsize Documento generado con \LaTeX{} -- \today
\end{center}

% Imprimir todas las soluciones al final
\newpage
\chapter*{Soluciones a los Ejercicios}
\addcontentsline{toc}{chapter}{Soluciones a los Ejercicios}

\printsolutions

\end{document}
