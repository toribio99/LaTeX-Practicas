\documentclass[11pt,a4paper]{book}

% Paquetes básicos
\usepackage[utf8]{inputenc}
\usepackage[spanish,es-tabla]{babel}
\usepackage[margin=2.5cm]{geometry}
\usepackage{xcolor}
\usepackage{tcolorbox}
\usepackage{enumitem}
\usepackage{fontawesome5}
\usepackage{listings}
\usepackage{graphicx}
\usepackage{lipsum}

% Paquetes para personalización de estructura
\usepackage{titlesec}
\usepackage{titletoc}
\usepackage{tocloft}
\usepackage{sectsty}

% Paquetes para ejercicios
\usepackage{amsthm}

% Colores personalizados
\definecolor{commandcolor}{RGB}{39,174,96}
\definecolor{codebackground}{RGB}{245,245,245}
\definecolor{chaptercolor}{RGB}{70,130,180}

% Configuración de listings con soporte para español
\lstset{
	basicstyle=\ttfamily\footnotesize,
	backgroundcolor=\color{codebackground},
	breaklines=true,
	columns=fullflexible,
	keepspaces=true,
	frame=single,
	rulecolor=\color{gray!30},
	inputencoding=utf8,
	extendedchars=true,
	literate=
		{á}{{\'a}}1 {é}{{\'e}}1 {í}{{\'i}}1 {ó}{{\'o}}1 {ú}{{\'u}}1
		{Á}{{\'A}}1 {É}{{\'E}}1 {Í}{{\'I}}1 {Ó}{{\'O}}1 {Ú}{{\'U}}1
		{ñ}{{\~n}}1 {Ñ}{{\~N}}1
		{ü}{{\"u}}1 {Ü}{{\"U}}1
		{¿}{{?`}}1 {¡}{{!`}}1
}

% Título
\title{\textbf{\Huge Personalización de Estructura en \LaTeX{}}\\\large Guía Completa de Capítulos, Secciones, Portadas, TOC y Ejercicios}
\author{}
\date{\today}
 \usepackage[
%colorlinks=true,        % Enlaces con color (en lugar de cajas)
linkcolor=blue,         % Color de enlaces internos
urlcolor=cyan,          % Color de URLs
citecolor=green,        % Color de citas bibliográficas
filecolor=magenta,      % Color de enlaces a archivos
pdfborder={0 0 0},      % Sin bordes en los enlaces
bookmarks=true,         % Crear marcadores en el PDF
bookmarksopen=true,     % Marcadores expandidos al abrir
pdftitle={Mi Título},   % Título del PDF
pdfauthor={Mi Nombre},  % Autor del PDF
pdfsubject={Tema},      % Tema del documento
pdfkeywords={palabra1, palabra2}, % Palabras clave
%hidelinks,              % Ocultar todos los bordes/colores de enlaces
unicode=true,           % Permitir caracteres Unicode en marcadores
breaklinks=true         % Permitir saltos de línea en enlaces
]{hyperref}

\begin{document}

\maketitle
\thispagestyle{empty}

\begin{tcolorbox}[colback=blue!5,colframe=blue!75!black,title=\faInfoCircle\ Introducción]
Esta guía exhaustiva cubre todos los paquetes, comandos y entornos disponibles en \LaTeX{} para personalizar la estructura de documentos, incluyendo capítulos, secciones, portadas, tablas de contenido, y sistemas de ejercicios con respuestas.
\end{tcolorbox}

\tableofcontents

\chapter{Personalización de Capítulos}

\section{Paquete titlesec}

\subsection{Introducción a titlesec}

\begin{tcolorbox}[colback=green!5,colframe=green!50!black]
	\textbf{Descripción:} El paquete \texttt{titlesec} permite personalizar completamente el formato de capítulos, secciones y subsecciones.

	\textbf{Carga:}
	\begin{lstlisting}[language=TeX]
\usepackage{titlesec}
	\end{lstlisting}
\end{tcolorbox}

\subsection{Comando titleformat}

\subsubsection*{\texttt{\textbackslash titleformat}}
\begin{tcolorbox}[colback=green!5,colframe=green!50!black]
	\textbf{Descripción:} Define el formato de títulos (capítulos, secciones, etc.)

	\textbf{Sintaxis:}
	\begin{lstlisting}[language=TeX]
\titleformat{comando}[forma]{formato}{etiqueta}{separación}{antes}[después]
	\end{lstlisting}

	\textbf{Parámetros:}
	\begin{itemize}[nosep]
		\item \texttt{comando} -- \textbackslash chapter, \textbackslash section, etc.
		\item \texttt{forma} -- hang, block, display, runin, leftmargin, etc.
		\item \texttt{formato} -- Formato del título completo
		\item \texttt{etiqueta} -- Formato del número (ej: Capítulo 1)
		\item \texttt{separación} -- Espacio entre etiqueta y título
		\item \texttt{antes} -- Código antes del título
		\item \texttt{después} -- Código después del título (opcional)
	\end{itemize}
\end{tcolorbox}

\subsection{Ejemplos de Capítulos Personalizados}

\subsubsection*{Ejemplo 1: Capítulo con línea decorativa}
\begin{tcolorbox}[colback=cyan!10,colframe=cyan!75!black,title=\faCode\ Código]
	\begin{lstlisting}[language=TeX]
\titleformat{\chapter}[display]
  {\normalfont\huge\bfseries}
  {\chaptertitlename\ \thechapter}
  {20pt}
  {\Huge}
	\end{lstlisting}
\end{tcolorbox}

\subsubsection*{Ejemplo 2: Capítulo con caja de color}
\begin{tcolorbox}[colback=cyan!10,colframe=cyan!75!black,title=\faCode\ Código]
	\begin{lstlisting}[language=TeX]
\titleformat{\chapter}[display]
  {\normalfont\huge\bfseries\color{blue}}
  {\filleft\MakeUppercase{\chaptertitlename} \Huge\thechapter}
  {4ex}
  {\titlerule\vspace{2ex}\filleft}
  [\vspace{2ex}\titlerule]
	\end{lstlisting}
\end{tcolorbox}

\subsubsection*{Ejemplo 3: Capítulo con número grande}
\begin{tcolorbox}[colback=cyan!10,colframe=cyan!75!black,title=\faCode\ Código]
	\begin{lstlisting}[language=TeX]
\titleformat{\chapter}[display]
  {\normalfont\bfseries\filright}
  {\LARGE\thechapter}
  {1ex}
  {\titlerule[2pt]\vspace{2ex}\huge}
  [\vspace{1ex}\titlerule]
	\end{lstlisting}
\end{tcolorbox}

\subsubsection*{Ejemplo 4: Capítulo estilo moderno}
\begin{tcolorbox}[colback=cyan!10,colframe=cyan!75!black,title=\faCode\ Código]
	\begin{lstlisting}[language=TeX]
\usepackage{xcolor}

\titleformat{\chapter}[display]
  {\normalfont\huge\bfseries}
  {\colorbox{blue!80}{\parbox{3cm}{\centering\color{white}
   \fontsize{60}{60}\selectfont\thechapter}}}
  {10pt}
  {\Huge\raggedright}
	\end{lstlisting}
\end{tcolorbox}

\subsection{Comando titlespacing}

\subsubsection*{\texttt{\textbackslash titlespacing}}
\begin{tcolorbox}[colback=green!5,colframe=green!50!black]
	\textbf{Descripción:} Ajusta los espacios antes y después de títulos

	\textbf{Sintaxis:}
	\begin{lstlisting}[language=TeX]
\titlespacing*{comando}{izquierda}{antes}{después}[derecha]
	\end{lstlisting}

	\textbf{Ejemplo:}
	\begin{lstlisting}[language=TeX]
\titlespacing*{\chapter}{0pt}{50pt}{40pt}
\titlespacing*{\section}{0pt}{20pt}{10pt}
\titlespacing*{\subsection}{0pt}{15pt}{8pt}
	\end{lstlisting}
\end{tcolorbox}

\section{Personalización de Secciones}

\subsection{Formato de Secciones con titlesec}

\subsubsection*{Ejemplo 1: Sección con caja de color}
\begin{tcolorbox}[colback=cyan!10,colframe=cyan!75!black,title=\faCode\ Código]
	\begin{lstlisting}[language=TeX]
\titleformat{\section}
  {\normalfont\Large\bfseries\color{white}}
  {\colorbox{blue!80}{\parbox{1.5cm}{\centering\thesection}}}
  {1em}
  {\colorbox{blue!20}{\parbox{\dimexpr\textwidth-3cm\relax}{#1}}}
	\end{lstlisting}
\end{tcolorbox}

\subsubsection*{Ejemplo 2: Sección con línea inferior}
\begin{tcolorbox}[colback=cyan!10,colframe=cyan!75!black,title=\faCode\ Código]
	\begin{lstlisting}[language=TeX]
\titleformat{\section}
  {\normalfont\Large\bfseries}
  {\thesection}
  {1em}
  {#1}
  [\vspace{0.5ex}\titlerule]
	\end{lstlisting}
\end{tcolorbox}

\subsubsection*{Ejemplo 3: Subsección con círculo}
\begin{tcolorbox}[colback=cyan!10,colframe=cyan!75!black,title=\faCode\ Código]
	\begin{lstlisting}[language=TeX]
\usepackage{tikz}

\titleformat{\subsection}
  {\normalfont\large\bfseries}
  {\tikz[baseline=(char.base)]{
    \node[circle,fill=blue!80,text=white,inner sep=2pt] (char)
         {\thesubsection};}}
  {1em}
  {#1}
	\end{lstlisting}
\end{tcolorbox}

\section{Paquete sectsty}

\subsection{Introducción a sectsty}

\begin{tcolorbox}[colback=green!5,colframe=green!50!black]
	\textbf{Descripción:} \texttt{sectsty} permite cambiar estilos de secciones de forma más simple que titlesec.

	\textbf{Carga:}
	\begin{lstlisting}[language=TeX]
\usepackage{sectsty}
	\end{lstlisting}
\end{tcolorbox}

\subsection{Comandos de sectsty}

\begin{tcolorbox}[colback=green!5,colframe=green!50!black]
	\textbf{Comandos disponibles:}
	\begin{itemize}[nosep]
		\item \texttt{\textbackslash allsectionsfont\{formato\}} -- Todas las secciones
		\item \texttt{\textbackslash chapterfont\{formato\}} -- Solo capítulos
		\item \texttt{\textbackslash sectionfont\{formato\}} -- Solo secciones
		\item \texttt{\textbackslash subsectionfont\{formato\}} -- Solo subsecciones
		\item \texttt{\textbackslash subsubsectionfont\{formato\}} -- Subsubsecciones
		\item \texttt{\textbackslash paragraphfont\{formato\}} -- Párrafos
		\item \texttt{\textbackslash subparagraphfont\{formato\}} -- Subpárrafos
	\end{itemize}

	\textbf{Ejemplo:}
	\begin{lstlisting}[language=TeX]
\allsectionsfont{\sffamily}  % Todas en sans-serif
\chapterfont{\color{blue}\Huge}
\sectionfont{\color{red}\Large}
	\end{lstlisting}
\end{tcolorbox}

\chapter{Tablas de Contenido (TOC)}

\section{Paquete tocloft}

\subsection{Introducción a tocloft}

\begin{tcolorbox}[colback=green!5,colframe=green!50!black]
	\textbf{Descripción:} \texttt{tocloft} permite control completo sobre el formato de la tabla de contenidos (TOC), lista de figuras (LOF) y lista de tablas (LOT).

	\textbf{Carga:}
	\begin{lstlisting}[language=TeX]
\usepackage{tocloft}
	\end{lstlisting}
\end{tcolorbox}

\subsection{Personalización del Título del TOC}

\begin{tcolorbox}[colback=green!5,colframe=green!50!black]
	\textbf{Comandos:}
	\begin{lstlisting}[language=TeX]
\renewcommand{\contentsname}{Índice General}
\renewcommand{\listfigurename}{Índice de Figuras}
\renewcommand{\listtablename}{Índice de Tablas}
	\end{lstlisting}
\end{tcolorbox}

\subsection{Formato de Entradas del TOC}

\subsubsection*{Formato de capítulos en TOC}
\begin{tcolorbox}[colback=green!5,colframe=green!50!black]
	\textbf{Comandos disponibles:}
	\begin{itemize}[nosep]
		\item \texttt{\textbackslash cftchapfont} -- Fuente del título de capítulo
		\item \texttt{\textbackslash cftchappagefont} -- Fuente del número de página
		\item \texttt{\textbackslash cftchappresnum} -- Texto antes del número
		\item \texttt{\textbackslash cftchapaftersnum} -- Texto después del número
		\item \texttt{\textbackslash cftchapnumwidth} -- Ancho del número
		\item \texttt{\textbackslash cftbeforechapskip} -- Espacio antes del capítulo
	\end{itemize}

	\textbf{Ejemplo:}
	\begin{lstlisting}[language=TeX]
\renewcommand{\cftchapfont}{\bfseries\large}
\renewcommand{\cftchappagefont}{\bfseries\large}
\renewcommand{\cftchappresnum}{Capítulo }
\renewcommand{\cftchapaftersnum}{:}
\setlength{\cftchapnumwidth}{5em}
	\end{lstlisting}
\end{tcolorbox}

\subsubsection*{Formato de secciones en TOC}
\begin{tcolorbox}[colback=green!5,colframe=green!50!black]
	\textbf{Comandos similares para secciones:}
	\begin{lstlisting}[language=TeX]
\renewcommand{\cftsecfont}{\normalfont}
\renewcommand{\cftsecpagefont}{\normalfont}
\setlength{\cftsecindent}{2em}
\setlength{\cftsecnumwidth}{3em}
	\end{lstlisting}
\end{tcolorbox}

\subsection{Puntos Guía (Leaders)}

\begin{tcolorbox}[colback=green!5,colframe=green!50!black]
	\textbf{Descripción:} Personalizar los puntos entre título y número de página

	\textbf{Comandos:}
	\begin{lstlisting}[language=TeX]
% Eliminar puntos guía
\renewcommand{\cftchapdotsep}{\cftnodots}
\renewcommand{\cftsecdotsep}{\cftnodots}

% Cambiar separación de puntos
\renewcommand{\cftchapdotsep}{1}  % Más denso
\renewcommand{\cftsecdotsep}{4.5} % Por defecto
	\end{lstlisting}
\end{tcolorbox}

\subsection{Profundidad del TOC}

\begin{tcolorbox}[colback=green!5,colframe=green!50!black]
	\textbf{Descripción:} Controlar qué niveles aparecen en el TOC

	\textbf{Comando:}
	\begin{lstlisting}[language=TeX]
\setcounter{tocdepth}{3}
% 0 = solo capítulos
% 1 = capítulos y secciones
% 2 = hasta subsecciones
% 3 = hasta subsubsecciones
	\end{lstlisting}
\end{tcolorbox}

\section{Paquete titletoc}

\subsection{Introducción a titletoc}

\begin{tcolorbox}[colback=green!5,colframe=green!50!black]
	\textbf{Descripción:} \texttt{titletoc} trabaja junto con \texttt{titlesec} para personalizar el TOC.

	\textbf{Carga:}
	\begin{lstlisting}[language=TeX]
\usepackage{titletoc}
	\end{lstlisting}
\end{tcolorbox}

\subsection{Comando titlecontents}

\begin{tcolorbox}[colback=green!5,colframe=green!50!black]
	\textbf{Sintaxis:}
	\begin{lstlisting}[language=TeX]
\titlecontents{sección}[margen izq]{código antes}
              {código numerado}{código sin número}
              {código después}[código al final]
	\end{lstlisting}

	\textbf{Ejemplo:}
	\begin{lstlisting}[language=TeX]
\titlecontents{chapter}[0em]
  {\vspace{1em}\bfseries\large}
  {\contentslabel{2em}}
  {}
  {\hfill\contentspage}
	\end{lstlisting}
\end{tcolorbox}

\section{Mini Tablas de Contenido}

\subsection{Paquete minitoc}

\begin{tcolorbox}[colback=green!5,colframe=green!50!black]
	\textbf{Descripción:} Crea mini TOC al inicio de cada capítulo

	\textbf{Carga y uso:}
	\begin{lstlisting}[language=TeX]
\usepackage{minitoc}
\dominitoc    % En el preámbulo

% En el documento
\tableofcontents
\chapter{Primer Capítulo}
\minitoc      % Muestra mini-TOC del capítulo
	\end{lstlisting}
\end{tcolorbox}

\section{Ejemplo Completo de TOC Personalizado}

\begin{tcolorbox}[colback=purple!10,colframe=purple!75!black,title=\faCode\ Ejemplo Integrado]
	\begin{lstlisting}[language=TeX]
\usepackage{tocloft}
\usepackage{xcolor}

% Título del TOC
\renewcommand{\contentsname}{%
  \begin{center}
    \color{blue}\Huge\bfseries Tabla de Contenidos
  \end{center}}

% Capítulos en negrita y azul
\renewcommand{\cftchapfont}{\bfseries\color{blue}}
\renewcommand{\cftchappagefont}{\bfseries\color{blue}}
\renewcommand{\cftchappresnum}{Cap. }
\setlength{\cftchapnumwidth}{3.5em}

% Secciones indentadas
\setlength{\cftsecindent}{3.5em}
\setlength{\cftsecnumwidth}{2.5em}

% Sin puntos guía para capítulos
\renewcommand{\cftchapdotsep}{\cftnodots}
	\end{lstlisting}
\end{tcolorbox}

\chapter{Creación de Portadas}

\section{Entorno titlepage}

\subsection{Uso Básico}

\begin{tcolorbox}[colback=green!5,colframe=green!50!black]
	\textbf{Descripción:} El entorno \texttt{titlepage} crea una página de título personalizada.

	\textbf{Sintaxis:}
	\begin{lstlisting}[language=TeX]
\begin{titlepage}
  % Contenido de la portada
\end{titlepage}
	\end{lstlisting}
\end{tcolorbox}

\subsection{Ejemplo 1: Portada Simple}

\begin{tcolorbox}[colback=cyan!10,colframe=cyan!75!black,title=\faCode\ Código]
	\begin{lstlisting}[language=TeX]
\begin{titlepage}
  \centering
  \vspace*{2cm}

  {\Huge\bfseries Título del Documento\par}
  \vspace{1cm}
  {\Large Subtítulo\par}
  \vspace{2cm}
  {\large Autor: Juan Pérez\par}
  \vfill
  {\large Universidad XYZ\par}
  {\large \today\par}
\end{titlepage}
	\end{lstlisting}
\end{tcolorbox}

\subsection{Ejemplo 2: Portada con Logo}

\begin{tcolorbox}[colback=cyan!10,colframe=cyan!75!black,title=\faCode\ Código]
	\begin{lstlisting}[language=TeX]
\begin{titlepage}
  \centering
  \includegraphics[width=0.3\textwidth]{logo.png}\par
  \vspace{1cm}
  {\scshape\LARGE Universidad XYZ \par}
  \vspace{1.5cm}
  {\huge\bfseries Título del Trabajo\par}
  \vspace{2cm}
  {\Large\itshape Juan Pérez\par}
  \vfill
  Presentado como requisito para\par
  el título de Licenciado en Ciencias
  \vfill
  {\large \today\par}
\end{titlepage}
	\end{lstlisting}
\end{tcolorbox}

\subsection{Ejemplo 3: Portada con Color de Fondo}

\begin{tcolorbox}[colback=cyan!10,colframe=cyan!75!black,title=\faCode\ Código]
	\begin{lstlisting}[language=TeX]
\usepackage{pagecolor}
\usepackage{afterpage}

\begin{titlepage}
  \pagecolor{blue!10}
  \centering
  \vspace*{3cm}

  \colorbox{blue!80}{%
    \parbox{0.8\textwidth}{%
      \centering
      \color{white}
      {\Huge\bfseries Título Principal\par}
      \vspace{0.5cm}
      {\Large Subtítulo del Documento\par}
    }
  }

  \vspace{3cm}
  {\LARGE Autor: María García\par}
  \vfill
  {\Large Institución\par}
  {\large \today\par}
\end{titlepage}
\afterpage{\nopagecolor}
	\end{lstlisting}
\end{tcolorbox}

\section{Paquete pdfpages}

\subsection{Inserción de PDFs Externos como Portada}

\begin{tcolorbox}[colback=green!5,colframe=green!50!black]
	\textbf{Descripción:} \texttt{pdfpages} permite insertar páginas de PDFs externos.

	\textbf{Carga:}
	\begin{lstlisting}[language=TeX]
\usepackage{pdfpages}
	\end{lstlisting}

	\textbf{Uso básico:}
	\begin{lstlisting}[language=TeX]
% Incluir una página
\includepdf{portada.pdf}

% Incluir varias páginas
\includepdf[pages={1-3}]{documento.pdf}

% Incluir todas las páginas
\includepdf[pages=-]{documento.pdf}
	\end{lstlisting}
\end{tcolorbox}

\subsection{Opciones de includepdf}

\begin{tcolorbox}[colback=green!5,colframe=green!50!black]
	\textbf{Opciones disponibles:}
	\begin{itemize}[nosep]
		\item \texttt{pages=\{lista\}} -- Páginas a incluir
		\item \texttt{nup=columnas x filas} -- Múltiples páginas por hoja
		\item \texttt{landscape} -- Orientación horizontal
		\item \texttt{angle=grados} -- Rotar páginas
		\item \texttt{scale=factor} -- Escalar páginas
		\item \texttt{frame=true} -- Añadir marco
		\item \texttt{pagecommand=comando} -- Ejecutar comando en cada página
	\end{itemize}

	\textbf{Ejemplo:}
	\begin{lstlisting}[language=TeX]
\includepdf[pages={1,3,5-7},nup=2x2,frame=true]{doc.pdf}
	\end{lstlisting}
\end{tcolorbox}

\section{Paquetes para Fondos y Marcas de Agua}

\subsection{Paquete background}

\begin{tcolorbox}[colback=green!5,colframe=green!50!black]
	\textbf{Descripción:} Añade fondos y marcas de agua a las páginas.

	\textbf{Carga y configuración:}
	\begin{lstlisting}[language=TeX]
\usepackage{background}

\backgroundsetup{
  scale=1,
  angle=0,
  opacity=0.1,
  contents={%
    \includegraphics[width=\paperwidth,height=\paperheight]{fondo.png}
  }
}
	\end{lstlisting}
\end{tcolorbox}

\subsection{Paquete eso-pic}

\begin{tcolorbox}[colback=green!5,colframe=green!50!black]
	\textbf{Descripción:} Añade imágenes o texto en posiciones absolutas.

	\textbf{Uso:}
	\begin{lstlisting}[language=TeX]
\usepackage{eso-pic}

% Añadir en todas las páginas
\AddToShipoutPictureBG{%
  \AtPageCenter{%
    \makebox(0,0){\rotatebox{45}{\textcolor{gray!30}{%
      \fontsize{5cm}{5cm}\selectfont BORRADOR}}}}
}

% Añadir solo en la página actual
\AddToShipoutPictureBG*{contenido}
	\end{lstlisting}
\end{tcolorbox}

\chapter{Listas de Ejercicios y Respuestas}

\section{Entornos Básicos con amsthm}

\subsection{Configuración Inicial}

\begin{tcolorbox}[colback=green!5,colframe=green!50!black]
	\textbf{Carga:}
	\begin{lstlisting}[language=TeX]
\usepackage{amsthm}

% Definir estilos
\theoremstyle{definition}
\newtheorem{ejercicio}{Ejercicio}[chapter]
\newtheorem{problema}{Problema}[chapter]

\theoremstyle{remark}
\newtheorem*{solucion}{Solución}
	\end{lstlisting}

	\textbf{Uso:}
	\begin{lstlisting}[language=TeX]
\begin{ejercicio}
  Calcular la derivada de $f(x) = x^2 + 3x$.
\end{ejercicio}

\begin{solucion}
  $f'(x) = 2x + 3$
\end{solucion}
	\end{lstlisting}
\end{tcolorbox}

\section{Paquete exercise}

\subsection{Introducción a exercise}

\begin{tcolorbox}[colback=green!5,colframe=green!50!black]
	\textbf{Descripción:} Sistema completo para ejercicios con respuestas automáticas.

	\textbf{Carga:}
	\begin{lstlisting}[language=TeX]
\usepackage{exercise}
	\end{lstlisting}
\end{tcolorbox}

\subsection{Entornos de exercise}

\begin{tcolorbox}[colback=green!5,colframe=green!50!black]
	\textbf{Entornos disponibles:}
	\begin{itemize}[nosep]
		\item \texttt{Exercise} -- Ejercicio principal
		\item \texttt{Answer} -- Respuesta al ejercicio
		\item \texttt{ExerciseList} -- Lista de ejercicios
	\end{itemize}

	\textbf{Ejemplo:}
	\begin{lstlisting}[language=TeX]
\begin{Exercise}[title={Derivadas}, label={ex:deriv}]
  Calcular $\frac{d}{dx}(x^3)$.
\end{Exercise}

\begin{Answer}[ref={ex:deriv}]
  $\frac{d}{dx}(x^3) = 3x^2$
\end{Answer}

% Mostrar todas las respuestas
\shipoutAnswer
	\end{lstlisting}
\end{tcolorbox}

\subsection{Personalización de exercise}

\begin{tcolorbox}[colback=green!5,colframe=green!50!black]
	\textbf{Opciones de Exercise:}
	\begin{itemize}[nosep]
		\item \texttt{title} -- Título del ejercicio
		\item \texttt{label} -- Etiqueta para referencia
		\item \texttt{difficulty} -- Nivel de dificultad
		\item \texttt{counter} -- Contador personalizado
	\end{itemize}

	\textbf{Comandos de configuración:}
	\begin{lstlisting}[language=TeX]
\renewcommand{\ExerciseName}{Ejercicio}
\renewcommand{\AnswerName}{Respuesta}
\renewcommand{\ExerciseHeader}{\textbf{\ExerciseName\ \ExerciseHeaderNB}}
	\end{lstlisting}
\end{tcolorbox}

\section{Paquete exsheets}

\subsection{Introducción a exsheets}

\begin{tcolorbox}[colback=green!5,colframe=green!50!black]
	\textbf{Descripción:} Sistema moderno y flexible para ejercicios y soluciones.

	\textbf{Carga:}
	\begin{lstlisting}[language=TeX]
\usepackage{exsheets}
	\end{lstlisting}
\end{tcolorbox}

\subsection{Entornos de exsheets}

\begin{tcolorbox}[colback=green!5,colframe=green!50!black]
	\textbf{Entornos:}
	\begin{itemize}[nosep]
		\item \texttt{question} -- Pregunta/ejercicio
		\item \texttt{solution} -- Solución
	\end{itemize}

	\textbf{Ejemplo básico:}
	\begin{lstlisting}[language=TeX]
\begin{question}
  ¿Cuánto es $2 + 2$?
\end{question}

\begin{solution}
  $2 + 2 = 4$
\end{solution}

% Imprimir todas las soluciones
\printsolutions
	\end{lstlisting}
\end{tcolorbox}

\subsection{Opciones de question}

\begin{tcolorbox}[colback=green!5,colframe=green!50!black]
	\textbf{Opciones disponibles:}
	\begin{lstlisting}[language=TeX]
\begin{question}[
  name={Derivadas},
  subtitle={Cálculo básico},
  points={5},
  topic={Cálculo}
]
  Contenido del ejercicio
\end{question}
	\end{lstlisting}
\end{tcolorbox}

\subsection{Configuración de exsheets}

\begin{tcolorbox}[colback=green!5,colframe=green!50!black]
	\textbf{Personalización:}
	\begin{lstlisting}[language=TeX]
% Configurar formato
\SetupExSheets{
  headings = block-subtitle,
  counter-format = ch.qu,
  counter-within = chapter
}

% Configurar soluciones
\SetupExSheets{solution/print = true}  % Mostrar soluciones
\SetupExSheets{solution/print = false} % Ocultar soluciones
	\end{lstlisting}
\end{tcolorbox}

\section{Paquete answers}

\subsection{Introducción a answers}

\begin{tcolorbox}[colback=green!5,colframe=green!50!black]
	\textbf{Descripción:} Gestiona ejercicios y respuestas en archivos separados.

	\textbf{Carga:}
	\begin{lstlisting}[language=TeX]
\usepackage{answers}
\Newassociation{sol}{Solution}{ans}

% En el preámbulo
\Opensolutionfile{ans}[respuestas]
	\end{lstlisting}
\end{tcolorbox}

\subsection{Uso de answers}

\begin{tcolorbox}[colback=green!5,colframe=green!50!black]
	\textbf{Ejemplo:}
	\begin{lstlisting}[language=TeX]
% Ejercicio con solución
\begin{enumerate}
\item ¿Cuánto es $5 \times 5$?
\begin{sol}
  $5 \times 5 = 25$
\end{sol}

\item Resolver $x^2 = 9$
\begin{sol}
  $x = \pm 3$
\end{sol}
\end{enumerate}

% Cerrar archivo de soluciones
\Closesolutionfile{ans}

% Más adelante, mostrar todas las soluciones
\section*{Soluciones}
\input{respuestas}
	\end{lstlisting}
\end{tcolorbox}

\section{Paquete xsim}

\subsection{Introducción a xsim}

\begin{tcolorbox}[colback=green!5,colframe=green!50!black]
	\textbf{Descripción:} Sistema moderno y extensible para ejercicios (sucesor de exsheets).

	\textbf{Carga:}
	\begin{lstlisting}[language=TeX]
\usepackage{xsim}
	\end{lstlisting}
\end{tcolorbox}

\subsection{Entornos de xsim}

\begin{tcolorbox}[colback=green!5,colframe=green!50!black]
	\textbf{Ejemplo:}
	\begin{lstlisting}[language=TeX]
\begin{exercise}[subtitle={Álgebra Básica}]
  Resolver la ecuación $2x + 5 = 13$
\end{exercise}

\begin{solution}
  \begin{align*}
    2x + 5 &= 13 \\
    2x &= 8 \\
    x &= 4
  \end{align*}
\end{solution}

% Imprimir soluciones
\printsolutions[print=all]
	\end{lstlisting}
\end{tcolorbox}

\subsection{Configuración de xsim}

\begin{tcolorbox}[colback=green!5,colframe=green!50!black]
	\textbf{Opciones de configuración:}
	\begin{lstlisting}[language=TeX]
\xsimsetup{
  exercise/template = default,
  solution/template = default,
  exercise/name = Ejercicio,
  solution/name = Solución,
  path = ejercicios  % Directorio para archivos
}
	\end{lstlisting}
\end{tcolorbox}

\section{Crear Lista de Ejercicios Personalizada}

\subsection{Lista Manual}

\begin{tcolorbox}[colback=cyan!10,colframe=cyan!75!black,title=\faCode\ Ejemplo]
	\begin{lstlisting}[language=TeX]
\newcounter{ejercicio}[chapter]
\newenvironment{miEjercicio}[1][]{%
  \refstepcounter{ejercicio}%
  \par\medskip\noindent
  \textbf{Ejercicio \theejercicio\ #1}\rmfamily
}{%
  \par\medskip
}

% Uso
\begin{miEjercicio}[Derivadas]
  Calcular la derivada de $f(x) = x^2$.
\end{miEjercicio}
	\end{lstlisting}
\end{tcolorbox}

\section{Ejemplo Completo Integrado}

\begin{tcolorbox}[colback=purple!10,colframe=purple!75!black,title=\faCode\ Sistema Completo]
	\begin{lstlisting}[language=TeX]
\documentclass{book}
\usepackage{xsim}
\usepackage{tcolorbox}

% Configuración
\xsimsetup{
  exercise/name = Ejercicio,
  solution/name = Solución,
  exercise/print = true,
  solution/print = false
}

% Plantilla personalizada
\DeclareExerciseEnvironmentTemplate{colorbox}{
  \tcolorbox[colback=blue!5,colframe=blue!75!black,
             title=\XSIMmixedcase{\GetExerciseName}~\GetExerciseProperty{counter}]
}{
  \endtcolorbox
}

\xsimsetup{exercise/template = colorbox}

\begin{document}

\chapter{Ejercicios de Cálculo}

\begin{exercise}[subtitle={Límites}]
  Calcular $\lim_{x \to 0} \frac{\sin x}{x}$
\end{exercise}

\begin{solution}
  $\lim_{x \to 0} \frac{\sin x}{x} = 1$
\end{solution}

\begin{exercise}[subtitle={Derivadas}]
  Derivar $f(x) = e^{2x}$
\end{exercise}

\begin{solution}
  $f'(x) = 2e^{2x}$
\end{solution}

% Mostrar todas las soluciones al final
\chapter*{Soluciones}
\printsolutions

\end{document}
	\end{lstlisting}
\end{tcolorbox}

\chapter{Ejemplos Completos y Plantillas}

\section{Plantilla de Libro Académico}

\begin{tcolorbox}[colback=purple!10,colframe=purple!75!black,title=\faCode\ Plantilla Completa]
	\begin{lstlisting}[language=TeX]
\documentclass[11pt,a4paper,twoside]{book}
\usepackage[utf8]{inputenc}
\usepackage[spanish]{babel}
\usepackage{titlesec}
\usepackage{tocloft}
\usepackage{xsim}
\usepackage{xcolor}
\usepackage{graphicx}

% Personalizar capítulos
\titleformat{\chapter}[display]
  {\normalfont\huge\bfseries\color{blue!80}}
  {\filleft\MakeUppercase{\chaptertitlename}\ \thechapter}
  {4ex}
  {\titlerule\vspace{2ex}\filleft}
  [\vspace{2ex}\titlerule]

% Personalizar TOC
\renewcommand{\cftchapfont}{\bfseries\color{blue!80}}
\renewcommand{\cftchappagefont}{\bfseries\color{blue!80}}

% Configurar ejercicios
\xsimsetup{
  exercise/name = Ejercicio,
  solution/name = Solución
}

\title{Título del Libro}
\author{Autor}
\date{\today}

\begin{document}

% Portada personalizada
\begin{titlepage}
  \centering
  \includegraphics[width=0.2\textwidth]{logo.png}\par
  \vspace{1cm}
  {\scshape\LARGE Universidad XYZ \par}
  \vspace{1.5cm}
  {\huge\bfseries Título del Libro\par}
  \vspace{2cm}
  {\Large\itshape Nombre del Autor\par}
  \vfill
  {\large \today\par}
\end{titlepage}

\tableofcontents

\chapter{Primer Capítulo}

\section{Introducción}
Contenido...

\begin{exercise}
  Ejercicio de ejemplo
\end{exercise}

\begin{solution}
  Solución del ejercicio
\end{solution}

% Al final del libro
\chapter*{Soluciones}
\addcontentsline{toc}{chapter}{Soluciones}
\printsolutions

\end{document}
	\end{lstlisting}
\end{tcolorbox}

\section{Plantilla de Reporte con Portada}

\begin{tcolorbox}[colback=purple!10,colframe=purple!75!black,title=\faCode\ Reporte Profesional]
	\begin{lstlisting}[language=TeX]
\documentclass[12pt,a4paper]{report}
\usepackage[utf8]{inputenc}
\usepackage[spanish]{babel}
\usepackage[margin=2.5cm]{geometry}
\usepackage{titlesec}
\usepackage{xcolor}
\usepackage{graphicx}

% Formato de capítulos
\titleformat{\chapter}[block]
  {\normalfont\huge\bfseries}
  {\colorbox{blue!80}{\color{white}\thechapter}}
  {10pt}
  {\huge}

\begin{document}

% Portada
\begin{titlepage}
  \centering
  \vspace*{2cm}

  \colorbox{blue!80}{%
    \parbox{0.9\textwidth}{%
      \centering\color{white}
      \vspace{1cm}
      {\Huge\bfseries REPORTE TÉCNICO\par}
      \vspace{0.5cm}
      {\Large Análisis de Resultados 2024\par}
      \vspace{1cm}
    }
  }

  \vspace{3cm}
  {\LARGE Preparado por:\par}
  {\Large Juan Pérez\par}
  \vspace{2cm}
  {\large Departamento de Ingeniería\par}
  {\large Universidad XYZ\par}
  \vfill
  {\large \today\par}
\end{titlepage}

\tableofcontents

\chapter{Introducción}
Contenido del reporte...

\end{document}
	\end{lstlisting}
\end{tcolorbox}

\chapter{Tips y Mejores Prácticas}

\section*{\faLightbulb\ Recomendaciones Generales}

\begin{tcolorbox}[colback=blue!10,colframe=blue!75!black]
	\textbf{Para personalización de capítulos:}
	\begin{itemize}[leftmargin=*]
		\item Usa \texttt{titlesec} para control completo
		\item Mantén consistencia en todo el documento
		\item Prueba diferentes formas: display, block, hang
		\item Combina con colores para mayor impacto visual
	\end{itemize}

	\textbf{Para tablas de contenido:}
	\begin{itemize}[leftmargin=*]
		\item \texttt{tocloft} es más simple, \texttt{titletoc} más potente
		\item Ajusta \texttt{tocdepth} según necesidad
		\item Personaliza cada nivel independientemente
		\item Considera mini-TOC para documentos largos
	\end{itemize}

	\textbf{Para portadas:}
	\begin{itemize}[leftmargin=*]
		\item Usa \texttt{titlepage} para portadas desde cero
		\item \texttt{pdfpages} para portadas pre-diseñadas
		\item Aprovecha \texttt{\textbackslash vfill} para espaciado vertical
		\item Combina con \texttt{xcolor} para fondos atractivos
	\end{itemize}

	\textbf{Para ejercicios:}
	\begin{itemize}[leftmargin=*]
		\item \texttt{xsim} es la opción más moderna
		\item \texttt{exsheets} también es excelente
		\item Usa \texttt{amsthm} para casos simples
		\item Considera separar respuestas en sección aparte
	\end{itemize}
\end{tcolorbox}

\section*{\faExclamationTriangle\ Errores Comunes}

\begin{tcolorbox}[colback=red!10,colframe=red!75!black]
	\textbf{Problemas frecuentes:}
	\begin{itemize}[leftmargin=*]
		\item \textbf{titlesec y KOMA-Script:} No son compatibles, elige uno
		\item \textbf{tocloft y memoir:} memoir tiene sus propios comandos
		\item \textbf{Recompilar:} TOC requiere 2 compilaciones para actualizarse
		\item \textbf{Numeración:} Verifica contadores antes de personalizarlos
		\item \textbf{Portadas:} Usa \texttt{\textbackslash thispagestyle\{empty\}} para quitar número
		\item \textbf{Ejercicios:} Cierra archivos de soluciones antes de imprimirlas
		\item \textbf{Referencias:} Usa \texttt{\textbackslash label} para referenciar ejercicios
	\end{itemize}
\end{tcolorbox}

\section*{\faBook\ Recursos Adicionales}

\begin{tcolorbox}[colback=yellow!10,colframe=orange!75!black]
	\textbf{Documentación oficial:}
	\begin{itemize}[leftmargin=*]
		\item \texttt{texdoc titlesec} -- Manual de titlesec
		\item \texttt{texdoc tocloft} -- Guía de tocloft
		\item \texttt{texdoc xsim} -- Documentación de xsim
		\item \texttt{texdoc pdfpages} -- Opciones de pdfpages
		\item \texttt{texdoc memoir} -- Clase memoir (todo integrado)
	\end{itemize}

	\textbf{Alternativas y complementos:}
	\begin{itemize}[leftmargin=*]
		\item Clase \texttt{memoir} -- Incluye todo integrado
		\item KOMA-Script (\texttt{scrbook}) -- Alternativa alemana potente
		\item \texttt{fancyhdr} -- Para encabezados personalizados
		\item \texttt{geometry} -- Control preciso de márgenes
		\item \texttt{hyperref} -- TOC clickeable en PDF
	\end{itemize}
\end{tcolorbox}

\vspace{2cm}

\begin{center}
	\begin{tcolorbox}[colback=blue!5,colframe=blue!75!black,width=0.9\textwidth]
		\centering
		\large\textbf{Documento generado con \LaTeX{}}\\[0.5em]
		\normalsize\textit{Guía Completa de Personalización de Estructura}\\[0.5em]
		\small\today
	\end{tcolorbox}

	\vspace{1cm}

	\textit{Esta guía cubre los principales paquetes para personalizar}\\
	\textit{la estructura de documentos en \LaTeX{}.}\\[0.5em]
	\textit{Para más información, consulta la documentación oficial de cada paquete.}
\end{center}

\end{document}
