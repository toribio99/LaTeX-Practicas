\documentclass[11pt,a4paper]{article}

% =====================================================
% PREÁMBULO: PAQUETES NECESARIOS
% =====================================================

% Detección de motor de compilación
\usepackage{iftex}

% Configuración según motor
\ifluatex
  % Si compila con LuaLaTeX
  \usepackage{fontspec}
  \usepackage{polyglossia}
  \setdefaultlanguage[variant=mexican]{spanish}
\else
  \ifxetex
    % Si compila con XeLaTeX
    \usepackage{fontspec}
    \usepackage{polyglossia}
    \setdefaultlanguage[variant=mexican]{spanish}
  \else
    % Si compila con pdfLaTeX
    \usepackage[T1]{fontenc}
    \usepackage[spanish,es-tabla]{babel}
  \fi
\fi

% Paquetes comunes
\usepackage[margin=2.5cm]{geometry}
\usepackage{amsmath}
\usepackage{xcolor}
\usepackage{tcolorbox}
\tcbuselibrary{listings,skins,breakable}
\usepackage{enumitem}
\usepackage{booktabs}
\usepackage{array}
\usepackage{graphicx}

% Definición de colores
\definecolor{azuloscuro}{RGB}{0,51,102}
\definecolor{azulclaro}{RGB}{230,240,250}
\definecolor{verdeoscuro}{RGB}{0,100,0}
\definecolor{verdeclaro}{RGB}{230,255,230}
\definecolor{rojoclaro}{RGB}{255,230,230}
\definecolor{amarilloclaro}{RGB}{255,255,200}
\definecolor{grisclaro}{RGB}{240,240,240}
\definecolor{naranjaoscuro}{RGB}{230,115,0}

% Configuración de listings
\lstset{
    basicstyle=\ttfamily\small,
    breaklines=true,
    columns=flexible,
    showstringspaces=false,
    frame=single,
    backgroundcolor=\color{grisclaro},
    xleftmargin=5pt,
    xrightmargin=5pt
}

% Hyperref al final
\usepackage[
    colorlinks=true,
    linkcolor=azuloscuro,
    urlcolor=blue,
    citecolor=verdeoscuro,
    pdftitle={Guía: pdfLaTeX vs LuaLaTeX vs XeLaTeX},
    pdfauthor={Documentación LaTeX},
    bookmarks=true,
    unicode=true,
    breaklinks=true
]{hyperref}

% =====================================================
% TÍTULO
% =====================================================
\title{\textbf{\Huge Guía Completa:}\\[0.5cm]
\Large pdfLaTeX vs XeLaTeX vs LuaLaTeX\\[0.3cm]
\large Codificación, Fuentes y Modernización de Documentos}
\author{Documentación técnica de LaTeX}
\date{\today}

% =====================================================
% INICIO DEL DOCUMENTO
% =====================================================
\begin{document}

\maketitle
\tableofcontents
\newpage

% =====================================================
% SECCIÓN 1: INTRODUCCIÓN E HISTORIA
% =====================================================
\section{Introducción: La evolución de LaTeX}

\begin{tcolorbox}[colback=azulclaro,colframe=azuloscuro,title=\textbf{Contexto histórico}]
LaTeX fue creado en los años 80, cuando la codificación ASCII (solo caracteres en inglés) era el estándar. Los caracteres acentuados como \textbf{á, é, í, ñ} no existían nativamente en el sistema, lo que creó la necesidad de paquetes especiales para manejar diferentes idiomas.
\end{tcolorbox}

\subsection{Línea de tiempo}

\begin{itemize}[leftmargin=*]
    \item \textbf{1984:} Donald Knuth crea TeX (solo ASCII)
    \item \textbf{1985:} Leslie Lamport crea LaTeX sobre TeX
    \item \textbf{1990s:} Aparecen \texttt{inputenc} y \texttt{fontenc} para manejar caracteres especiales
    \item \textbf{2000s:} Surgen XeTeX y LuaTeX como motores modernos
    \item \textbf{2004:} Aparece \texttt{fontspec} para fuentes OpenType/TrueType
    \item \textbf{2018:} \textbf{LaTeX hace UTF-8 el estándar por defecto}
    \item \textbf{2020-2025:} LuaLaTeX se consolida como motor moderno recomendado
\end{itemize}

\begin{tcolorbox}[colback=amarilloclaro,colframe=naranjaoscuro,title=\textbf{Punto clave de 2018}]
\textbf{Cambio fundamental:} Desde LaTeX 2018, UTF-8 es la codificación por defecto. Esto significa que \texttt{\textbackslash usepackage[utf8]\{inputenc\}} ya no es necesario en versiones modernas de LaTeX.
\end{tcolorbox}

% =====================================================
% SECCIÓN 2: LOS TRES MOTORES
% =====================================================
\section{Los tres motores de compilación}

\subsection{¿Qué es un motor de compilación?}

\begin{tcolorbox}[colback=verdeclaro,colframe=verdeoscuro]
Un \textbf{motor de compilación} (o \textit{engine}) es el programa que convierte tu código LaTeX (.tex) en un documento PDF. Cada motor tiene diferentes capacidades y limitaciones.
\end{tcolorbox}

\subsection{pdfLaTeX: El motor tradicional}

\begin{tcolorbox}[colback=azulclaro,colframe=azuloscuro,title=\textbf{pdfLaTeX},breakable]
\textbf{Año de aparición:} 1996-2000

\textbf{Características:}
\begin{itemize}
    \item Motor tradicional más antiguo y estable
    \item Genera PDFs directamente (antes se generaba DVI primero)
    \item Usa fuentes Type 1 (formatos antiguos)
    \item Requería \texttt{inputenc} para UTF-8 (hasta 2018)
    \item Requiere \texttt{fontenc} para salida correcta de acentos
\end{itemize}

\textbf{Ventajas:}
\begin{itemize}
    \item Más rápido en compilación
    \item Compatible con todo el código LaTeX histórico
    \item Ampliamente documentado
    \item Consume menos memoria
\end{itemize}

\textbf{Desventajas:}
\begin{itemize}
    \item No puede usar fuentes del sistema directamente
    \item Soporte limitado para Unicode avanzado
    \item Requiere configuración adicional para idiomas no latinos
    \item No puede usar fuentes OpenType/TrueType sin conversión
\end{itemize}

\textbf{Comando de compilación:}
\begin{lstlisting}[language=bash]
pdflatex documento.tex
\end{lstlisting}
\end{tcolorbox}

\subsection{XeLaTeX: El puente hacia la modernidad}

\begin{tcolorbox}[colback=azulclaro,colframe=azuloscuro,title=\textbf{XeLaTeX},breakable]
\textbf{Año de aparición:} 2004

\textbf{Características:}
\begin{itemize}
    \item Motor moderno basado en XeTeX
    \item UTF-8 nativo (no necesita \texttt{inputenc})
    \item Usa fuentes OpenType y TrueType directamente
    \item Acceso a todas las fuentes del sistema operativo
    \item No necesita \texttt{fontenc}
\end{itemize}

\textbf{Ventajas:}
\begin{itemize}
    \item UTF-8 nativo desde el principio
    \item Acceso directo a fuentes del sistema
    \item Excelente soporte multilingüe
    \item Manejo avanzado de tipografía
    \item Usa \texttt{fontspec} para configuración de fuentes
\end{itemize}

\textbf{Desventajas:}
\begin{itemize}
    \item Ligeramente más lento que pdfLaTeX
    \item Algunos paquetes antiguos incompatibles
    \item Menor optimización de microtipografía que LuaLaTeX
\end{itemize}

\textbf{Comando de compilación:}
\begin{lstlisting}[language=bash]
xelatex documento.tex
\end{lstlisting}
\end{tcolorbox}

\subsection{LuaLaTeX: El motor del futuro}

\begin{tcolorbox}[colback=verdeclaro,colframe=verdeoscuro,title=\textbf{LuaLaTeX (RECOMENDADO)},breakable]
\textbf{Año de aparición:} 2007-2010 (estable desde ~2013)

\textbf{Características:}
\begin{itemize}
    \item Motor más moderno basado en LuaTeX
    \item UTF-8 nativo (no necesita \texttt{inputenc})
    \item Integra lenguaje Lua para programación avanzada
    \item Usa fuentes OpenType y TrueType directamente
    \item No necesita \texttt{fontenc}
    \item Mejor manejo de memoria que XeLaTeX
\end{itemize}

\textbf{Ventajas:}
\begin{itemize}
    \item UTF-8 nativo completo
    \item Todas las fuentes del sistema disponibles
    \item Programable con Lua (automatizaciones complejas)
    \item Mejor microtipografía que XeLaTeX
    \item Desarrollo activo y comunidad creciente
    \item Maneja mejor documentos grandes
    \item Futuro oficial de LaTeX según el proyecto LaTeX3
\end{itemize}

\textbf{Desventajas:}
\begin{itemize}
    \item Más lento que pdfLaTeX (pero más rápido que XeLaTeX)
    \item Mayor consumo de memoria
    \item Algunos paquetes muy antiguos incompatibles
\end{itemize}

\textbf{Comando de compilación:}
\begin{lstlisting}[language=bash]
lualatex documento.tex
\end{lstlisting}
\end{tcolorbox}

\begin{tcolorbox}[colback=amarilloclaro,colframe=naranjaoscuro,title=\textbf{Recomendación oficial}]
El proyecto LaTeX recomienda migrar gradualmente a \textbf{LuaLaTeX} para nuevos documentos. Es considerado el motor del futuro y recibirá las innovaciones más importantes.
\end{tcolorbox}

% =====================================================
% SECCIÓN 3: TABLA COMPARATIVA
% =====================================================
\section{Tabla comparativa de motores}

\begin{table}[htbp]
\centering
\small
\begin{tabular}{|>{\raggedright\arraybackslash}p{3cm}|c|c|c|}
\hline
\textbf{Característica} & \textbf{pdfLaTeX} & \textbf{XeLaTeX} & \textbf{LuaLaTeX} \\
\hline
\hline
Año de origen & 1996 & 2004 & 2007 \\
\hline
UTF-8 nativo & Solo desde 2018 & Sí (siempre) & Sí (siempre) \\
\hline
Necesita \texttt{inputenc} & No (desde 2018) & No & No \\
\hline
Necesita \texttt{fontenc} & Sí & No & No \\
\hline
Usa \texttt{fontspec} & No & Sí & Sí \\
\hline
Fuentes sistema & No & Sí & Sí \\
\hline
OpenType/TrueType & No directo & Sí & Sí \\
\hline
Velocidad & Rápido & Medio & Medio-Lento \\
\hline
Memoria & Baja & Media & Alta \\
\hline
Microtipografía & Buena & Buena & Excelente \\
\hline
Programable & No & No & Sí (Lua) \\
\hline
Futuro LaTeX3 & Mantenimiento & Mantenimiento & Principal \\
\hline
Multilingüe avanzado & Limitado & Excelente & Excelente \\
\hline
Recomendado para & Docs simples & Tipografía & Todo uso \\
\hline
\end{tabular}
\caption{Comparación detallada de los tres motores principales}
\end{table}

% =====================================================
% SECCIÓN 4: INPUTENC Y FONTENC EN PROFUNDIDAD
% =====================================================
\section{Los paquetes \texttt{inputenc} y \texttt{fontenc}}

\subsection{¿Qué es \texttt{inputenc}?}

\begin{tcolorbox}[colback=azulclaro,colframe=azuloscuro,title=\textbf{inputenc: Codificación de entrada},breakable]
\textbf{Propósito:} Le dice a LaTeX cómo interpretar los caracteres en tu archivo .tex

\textbf{Historia:}
\begin{itemize}
    \item Antes de 2018, LaTeX asumía ASCII (solo inglés)
    \item Para escribir "café", tenías que usar códigos especiales o declarar codificación
    \item \texttt{\textbackslash usepackage[utf8]\{inputenc\}} permitía escribir "café" directamente
\end{itemize}

\textbf{Estado actual (2018+):}
\begin{itemize}
    \item UTF-8 es la codificación por defecto
    \item Ya NO es necesario incluir \texttt{inputenc} en pdfLaTeX moderno
    \item NO hacer nada es equivalente a \texttt{\textbackslash usepackage[utf8]\{inputenc\}}
\end{itemize}

\textbf{Cuándo aún es necesario:}
\begin{itemize}
    \item Archivos antiguos con codificación latin1 o iso-8859-1
    \item Sistemas con LaTeX anterior a 2018 (muy raro hoy)
\end{itemize}
\end{tcolorbox}

\subsection{¿Qué es \texttt{fontenc}?}

\begin{tcolorbox}[colback=azulclaro,colframe=azuloscuro,title=\textbf{fontenc: Codificación de salida},breakable]
\textbf{Propósito:} Le dice a LaTeX cómo representar los caracteres en el PDF final

\textbf{Historia:}
\begin{itemize}
    \item Por defecto, pdfLaTeX usa codificación OT1 (Original TeX encoding)
    \item OT1 tiene solo 128 caracteres (no incluye acentos directamente)
    \item Con OT1, "café" se construye con acentos superpuestos (problemas en copiar/pegar)
    \item T1 tiene 256 caracteres con acentos nativos
\end{itemize}

\textbf{Problema sin T1:}
\begin{verbatim}
% Sin fontenc o con OT1:
café → "caf\'e" en el PDF (no se puede buscar o copiar bien)

% Con T1:
café → "café" en el PDF (carácter nativo, buscable)
\end{verbatim}

\textbf{Estado actual:}
\begin{itemize}
    \item \textbf{pdfLaTeX:} Todavía NECESITA \texttt{\textbackslash usepackage[T1]\{fontenc\}}
    \item \textbf{XeLaTeX/LuaLaTeX:} NO necesitan \texttt{fontenc} (usan TU encoding automático)
\end{itemize}
\end{tcolorbox}

\subsection{Diferencia entre entrada y salida}

\begin{tcolorbox}[colback=verdeclaro,colframe=verdeoscuro,title=\textbf{Analogía conceptual}]
\textbf{inputenc:} Cómo \textit{lees} el archivo (lo que escribiste en tu editor)

\textbf{fontenc:} Cómo se \textit{imprime} en el PDF (lo que ve el lector)

\textbf{Ejemplo:}
\begin{itemize}
    \item Escribes "ñ" en tu editor → \texttt{inputenc} interpreta ese carácter
    \item LaTeX procesa el documento
    \item \texttt{fontenc} decide cómo representar "ñ" en el PDF final
\end{itemize}
\end{tcolorbox}

% =====================================================
% SECCIÓN 5: FONTSPEC
% =====================================================
\section{El paquete \texttt{fontspec}: La solución moderna}

\subsection{¿Qué es \texttt{fontspec}?}

\begin{tcolorbox}[colback=verdeclaro,colframe=verdeoscuro,title=\textbf{fontspec},breakable]
\textbf{Aparición:} 2004 (junto con XeTeX)

\textbf{Propósito:} Reemplaza tanto \texttt{inputenc} como \texttt{fontenc} con un sistema moderno

\textbf{Características:}
\begin{itemize}
    \item Solo funciona con XeLaTeX y LuaLaTeX
    \item Acceso directo a fuentes del sistema (OpenType, TrueType)
    \item No necesita conversión de fuentes
    \item UTF-8 nativo total
    \item Configuración avanzada de tipografía
\end{itemize}

\textbf{Ventajas:}
\begin{itemize}
    \item Usa las mismas fuentes que Word, Photoshop, etc.
    \item Configuración intuitiva
    \item Soporte completo para ligaduras, kerning avanzado
    \item Números oldstyle, variantes estilísticas
    \item Funciones OpenType completas
\end{itemize}
\end{tcolorbox}

\subsection{Ejemplo básico de \texttt{fontspec}}

\begin{tcolorbox}[colback=grisclaro,colframe=black,title=\textbf{Código con fontspec}]
\begin{lstlisting}[language=TeX]
\documentclass{article}
\usepackage{fontspec}

% Seleccionar fuentes del sistema
\setmainfont{Times New Roman}    % Fuente principal (serif)
\setsansfont{Arial}               % Fuente sans-serif
\setmonofont{Courier New}         % Fuente monoespaciada

% O fuentes TeX modernas
\setmainfont{TeX Gyre Pagella}
\setsansfont{TeX Gyre Heros}
\setmonofont{TeX Gyre Cursor}

\begin{document}
Este texto usa Times New Roman.

\textsf{Este texto usa Arial.}

\texttt{Este texto usa Courier New.}
\end{document}
\end{lstlisting}
\end{tcolorbox}

\subsection{Configuración avanzada con \texttt{fontspec}}

\begin{tcolorbox}[colback=grisclaro,colframe=black,title=\textbf{Opciones avanzadas}]
\begin{lstlisting}[language=TeX]
\setmainfont{TeX Gyre Pagella}[
    Ligatures=TeX,           % Ligaduras LaTeX (-- -> en-dash, etc.)
    Numbers=OldStyle,        % Números oldstyle (1234 con descendentes)
    Scale=1.0,               % Escala de la fuente
    BoldFont={* Bold},       % Fuente negrita
    ItalicFont={* Italic},   % Fuente cursiva
]

% Verificar si una fuente existe antes de usarla
\IfFontExistsTF{TeX Gyre Pagella}{
    \setmainfont{TeX Gyre Pagella}
}{
    \setmainfont{Latin Modern Roman}  % Fallback
}
\end{lstlisting}
\end{tcolorbox}

% =====================================================
% SECCIÓN 6: PREÁMBULOS RECOMENDADOS
% =====================================================
\section{Preámbulos recomendados según caso de uso}

\subsection{Caso 1: Solo pdfLaTeX (simple y rápido)}

\begin{tcolorbox}[colback=azulclaro,colframe=azuloscuro,title=\textbf{Preámbulo para pdfLaTeX},breakable]
\begin{lstlisting}[language=TeX]
\documentclass[11pt,a4paper]{article}

% SOLO para pdfLaTeX
\usepackage[T1]{fontenc}
% inputenc ya no necesario en LaTeX 2018+

\usepackage[spanish]{babel}
\usepackage{amsmath}
\usepackage{graphicx}

\begin{document}
Contenido con acentos: áéíóú ñ
\end{document}
\end{lstlisting}

\textbf{Cuándo usar:}
\begin{itemize}
    \item Documentos simples sin requisitos tipográficos especiales
    \item Máxima velocidad de compilación
    \item Compatibilidad con código LaTeX muy antiguo
\end{itemize}
\end{tcolorbox}

\subsection{Caso 2: Solo LuaLaTeX (moderno recomendado)}

\begin{tcolorbox}[colback=verdeclaro,colframe=verdeoscuro,title=\textbf{Preámbulo para LuaLaTeX},breakable]
\begin{lstlisting}[language=TeX]
\documentclass[11pt,a4paper]{article}

% SOLO para LuaLaTeX/XeLaTeX
\usepackage{fontspec}
\usepackage{polyglossia}
\setdefaultlanguage[variant=mexican]{spanish}

% Fuentes modernas
\setmainfont{TeX Gyre Pagella}
\setsansfont{TeX Gyre Heros}
\setmonofont{TeX Gyre Cursor}

\usepackage{amsmath}
\usepackage{graphicx}

\begin{document}
Contenido con acentos: áéíóú ñ
Caracteres especiales: € £ ¥ © ® ™
\end{document}
\end{lstlisting}

\textbf{Cuándo usar:}
\begin{itemize}
    \item Documentos nuevos (2020+)
    \item Necesitas fuentes del sistema
    \item Documentos multilingües complejos
    \item Quieres aprovechar tipografía moderna
\end{itemize}
\end{tcolorbox}

\subsection{Caso 3: Compatibilidad universal (recomendado)}

\begin{tcolorbox}[colback=amarilloclaro,colframe=naranjaoscuro,title=\textbf{Preámbulo universal (funciona con los 3 motores)},breakable]
\begin{lstlisting}[language=TeX]
\documentclass[11pt,a4paper]{article}

% ============================================
% DETECCIÓN AUTOMÁTICA DEL MOTOR
% ============================================
\usepackage{iftex}

\ifluatex
  % Configuración para LuaLaTeX
  \usepackage{fontspec}
  \usepackage{polyglossia}
  \setdefaultlanguage[variant=mexican]{spanish}

  % Fuentes con fallback
  \IfFontExistsTF{TeX Gyre Pagella}{
    \setmainfont{TeX Gyre Pagella}
  }{
    \setmainfont{Latin Modern Roman}
  }
\else
  \ifxetex
    % Configuración para XeLaTeX
    \usepackage{fontspec}
    \usepackage{polyglossia}
    \setdefaultlanguage[variant=mexican]{spanish}
    \setmainfont{Latin Modern Roman}
  \else
    % Configuración para pdfLaTeX
    \usepackage[T1]{fontenc}
    \usepackage[spanish]{babel}
  \fi
\fi

% Paquetes comunes a todos los motores
\usepackage{amsmath}
\usepackage{graphicx}

\begin{document}
Este documento compila con pdfLaTeX, XeLaTeX o LuaLaTeX.

Acentos: áéíóú ñ ¿¡
\end{document}
\end{lstlisting}

\textbf{Cuándo usar:}
\begin{itemize}
    \item Compartirás el documento con otros usuarios
    \item No sabes qué motor usará el destinatario
    \item Quieres flexibilidad máxima
    \item Documentos colaborativos
\end{itemize}
\end{tcolorbox}

% =====================================================
% SECCIÓN 7: MIGRACIÓN
% =====================================================
\section{Migración de documentos antiguos}

\subsection{De pdfLaTeX a LuaLaTeX}

\begin{tcolorbox}[colback=verdeclaro,colframe=verdeoscuro,title=\textbf{Pasos para migrar},breakable]
\textbf{Paso 1: Cambiar el preámbulo}
\begin{lstlisting}[language=TeX]
% ANTES (pdfLaTeX):
\usepackage[utf8]{inputenc}  % Eliminar
\usepackage[T1]{fontenc}     % Eliminar
\usepackage[spanish]{babel}

% DESPUÉS (LuaLaTeX):
\usepackage{fontspec}
\usepackage{polyglossia}
\setdefaultlanguage[variant=mexican]{spanish}
\end{lstlisting}

\textbf{Paso 2: Revisar paquetes incompatibles}
\begin{itemize}
    \item \texttt{libertine}, \texttt{mathpazo} (fuentes Type1) → Cambiar a \texttt{fontspec}
    \item \texttt{ucs} → Ya no necesario
    \item \texttt{ae}, \texttt{aecompl} → Ya no necesarios
\end{itemize}

\textbf{Paso 3: Cambiar comando de compilación}
\begin{lstlisting}[language=bash]
# ANTES:
pdflatex documento.tex

# DESPUÉS:
lualatex documento.tex
\end{lstlisting}

\textbf{Paso 4: Probar compilación}
\begin{itemize}
    \item Compilar 2-3 veces para actualizar referencias
    \item Verificar acentos y caracteres especiales
    \item Revisar saltos de página (pueden cambiar ligeramente)
\end{itemize}
\end{tcolorbox}

\subsection{Problemas comunes en la migración}

\begin{table}[htbp]
\centering
\small
\begin{tabular}{|p{5cm}|p{8cm}|}
\hline
\textbf{Problema} & \textbf{Solución} \\
\hline
\hline
Error con \texttt{inputenc} & Eliminar \texttt{\textbackslash usepackage[utf8]\{inputenc\}} \\
\hline
Fuentes no encontradas & Usar fuentes TeX Gyre o Latin Modern en lugar de Type1 \\
\hline
\texttt{babel} no funciona & Cambiar a \texttt{polyglossia} \\
\hline
Compilación muy lenta & Normal en primera compilación; mejora después \\
\hline
Paquete \texttt{libertine} error & Cambiar a \texttt{\textbackslash setmainfont\{Linux Libertine O\}} \\
\hline
Errores con \texttt{microtype} & Actualizar \texttt{microtype} o usar versión reciente \\
\hline
\end{tabular}
\caption{Problemas comunes al migrar a LuaLaTeX}
\end{table}

% =====================================================
% SECCIÓN 8: CONFIGURACIÓN TEXSTUDIO
% =====================================================
\section{Configuración en TeXstudio}

\subsection{Tu configuración actual}

\begin{tcolorbox}[colback=azulclaro,colframe=azuloscuro,title=\textbf{Configuración actual de TeXstudio}]
Según mencionaste, tienes configurado:
\begin{center}
\texttt{txs:///lualatex | txs:///pdflatex}
\end{center}

\textbf{Esto significa:}
\begin{itemize}
    \item \textbf{Primero intenta:} LuaLaTeX (moderno)
    \item \textbf{Si falla:} pdfLaTeX (fallback)
\end{itemize}

\textbf{Ventajas:}
\begin{itemize}
    \item Priorizas el motor moderno
    \item Tienes respaldo automático
\end{itemize}

\textbf{Desventajas:}
\begin{itemize}
    \item Si usas \texttt{fontspec}, el fallback fallará también
    \item Puede crear confusión sobre qué motor usó
\end{itemize}
\end{tcolorbox}

\subsection{Opciones de configuración recomendadas}

\begin{tcolorbox}[colback=verdeclaro,colframe=verdeoscuro,title=\textbf{Opción 1: Solo LuaLaTeX (recomendado)},breakable]
\textbf{Configuración en TeXstudio:}
\begin{center}
\texttt{txs:///lualatex}
\end{center}

\textbf{Ventajas:}
\begin{itemize}
    \item Consistencia total
    \item Puedes usar \texttt{fontspec} sin problemas
    \item Siempre sabes qué motor se usó
\end{itemize}

\textbf{Para configurar:}
\begin{enumerate}
    \item Menú: Opciones → Configurar TeXstudio
    \item Sección: Órdenes
    \item Compilación por defecto: Seleccionar "LuaLaTeX"
\end{enumerate}
\end{tcolorbox}

\begin{tcolorbox}[colback=amarilloclaro,colframe=naranjaoscuro,title=\textbf{Opción 2: Selector manual},breakable]
\textbf{Configuración:}
\begin{itemize}
    \item Mantener compilador por defecto como LuaLaTeX
    \item En barra de herramientas, menú desplegable permite cambiar
    \item Para documento específico, elegir pdfLaTeX si es necesario
\end{itemize}

\textbf{Cuándo usar cada uno:}
\begin{itemize}
    \item \textbf{LuaLaTeX:} Documentos nuevos, con fuentes personalizadas
    \item \textbf{pdfLaTeX:} Documentos antiguos, máxima velocidad
    \item \textbf{XeLaTeX:} Casos especiales con scripts no latinos
\end{itemize}
\end{tcolorbox}

% =====================================================
% SECCIÓN 9: CASOS DE USO PRÁCTICOS
% =====================================================
\section{Casos de uso prácticos}

\subsection{Caso: Documento académico simple}

\begin{tcolorbox}[colback=grisclaro,colframe=black,title=\textbf{Tesis o artículo académico}]
\textbf{Requisitos:}
\begin{itemize}
    \item Texto principalmente en español
    \item Ecuaciones matemáticas
    \item Figuras y tablas
    \item Sin requisitos tipográficos especiales
\end{itemize}

\textbf{Motor recomendado:} pdfLaTeX o LuaLaTeX

\textbf{Razón:}
\begin{itemize}
    \item pdfLaTeX: Máxima velocidad, compatible con plantillas antiguas
    \item LuaLaTeX: Moderno, mejor tipografía, futuro-compatible
\end{itemize}
\end{tcolorbox}

\subsection{Caso: Libro con diseño tipográfico}

\begin{tcolorbox}[colback=grisclaro,colframe=black,title=\textbf{Libro o manual con diseño cuidado}]
\textbf{Requisitos:}
\begin{itemize}
    \item Múltiples fuentes personalizadas
    \item Diseño tipográfico profesional
    \item Números oldstyle, ligaduras avanzadas
    \item Varios idiomas
\end{itemize}

\textbf{Motor recomendado:} LuaLaTeX

\textbf{Razón:}
\begin{itemize}
    \item Acceso completo a fuentes OpenType
    \item Control tipográfico avanzado
    \item Mejor microtipografía
\end{itemize}

\textbf{Ejemplo de preámbulo:}
\begin{lstlisting}[language=TeX]
\usepackage{fontspec}
\usepackage{polyglossia}

\setmainfont{EB Garamond}[
    Ligatures={TeX,Common},
    Numbers=OldStyle
]
\setsansfont{Fira Sans}
\end{lstlisting}
\end{tcolorbox}

\subsection{Caso: Documento multilingüe}

\begin{tcolorbox}[colback=grisclaro,colframe=black,title=\textbf{Documento en español, inglés, árabe, chino}]
\textbf{Requisitos:}
\begin{itemize}
    \item Múltiples idiomas con scripts diferentes
    \item Escritura de derecha a izquierda (árabe)
    \item Caracteres CJK (chino, japonés, coreano)
\end{itemize}

\textbf{Motor recomendado:} LuaLaTeX o XeLaTeX

\textbf{Razón:}
\begin{itemize}
    \item UTF-8 nativo completo
    \item Soporte para todos los scripts Unicode
    \item pdfLaTeX tiene limitaciones severas para esto
\end{itemize}

\textbf{Ejemplo:}
\begin{lstlisting}[language=TeX]
\usepackage{fontspec}
\usepackage{polyglossia}

\setdefaultlanguage{spanish}
\setotherlanguages{english,arabic}

\newfontfamily\arabicfont{Amiri}[Script=Arabic]
\end{lstlisting}
\end{tcolorbox}

\subsection{Caso: Documentos de física con TikZ}

\begin{tcolorbox}[colback=grisclaro,colframe=black,title=\textbf{Tus ejercicios de física con gráficas}]
\textbf{Requisitos:}
\begin{itemize}
    \item Ecuaciones matemáticas complejas
    \item Gráficas TikZ/PGFPlots
    \item Velocidad de compilación importante
    \item Español con acentos
\end{itemize}

\textbf{Motor recomendado:} LuaLaTeX (lo que ya usas)

\textbf{Razón:}
\begin{itemize}
    \item TikZ funciona bien con todos, pero LuaLaTeX tiene ventajas
    \item Mejor manejo de memoria para gráficas complejas
    \item UTF-8 nativo para comentarios y texto
    \item Futuro-compatible
\end{itemize}

\textbf{Tu configuración actual es óptima para este caso.}
\end{tcolorbox}

% =====================================================
% SECCIÓN 10: RECOMENDACIONES FINALES
% =====================================================
\section{Recomendaciones finales}

\subsection{Para proyectos nuevos (2025 en adelante)}

\begin{tcolorbox}[colback=verdeclaro,colframe=verdeoscuro,title=\textbf{Recomendación general}]
\textbf{Usa LuaLaTeX con este preámbulo:}
\begin{lstlisting}[language=TeX]
\documentclass[11pt,a4paper]{article}
\usepackage{fontspec}
\usepackage{polyglossia}
\setdefaultlanguage[variant=mexican]{spanish}
\usepackage{amsmath}
\usepackage{graphicx}
\end{lstlisting}

\textbf{Razones:}
\begin{itemize}
    \item Es el futuro oficial de LaTeX
    \item Desarrollo más activo
    \item Mejor tipografía
    \item No necesitas pensar en codificaciones
\end{itemize}
\end{tcolorbox}

\subsection{Para proyectos existentes}

\begin{tcolorbox}[colback=azulclaro,colframe=azuloscuro,title=\textbf{Migración gradual}]
\textbf{No apresures la migración:}
\begin{itemize}
    \item Si tu documento pdfLaTeX funciona, déjalo así
    \item Migra solo si necesitas funciones modernas
    \item Prueba primero en copia del documento
\end{itemize}

\textbf{Actualiza el preámbulo pdfLaTeX a:}
\begin{lstlisting}[language=TeX]
\usepackage[T1]{fontenc}
% inputenc ya no necesario
\usepackage[spanish]{babel}
\end{lstlisting}
\end{tcolorbox}

\subsection{Resumen de decisiones}

\begin{table}[htbp]
\centering
\small
\begin{tabular}{|p{5cm}|p{3cm}|p{5cm}|}
\hline
\textbf{Situación} & \textbf{Motor} & \textbf{Paquetes clave} \\
\hline
\hline
Documento nuevo simple & LuaLaTeX & \texttt{fontspec}, \texttt{polyglossia} \\
\hline
Documento nuevo complejo & LuaLaTeX & \texttt{fontspec}, \texttt{polyglossia} \\
\hline
Máxima velocidad & pdfLaTeX & \texttt{fontenc}, \texttt{babel} \\
\hline
Documento antiguo & pdfLaTeX & Mantener original \\
\hline
Múltiples idiomas & LuaLaTeX & \texttt{fontspec}, \texttt{polyglossia} \\
\hline
Compartir con otros & Universal & \texttt{iftex} + condicionales \\
\hline
Tipografía avanzada & LuaLaTeX & \texttt{fontspec} \\
\hline
\end{tabular}
\caption{Guía rápida de decisión}
\end{table}

% =====================================================
% SECCIÓN 11: RECURSOS Y REFERENCIAS
% =====================================================
\section{Recursos adicionales}

\subsection{Documentación oficial}

\begin{itemize}
    \item \textbf{fontspec:} \texttt{texdoc fontspec} (en terminal)
    \item \textbf{polyglossia:} \texttt{texdoc polyglossia}
    \item \textbf{iftex:} \texttt{texdoc iftex}
    \item \textbf{LuaTeX:} \url{http://www.luatex.org/}
    \item \textbf{LaTeX Project:} \url{https://www.latex-project.org/}
\end{itemize}

\subsection{Fuentes recomendadas}

\subsubsection{Fuentes TeX Gyre (incluidas en TeX Live)}

\begin{itemize}
    \item \textbf{TeX Gyre Pagella} (basada en Palatino)
    \item \textbf{TeX Gyre Termes} (basada en Times)
    \item \textbf{TeX Gyre Heros} (basada en Helvetica)
    \item \textbf{TeX Gyre Cursor} (basada en Courier)
    \item \textbf{TeX Gyre Bonum} (basada en Bookman)
\end{itemize}

\subsubsection{Fuentes OpenType gratuitas recomendadas}

\begin{itemize}
    \item \textbf{Linux Libertine} (serif elegante)
    \item \textbf{EB Garamond} (clásica para libros)
    \item \textbf{Fira Sans} (sans-serif moderna)
    \item \textbf{Source Serif Pro / Source Sans Pro} (Adobe)
    \item \textbf{Crimson Text} (estilo Garamond)
\end{itemize}

% =====================================================
% CONCLUSIÓN
% =====================================================
\section{Conclusión}

\begin{tcolorbox}[colback=amarilloclaro,colframe=naranjaoscuro,title=\textbf{Conclusión}]
\textbf{Lo más importante:}

\begin{enumerate}
    \item \textbf{inputenc ya no es necesario} en LaTeX moderno (2018+)
    \item \textbf{fontenc todavía es necesario} para pdfLaTeX
    \item \textbf{LuaLaTeX es el futuro} de LaTeX
    \item \textbf{fontspec reemplaza ambos} en motores modernos
    \item \textbf{Tu configuración actual es excelente}
\end{enumerate}

\textbf{Recomendación personal para ti:}

Dado que ya usas \texttt{txs:///lualatex | txs:///pdflatex}, estás en el camino correcto. Para nuevos documentos, usa directamente:

\begin{lstlisting}[language=TeX]
\usepackage{fontspec}
\usepackage{polyglossia}
\setdefaultlanguage[variant=mexican]{spanish}
\end{lstlisting}

Y configura TeXstudio para usar solo LuaLaTeX sin fallback.
\end{tcolorbox}

% =====================================================
% PIE DE PÁGINA
% =====================================================
\vfill
\begin{center}
\textit{Documento generado con \LaTeX{} -- \today}\\[0.2cm]
\small Compilado con \ifluatex LuaLaTeX \else \ifxetex XeLaTeX \else pdfLaTeX \fi\fi
\end{center}

\end{document}
