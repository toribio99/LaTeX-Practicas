\documentclass[11pt,a4paper]{article}

% Paquetes necesarios
\usepackage[utf8]{inputenc}
\usepackage[spanish]{babel}
\usepackage[margin=2.5cm]{geometry}
\usepackage{xcolor}
\usepackage{tcolorbox}
\usepackage{enumitem}
\usepackage{fontawesome5}
\usepackage{listings}
\usepackage{graphicx}
\usepackage{lipsum}

% Paquetes Fancy principales
\usepackage{fancyhdr}
\usepackage{fancybox}
\usepackage{fancyvrb}

% Colores personalizados
\definecolor{commandcolor}{RGB}{39,174,96}
\definecolor{codebackground}{RGB}{245,245,245}

% Configuración de listings con soporte para español
\lstset{
	basicstyle=\ttfamily\footnotesize,
	backgroundcolor=\color{codebackground},
	breaklines=true,
	columns=fullflexible,
	keepspaces=true,
	frame=single,
	rulecolor=\color{gray!30},
	inputencoding=utf8,
	extendedchars=true,
	literate=
		{á}{{\'a}}1 {é}{{\'e}}1 {í}{{\'i}}1 {ó}{{\'o}}1 {ú}{{\'u}}1
		{Á}{{\'A}}1 {É}{{\'E}}1 {Í}{{\'I}}1 {Ó}{{\'O}}1 {Ú}{{\'U}}1
		{ñ}{{\~n}}1 {Ñ}{{\~N}}1
		{ü}{{\"u}}1 {Ü}{{\"U}}1
		{¿}{{?`}}1 {¡}{{!`}}1
}

% Configuración de fancyhdr para este documento
\pagestyle{fancy}
\fancyhf{}
\fancyhead[L]{\leftmark}
\fancyhead[R]{\thepage}
\fancyfoot[C]{\textit{Guía de Paquetes Fancy en \LaTeX{}}}
\renewcommand{\headrulewidth}{0.4pt}
\renewcommand{\footrulewidth}{0.4pt}
\setlength{\headheight}{15pt}
\addtolength{\topmargin}{-3pt}

% Título
\title{\textbf{\Huge Paquetes Fancy en \LaTeX{}}\\\large Guía Completa de Comandos, Entornos y Personalización}
\author{}
\date{\today}
 \usepackage[
%colorlinks=true,        % Enlaces con color (en lugar de cajas)
linkcolor=blue,         % Color de enlaces internos
urlcolor=cyan,          % Color de URLs
citecolor=green,        % Color de citas bibliográficas
filecolor=magenta,      % Color de enlaces a archivos
pdfborder={0 0 0},      % Sin bordes en los enlaces
bookmarks=true,         % Crear marcadores en el PDF
bookmarksopen=true,     % Marcadores expandidos al abrir
pdftitle={Mi Título},   % Título del PDF
pdfauthor={Mi Nombre},  % Autor del PDF
pdfsubject={Tema},      % Tema del documento
pdfkeywords={palabra1, palabra2}, % Palabras clave
%hidelinks,              % Ocultar todos los bordes/colores de enlaces
unicode=true,           % Permitir caracteres Unicode en marcadores
breaklinks=true         % Permitir saltos de línea en enlaces
]{hyperref}

\begin{document}

\maketitle
\thispagestyle{empty}

\begin{tcolorbox}[colback=blue!5,colframe=blue!75!black,title=\faInfoCircle\ Introducción]
Los paquetes \textbf{Fancy} de \LaTeX{} proporcionan herramientas avanzadas para personalizar y mejorar la presentación de documentos. Esta guía cubre todos los paquetes disponibles con el prefijo "fancy", incluyendo comandos, entornos, opciones y ejemplos prácticos.
\end{tcolorbox}

\tableofcontents
%\newpage

\section{Paquete fancyhdr}

\subsection{Introducción a fancyhdr}

\begin{tcolorbox}[colback=green!5,colframe=green!50!black]
	\textbf{Descripción:} El paquete \texttt{fancyhdr} permite personalizar completamente los encabezados y pies de página de un documento.

	\textbf{Carga:}
	\begin{lstlisting}[language=TeX]
\usepackage{fancyhdr}
	\end{lstlisting}
\end{tcolorbox}

\subsection{Estilos de Página}

\subsubsection*{\texttt{\textbackslash pagestyle\{fancy\}}}
\begin{tcolorbox}[colback=green!5,colframe=green!50!black]
	\textbf{Descripción:} Activa el estilo fancy para todas las páginas

	\textbf{Ejemplo:}
	\begin{lstlisting}[language=TeX]
\pagestyle{fancy}
	\end{lstlisting}
\end{tcolorbox}

\subsubsection*{\texttt{\textbackslash thispagestyle\{fancy\}}}
\begin{tcolorbox}[colback=green!5,colframe=green!50!black]
	\textbf{Descripción:} Aplica el estilo fancy solo a la página actual

	\textbf{Ejemplo:}
	\begin{lstlisting}[language=TeX]
\thispagestyle{fancy}
	\end{lstlisting}
\end{tcolorbox}

\subsection{Comandos de Configuración}

\subsubsection*{\texttt{\textbackslash fancyhf\{\}}}
\begin{tcolorbox}[colback=green!5,colframe=green!50!black]
	\textbf{Descripción:} Limpia todos los campos de encabezado y pie de página

	\textbf{Ejemplo:}
	\begin{lstlisting}[language=TeX]
\fancyhf{} % Limpia todo
\fancyhf[L]{Texto} % Solo limpia y establece izquierda
	\end{lstlisting}
\end{tcolorbox}

\subsection{Encabezados (Headers)}

\subsubsection*{\texttt{\textbackslash fancyhead[posición]\{contenido\}}}
\begin{tcolorbox}[colback=green!5,colframe=green!50!black]
	\textbf{Descripción:} Define el contenido del encabezado en una posición específica

	\textbf{Posiciones disponibles:}
	\begin{itemize}[nosep]
		\item \texttt{L} -- Left (izquierda)
		\item \texttt{C} -- Center (centro)
		\item \texttt{R} -- Right (derecha)
		\item \texttt{E} -- Even pages (páginas pares)
		\item \texttt{O} -- Odd pages (páginas impares)
	\end{itemize}

	\textbf{Ejemplo:}
	\begin{lstlisting}[language=TeX]
\fancyhead[L]{Capítulo \thechapter}
\fancyhead[C]{Título del Documento}
\fancyhead[R]{\thepage}
\fancyhead[LE,RO]{\thepage} % Izquierda par, derecha impar
\fancyhead[LO,RE]{Autor}
	\end{lstlisting}
\end{tcolorbox}

\subsection{Pies de Página (Footers)}

\subsubsection*{\texttt{\textbackslash fancyfoot[posición]\{contenido\}}}
\begin{tcolorbox}[colback=green!5,colframe=green!50!black]
	\textbf{Descripción:} Define el contenido del pie de página

	\textbf{Ejemplo:}
	\begin{lstlisting}[language=TeX]
\fancyfoot[L]{Universidad XYZ}
\fancyfoot[C]{\thepage}
\fancyfoot[R]{\today}
\fancyfoot[LE,RO]{\thepage}
\fancyfoot[LO,RE]{\leftmark}
	\end{lstlisting}
\end{tcolorbox}

\subsection{Líneas Decorativas}

\subsubsection*{\texttt{\textbackslash renewcommand\{\textbackslash headrulewidth\}\{grosor\}}}
\begin{tcolorbox}[colback=green!5,colframe=green!50!black]
	\textbf{Descripción:} Define el grosor de la línea del encabezado

	\textbf{Ejemplo:}
	\begin{lstlisting}[language=TeX]
\renewcommand{\headrulewidth}{0.4pt} % Línea visible
\renewcommand{\headrulewidth}{0pt}   % Sin línea
\renewcommand{\headrulewidth}{2pt}   % Línea gruesa
	\end{lstlisting}
\end{tcolorbox}

\subsubsection*{\texttt{\textbackslash renewcommand\{\textbackslash footrulewidth\}\{grosor\}}}
\begin{tcolorbox}[colback=green!5,colframe=green!50!black]
	\textbf{Descripción:} Define el grosor de la línea del pie de página

	\textbf{Ejemplo:}
	\begin{lstlisting}[language=TeX]
\renewcommand{\footrulewidth}{0.4pt} % Línea visible
\renewcommand{\footrulewidth}{0pt}   % Sin línea
	\end{lstlisting}
\end{tcolorbox}

\subsection{Variables Útiles}

\begin{tcolorbox}[colback=green!5,colframe=green!50!black]
	\textbf{Variables predefinidas para usar en encabezados/pies:}
	\begin{itemize}[nosep]
		\item \texttt{\textbackslash thepage} -- Número de página actual
		\item \texttt{\textbackslash thechapter} -- Número de capítulo actual
		\item \texttt{\textbackslash thesection} -- Número de sección actual
		\item \texttt{\textbackslash leftmark} -- Marca izquierda (capítulo)
		\item \texttt{\textbackslash rightmark} -- Marca derecha (sección)
		\item \texttt{\textbackslash today} -- Fecha actual
		\item \texttt{\textbackslash chaptername} -- Nombre "Capítulo"
		\item \texttt{\textbackslash sectionname} -- Nombre "Sección"
	\end{itemize}
\end{tcolorbox}

\subsection{Ejemplos Completos de fancyhdr}

\subsubsection*{Ejemplo 1: Estilo Simple}
\begin{tcolorbox}[colback=cyan!10,colframe=cyan!75!black,title=\faCode\ Código]
	\begin{lstlisting}[language=TeX]
\usepackage{fancyhdr}
\pagestyle{fancy}
\fancyhf{}
\fancyhead[L]{\leftmark}
\fancyhead[R]{\thepage}
\renewcommand{\headrulewidth}{0.4pt}
	\end{lstlisting}
\end{tcolorbox}

\subsubsection*{Ejemplo 2: Estilo Doble Cara (twoside)}
\begin{tcolorbox}[colback=cyan!10,colframe=cyan!75!black,title=\faCode\ Código]
	\begin{lstlisting}[language=TeX]
\usepackage{fancyhdr}
\pagestyle{fancy}
\fancyhf{}
\fancyhead[LE,RO]{\thepage}
\fancyhead[LO]{\rightmark}
\fancyhead[RE]{\leftmark}
\fancyfoot[C]{Mi Documento}
\renewcommand{\headrulewidth}{0.4pt}
\renewcommand{\footrulewidth}{0.4pt}
	\end{lstlisting}
\end{tcolorbox}

\subsubsection*{Ejemplo 3: Estilo Personalizado Avanzado}
\begin{tcolorbox}[colback=cyan!10,colframe=cyan!75!black,title=\faCode\ Código]
	\begin{lstlisting}[language=TeX]
\usepackage{fancyhdr}
\usepackage{graphicx}

\pagestyle{fancy}
\fancyhf{}

% Encabezado con logo
\fancyhead[L]{\includegraphics[height=1cm]{logo.png}}
\fancyhead[C]{\textbf{Título del Documento}}
\fancyhead[R]{Página \thepage}

% Pie de página
\fancyfoot[L]{Autor: Juan Pérez}
\fancyfoot[C]{\today}
\fancyfoot[R]{Versión 1.0}

% Líneas decorativas
\renewcommand{\headrulewidth}{2pt}
\renewcommand{\footrulewidth}{1pt}
	\end{lstlisting}
\end{tcolorbox}

\subsection{Personalización de Marcas}

\subsubsection*{\texttt{\textbackslash markboth\{izquierda\}\{derecha\}}}
\begin{tcolorbox}[colback=green!5,colframe=green!50!black]
	\textbf{Descripción:} Establece manualmente las marcas izquierda y derecha

	\textbf{Ejemplo:}
	\begin{lstlisting}[language=TeX]
\markboth{Capítulo 1: Introducción}{1.1 Conceptos Básicos}
	\end{lstlisting}
\end{tcolorbox}

\subsubsection*{\texttt{\textbackslash markright\{derecha\}}}
\begin{tcolorbox}[colback=green!5,colframe=green!50!black]
	\textbf{Descripción:} Establece solo la marca derecha

	\textbf{Ejemplo:}
	\begin{lstlisting}[language=TeX]
\markright{Sección Actual}
	\end{lstlisting}
\end{tcolorbox}

\subsection{Ancho de Encabezado/Pie}

\subsubsection*{\texttt{\textbackslash fancyhfoffset[posición]\{offset\}}}
\begin{tcolorbox}[colback=green!5,colframe=green!50!black]
	\textbf{Descripción:} Ajusta el desplazamiento horizontal del encabezado/pie

	\textbf{Ejemplo:}
	\begin{lstlisting}[language=TeX]
\fancyhfoffset[L]{1cm} % Extiende 1cm a la izquierda
\fancyhfoffset[R]{2cm} % Extiende 2cm a la derecha
	\end{lstlisting}
\end{tcolorbox}

\newpage

\section{Paquete fancybox}

\subsection{Introducción a fancybox}

\begin{tcolorbox}[colback=green!5,colframe=green!50!black]
	\textbf{Descripción:} El paquete \texttt{fancybox} proporciona cajas decorativas con diversos estilos (sombra, doble borde, ovaladas, etc.)

	\textbf{Carga:}
	\begin{lstlisting}[language=TeX]
\usepackage{fancybox}
	\end{lstlisting}
\end{tcolorbox}

\subsection{Cajas con Estilo}

\subsubsection*{\texttt{\textbackslash shadowbox\{texto\}}}
\begin{tcolorbox}[colback=green!5,colframe=green!50!black]
	\textbf{Descripción:} Caja con sombra

	\textbf{Ejemplo:}
	\begin{lstlisting}[language=TeX]
\shadowbox{Texto con sombra}
	\end{lstlisting}

	\textbf{Resultado:} \shadowbox{Texto con sombra}
\end{tcolorbox}

\subsubsection*{\texttt{\textbackslash doublebox\{texto\}}}
\begin{tcolorbox}[colback=green!5,colframe=green!50!black]
	\textbf{Descripción:} Caja con doble borde

	\textbf{Ejemplo:}
	\begin{lstlisting}[language=TeX]
\doublebox{Texto con doble borde}
	\end{lstlisting}

	\textbf{Resultado:} \doublebox{Texto con doble borde}
\end{tcolorbox}

\subsubsection*{\texttt{\textbackslash ovalbox\{texto\}}}
\begin{tcolorbox}[colback=green!5,colframe=green!50!black]
	\textbf{Descripción:} Caja con bordes redondeados

	\textbf{Ejemplo:}
	\begin{lstlisting}[language=TeX]
\ovalbox{Texto con bordes ovalados}
	\end{lstlisting}

	\textbf{Resultado:} \ovalbox{Texto con bordes ovalados}
\end{tcolorbox}

\subsubsection*{\texttt{\textbackslash Ovalbox\{texto\}}}
\begin{tcolorbox}[colback=green!5,colframe=green!50!black]
	\textbf{Descripción:} Caja oval con borde más grueso

	\textbf{Ejemplo:}
	\begin{lstlisting}[language=TeX]
\Ovalbox{Texto con bordes ovalados gruesos}
	\end{lstlisting}

	\textbf{Resultado:} \Ovalbox{Texto con bordes ovalados gruesos}
\end{tcolorbox}

\subsection{Parámetros de Personalización}

\subsubsection*{\texttt{\textbackslash setlength\{\textbackslash fboxrule\}\{grosor\}}}
\begin{tcolorbox}[colback=green!5,colframe=green!50!black]
	\textbf{Descripción:} Define el grosor del borde de las cajas

	\textbf{Ejemplo:}
	\begin{lstlisting}[language=TeX]
\setlength{\fboxrule}{2pt}
\ovalbox{Borde grueso}
	\end{lstlisting}
\end{tcolorbox}

\subsubsection*{\texttt{\textbackslash setlength\{\textbackslash fboxsep\}\{separación\}}}
\begin{tcolorbox}[colback=green!5,colframe=green!50!black]
	\textbf{Descripción:} Define la separación entre el texto y el borde

	\textbf{Ejemplo:}
	\begin{lstlisting}[language=TeX]
\setlength{\fboxsep}{5pt}
\shadowbox{Más espacio interno}
	\end{lstlisting}
\end{tcolorbox}

\subsubsection*{\texttt{\textbackslash setlength\{\textbackslash shadowsize\}\{tamaño\}}}
\begin{tcolorbox}[colback=green!5,colframe=green!50!black]
	\textbf{Descripción:} Define el tamaño de la sombra en shadowbox

	\textbf{Ejemplo:}
	\begin{lstlisting}[language=TeX]
\setlength{\shadowsize}{5pt}
\shadowbox{Sombra más grande}
	\end{lstlisting}
\end{tcolorbox}

\subsection{Entornos de fancybox}

\subsubsection*{Entorno \texttt{Sbox}}
\begin{tcolorbox}[colback=green!5,colframe=green!50!black]
	\textbf{Descripción:} Guarda contenido en una caja para luego aplicarle estilo

	\textbf{Ejemplo:}
	\begin{lstlisting}[language=TeX]
\newsavebox{\mybox}
\begin{Sbox}
  \begin{minipage}{0.5\textwidth}
    Contenido de la caja con texto más largo
    que puede ocupar varias líneas.
  \end{minipage}
\end{Sbox}
\shadowbox{\TheSbox}
	\end{lstlisting}
\end{tcolorbox}

\subsection{Ejemplos Combinados}

\begin{tcolorbox}[colback=cyan!10,colframe=cyan!75!black,title=\faCode\ Ejemplo Avanzado]
	\begin{lstlisting}[language=TeX]
% Ajustar parámetros
\setlength{\fboxrule}{3pt}
\setlength{\fboxsep}{10pt}
\setlength{\shadowsize}{6pt}

% Crear cajas personalizadas
\shadowbox{\textbf{Importante:} Este es un aviso con sombra}

\doublebox{\textcolor{red}{¡Atención!} Texto destacado}

\Ovalbox{\large Texto grande en caja oval}
	\end{lstlisting}
\end{tcolorbox}

\newpage

\section{Paquete fancyvrb}

\subsection{Introducción a fancyvrb}

\begin{tcolorbox}[colback=green!5,colframe=green!50!black]
	\textbf{Descripción:} El paquete \texttt{fancyvrb} extiende el entorno \texttt{verbatim} estándar con opciones avanzadas de formateo, numeración de líneas, colores, marcos y más.

	\textbf{Carga:}
	\begin{lstlisting}[language=TeX]
\usepackage{fancyvrb}
	\end{lstlisting}
\end{tcolorbox}

\subsection{Entorno Verbatim Mejorado}

\subsubsection*{\texttt{Verbatim}}
\begin{tcolorbox}[colback=green!5,colframe=green!50!black]
	\textbf{Descripción:} Entorno verbatim mejorado con opciones

	\textbf{Ejemplo básico:}
	\begin{lstlisting}[language=TeX]
\begin{Verbatim}
Código fuente sin formato
  con indentación preservada
\end{Verbatim}
	\end{lstlisting}
\end{tcolorbox}

\subsection{Opciones de Formato}

\subsubsection*{frame -- Marco alrededor del código}
\begin{tcolorbox}[colback=green!5,colframe=green!50!black]
	\textbf{Opciones:} \texttt{none}, \texttt{single}, \texttt{lines}, \texttt{topline}, \texttt{bottomline}, \texttt{leftline}

	\textbf{Ejemplo:}
	\begin{lstlisting}[language=TeX]
\begin{Verbatim}[frame=single]
def hello():
    print("Hello, World!")
\end{Verbatim}
	\end{lstlisting}
\end{tcolorbox}

\subsubsection*{numbers -- Numeración de líneas}
\begin{tcolorbox}[colback=green!5,colframe=green!50!black]
	\textbf{Opciones:} \texttt{none}, \texttt{left}, \texttt{right}

	\textbf{Ejemplo:}
	\begin{lstlisting}[language=TeX]
\begin{Verbatim}[numbers=left]
Línea 1
Línea 2
Línea 3
\end{Verbatim}
	\end{lstlisting}
\end{tcolorbox}

\subsubsection*{numbersep -- Separación de números}
\begin{tcolorbox}[colback=green!5,colframe=green!50!black]
	\textbf{Descripción:} Distancia entre números de línea y texto

	\textbf{Ejemplo:}
	\begin{lstlisting}[language=TeX]
\begin{Verbatim}[numbers=left,numbersep=5pt]
Código con números
\end{Verbatim}
	\end{lstlisting}
\end{tcolorbox}

\subsubsection*{firstnumber -- Número inicial}
\begin{tcolorbox}[colback=green!5,colframe=green!50!black]
	\textbf{Descripción:} Define el número de la primera línea

	\textbf{Ejemplo:}
	\begin{lstlisting}[language=TeX]
\begin{Verbatim}[numbers=left,firstnumber=100]
Línea 100
Línea 101
\end{Verbatim}
	\end{lstlisting}
\end{tcolorbox}

\subsubsection*{stepnumber -- Paso de numeración}
\begin{tcolorbox}[colback=green!5,colframe=green!50!black]
	\textbf{Descripción:} Numera cada N líneas

	\textbf{Ejemplo:}
	\begin{lstlisting}[language=TeX]
\begin{Verbatim}[numbers=left,stepnumber=5]
Solo cada 5 líneas tendrá número
\end{Verbatim}
	\end{lstlisting}
\end{tcolorbox}

\subsection{Opciones de Tamaño y Fuente}

\subsubsection*{fontsize -- Tamaño de fuente}
\begin{tcolorbox}[colback=green!5,colframe=green!50!black]
	\textbf{Ejemplo:}
	\begin{lstlisting}[language=TeX]
\begin{Verbatim}[fontsize=\small]
Código en tamaño pequeño
\end{Verbatim}

\begin{Verbatim}[fontsize=\large]
Código en tamaño grande
\end{Verbatim}
	\end{lstlisting}
\end{tcolorbox}

\subsubsection*{fontfamily -- Familia de fuente}
\begin{tcolorbox}[colback=green!5,colframe=green!50!black]
	\textbf{Ejemplo:}
	\begin{lstlisting}[language=TeX]
\begin{Verbatim}[fontfamily=courier]
Código en Courier
\end{Verbatim}
	\end{lstlisting}
\end{tcolorbox}

\subsection{Opciones de Color}

\subsubsection*{formatcom -- Formato personalizado}
\begin{tcolorbox}[colback=green!5,colframe=green!50!black]
	\textbf{Ejemplo:}
	\begin{lstlisting}[language=TeX]
\begin{Verbatim}[formatcom=\color{blue}]
Código en color azul
\end{Verbatim}
	\end{lstlisting}
\end{tcolorbox}

\subsection{Opciones de Espaciado}

\subsubsection*{xleftmargin -- Margen izquierdo}
\begin{tcolorbox}[colback=green!5,colframe=green!50!black]
	\textbf{Ejemplo:}
	\begin{lstlisting}[language=TeX]
\begin{Verbatim}[xleftmargin=2cm]
Código con margen izquierdo
\end{Verbatim}
	\end{lstlisting}
\end{tcolorbox}

\subsubsection*{xrightmargin -- Margen derecho}
\begin{tcolorbox}[colback=green!5,colframe=green!50!black]
	\textbf{Ejemplo:}
	\begin{lstlisting}[language=TeX]
\begin{Verbatim}[xrightmargin=1cm]
Código con margen derecho
\end{Verbatim}
	\end{lstlisting}
\end{tcolorbox}

\subsection{Otras Opciones Útiles}

\subsubsection*{label -- Etiqueta personalizada}
\begin{tcolorbox}[colback=green!5,colframe=green!50!black]
	\textbf{Ejemplo:}
	\begin{lstlisting}[language=TeX]
\begin{Verbatim}[frame=single,label=Código Python]
def ejemplo():
    pass
\end{Verbatim}
	\end{lstlisting}
\end{tcolorbox}

\subsubsection*{labelposition -- Posición de etiqueta}
\begin{tcolorbox}[colback=green!5,colframe=green!50!black]
	\textbf{Opciones:} \texttt{none}, \texttt{topline}, \texttt{bottomline}, \texttt{all}

	\textbf{Ejemplo:}
	\begin{lstlisting}[language=TeX]
\begin{Verbatim}[frame=single,label=Ejemplo,labelposition=topline]
Código con etiqueta arriba
\end{Verbatim}
	\end{lstlisting}
\end{tcolorbox}

\subsubsection*{samepage -- Evitar salto de página}
\begin{tcolorbox}[colback=green!5,colframe=green!50!black]
	\textbf{Ejemplo:}
	\begin{lstlisting}[language=TeX]
\begin{Verbatim}[samepage=true]
Este código no se dividirá entre páginas
\end{Verbatim}
	\end{lstlisting}
\end{tcolorbox}

\subsection{Comando Inline}

\subsubsection*{\texttt{\textbackslash Verb}}
\begin{tcolorbox}[colback=green!5,colframe=green!50!black]
	\textbf{Descripción:} Versión inline de Verbatim

	\textbf{Ejemplo:}
	\begin{lstlisting}[language=TeX]
El comando \Verb|printf("Hello")| imprime un mensaje.
	\end{lstlisting}
\end{tcolorbox}

\subsection{Definir Estilos Personalizados}

\subsubsection*{\texttt{\textbackslash DefineVerbatimEnvironment}}
\begin{tcolorbox}[colback=green!5,colframe=green!50!black]
	\textbf{Descripción:} Define un nuevo entorno verbatim personalizado

	\textbf{Ejemplo:}
	\begin{lstlisting}[language=TeX]
\DefineVerbatimEnvironment{MyCode}{Verbatim}{
  frame=single,
  numbers=left,
  numbersep=3pt,
  fontsize=\small,
  formatcom=\color{blue}
}

\begin{MyCode}
// Mi código personalizado
int x = 10;
\end{MyCode}
	\end{lstlisting}
\end{tcolorbox}

\subsection{Guardar y Reutilizar Código}

\subsubsection*{\texttt{\textbackslash SaveVerb}}
\begin{tcolorbox}[colback=green!5,colframe=green!50!black]
	\textbf{Descripción:} Guarda código verbatim para reutilizarlo

	\textbf{Ejemplo:}
	\begin{lstlisting}[language=TeX]
\SaveVerb{micodigo}|printf("Hola")|
Luego puedes usar: \UseVerb{micodigo}
	\end{lstlisting}
\end{tcolorbox}

\subsection{Ejemplo Completo de fancyvrb}

\begin{tcolorbox}[colback=cyan!10,colframe=cyan!75!black,title=\faCode\ Ejemplo Avanzado]
	\begin{lstlisting}[language=TeX]
% Definir estilo personalizado
\DefineVerbatimEnvironment{PythonCode}{Verbatim}{
  frame=lines,
  numbers=left,
  numbersep=5pt,
  fontsize=\small,
  formatcom=\color{blue!50!black},
  label=Código Python,
  labelposition=topline,
  xleftmargin=1cm
}

% Usar el estilo
\begin{PythonCode}
def factorial(n):
    if n == 0:
        return 1
    else:
        return n * factorial(n-1)

print(factorial(5))
\end{PythonCode}
	\end{lstlisting}
\end{tcolorbox}

\newpage

\section{Paquete fancychap (No disponible en distribuciones modernas)}

\begin{tcolorbox}[colback=yellow!10,colframe=orange!75!black,title=\faExclamationTriangle\ Nota Importante]
El paquete \texttt{fancychap} está obsoleto y no está disponible en distribuciones modernas de \LaTeX{}. Para capítulos decorativos, se recomienda usar alternativas modernas como:
\begin{itemize}
	\item \texttt{titlesec} -- Personalización completa de títulos
	\item \texttt{memoir} -- Clase de documento con estilos integrados
	\item \texttt{KOMA-Script} -- Clases alternativas con opciones avanzadas
\end{itemize}
\end{tcolorbox}

\section{Alternativa: Paquete titlesec}

\subsection{Introducción a titlesec}

\begin{tcolorbox}[colback=green!5,colframe=green!50!black]
	\textbf{Descripción:} \texttt{titlesec} permite personalizar completamente el formato de capítulos, secciones y subsecciones.

	\textbf{Carga:}
	\begin{lstlisting}[language=TeX]
\usepackage{titlesec}
	\end{lstlisting}
\end{tcolorbox}

\subsection{Personalizar Capítulos}

\subsubsection*{\texttt{\textbackslash titleformat}}
\begin{tcolorbox}[colback=green!5,colframe=green!50!black]
	\textbf{Descripción:} Define el formato de un título

	\textbf{Sintaxis:}
	\begin{lstlisting}[language=TeX]
\titleformat{comando}[forma]{formato}{etiqueta}{sep}{antes}[después]
	\end{lstlisting}

	\textbf{Ejemplo - Capítulo decorativo:}
	\begin{lstlisting}[language=TeX]
\titleformat{\chapter}[display]
  {\normalfont\huge\bfseries}
  {\chaptertitlename\ \thechapter}
  {20pt}
  {\Huge}
	\end{lstlisting}
\end{tcolorbox}

\subsubsection*{Ejemplo - Capítulo con línea decorativa}
\begin{tcolorbox}[colback=cyan!10,colframe=cyan!75!black,title=\faCode\ Código]
	\begin{lstlisting}[language=TeX]
\usepackage{titlesec}

\titleformat{\chapter}[display]
  {\normalfont\huge\bfseries\color{blue}}
  {\filright\MakeUppercase{\chaptertitlename}\ \thechapter}
  {1ex}
  {\titlerule\vspace{1ex}\filleft}
  [\vspace{1ex}\titlerule]
	\end{lstlisting}
\end{tcolorbox}

\subsection{Personalizar Secciones}

\begin{tcolorbox}[colback=cyan!10,colframe=cyan!75!black,title=\faCode\ Ejemplo]
	\begin{lstlisting}[language=TeX]
% Sección con caja de color
\titleformat{\section}
  {\normalfont\Large\bfseries\color{white}}
  {\colorbox{blue}{\parbox{1cm}{\centering\thesection}}}
  {1em}
  {\colorbox{blue!20}{\parbox{\dimexpr\textwidth-2cm}{\raggedright #1}}}

% Subsección con línea
\titleformat{\subsection}
  {\normalfont\large\bfseries}
  {\thesubsection}
  {1em}
  {#1\\\titlerule}
	\end{lstlisting}
\end{tcolorbox}

\subsection{Espaciado de Títulos}

\subsubsection*{\texttt{\textbackslash titlespacing}}
\begin{tcolorbox}[colback=green!5,colframe=green!50!black]
	\textbf{Descripción:} Ajusta el espacio antes y después de títulos

	\textbf{Sintaxis:}
	\begin{lstlisting}[language=TeX]
\titlespacing{comando}{izquierda}{antes}{después}[derecha]
	\end{lstlisting}

	\textbf{Ejemplo:}
	\begin{lstlisting}[language=TeX]
\titlespacing{\chapter}{0pt}{50pt}{40pt}
\titlespacing{\section}{0pt}{20pt}{10pt}
	\end{lstlisting}
\end{tcolorbox}

\newpage

\section{Otros Paquetes Fancy Útiles}

\subsection{fancyref -- Referencias Mejoradas}

\begin{tcolorbox}[colback=green!5,colframe=green!50!black]
	\textbf{Descripción:} Crea referencias automáticas con formato inteligente

	\textbf{Carga:}
	\begin{lstlisting}[language=TeX]
\usepackage{fancyref}
	\end{lstlisting}

	\textbf{Uso:}
	\begin{lstlisting}[language=TeX]
\fref{eq:example}  % Genera "Ecuación 1.2"
\Fref{fig:graph}   % Genera "Figura 2.3"
	\end{lstlisting}
\end{tcolorbox}

\subsection{fancypar -- Párrafos Decorativos}

\begin{tcolorbox}[colback=green!5,colframe=green!50!black]
	\textbf{Descripción:} Crea párrafos con letras capitales decorativas y otros efectos

	\textbf{Nota:} Paquete no muy común, considerar alternativas como \texttt{lettrine}
\end{tcolorbox}

\subsection{lettrine -- Letra Capital (Alternativa)}

\begin{tcolorbox}[colback=green!5,colframe=green!50!black]
	\textbf{Descripción:} Crea letras capitales decorativas al inicio de párrafos

	\textbf{Carga:}
	\begin{lstlisting}[language=TeX]
\usepackage{lettrine}
	\end{lstlisting}

	\textbf{Ejemplo:}
	\begin{lstlisting}[language=TeX]
\lettrine{E}{rase} una vez en un lugar muy lejano...
% La 'E' será grande y decorativa
	\end{lstlisting}
\end{tcolorbox}

\subsection{fancytabs -- Pestañas Laterales}

\begin{tcolorbox}[colback=green!5,colframe=green!50!black]
	\textbf{Descripción:} Añade pestañas en los márgenes del documento

	\textbf{Carga:}
	\begin{lstlisting}[language=TeX]
\usepackage{fancytabs}
	\end{lstlisting}

	\textbf{Ejemplo:}
	\begin{lstlisting}[language=TeX]
\fancytab{Capítulo 1}
\fancytab[r]{Derecha}  % Pestaña a la derecha
	\end{lstlisting}
\end{tcolorbox}

\subsection{fancytooltips -- Tooltips en PDF}

\begin{tcolorbox}[colback=green!5,colframe=green!50!black]
	\textbf{Descripción:} Crea tooltips interactivos en documentos PDF

	\textbf{Carga:}
	\begin{lstlisting}[language=TeX]
\usepackage{fancytooltips}
	\end{lstlisting}

	\textbf{Ejemplo:}
	\begin{lstlisting}[language=TeX]
\tooltip{texto visible}{texto del tooltip}
	\end{lstlisting}
\end{tcolorbox}

\newpage

\section{Combinaciones Avanzadas}

\subsection{Documento Completo con fancyhdr + fancybox}

\begin{tcolorbox}[colback=purple!10,colframe=purple!75!black,title=\faCode\ Ejemplo Integrado]
	\begin{lstlisting}[language=TeX]
\documentclass{article}
\usepackage{fancyhdr}
\usepackage{fancybox}
\usepackage{xcolor}

% Configurar encabezados
\pagestyle{fancy}
\fancyhf{}
\fancyhead[L]{\shadowbox{\textbf{Mi Documento}}}
\fancyhead[R]{\ovalbox{Página \thepage}}
\fancyfoot[C]{\doublebox{\today}}
\renewcommand{\headrulewidth}{2pt}
\renewcommand{\footrulewidth}{2pt}

\begin{document}

\section{Introducción}

\shadowbox{\parbox{0.9\textwidth}{
  Este es un párrafo dentro de una caja con sombra.
  Combina múltiples paquetes fancy para crear
  documentos con estilo profesional.
}}

\end{document}
	\end{lstlisting}
\end{tcolorbox}

\subsection{Plantilla de Reporte con Fancy}

\begin{tcolorbox}[colback=purple!10,colframe=purple!75!black,title=\faCode\ Plantilla Profesional]
	\begin{lstlisting}[language=TeX]
\documentclass[11pt,a4paper]{report}
\usepackage{fancyhdr}
\usepackage{fancybox}
\usepackage{fancyvrb}
\usepackage{xcolor}
\usepackage{graphicx}

% Header personalizado
\pagestyle{fancy}
\fancyhf{}
\fancyhead[LE,RO]{\thepage}
\fancyhead[LO]{\nouppercase{\rightmark}}
\fancyhead[RE]{\nouppercase{\leftmark}}
\fancyfoot[C]{\shadowbox{\small Reporte Técnico 2024}}
\renewcommand{\headrulewidth}{0.5pt}
\renewcommand{\footrulewidth}{0.5pt}

% Estilo verbatim personalizado
\DefineVerbatimEnvironment{Code}{Verbatim}{
  frame=single,
  numbers=left,
  fontsize=\small,
  formatcom=\color{blue!60!black}
}

\begin{document}

\chapter{Análisis}

\shadowbox{\parbox{\textwidth}{
  \textbf{Resumen Ejecutivo:}
  Este capítulo presenta los hallazgos principales...
}}

\section{Metodología}

\begin{Code}
# Código de ejemplo
def proceso():
    return resultado
\end{Code}

\end{document}
	\end{lstlisting}
\end{tcolorbox}

\newpage

\section*{\faLightbulb\ Tips y Mejores Prácticas}

\begin{tcolorbox}[colback=blue!10,colframe=blue!75!black]
	\textbf{Recomendaciones para usar paquetes Fancy:}
	\begin{itemize}[leftmargin=*]
		\item \textbf{fancyhdr:} Siempre usa \texttt{\textbackslash fancyhf\{\}} antes de configurar para limpiar configuraciones previas
		\item \textbf{fancyhdr:} Para documentos twoside, usa posiciones \texttt{LE,RO} y \texttt{LO,RE} para alternar
		\item \textbf{fancybox:} Ajusta \texttt{\textbackslash fboxsep} y \texttt{\textbackslash fboxrule} antes de usar las cajas
		\item \textbf{fancyvrb:} Define estilos personalizados con \texttt{\textbackslash DefineVerbatimEnvironment} para consistencia
		\item \textbf{fancyvrb:} Usa \texttt{fontsize=\textbackslash small} para código largo que debe caber en la página
		\item \textbf{Combinaciones:} Los paquetes fancy se complementan bien entre sí
		\item \textbf{Compilación:} Algunos efectos requieren dos compilaciones para verse correctamente
	\end{itemize}
\end{tcolorbox}

\section*{\faExclamationTriangle\ Errores Comunes}

\begin{tcolorbox}[colback=red!10,colframe=red!75!black]
	\textbf{Problemas frecuentes y soluciones:}
	\begin{itemize}[leftmargin=*]
		\item \textbf{Headers no aparecen:} Asegúrate de usar \texttt{\textbackslash pagestyle\{fancy\}}
		\item \textbf{Conflicto con otros paquetes:} Carga fancyhdr después de geometry
		\item \textbf{Líneas muy gruesas:} Reduce \texttt{\textbackslash headrulewidth} a 0.4pt o menos
		\item \textbf{Texto se sale de la caja:} Usa \texttt{parbox} o \texttt{minipage} dentro de las cajas fancy
		\item \textbf{Verbatim no funciona en notas al pie:} Usa \texttt{\textbackslash SaveVerb} primero
		\item \textbf{Capítulos sin estilo:} Verifica que estés usando clase \texttt{book} o \texttt{report}
		\item \textbf{Páginas especiales sin fancy:} Usa \texttt{\textbackslash thispagestyle\{fancy\}} después de \texttt{\textbackslash maketitle}
	\end{itemize}
\end{tcolorbox}

\section*{\faBook\ Recursos Adicionales}

\begin{tcolorbox}[colback=yellow!10,colframe=orange!75!black]
	\textbf{Documentación oficial:}
	\begin{itemize}[leftmargin=*]
		\item \texttt{texdoc fancyhdr} -- Manual completo de fancyhdr
		\item \texttt{texdoc fancybox} -- Documentación de fancybox
		\item \texttt{texdoc fancyvrb} -- Guía de fancyvrb
		\item \texttt{texdoc titlesec} -- Para personalizar títulos (alternativa a fancychap)
	\end{itemize}

	\textbf{Paquetes complementarios:}
	\begin{itemize}[leftmargin=*]
		\item \texttt{geometry} -- Ajustar márgenes y tamaño de página
		\item \texttt{xcolor} -- Añadir colores a encabezados y cajas
		\item \texttt{graphicx} -- Incluir imágenes en headers
		\item \texttt{tcolorbox} -- Cajas avanzadas (alternativa moderna a fancybox)
		\item \texttt{listings} -- Alternativa a fancyvrb para resaltado de sintaxis
	\end{itemize}
\end{tcolorbox}

\section*{\faKeyboard\ Plantillas Rápidas}

\subsection*{Header Simple}
\begin{tcolorbox}[colback=teal!10,colframe=teal!75!black]
	\begin{lstlisting}[language=TeX]
\pagestyle{fancy}
\fancyhf{}
\fancyhead[L]{\leftmark}
\fancyhead[R]{\thepage}
\renewcommand{\headrulewidth}{0.4pt}
	\end{lstlisting}
\end{tcolorbox}

\subsection*{Header Doble Cara}
\begin{tcolorbox}[colback=teal!10,colframe=teal!75!black]
	\begin{lstlisting}[language=TeX]
\pagestyle{fancy}
\fancyhf{}
\fancyhead[LE,RO]{\thepage}
\fancyhead[LO]{\rightmark}
\fancyhead[RE]{\leftmark}
\renewcommand{\headrulewidth}{0.4pt}
	\end{lstlisting}
\end{tcolorbox}

\subsection*{Caja de Advertencia}
\begin{tcolorbox}[colback=teal!10,colframe=teal!75!black]
	\begin{lstlisting}[language=TeX]
\setlength{\fboxrule}{2pt}
\doublebox{\textbf{¡Atención!} Mensaje importante}
	\end{lstlisting}
\end{tcolorbox}

\subsection*{Código con Numeración}
\begin{tcolorbox}[colback=teal!10,colframe=teal!75!black]
	\begin{lstlisting}[language=TeX]
\begin{Verbatim}[frame=single,numbers=left,fontsize=\small]
// Tu código aquí
\end{Verbatim}
	\end{lstlisting}
\end{tcolorbox}

\vspace{2cm}

\begin{center}
	\shadowbox{\parbox{0.8\textwidth}{\centering
		\large\textbf{Documento generado con \LaTeX{}}\\[0.5em]
		\normalsize\textit{Guía completa de paquetes Fancy}\\[0.5em]
		\small\today
	}}

	\vspace{1cm}

	\textit{Esta guía cubre los principales paquetes fancy disponibles en \LaTeX{}.}\\
	\textit{Para más información, consulta la documentación oficial de cada paquete.}
\end{center}

\end{document}
