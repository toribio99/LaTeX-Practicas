\documentclass[12pt,a4paper]{article}
\usepackage[utf8]{inputenc}
\usepackage[spanish]{babel}
\usepackage{geometry}
\geometry{margin=2.5cm}
\usepackage{xcolor}
\usepackage{fancyvrb}
\usepackage{lipsum}
\usepackage{tcolorbox}

% Paquetes para columnas
\usepackage{multicol}
\usepackage{parallel}
\usepackage{parcolumns}
\usepackage{paracol}
\usepackage{tabularx}
\usepackage{adjustbox}
\usepackage{graphicx}

% Configuración de títulos
\usepackage{titlesec}
\titleformat{\section}{\LARGE\bfseries\color{blue!70!black}}{\thesection}{1em}{}
\titleformat{\subsection}{\Large\bfseries\color{green!60!black}}{\thesubsection}{1em}{}
\titleformat{\subsubsection}{\large\bfseries\color{orange!70!black}}{\thesubsubsection}{1em}{}

\title{\textbf{Aplicación Expandida de los 8 Métodos\\para Crear Columnas en LaTeX}}
\author{Ejemplos Detallados con Lipsum}
\date{\today}
\usepackage[hidelinks]{hyperref}

\begin{document}

\maketitle
\tableofcontents
\newpage

%=================================================================
\section{Introducción}
%=================================================================

Este documento presenta aplicaciones expandidas y detalladas de los 8 métodos principales para crear columnas en LaTeX. Cada método incluye múltiples ejemplos con texto lipsum para demostrar el comportamiento real del flujo de texto y las diferentes opciones disponibles.

\vspace{0.5cm}

\subsection{Objetivos del Documento}

\begin{itemize}
    \item Demostrar cada método con ejemplos prácticos extensos
    \item Explorar variantes de parámetros y opciones
    \item Usar texto lipsum para visualizar el comportamiento real
    \item Comparar ventajas y limitaciones de cada método
\end{itemize}

\newpage

%=================================================================
\section{Método 1: \texttt{minipage}}
%=================================================================

\subsection{Descripción General}

El entorno \texttt{minipage} es el método más básico y versátil para crear columnas en LaTeX. Cada minipage es esencialmente una caja independiente que puede contener cualquier elemento de LaTeX.

\subsection{Sintaxis Básica}

\begin{Verbatim}[frame=single,fontsize=\small]
\begin{minipage}[alineación]{ancho}
    Contenido...
\end{minipage}
\end{Verbatim}

\textbf{Opciones de alineación:} t (top), c (center), b (bottom)

\subsubsection{Ejemplo 1: Dos columnas básicas con texto lipsum}

\noindent
\begin{minipage}[t]{0.48\textwidth}
    \textbf{Columna Izquierda}

    \lipsum[1]
\end{minipage}
\hfill
\begin{minipage}[t]{0.48\textwidth}
    \textbf{Columna Derecha}

    \lipsum[2]
\end{minipage}

\vspace{1cm}

\subsubsection{Ejemplo 2: Tres columnas con anchos diferentes}

\noindent
\begin{minipage}[t]{0.30\textwidth}
    \textbf{30\% - Estrecha}

    \lipsum[3][1-4]
\end{minipage}
\hfill
\begin{minipage}[t]{0.40\textwidth}
    \textbf{40\% - Media}

    \lipsum[4][1-4]
\end{minipage}
\hfill
\begin{minipage}[t]{0.25\textwidth}
    \textbf{25\% - Más estrecha}

    \lipsum[5][1-4]
\end{minipage}

\vspace{1cm}

\subsubsection{Ejemplo 3: Alineación vertical (top vs center vs bottom)}

\noindent
\fbox{\begin{minipage}[t]{0.30\textwidth}
    \textbf{Alineación TOP [t]}

    \lipsum[6][1-3]
\end{minipage}}
\hfill
\fbox{\begin{minipage}[c]{0.30\textwidth}
    \textbf{Alineación CENTER [c]}

    Texto centrado.
\end{minipage}}
\hfill
\fbox{\begin{minipage}[b]{0.30\textwidth}
    \textbf{Alineación BOTTOM [b]}

    Texto alineado abajo.
\end{minipage}}

\vspace{1cm}

\subsubsection{Ejemplo 4: Con líneas separadoras verticales}

\noindent
\begin{minipage}[t]{0.32\textwidth}
    \textbf{Primera Columna}

    \lipsum[7][1-5]
\end{minipage}
\hfill
\vrule width 1pt
\hfill
\begin{minipage}[t]{0.32\textwidth}
    \textbf{Segunda Columna}

    \lipsum[8][1-5]
\end{minipage}
\hfill
\vrule width 1pt
\hfill
\begin{minipage}[t]{0.32\textwidth}
    \textbf{Tercera Columna}

    \lipsum[9][1-5]
\end{minipage}

\vspace{1cm}

\subsubsection{Ejemplo 5: Con colores de fondo}

\noindent
\colorbox{yellow!20}{\begin{minipage}{0.48\textwidth}
    \textbf{Columna con fondo amarillo}

    \lipsum[10][1-6]
\end{minipage}}
\hfill
\colorbox{cyan!20}{\begin{minipage}{0.48\textwidth}
    \textbf{Columna con fondo cyan}

    \lipsum[11][1-6]
\end{minipage}}

\vspace{1cm}

\subsubsection{Ejemplo 6: Minipages anidadas}

\noindent
\begin{minipage}[t]{0.48\textwidth}
    \textbf{Minipage Externa Izquierda}

    \lipsum[12][1-3]

    \vspace{0.3cm}

    \noindent
    \colorbox{green!10}{\begin{minipage}{0.9\textwidth}
        \small\textbf{Minipage Anidada}

        \lipsum[13][1-2]
    \end{minipage}}
\end{minipage}
\hfill
\begin{minipage}[t]{0.48\textwidth}
    \textbf{Minipage Externa Derecha}

    \lipsum[14][1-3]

    \vspace{0.3cm}

    \noindent
    \colorbox{blue!10}{\begin{minipage}{0.9\textwidth}
        \small\textbf{Minipage Anidada}

        \lipsum[15][1-2]
    \end{minipage}}
\end{minipage}

\newpage

%=================================================================
\section{Método 2: \texttt{multicol}}
%=================================================================

\subsection{Descripción General}

El paquete \texttt{multicol} permite crear múltiples columnas que se balancean automáticamente. El texto fluye de una columna a otra de forma continua.

\subsection{Sintaxis Básica}

\begin{Verbatim}[frame=single,fontsize=\small]
\begin{multicols}{número_de_columnas}
    Contenido...
\end{multicols}
\end{Verbatim}

\subsubsection{Ejemplo 1: Dos columnas balanceadas con texto largo}

\begin{multicols}{2}
    \textbf{Texto en dos columnas balanceadas:}

    \lipsum[1-3]
\end{multicols}

\vspace{1cm}

\subsubsection{Ejemplo 2: Tres columnas balanceadas}

\begin{multicols}{3}
    \textbf{Texto distribuido en tres columnas:}

    \lipsum[4-5]
\end{multicols}

\vspace{1cm}

\subsubsection{Ejemplo 3: Cuatro columnas (diseño tipo periódico)}

\begin{multicols}{4}
    \textbf{Cuatro columnas estrechas:}

    \lipsum[6-7]
\end{multicols}

\vspace{1cm}

\subsubsection{Ejemplo 4: Con línea separadora entre columnas}

\setlength{\columnseprule}{0.4pt}
\setlength{\columnsep}{1cm}
\begin{multicols}{2}
    \textbf{Columnas con línea divisoria y separación de 1cm:}

    \lipsum[8-10]
\end{multicols}
\setlength{\columnseprule}{0pt}
\setlength{\columnsep}{10pt}

\vspace{1cm}

\subsubsection{Ejemplo 5: Con línea separadora gruesa y color}

\setlength{\columnseprule}{2pt}
\renewcommand{\columnseprulecolor}{\color{blue}}
\setlength{\columnsep}{1.5cm}
\begin{multicols}{2}
    \textbf{Columnas con línea azul gruesa y separación amplia:}

    \lipsum[11-12]
\end{multicols}
\setlength{\columnseprule}{0pt}
\renewcommand{\columnseprulecolor}{\color{black}}
\setlength{\columnsep}{10pt}

\vspace{1cm}

\subsubsection{Ejemplo 6: Texto con listas en columnas}

\begin{multicols}{2}
    \textbf{Listas en múltiples columnas:}

    \lipsum[13][1-3]

    \begin{itemize}
        \item Primer elemento de la lista
        \item Segundo elemento
        \item Tercer elemento con texto más largo
        \item Cuarto elemento
        \item Quinto elemento
    \end{itemize}

    \lipsum[14][1-3]

    \begin{enumerate}
        \item Primero
        \item Segundo
        \item Tercero
        \item Cuarto
    \end{enumerate}

    \lipsum[15][1-2]
\end{multicols}

\vspace{1cm}

\subsubsection{Ejemplo 7: Con espaciado entre columnas variado}

\setlength{\columnsep}{0.3cm}
\begin{multicols}{3}
    \textbf{Separación mínima (0.3cm):}

    \lipsum[16-17]
\end{multicols}

\setlength{\columnsep}{2cm}
\begin{multicols}{2}
    \textbf{Separación amplia (2cm):}

    \lipsum[18]
\end{multicols}
\setlength{\columnsep}{10pt}

\newpage

%=================================================================
\section{Método 3: \texttt{parallel}}
%=================================================================

\subsection{Descripción General}

El paquete \texttt{parallel} permite crear contenido paralelo con control independiente de cada columna. Es ideal para comparaciones lado a lado.

\subsection{Sintaxis Básica}

\begin{Verbatim}[frame=single,fontsize=\small]
\begin{Parallel}{ancho_izq}{ancho_der}
    \ParallelLText{Contenido izquierdo}
    \ParallelRText{Contenido derecho}
    \ParallelPar
\end{Parallel}
\end{Verbatim}

\subsubsection{Ejemplo 1: Dos textos paralelos básicos}

\begin{Parallel}{0.48\textwidth}{0.48\textwidth}
    \ParallelLText{
        \textbf{Texto Paralelo Izquierdo}

        \lipsum[1][1-5]
    }

    \ParallelRText{
        \textbf{Texto Paralelo Derecho}

        \lipsum[2][1-5]
    }

    \ParallelPar
\end{Parallel}

\vspace{1cm}

\subsubsection{Ejemplo 2: Múltiples párrafos paralelos}

\begin{Parallel}{0.48\textwidth}{0.48\textwidth}
    \ParallelLText{
        \textbf{Primer Párrafo Izquierdo}

        \lipsum[3][1-3]
    }

    \ParallelRText{
        \textbf{Primer Párrafo Derecho}

        \lipsum[4][1-3]
    }

    \ParallelPar

    \ParallelLText{
        \textbf{Segundo Párrafo Izquierdo}

        \lipsum[5][1-3]
    }

    \ParallelRText{
        \textbf{Segundo Párrafo Derecho}

        \lipsum[6][1-3]
    }

    \ParallelPar
\end{Parallel}

\vspace{1cm}

\subsubsection{Ejemplo 3: Anchos asimétricos}

\begin{Parallel}{0.35\textwidth}{0.60\textwidth}
    \ParallelLText{
        \textbf{Columna Estrecha (35\%)}

        \lipsum[7][1-4]
    }

    \ParallelRText{
        \textbf{Columna Ancha (60\%)}

        \lipsum[8][1-7]
    }

    \ParallelPar
\end{Parallel}

\vspace{1cm}

\subsubsection{Ejemplo 4: Comparación de versiones de texto}

\begin{Parallel}{0.48\textwidth}{0.48\textwidth}
    \ParallelLText{
        \textbf{Versión Original}

        \lipsum[9][1-6]
    }

    \ParallelRText{
        \textbf{Versión Revisada}

        \lipsum[10][1-6]
    }

    \ParallelPar
\end{Parallel}

\vspace{1cm}

\subsubsection{Ejemplo 5: Con listas paralelas}

\begin{Parallel}{0.48\textwidth}{0.48\textwidth}
    \ParallelLText{
        \textbf{Lista de Características}

        \begin{itemize}
            \item Característica A
            \item Característica B
            \item Característica C
            \item Característica D
        \end{itemize}

        \lipsum[11][1-2]
    }

    \ParallelRText{
        \textbf{Lista de Ventajas}

        \begin{enumerate}
            \item Ventaja 1
            \item Ventaja 2
            \item Ventaja 3
            \item Ventaja 4
        \end{enumerate}

        \lipsum[12][1-2]
    }

    \ParallelPar
\end{Parallel}

\newpage

%=================================================================
\section{Método 4: \texttt{parcolumns}}
%=================================================================

\subsection{Descripción General}

El paquete \texttt{parcolumns} permite crear columnas paralelas organizadas en chunks (fragmentos). Cada chunk se alinea verticalmente con su contraparte.

\subsection{Sintaxis Básica}

\begin{Verbatim}[frame=single,fontsize=\small]
\begin{parcolumns}[opciones]{número_columnas}
    \colchunk{Contenido columna 1}
    \colchunk{Contenido columna 2}
\end{parcolumns}
\end{Verbatim}

\subsubsection{Ejemplo 1: Dos columnas con chunks básicos}

\begin{parcolumns}[nofirstindent]{2}
    \colchunk{
        \textbf{Primera Columna - Chunk 1}

        \lipsum[1][1-5]
    }

    \colchunk{
        \textbf{Segunda Columna - Chunk 1}

        \lipsum[2][1-5]
    }
\end{parcolumns}

\vspace{1cm}

\subsubsection{Ejemplo 2: Múltiples chunks consecutivos}

\begin{parcolumns}[nofirstindent]{2}
    \colchunk{
        \textbf{Chunk 1 - Izquierda}

        \lipsum[3][1-2]
    }

    \colchunk{
        \textbf{Chunk 1 - Derecha}

        \lipsum[4][1-2]
    }

    \colchunk{
        \textbf{Chunk 2 - Izquierda}

        \lipsum[5][1-2]
    }

    \colchunk{
        \textbf{Chunk 2 - Derecha}

        \lipsum[6][1-2]
    }
\end{parcolumns}

\vspace{1cm}

\subsubsection{Ejemplo 3: Tres columnas con chunks}

\begin{parcolumns}[nofirstindent]{3}
    \colchunk{
        \textbf{Columna 1}

        Texto breve para la primera columna.
    }

    \colchunk{
        \textbf{Columna 2}

        Texto breve para la segunda columna.
    }

    \colchunk{
        \textbf{Columna 3}

        Texto breve para la tercera columna.
    }
\end{parcolumns}

\vspace{1cm}

\subsubsection{Ejemplo 4: Con sangría en primera línea}

\begin{parcolumns}{2}
    \colchunk{
        \textbf{Con sangría automática}

        \lipsum[12][1-5]
    }

    \colchunk{
        \textbf{También con sangría}

        \lipsum[13][1-5]
    }
\end{parcolumns}

\vspace{1cm}

\subsubsection{Ejemplo 5: Chunks con listas}

\begin{parcolumns}[nofirstindent]{2}
    \colchunk{
        \textbf{Lista Izquierda}

        \begin{itemize}
            \item Elemento A
            \item Elemento B
            \item Elemento C
        \end{itemize}

        \lipsum[14][1-2]
    }

    \colchunk{
        \textbf{Lista Derecha}

        \begin{enumerate}
            \item Primero
            \item Segundo
            \item Tercero
        \end{enumerate}

        \lipsum[15][1-2]
    }
\end{parcolumns}

\newpage

%=================================================================
\section{Método 5: \texttt{paracol}}
%=================================================================

\subsection{Descripción General}

El paquete \texttt{paracol} es uno de los más avanzados, permitiendo cambiar entre columnas explícitamente y con soporte para documentos largos que abarcan múltiples páginas.

\subsection{Sintaxis Básica}

\begin{Verbatim}[frame=single,fontsize=\small]
\begin{paracol}{número_columnas}
    Contenido columna 1
    \switchcolumn
    Contenido columna 2
\end{paracol}
\end{Verbatim}

\subsubsection{Ejemplo 1: Dos columnas básicas con switchcolumn}

\begin{paracol}{2}
    \textbf{Primera Columna}

    \lipsum[1-2]

    \switchcolumn

    \textbf{Segunda Columna}

    \lipsum[3-4]
\end{paracol}

\vspace{1cm}

\subsubsection{Ejemplo 2: Con línea separadora}

\setlength{\columnsep}{1cm}
\setlength{\columnseprule}{0.4pt}

\begin{paracol}{2}
    \textbf{Columna con Línea Separadora}

    \lipsum[5][1-8]

    \switchcolumn

    \textbf{Segunda Columna}

    \lipsum[6][1-8]
\end{paracol}

\setlength{\columnsep}{10pt}
\setlength{\columnseprule}{0pt}

\vspace{1cm}

\subsubsection{Ejemplo 3: Múltiples cambios de columna}

\begin{paracol}{2}
    \textbf{Parte 1 - Izquierda}

    \lipsum[7][1-4]

    \switchcolumn

    \textbf{Parte 1 - Derecha}

    \lipsum[8][1-4]

    \switchcolumn

    \textbf{Parte 2 - Izquierda}

    \lipsum[9][1-4]

    \switchcolumn

    \textbf{Parte 2 - Derecha}

    \lipsum[10][1-4]
\end{paracol}

\vspace{1cm}

\subsubsection{Ejemplo 4: Tres columnas con paracol}

\begin{paracol}{3}
    \textbf{Columna 1}

    \lipsum[11][1-5]

    \switchcolumn

    \textbf{Columna 2}

    \lipsum[12][1-5]

    \switchcolumn

    \textbf{Columna 3}

    \lipsum[13][1-5]
\end{paracol}

\vspace{1cm}

\subsubsection{Ejemplo 5: Con línea separadora gruesa y color}

\setlength{\columnseprule}{2pt}
\renewcommand{\columnseprulecolor}{\color{red!50}}

\begin{paracol}{2}
    \textbf{Línea Roja Gruesa}

    \lipsum[14][1-6]

    \switchcolumn

    \textbf{Segunda Parte}

    \lipsum[15][1-6]
\end{paracol}

\setlength{\columnseprule}{0pt}
\renewcommand{\columnseprulecolor}{\color{black}}

\vspace{1cm}

\subsubsection{Ejemplo 6: Con listas y enumeraciones}

\begin{paracol}{2}
    \textbf{Lista de Ventajas}

    \begin{itemize}
        \item Flexibilidad alta
        \item Control preciso
        \item Soporte multi-página
        \item Líneas separadoras
    \end{itemize}

    \lipsum[16][1-3]

    \switchcolumn

    \textbf{Pasos de Uso}

    \begin{enumerate}
        \item Iniciar entorno paracol
        \item Escribir contenido primera columna
        \item Usar switchcolumn
        \item Escribir contenido segunda columna
        \item Cerrar entorno
    \end{enumerate}

    \lipsum[17][1-3]
\end{paracol}

\newpage

%=================================================================
\section{Método 6: \texttt{tabularx}}
%=================================================================

\subsection{Descripción General}

Usar tablas sin bordes visibles es una forma simple y efectiva de crear columnas, especialmente para contenido estructurado y comparaciones.

\subsection{Sintaxis Básica}

\begin{Verbatim}[frame=single,fontsize=\small]
\begin{tabularx}{\textwidth}{XX}
    Columna 1 & Columna 2 \\
\end{tabularx}
\end{Verbatim}

\subsubsection{Ejemplo 1: Dos columnas sin bordes}

\noindent
\begin{tabularx}{\textwidth}{XX}
    \textbf{Columna A} & \textbf{Columna B} \\[0.5em]
    \lipsum[1][1-5] & \lipsum[2][1-5] \\[0.5em]
    \lipsum[3][1-4] & \lipsum[4][1-4]
\end{tabularx}

\vspace{1cm}

\subsubsection{Ejemplo 2: Con línea separadora vertical}

\noindent
\begin{tabularx}{\textwidth}{X|X}
    \textbf{Parte Izquierda} & \textbf{Parte Derecha} \\[0.5em]
    \hline
    \lipsum[5][1-6] & \lipsum[6][1-6] \\[0.5em]
    \lipsum[7][1-5] & \lipsum[8][1-5]
\end{tabularx}

\vspace{1cm}

\subsubsection{Ejemplo 3: Tres columnas con separadores}

\noindent
\begin{tabularx}{\textwidth}{X|X|X}
    \textbf{Columna 1} & \textbf{Columna 2} & \textbf{Columna 3} \\
    \hline
    \lipsum[9][1-4] & \lipsum[10][1-4] & \lipsum[11][1-4]
\end{tabularx}

\vspace{1cm}

\subsubsection{Ejemplo 4: Anchos personalizados}

\noindent
\begin{tabularx}{\textwidth}{p{0.3\textwidth}|X}
    \textbf{Columna Fija 30\%} & \textbf{Columna Flexible} \\
    \hline
    \lipsum[12][1-5] & \lipsum[13][1-8] \\[0.5em]
    \lipsum[14][1-4] & \lipsum[15][1-7]
\end{tabularx}

\vspace{1cm}

\subsubsection{Ejemplo 5: Con colores de fondo}

\noindent
\begin{tabularx}{\textwidth}{XX}
    \cellcolor{yellow!20}\textbf{Fondo Amarillo} & \cellcolor{cyan!20}\textbf{Fondo Cyan} \\[0.5em]
    \cellcolor{yellow!10}\lipsum[16][1-5] & \cellcolor{cyan!10}\lipsum[17][1-5]
\end{tabularx}

\vspace{1cm}

\subsubsection{Ejemplo 6: Tabla comparativa}

\noindent
\begin{tabularx}{\textwidth}{|X|X|}
    \hline
    \cellcolor{green!20}\textbf{Ventajas} & \cellcolor{red!20}\textbf{Desventajas} \\
    \hline
    \lipsum[18][1-4] & \lipsum[19][1-4] \\
    \hline
    \lipsum[20][1-4] & \lipsum[21][1-4] \\
    \hline
\end{tabularx}

\newpage

%=================================================================
\section{Método 7: \texttt{\textbackslash twocolumn}}
%=================================================================

\subsection{Descripción General}

El comando \texttt{\textbackslash twocolumn} es nativo de LaTeX y cambia todo el documento (o una sección) a dos columnas. Es útil para documentos tipo artículo científico o periódico.

\subsection{Sintaxis Básica}

\begin{Verbatim}[frame=single,fontsize=\small]
% En el preámbulo:
\documentclass[twocolumn]{article}

% O dentro del documento:
\twocolumn
Contenido en dos columnas...
\onecolumn  % Volver a una columna
\end{Verbatim}

\subsubsection{Ejemplo: Demostración con texto lipsum}

\textbf{Nota:} Para evitar afectar todo el documento, aquí mostramos el código sin ejecutarlo. En su lugar, usamos \texttt{multicol} para simular el efecto:

\begin{Verbatim}[frame=single,fontsize=\small]
\twocolumn
\section{Título de Sección}
\lipsum[1-5]
\onecolumn
\end{Verbatim}

\textbf{Simulación del resultado:}

\begin{multicols}{2}
    \subsection*{Título de Sección}

    \lipsum[1-5]
\end{multicols}

\vspace{1cm}

\subsubsection{Características de twocolumn}

\begin{itemize}
    \item Afecta todo el documento desde donde se invoca
    \item Balancea automáticamente las columnas
    \item Permite títulos que ocupen todo el ancho con \texttt{\textbackslash twocolumn[Título]}
    \item Se puede configurar el espaciado con \texttt{\textbackslash columnsep}
    \item Menos flexible que otros métodos
\end{itemize}

\vspace{1cm}

\subsubsection{Ejemplo simulado: Artículo científico}

\setlength{\columnseprule}{0.4pt}
\begin{multicols}{2}
    \subsection*{Abstract}

    \lipsum[6][1-4]

    \subsection*{Introduction}

    \lipsum[7-9]

    \subsection*{Methodology}

    \lipsum[10-11]

    \subsection*{Results}

    \lipsum[12-13]
\end{multicols}
\setlength{\columnseprule}{0pt}

\newpage

%=================================================================
\section{Método 8: \texttt{adjustbox}}
%=================================================================

\subsection{Descripción General}

El paquete \texttt{adjustbox} mejora las capacidades de \texttt{minipage} añadiendo opciones avanzadas de alineación, marcos, fondos, rotación y escalado.

\subsection{Sintaxis Básica}

\begin{Verbatim}[frame=single,fontsize=\small]
\begin{adjustbox}{minipage=ancho, opciones}
    Contenido...
\end{adjustbox}
\end{Verbatim}

\subsubsection{Ejemplo 1: Cajas básicas con marcos}

\noindent
\begin{adjustbox}{minipage=0.48\textwidth, valign=t, frame}
    \textbf{Caja 1 con Marco}

    \lipsum[1][1-5]
\end{adjustbox}
\hfill
\begin{adjustbox}{minipage=0.48\textwidth, valign=t, frame}
    \textbf{Caja 2 con Marco}

    \lipsum[2][1-5]
\end{adjustbox}

\vspace{1cm}

\subsubsection{Ejemplo 2: Con fondos de colores}

\noindent
\begin{adjustbox}{minipage=0.32\textwidth, valign=t, bgcolor=yellow!20}
    \textbf{Fondo Amarillo}

    \lipsum[3][1-4]
\end{adjustbox}
\hfill
\begin{adjustbox}{minipage=0.32\textwidth, valign=t, bgcolor=green!20}
    \textbf{Fondo Verde}

    \lipsum[4][1-4]
\end{adjustbox}
\hfill
\begin{adjustbox}{minipage=0.32\textwidth, valign=t, bgcolor=blue!20}
    \textbf{Fondo Azul}

    \lipsum[5][1-4]
\end{adjustbox}

\vspace{1cm}

\subsubsection{Ejemplo 3: Marco y fondo combinados}

\noindent
\begin{adjustbox}{minipage=0.48\textwidth, valign=t, frame, bgcolor=cyan!10}
    \textbf{Marco + Fondo Cyan}

    \lipsum[6][1-6]
\end{adjustbox}
\hfill
\begin{adjustbox}{minipage=0.48\textwidth, valign=t, frame, bgcolor=orange!10}
    \textbf{Marco + Fondo Naranja}

    \lipsum[7][1-6]
\end{adjustbox}

\vspace{1cm}

\subsubsection{Ejemplo 4: Diferentes alturas con alineación top}

\noindent
\begin{adjustbox}{minipage=0.23\textwidth, valign=t, frame, bgcolor=red!10}
    \textbf{Corta}

    \lipsum[8][1-2]
\end{adjustbox}
\hfill
\begin{adjustbox}{minipage=0.23\textwidth, valign=t, frame, bgcolor=green!10}
    \textbf{Media}

    \lipsum[9][1-4]
\end{adjustbox}
\hfill
\begin{adjustbox}{minipage=0.23\textwidth, valign=t, frame, bgcolor=blue!10}
    \textbf{Larga}

    \lipsum[10][1-6]
\end{adjustbox}
\hfill
\begin{adjustbox}{minipage=0.23\textwidth, valign=t, frame, bgcolor=yellow!10}
    \textbf{Muy Larga}

    \lipsum[11][1-8]
\end{adjustbox}

\vspace{1cm}

\subsubsection{Ejemplo 5: Con margen interno (padding)}

\noindent
\begin{adjustbox}{minipage=0.48\textwidth, valign=t, frame, margin=10pt, bgcolor=purple!10}
    \textbf{Con Margen Interno 10pt}

    \lipsum[12][1-5]
\end{adjustbox}
\hfill
\begin{adjustbox}{minipage=0.48\textwidth, valign=t, frame, margin=5pt, bgcolor=pink!20}
    \textbf{Con Margen Interno 5pt}

    \lipsum[13][1-5]
\end{adjustbox}

\vspace{1cm}

\subsubsection{Ejemplo 6: Cajas anidadas con adjustbox}

\noindent
\begin{adjustbox}{minipage=0.48\textwidth, valign=t, frame=2pt, bgcolor=cyan!5}
    \textbf{Caja Externa Izquierda}

    \lipsum[14][1-3]

    \vspace{0.3cm}

    \begin{adjustbox}{minipage=0.9\textwidth, frame, bgcolor=white}
        \small\textbf{Caja Interna}

        \lipsum[15][1-2]
    \end{adjustbox}
\end{adjustbox}
\hfill
\begin{adjustbox}{minipage=0.48\textwidth, valign=t, frame=2pt, bgcolor=orange!5}
    \textbf{Caja Externa Derecha}

    \lipsum[16][1-3]

    \vspace{0.3cm}

    \begin{adjustbox}{minipage=0.9\textwidth, frame, bgcolor=white}
        \small\textbf{Caja Interna}

        \lipsum[17][1-2]
    \end{adjustbox}
\end{adjustbox}

\vspace{1cm}

\subsubsection{Ejemplo 7: Con listas y diferentes opciones}

\noindent
\begin{adjustbox}{minipage=0.48\textwidth, valign=t, frame, bgcolor=green!10, margin=8pt}
    \textbf{Lista con Marco Verde}

    \begin{itemize}
        \item Característica principal A
        \item Característica principal B
        \item Característica principal C
    \end{itemize}

    \lipsum[18][1-3]
\end{adjustbox}
\hfill
\begin{adjustbox}{minipage=0.48\textwidth, valign=t, frame, bgcolor=blue!10, margin=8pt}
    \textbf{Enumeración con Marco Azul}

    \begin{enumerate}
        \item Primer punto importante
        \item Segundo punto importante
        \item Tercer punto importante
    \end{enumerate}

    \lipsum[19][1-3]
\end{adjustbox}

\newpage

%=================================================================
\section{Comparación Detallada de los 8 Métodos}
%=================================================================

\subsection{Tabla Comparativa Completa}

\begin{center}
\small
\begin{tabular}{|l|c|c|c|p{3.5cm}|}
\hline
\textbf{Método} & \textbf{Facilidad} & \textbf{Flexibilidad} & \textbf{Balance Auto} & \textbf{Mejor Uso} \\
\hline
minipage & ★★★★★ & ★★★★★ & No & Control total, cajas independientes \\
\hline
multicol & ★★★★★ & ★★★☆☆ & Sí & Texto continuo, artículos \\
\hline
parallel & ★★★☆☆ & ★★★★☆ & No & Comparaciones, traducciones \\
\hline
parcolumns & ★★★☆☆ & ★★★☆☆ & No & Contenido estructurado en chunks \\
\hline
paracol & ★★★☆☆ & ★★★★★ & No & Docs largos, control fino \\
\hline
tabularx & ★★★★★ & ★★☆☆☆ & No & Comparaciones simples \\
\hline
twocolumn & ★★☆☆☆ & ★☆☆☆☆ & Sí & Documento completo \\
\hline
adjustbox & ★★★★☆ & ★★★★★ & No & Efectos visuales, marcos \\
\hline
\end{tabular}
\end{center}

\vspace{1cm}

\subsection{Criterios de Selección}

\begin{multicols}{2}
\subsubsection*{Usa minipage si...}
\begin{itemize}
    \item Necesitas control total sobre cada columna
    \item Quieres contenido completamente independiente
    \item Necesitas alineación precisa
\end{itemize}

\subsubsection*{Usa multicol si...}
\begin{itemize}
    \item Tienes texto continuo largo
    \item Quieres balance automático
    \item Necesitas 2 o más columnas uniformes
\end{itemize}

\subsubsection*{Usa parallel si...}
\begin{itemize}
    \item Comparas dos versiones de texto
    \item Haces traducciones paralelas
    \item Necesitas sincronización por párrafos
\end{itemize}

\subsubsection*{Usa parcolumns si...}
\begin{itemize}
    \item Trabajas con fragmentos estructurados
    \item Necesitas alineación vertical entre chunks
    \item El contenido está organizado en bloques
\end{itemize}

\subsubsection*{Usa paracol si...}
\begin{itemize}
    \item Tu documento es largo y multi-página
    \item Necesitas líneas separadoras personalizadas
    \item Quieres control explícito de cambios
\end{itemize}

\subsubsection*{Usa tabularx si...}
\begin{itemize}
    \item Haces comparaciones simples
    \item No necesitas mucha complejidad
    \item Prefieres sintaxis de tabla
\end{itemize}

\subsubsection*{Usa twocolumn si...}
\begin{itemize}
    \item Todo el documento será en columnas
    \item Escribes artículos científicos
    \item No necesitas cambiar frecuentemente
\end{itemize}

\subsubsection*{Usa adjustbox si...}
\begin{itemize}
    \item Necesitas efectos visuales (marcos, fondos)
    \item Quieres más opciones que minipage
    \item Trabajas con diseños complejos
\end{itemize}
\end{multicols}

\newpage

%=================================================================
\section{Ejemplos Combinados}
%=================================================================

\subsection{Combinando Varios Métodos}

Es posible combinar diferentes métodos para crear diseños complejos.

\subsubsection{Ejemplo: minipage + adjustbox}

\noindent
\begin{minipage}[t]{0.48\textwidth}
    \textbf{Minipage Normal}

    \lipsum[1][1-4]

    \vspace{0.3cm}

    \begin{adjustbox}{minipage=0.9\textwidth, frame, bgcolor=yellow!20}
        \small Adjustbox dentro de minipage

        \lipsum[2][1-2]
    \end{adjustbox}
\end{minipage}
\hfill
\begin{minipage}[t]{0.48\textwidth}
    \textbf{Otra Minipage}

    \lipsum[3][1-4]

    \vspace{0.3cm}

    \begin{adjustbox}{minipage=0.9\textwidth, frame, bgcolor=cyan!20}
        \small Otro adjustbox anidado

        \lipsum[4][1-2]
    \end{adjustbox}
\end{minipage}

\vspace{1cm}

\subsubsection{Ejemplo: multicol dentro de minipage}

\noindent
\begin{minipage}[t]{0.48\textwidth}
    \textbf{Minipage con Multicol Interno}

    \begin{multicols}{2}
        \lipsum[5][1-6]
    \end{multicols}
\end{minipage}
\hfill
\begin{minipage}[t]{0.48\textwidth}
    \textbf{Otra Minipage Normal}

    \lipsum[6][1-8]
\end{minipage}

\vspace{1cm}

\subsubsection{Ejemplo: tabularx con adjustbox}

\noindent
\begin{tabularx}{\textwidth}{XX}
    \begin{adjustbox}{minipage=\linewidth, frame, bgcolor=green!10}
        \textbf{Adjustbox en Celda 1}

        \lipsum[7][1-4]
    \end{adjustbox}
    &
    \begin{adjustbox}{minipage=\linewidth, frame, bgcolor=blue!10}
        \textbf{Adjustbox en Celda 2}

        \lipsum[8][1-4]
    \end{adjustbox}
\end{tabularx}

\newpage

%=================================================================
\section{Conclusiones y Recomendaciones Finales}
%=================================================================

\subsection{Resumen de Aplicaciones}

A lo largo de este documento hemos explorado en detalle los 8 métodos principales para crear columnas en LaTeX. Cada método tiene su propósito específico:

\begin{multicols}{2}
\begin{enumerate}
    \item \textbf{minipage}: El más versátil y fundamental
    \item \textbf{multicol}: Ideal para texto fluido
    \item \textbf{parallel}: Perfecto para comparaciones
    \item \textbf{parcolumns}: Bueno para estructuras por bloques
    \item \textbf{paracol}: El más potente para documentos largos
    \item \textbf{tabularx}: Simple para comparaciones básicas
    \item \textbf{twocolumn}: Para documentos completos
    \item \textbf{adjustbox}: El más visual y decorativo
\end{enumerate}
\end{multicols}

\subsection{Consejos Prácticos}

\begin{itemize}
    \item \textbf{Experimenta}: Prueba diferentes métodos para el mismo contenido
    \item \textbf{Combina}: No temas combinar métodos cuando sea necesario
    \item \textbf{Mantén simplicidad}: Usa el método más simple que cumpla tus necesidades
    \item \textbf{Considera el mantenimiento}: Elige métodos que sean fáciles de mantener
    \item \textbf{Piensa en el flujo}: Para texto continuo, usa multicol o paracol
    \item \textbf{Control fino}: Para diseños precisos, usa minipage o adjustbox
\end{itemize}

\subsection{Recursos Adicionales}

Para profundizar en cada método, consulta:
\begin{itemize}
    \item Documentación oficial de cada paquete en CTAN
    \item TeX StackExchange para casos de uso específicos
    \item Ejemplos en Overleaf Learn
\end{itemize}

\vspace{1cm}

\begin{center}
\begin{tcolorbox}[colback=blue!5,colframe=blue!70,width=0.8\textwidth,arc=3mm]
\centering
\Large\textbf{¡Practica con cada método!}

\vspace{0.3cm}

\normalsize
La mejor manera de dominar estos métodos es experimentando con ellos en tus propios documentos.
\end{tcolorbox}
\end{center}

\end{document}
