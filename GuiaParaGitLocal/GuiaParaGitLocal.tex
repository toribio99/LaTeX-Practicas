\documentclass[11pt,a4paper]{article}

% =====================================================
% PREÁMBULO: CONFIGURACIÓN DEL DOCUMENTO
% =====================================================

% Detección de motor de compilación
\usepackage{iftex}

\ifluatex
  \usepackage{fontspec}
  \usepackage{polyglossia}
  \setdefaultlanguage[variant=mexican]{spanish}
\else
  \ifxetex
    \usepackage{fontspec}
    \usepackage{polyglossia}
    \setdefaultlanguage[variant=mexican]{spanish}
  \else
    \usepackage[T1]{fontenc}
    \usepackage[spanish,es-tabla]{babel}
  \fi
\fi

% Paquetes esenciales
\usepackage[margin=2.5cm]{geometry}
\usepackage{amsmath}
\usepackage{xcolor}
\usepackage{tcolorbox}
\tcbuselibrary{listings,skins,breakable}
\usepackage{enumitem}
\usepackage{booktabs}
\usepackage{array}
\usepackage{graphicx}
\usepackage{fancyvrb}

% Definición de colores
\definecolor{azuloscuro}{RGB}{0,51,102}
\definecolor{azulclaro}{RGB}{230,240,250}
\definecolor{verdeoscuro}{RGB}{0,100,0}
\definecolor{verdeclaro}{RGB}{230,255,230}
\definecolor{rojoclaro}{RGB}{255,230,230}
\definecolor{amarilloclaro}{RGB}{255,255,200}
\definecolor{grisclaro}{RGB}{245,245,245}
\definecolor{naranjaoscuro}{RGB}{230,115,0}
\definecolor{moradoclaro}{RGB}{240,230,255}

% Configuración de listings
\lstset{
    basicstyle=\ttfamily\small,
    breaklines=true,
    columns=flexible,
    showstringspaces=false,
    frame=single,
    backgroundcolor=\color{grisclaro},
    xleftmargin=10pt,
    xrightmargin=10pt,
    numbers=left,
    numberstyle=\tiny\color{gray},
    stepnumber=1,
    numbersep=8pt
}

% Hyperref al final
\usepackage[
    colorlinks=true,
    linkcolor=azuloscuro,
    urlcolor=blue,
    citecolor=verdeoscuro,
    pdftitle={Guía de Git Local para Proyectos LaTeX},
    pdfauthor={Toribio Arrieta},
    bookmarks=true,
    unicode=true,
    breaklinks=true
]{hyperref}

% Título
\title{\textbf{\Huge Guía de Git Local}\\[0.5cm]
\Large Control de Versiones para Proyectos \LaTeX{}\\[0.3cm]
\large Guardar estados y recuperar versiones antiguas}
\author{Toribio Arrieta}
\date{\today}

% =====================================================
% INICIO DEL DOCUMENTO
% =====================================================
\begin{document}

\maketitle
\tableofcontents
\newpage

% =====================================================
% SECCIÓN 1: INTRODUCCIÓN
% =====================================================
\section{¿Qué es Git y por qué usarlo?}

\begin{tcolorbox}[colback=azulclaro,colframe=azuloscuro,title=\textbf{Git: Tu máquina del tiempo para documentos LaTeX}]
\textbf{Git} es un sistema de control de versiones que te permite:
\begin{itemize}
    \item Guardar ``fotografías'' de tu proyecto en diferentes momentos
    \item Volver a cualquier versión anterior
    \item Ver exactamente qué cambió entre versiones
    \item Nunca más perder trabajo importante
    \item Trabajar sin miedo a romper el documento
\end{itemize}
\end{tcolorbox}

\subsection{El problema sin Git}

\textbf{Situación típica sin Git:}

\begin{lstlisting}[language=bash,numbers=none]
Mi_Tesis.tex
Mi_Tesis_final.tex
Mi_Tesis_final_v2.tex
Mi_Tesis_final_v2_AHORA_SI.tex
Mi_Tesis_final_v2_AHORA_SI_revisor.tex
Mi_Tesis_DEFINITIVO.tex
Mi_Tesis_DEFINITIVO_FINAL.tex
\end{lstlisting}

\begin{tcolorbox}[colback=rojoclaro,colframe=red!75!black,title=\textbf{Problemas:}]
\begin{itemize}
    \item No sabes qué versión es la correcta
    \item Ocupa mucho espacio en disco
    \item No sabes qué cambió entre versiones
    \item Si borras algo importante, se perdió para siempre
\end{itemize}
\end{tcolorbox}

\subsection{La solución con Git}

\textbf{Con Git solo tienes:}

\begin{lstlisting}[language=bash,numbers=none]
Mi_Tesis.tex
\end{lstlisting}

\begin{tcolorbox}[colback=verdeclaro,colframe=verdeoscuro,title=\textbf{Ventajas:}]
\begin{itemize}
    \item Un solo archivo, múltiples versiones guardadas internamente
    \item Cada versión tiene fecha, hora y descripción
    \item Puedes ver cambios línea por línea
    \item Recuperar versiones anteriores en segundos
    \item Trabajar sin miedo: siempre puedes volver atrás
\end{itemize}
\end{tcolorbox}

% =====================================================
% SECCIÓN 2: CONCEPTOS BÁSICOS
% =====================================================
\section{Conceptos básicos de Git}

\subsection{Vocabulario esencial}

\begin{table}[htbp]
\centering
\begin{tabular}{|>{\raggedright\arraybackslash}p{3.5cm}|>{\raggedright\arraybackslash}p{9cm}|}
\hline
\textbf{Término} & \textbf{Significado} \\
\hline
\hline
\textbf{Repositorio} & Una carpeta de tu proyecto que Git vigila \\
\hline
\textbf{Commit} & Una ``fotografía'' guardada de tu proyecto en un momento específico \\
\hline
\textbf{Staging} & Preparar archivos antes de guardarlos (commit) \\
\hline
\textbf{Hash} & Código único de cada commit (ej: \texttt{a3f2b1c}) \\
\hline
\textbf{HEAD} & Dónde estás ahora (el commit actual) \\
\hline
\textbf{Branch} & Rama de desarrollo (por ahora usas solo \texttt{master}) \\
\hline
\textbf{Working directory} & Tus archivos actuales en la carpeta \\
\hline
\end{tabular}
\caption{Términos clave de Git}
\end{table}

\subsection{El flujo de trabajo en tres pasos}

\begin{tcolorbox}[colback=moradoclaro,colframe=purple!75!black,title=\textbf{Los tres pasos para guardar cambios:}]

\textbf{Paso 1: Working Directory} (tus archivos)
\begin{lstlisting}[language=bash,numbers=none]
Editas: Mi_Tesis.tex
\end{lstlisting}

\textbf{Paso 2: Staging Area} (preparación)
\begin{lstlisting}[language=bash,numbers=none]
git add Mi_Tesis.tex
\end{lstlisting}

\textbf{Paso 3: Repository} (guardado permanente)
\begin{lstlisting}[language=bash,numbers=none]
git commit -m "Agregue capitulo 3"
\end{lstlisting}

\end{tcolorbox}

% =====================================================
% SECCIÓN 3: CONFIGURACIÓN INICIAL
% =====================================================
\section{Configuración inicial de Git}

\subsection{Verificar que Git está instalado}

\begin{lstlisting}[language=bash]
git --version
\end{lstlisting}

\textbf{Salida esperada:}
\begin{lstlisting}[numbers=none]
git version 2.50.0
\end{lstlisting}

\subsection{Configurar tu identidad (solo una vez)}

\begin{lstlisting}[language=bash]
git config --global user.name "Tu Nombre"
git config --global user.email "tu@email.com"
\end{lstlisting}

\textbf{Verificar configuración:}
\begin{lstlisting}[language=bash]
git config user.name
git config user.email
\end{lstlisting}

\subsection{Crear el archivo .gitignore}

Este archivo le dice a Git qué archivos ignorar (archivos auxiliares de LaTeX):

\begin{tcolorbox}[colback=grisclaro,colframe=black,title=\textbf{Contenido del archivo .gitignore}]
\begin{lstlisting}[language=bash,numbers=none]
# Archivos auxiliares de LaTeX
*.aux
*.log
*.out
*.toc
*.lof
*.lot
*.fls
*.fdb_latexmk
*.synctex.gz
*.bbl
*.blg
*.idx
*.ilg
*.ind

# PDFs (opcional: comentar si quieres guardar PDFs)
*.pdf

# Archivos de sistema
.DS_Store
*~
*.swp
\end{lstlisting}
\end{tcolorbox}

% =====================================================
% SECCIÓN 4: INICIALIZAR REPOSITORIO
% =====================================================
\section{Inicializar un repositorio Git}

\subsection{Paso a paso para tu primer proyecto}

\begin{tcolorbox}[colback=verdeclaro,colframe=verdeoscuro,title=\textbf{Ejemplo: Configurar Git para tu tesis}]

\textbf{Paso 1: Ve a la carpeta de tu proyecto}
\begin{lstlisting}[language=bash]
cd /Users/toribioarrieta/Documents/MiTesis
\end{lstlisting}

\textbf{Paso 2: Inicializa Git}
\begin{lstlisting}[language=bash]
git init
\end{lstlisting}

\textbf{Salida esperada:}
\begin{lstlisting}[numbers=none]
Initialized empty Git repository in
/Users/toribioarrieta/Documents/MiTesis/.git/
\end{lstlisting}

\textbf{Paso 3: Crea el archivo .gitignore}
\begin{lstlisting}[language=bash]
# Crear archivo con contenido
cat > .gitignore << 'EOF'
*.aux
*.log
*.pdf
*.synctex.gz
.DS_Store
EOF
\end{lstlisting}

\textbf{Paso 4: Añade tus archivos importantes}
\begin{lstlisting}[language=bash]
git add *.tex
git add .gitignore
git add figuras/
\end{lstlisting}

\textbf{Paso 5: Haz tu primer commit}
\begin{lstlisting}[language=bash]
git commit -m "Version inicial de la tesis"
\end{lstlisting}

\end{tcolorbox}

% =====================================================
% SECCIÓN 5: USO DIARIO
% =====================================================
\section{Comandos para uso diario}

\subsection{Ver el estado actual}

\begin{lstlisting}[language=bash]
git status
\end{lstlisting}

\textbf{Interpretación de la salida:}

\begin{table}[htbp]
\centering
\small
\begin{tabular}{|p{5cm}|p{8cm}|}
\hline
\textbf{Mensaje} & \textbf{Significado} \\
\hline
\texttt{Changes not staged} & Archivos modificados pero no preparados \\
\hline
\texttt{Changes to be committed} & Archivos listos para guardar \\
\hline
\texttt{Untracked files} & Archivos nuevos que Git no vigila \\
\hline
\texttt{nothing to commit} & Todo está guardado, sin cambios \\
\hline
\end{tabular}
\end{table}

\subsection{Guardar cambios (workflow completo)}

\begin{tcolorbox}[colback=azulclaro,colframe=azuloscuro,title=\textbf{Flujo típico después de editar}]

\textbf{1. Editas tu documento en TeXstudio}
\begin{verbatim}
[Editas Mi_Tesis.tex]
[Guardas el archivo]
[Compilas y verificas que funciona]
\end{verbatim}

\textbf{2. Ves qué cambió}
\begin{lstlisting}[language=bash]
git status
git diff Mi_Tesis.tex
\end{lstlisting}

\textbf{3. Preparas el archivo}
\begin{lstlisting}[language=bash]
git add Mi_Tesis.tex
\end{lstlisting}

\textbf{4. Guardas con mensaje descriptivo}
\begin{lstlisting}[language=bash]
git commit -m "Agregue seccion sobre metodologia"
\end{lstlisting}

\end{tcolorbox}

\subsection{Mensajes de commit efectivos}

\begin{table}[htbp]
\centering
\begin{tabular}{|c|p{10cm}|}
\hline
\textbf{Estado} & \textbf{Mensaje} \\
\hline
\hline
✅ BIEN & \texttt{Agregue capitulo 3 sobre resultados} \\
\hline
✅ BIEN & \texttt{Corregi errores de ortografia en seccion 2.1} \\
\hline
✅ BIEN & \texttt{Anadi 5 figuras nuevas al capitulo 4} \\
\hline
✅ BIEN & \texttt{Reorganice estructura del preambulo} \\
\hline
❌ MAL & \texttt{cambios} \\
\hline
❌ MAL & \texttt{actualizacion} \\
\hline
❌ MAL & \texttt{asdf} \\
\hline
❌ MAL & \texttt{final} \\
\hline
\end{tabular}
\caption{Ejemplos de mensajes de commit}
\end{table}

% =====================================================
% SECCIÓN 6: VER HISTORIAL
% =====================================================
\section{Ver el historial de cambios}

\subsection{Comando básico}

\begin{lstlisting}[language=bash]
git log
\end{lstlisting}

\textbf{Salida ejemplo:}
\begin{lstlisting}[numbers=none]
commit a3f2b1c4d5e6f7a8b9c0d1e2f3a4b5c6d7e8f9a0
Author: Toribio Arrieta <toribio99@gmail.com>
Date:   Thu Oct 17 10:30:00 2025 -0600

    Agregue capitulo 3 sobre resultados

commit 9f8e7d6c5b4a3f2e1d0c9b8a7f6e5d4c3b2a1f0
Author: Toribio Arrieta <toribio99@gmail.com>
Date:   Wed Oct 16 15:45:00 2025 -0600

    Version inicial de la tesis
\end{lstlisting}

\subsection{Versión compacta (recomendada)}

\begin{lstlisting}[language=bash]
git log --oneline
\end{lstlisting}

\textbf{Salida ejemplo:}
\begin{lstlisting}[numbers=none]
a3f2b1c Agregue capitulo 3 sobre resultados
9f8e7d6 Version inicial de la tesis
\end{lstlisting}

\subsection{Ver últimos N commits}

\begin{lstlisting}[language=bash]
# Ver ultimos 5 commits
git log --oneline -5

# Ver ultimos 10 commits
git log --oneline -10
\end{lstlisting}

\subsection{Ver cambios detallados}

\begin{lstlisting}[language=bash]
# Ver que cambio en cada commit
git log -p

# Ver solo estadisticas
git log --stat
\end{lstlisting}

% =====================================================
% SECCIÓN 7: COMPARAR VERSIONES
% =====================================================
\section{Comparar versiones}

\subsection{Ver cambios NO guardados}

\begin{lstlisting}[language=bash]
# Ver diferencias en archivos modificados
git diff

# Ver diferencias en archivo especifico
git diff Mi_Tesis.tex
\end{lstlisting}

\subsection{Ver cambios preparados (staged)}

\begin{lstlisting}[language=bash]
git diff --staged
\end{lstlisting}

\subsection{Comparar con commit anterior}

\begin{lstlisting}[language=bash]
# Comparar con el commit anterior
git diff HEAD~1

# Comparar con 3 commits atras
git diff HEAD~3
\end{lstlisting}

\subsection{Comparar dos commits específicos}

\begin{lstlisting}[language=bash]
# Primero obtener los hashes
git log --oneline

# Luego comparar (ejemplo)
git diff a3f2b1c 9f8e7d6
\end{lstlisting}

% =====================================================
% SECCIÓN 8: RECUPERAR VERSIONES
% =====================================================
\section{Recuperar versiones anteriores}

\begin{tcolorbox}[colback=amarilloclaro,colframe=naranjaoscuro,title=\textbf{⚠️ IMPORTANTE}]
Antes de recuperar versiones antiguas, \textbf{asegúrate de haber guardado (commit) tus cambios actuales}. De lo contrario, los perderás.
\end{tcolorbox}

\subsection{Recuperar UN archivo específico}

\textbf{Escenario:} Borraste una sección importante de \texttt{Mi\_Tesis.tex} y quieres recuperarla.

\begin{tcolorbox}[colback=verdeclaro,colframe=verdeoscuro,title=\textbf{Procedimiento:}]

\textbf{Paso 1: Ver historial}
\begin{lstlisting}[language=bash]
git log --oneline
\end{lstlisting}

\textbf{Salida:}
\begin{lstlisting}[numbers=none]
a3f2b1c Agregue capitulo 3
b4c5d6e Corregi capitulo 2
e7f8g9h Version inicial
\end{lstlisting}

\textbf{Paso 2: Recuperar archivo de commit específico}
\begin{lstlisting}[language=bash]
# Recuperar Mi_Tesis.tex del commit b4c5d6e
git checkout b4c5d6e Mi_Tesis.tex
\end{lstlisting}

\textbf{Paso 3: Revisar el archivo recuperado}
\begin{verbatim}
[Abre Mi_Tesis.tex en TeXstudio]
[Verifica que tiene la sección que necesitabas]
\end{verbatim}

\textbf{Paso 4: Guardar la recuperación}
\begin{lstlisting}[language=bash]
git add Mi_Tesis.tex
git commit -m "Recupere seccion eliminada del capitulo 2"
\end{lstlisting}

\end{tcolorbox}

\subsection{Ver versión antigua SIN modificar la actual}

\begin{lstlisting}[language=bash]
# Ver el archivo como era en ese commit (solo lectura)
git show b4c5d6e:Mi_Tesis.tex
\end{lstlisting}

Puedes copiar la parte que necesitas y pegarla manualmente.

\subsection{Viajar en el tiempo (TODO el proyecto)}

\begin{tcolorbox}[colback=rojoclaro,colframe=red!75!black,title=\textbf{⚠️ CUIDADO: Modo avanzado}]

\textbf{Esto mueve TODO tu proyecto a una versión anterior:}
\begin{lstlisting}[language=bash]
# Viajar al commit b4c5d6e
git checkout b4c5d6e
\end{lstlisting}

\textbf{Para volver al presente:}
\begin{lstlisting}[language=bash]
git checkout master
\end{lstlisting}

\textbf{Nota:} Mientras estás en el pasado, estás en modo ``detached HEAD''. Solo observa, no hagas commits.
\end{tcolorbox}

% =====================================================
% SECCIÓN 9: DESHACER CAMBIOS
% =====================================================
\section{Deshacer cambios}

\subsection{Deshacer cambios NO guardados}

\textbf{Escenario:} Hiciste cambios en \texttt{Mi\_Tesis.tex} pero NO has hecho \texttt{git add} ni \texttt{git commit}.

\begin{lstlisting}[language=bash]
# Restaurar archivo a la ultima version guardada
git checkout -- Mi_Tesis.tex
\end{lstlisting}

\begin{tcolorbox}[colback=rojoclaro,colframe=red!75!black]
⚠️ \textbf{CUIDADO:} Esto BORRA todos los cambios no guardados en ese archivo.
\end{tcolorbox}

\subsection{Quitar archivo del staging}

\textbf{Escenario:} Hiciste \texttt{git add Mi\_Tesis.tex} pero NO quieres incluirlo en el commit.

\begin{lstlisting}[language=bash]
git reset Mi_Tesis.tex
\end{lstlisting}

El archivo sigue modificado, solo lo quitas del área de preparación.

\subsection{Deshacer el último commit}

\begin{tcolorbox}[colback=amarilloclaro,colframe=naranjaoscuro,title=\textbf{Deshacer commit manteniendo cambios:}]
\begin{lstlisting}[language=bash]
git reset --soft HEAD~1
\end{lstlisting}

\textbf{Efecto:}
\begin{itemize}
    \item Deshace el commit
    \item Los cambios quedan en staging (listos para nuevo commit)
    \item Puedes modificar y hacer commit de nuevo
\end{itemize}
\end{tcolorbox}

\begin{tcolorbox}[colback=rojoclaro,colframe=red!75!black,title=\textbf{Deshacer commit BORRANDO cambios:}]
\begin{lstlisting}[language=bash]
git reset --hard HEAD~1
\end{lstlisting}

⚠️ \textbf{MUY PELIGROSO:} Borra el commit Y los cambios permanentemente.
\end{tcolorbox}

% =====================================================
% SECCIÓN 10: CASOS DE USO PRÁCTICOS
% =====================================================
\section{Casos de uso prácticos}

\subsection{Caso 1: Trabajo diario en tu tesis}

\begin{tcolorbox}[colback=azulclaro,colframe=azuloscuro,title=\textbf{Flujo diario recomendado}]

\textbf{Mañana (inicio):}
\begin{lstlisting}[language=bash]
cd /Users/toribioarrieta/Documents/MiTesis
git status
\end{lstlisting}

\textbf{Durante el día:}
\begin{verbatim}
[Editas archivos .tex en TeXstudio]
[Compilas y verificas]
\end{verbatim}

\textbf{Cada 1-2 horas o al terminar una sección:}
\begin{lstlisting}[language=bash]
git add *.tex
git commit -m "Descripcion breve"
\end{lstlisting}

\textbf{Tarde (final del día):}
\begin{lstlisting}[language=bash]
git log --oneline -5
# Verificar que guardaste todo
\end{lstlisting}

\end{tcolorbox}

\subsection{Caso 2: Experimentar sin miedo}

\textbf{Escenario:} Quieres probar una nueva estructura pero no estás seguro si funcionará.

\begin{tcolorbox}[colback=verdeclaro,colframe=verdeoscuro]

\textbf{Paso 1: Guarda el estado actual}
\begin{lstlisting}[language=bash]
git add .
git commit -m "Version estable antes de experimento"
\end{lstlisting}

\textbf{Paso 2: Experimenta libremente}
\begin{verbatim}
[Haz cambios grandes]
[Reorganiza capítulos]
[Prueba nuevos enfoques]
\end{verbatim}

\textbf{Paso 3a: Si funciona, guárdalo}
\begin{lstlisting}[language=bash]
git add .
git commit -m "Nueva estructura - funciono bien"
\end{lstlisting}

\textbf{Paso 3b: Si NO funciona, descártalo}
\begin{lstlisting}[language=bash]
git checkout -- .
# Vuelve todo a como estaba
\end{lstlisting}

\end{tcolorbox}

\subsection{Caso 3: Recuperar trabajo perdido}

\textbf{Escenario:} Borraste accidentalmente 2 páginas importantes.

\begin{lstlisting}[language=bash]
# Ver cuando estaba bien
git log --oneline

# Recuperar version antigua
git checkout a3f2b1c Mi_Tesis.tex

# Copiar lo que necesitas a un archivo temporal
cp Mi_Tesis.tex Mi_Tesis_recuperado.tex

# Volver a la version actual
git checkout master

# Fusionar manualmente lo recuperado
\end{lstlisting}

% =====================================================
% SECCIÓN 11: TIPS Y BUENAS PRÁCTICAS
% =====================================================
\section{Tips y buenas prácticas}

\subsection{Frecuencia de commits}

\begin{table}[htbp]
\centering
\begin{tabular}{|c|p{10cm}|}
\hline
\textbf{Frecuencia} & \textbf{Cuándo hacer commit} \\
\hline
\hline
✅ Ideal & Cada sección completa (30 min - 1 hora de trabajo) \\
\hline
✅ Bueno & Al terminar el día de trabajo \\
\hline
⚠️ Aceptable & Cada cambio importante (nuevo capítulo, figuras, etc.) \\
\hline
❌ Muy poco & Solo al terminar TODO el documento \\
\hline
\end{tabular}
\end{table}

\subsection{Qué incluir en commits}

\begin{table}[htbp]
\centering
\begin{tabular}{|c|p{5cm}|p{5cm}|}
\hline
 & \textbf{SÍ guardar} & \textbf{NO guardar} \\
\hline
\hline
Archivos & \texttt{*.tex} & \texttt{*.aux} \\
 & Figuras (\texttt{.png, .jpg}) & \texttt{*.log} \\
 & \texttt{.bib} (bibliografía) & \texttt{*.pdf} (opcional) \\
 & \texttt{.gitignore} & \texttt{*.synctex.gz} \\
\hline
\end{tabular}
\caption{Qué archivos incluir}
\end{table}

\subsection{Comandos útiles rápidos}

\begin{lstlisting}[language=bash]
# Ver estado rapido
git status -s

# Anadir todos los .tex modificados
git add *.tex

# Ver historial bonito
git log --oneline --graph --all

# Ver ultimo commit
git log -1

# Ver archivos en el ultimo commit
git show --name-only

# Contar commits totales
git log --oneline | wc -l
\end{lstlisting}

% =====================================================
% SECCIÓN 12: SOLUCIÓN DE PROBLEMAS
% =====================================================
\section{Solución de problemas comunes}

\subsection{Error: ``Not a git repository''}

\textbf{Problema:}
\begin{lstlisting}[numbers=none]
fatal: not a git repository
\end{lstlisting}

\textbf{Solución:}
\begin{lstlisting}[language=bash]
# Estas en el directorio correcto?
pwd

# Si no, ve al directorio del proyecto
cd /Users/toribioarrieta/Documents/MiTesis

# Si nunca inicializaste Git:
git init
\end{lstlisting}

\subsection{Error: ``Please tell me who you are''}

\textbf{Solución:}
\begin{lstlisting}[language=bash]
git config --global user.name "Tu Nombre"
git config --global user.email "tu@email.com"
\end{lstlisting}

\subsection{Error: ``Changes not staged for commit''}

\textbf{No es error, es informativo.}

\textbf{Solución:}
\begin{lstlisting}[language=bash]
# Anadir archivos modificados
git add *.tex
git commit -m "Tu mensaje"
\end{lstlisting}

\subsection{¿Cómo saber en qué commit estoy?}

\begin{lstlisting}[language=bash]
git log -1 --oneline
# O simplemente:
git status
\end{lstlisting}

% =====================================================
% SECCIÓN 13: COMANDOS DE REFERENCIA RÁPIDA
% =====================================================
\section{Referencia rápida de comandos}

\begin{table}[htbp]
\centering
\small
\begin{tabular}{|p{6cm}|p{7cm}|}
\hline
\textbf{Comando} & \textbf{Qué hace} \\
\hline
\hline
\multicolumn{2}{|c|}{\textbf{CONFIGURACIÓN}} \\
\hline
\texttt{git init} & Inicializar repositorio \\
\hline
\texttt{git config user.name} & Ver tu nombre \\
\hline
\texttt{git config user.email} & Ver tu email \\
\hline
\hline
\multicolumn{2}{|c|}{\textbf{USO DIARIO}} \\
\hline
\texttt{git status} & Ver estado actual \\
\hline
\texttt{git add archivo.tex} & Preparar archivo \\
\hline
\texttt{git add *.tex} & Preparar todos los .tex \\
\hline
\texttt{git commit -m "msg"} & Guardar cambios \\
\hline
\hline
\multicolumn{2}{|c|}{\textbf{HISTORIAL}} \\
\hline
\texttt{git log} & Ver historial completo \\
\hline
\texttt{git log --oneline} & Ver historial compacto \\
\hline
\texttt{git log -5} & Ver últimos 5 commits \\
\hline
\texttt{git show} & Ver último commit en detalle \\
\hline
\hline
\multicolumn{2}{|c|}{\textbf{COMPARAR}} \\
\hline
\texttt{git diff} & Ver cambios no guardados \\
\hline
\texttt{git diff archivo.tex} & Ver cambios en archivo \\
\hline
\texttt{git diff --staged} & Ver cambios preparados \\
\hline
\hline
\multicolumn{2}{|c|}{\textbf{RECUPERAR}} \\
\hline
\texttt{git checkout hash archivo} & Recuperar archivo antiguo \\
\hline
\texttt{git checkout -{}-} archivo & Descartar cambios locales \\
\hline
\texttt{git reset -{}-soft HEAD\textasciitilde{}1} & Deshacer último commit \\
\hline
\end{tabular}
\caption{Comandos esenciales de Git}
\end{table}

% =====================================================
% SECCIÓN 14: EJERCICIO PRÁCTICO
% =====================================================
\section{Ejercicio práctico paso a paso}

\begin{tcolorbox}[colback=moradoclaro,colframe=purple!75!black,title=\textbf{Ejercicio: Tu primer repositorio Git}]

\textbf{Objetivo:} Crear un mini-proyecto LaTeX con Git desde cero.

\textbf{Paso 1: Crear directorio}
\begin{lstlisting}[language=bash]
mkdir ~/Documents/PracticaGit
cd ~/Documents/PracticaGit
\end{lstlisting}

\textbf{Paso 2: Crear archivo LaTeX simple}
\begin{lstlisting}[language=bash]
cat > documento.tex << 'EOF'
\documentclass{article}
\begin{document}
Hola, este es mi primer documento con Git.
\end{document}
EOF
\end{lstlisting}

\textbf{Paso 3: Inicializar Git}
\begin{lstlisting}[language=bash]
git init
\end{lstlisting}

\textbf{Paso 4: Crear .gitignore}
\begin{lstlisting}[language=bash]
cat > .gitignore << 'EOF'
*.aux
*.log
*.pdf
EOF
\end{lstlisting}

\textbf{Paso 5: Primer commit}
\begin{lstlisting}[language=bash]
git add documento.tex .gitignore
git commit -m "Version inicial del documento"
\end{lstlisting}

\textbf{Paso 6: Hacer cambios}
\begin{lstlisting}[language=bash]
cat >> documento.tex << 'EOF'

Este es un parrafo nuevo que estoy agregando.
EOF
\end{lstlisting}

\textbf{Paso 7: Ver diferencias}
\begin{lstlisting}[language=bash]
git diff documento.tex
\end{lstlisting}

\textbf{Paso 8: Guardar cambios}
\begin{lstlisting}[language=bash]
git add documento.tex
git commit -m "Agregue un parrafo nuevo"
\end{lstlisting}

\textbf{Paso 9: Ver historial}
\begin{lstlisting}[language=bash]
git log --oneline
\end{lstlisting}

\textbf{Paso 10: Recuperar versión anterior}
\begin{lstlisting}[language=bash]
git log --oneline
# Anota el hash del primer commit
git checkout [hash] documento.tex
# Verificar que volvio
cat documento.tex
# Regresar a la version mas reciente
git checkout master
\end{lstlisting}

\end{tcolorbox}

% =====================================================
% CONCLUSIÓN
% =====================================================
\section{Conclusión}

\begin{tcolorbox}[colback=verdeclaro,colframe=verdeoscuro,title=\textbf{Resumen final}]

\textbf{Lo más importante que debes recordar:}

\begin{enumerate}
    \item \textbf{Haz commits frecuentemente} - Mejor muchos pequeños que uno gigante
    \item \textbf{Mensajes descriptivos} - Tu yo del futuro te lo agradecerá
    \item \textbf{Git es local} - Todo está en tu computadora, no necesita internet
    \item \textbf{No temas experimentar} - Siempre puedes volver atrás
    \item \textbf{Usa .gitignore} - No guardes archivos auxiliares
\end{enumerate}

\textbf{Los 5 comandos que usarás el 90\% del tiempo:}
\begin{lstlisting}[language=bash,numbers=none]
git status           # Ver que cambio
git add *.tex        # Preparar archivos
git commit -m "..."  # Guardar cambios
git log --oneline    # Ver historial
git diff             # Ver diferencias
\end{lstlisting}

\end{tcolorbox}

\begin{tcolorbox}[colback=azulclaro,colframe=azuloscuro]
\textbf{Próximos pasos:}
\begin{itemize}
    \item Inicia Git en tu proyecto actual
    \item Haz tu primer commit hoy mismo
    \item Practica con el ejercicio de la sección 14
    \item Consulta esta guía cuando tengas dudas
\end{itemize}
\end{tcolorbox}

\vfill

\begin{center}
\textit{Documento generado con \LaTeX{} -- \today}\\[0.2cm]
\small Con Git, nunca más perderás tu trabajo importante 🎉
\end{center}

\end{document}
