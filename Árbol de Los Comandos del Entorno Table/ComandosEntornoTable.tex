% !TEX encoding = UTF-8 Unicode
\documentclass[11pt,a4paper]{article}

% Paquetes necesarios
\usepackage[utf8]{inputenc}
\usepackage[spanish]{babel}
\usepackage[margin=2cm]{geometry}
\usepackage{xcolor}
\usepackage{tcolorbox}
\usepackage{enumitem}
\usepackage{fontawesome5}
\usepackage{listings}
\usepackage{booktabs}

% Colores personalizados
\definecolor{categorycolor}{RGB}{41,128,185}
\definecolor{commandcolor}{RGB}{39,174,96}
\definecolor{examplecolor}{RGB}{149,165,166}

% Configuración de listings
\lstset{
	basicstyle=\ttfamily\small,
	breaklines=true,
	columns=fullflexible,
	keepspaces=true
}

% Título
\title{\textbf{\Huge Entornos \texttt{table} y \texttt{tabular} en \LaTeX{}}\\\large Guía Completa de Comandos y Opciones}
\author{}
\date{\today}
 \usepackage[
%colorlinks=true,        % Enlaces con color (en lugar de cajas)
linkcolor=blue,         % Color de enlaces internos
urlcolor=cyan,          % Color de URLs
citecolor=green,        % Color de citas bibliográficas
filecolor=magenta,      % Color de enlaces a archivos
pdfborder={0 0 0},      % Sin bordes en los enlaces
bookmarks=true,         % Crear marcadores en el PDF
bookmarksopen=true,     % Marcadores expandidos al abrir
pdftitle={Mi Título},   % Título del PDF
pdfauthor={Mi Nombre},  % Autor del PDF
pdfsubject={Tema},      % Tema del documento
pdfkeywords={palabra1, palabra2}, % Palabras clave
%hidelinks,              % Ocultar todos los bordes/colores de enlaces
unicode=true,           % Permitir caracteres Unicode en marcadores
breaklinks=true         % Permitir saltos de línea en enlaces
]{hyperref}

\begin{document}
	
	\maketitle
	
	\begin{tcolorbox}[colback=blue!5,colframe=blue!75!black,title=\faInfoCircle\ Introducción]
		El entorno \texttt{table} es un contenedor flotante para tablas (similar a \texttt{figure}), mientras que \texttt{tabular} crea la estructura real de la tabla con filas y columnas.
	\end{tcolorbox}
	
	\tableofcontents
	
	\section{Entorno table}
	
	\subsection*{Sintaxis básica}
	\begin{lstlisting}[language=TeX]
		\begin{table}[opciones]
			... contenido ...
		\end{table}
	\end{lstlisting}
	
	\subsection{Opciones de Posición}
	
	Controlan dónde \LaTeX{} coloca la tabla en el documento.
	
	\subsubsection*{\texttt{\textcolor{commandcolor}{[h]}} -- Here}
	\begin{tcolorbox}[colback=green!5,colframe=green!50!black]
		\textbf{Descripción:} Intenta colocar la tabla aquí
		
		\textbf{Ejemplo:}
		\begin{lstlisting}[language=TeX]
			\begin{table}[h]
				\centering
				\begin{tabular}{cc}
					...
				\end{tabular}
			\end{table}
		\end{lstlisting}
	\end{tcolorbox}
	
	\subsubsection*{\texttt{\textcolor{commandcolor}{[t]}} -- Top}
	\begin{tcolorbox}[colback=green!5,colframe=green!50!black]
		\textbf{Descripción:} Parte superior de la página
		
		\textbf{Ejemplo:}
		\begin{lstlisting}[language=TeX]
			\begin{table}[t]
				...
			\end{table}
		\end{lstlisting}
	\end{tcolorbox}
	
	\subsubsection*{\texttt{\textcolor{commandcolor}{[b]}} -- Bottom}
	\begin{tcolorbox}[colback=green!5,colframe=green!50!black]
		\textbf{Descripción:} Parte inferior de la página
		
		\textbf{Ejemplo:}
		\begin{lstlisting}[language=TeX]
			\begin{table}[b]
				...
			\end{table}
		\end{lstlisting}
	\end{tcolorbox}
	
	\subsubsection*{\texttt{\textcolor{commandcolor}{[p]}} -- Page}
	\begin{tcolorbox}[colback=green!5,colframe=green!50!black]
		\textbf{Descripción:} Página solo para flotantes
		
		\textbf{Ejemplo:}
		\begin{lstlisting}[language=TeX]
			\begin{table}[p]
				...
			\end{table}
		\end{lstlisting}
	\end{tcolorbox}
	
	\subsubsection*{\texttt{\textcolor{commandcolor}{[H]}} -- HERE (forzado)}
	\begin{tcolorbox}[colback=green!5,colframe=green!50!black]
		\textbf{Descripción:} Fuerza posición exacta (requiere paquete \texttt{float})
		
		\textbf{Ejemplo:}
		\begin{lstlisting}[language=TeX]
			\usepackage{float}
			\begin{table}[H]
				...
			\end{table}
		\end{lstlisting}
		
		\tcblower
		\faLightbulb\ \textbf{Nota:} Necesita \texttt{\textbackslash usepackage\{float\}}
	\end{tcolorbox}
	
	\subsubsection*{\texttt{\textcolor{commandcolor}{[htbp]}} -- Combinación}
	\begin{tcolorbox}[colback=green!5,colframe=green!50!black]
		\textbf{Descripción:} Múltiples opciones de posición (más flexible)
		
		\textbf{Ejemplo:}
		\begin{lstlisting}[language=TeX]
			\begin{table}[htbp]
				...
			\end{table}
		\end{lstlisting}
		
		\tcblower
		\faLightbulb\ \textbf{Nota:} Más flexible, \LaTeX{} elige la mejor opción
	\end{tcolorbox}
	
	\subsubsection*{\texttt{\textcolor{commandcolor}{[!h]}} -- Override}
	\begin{tcolorbox}[colback=green!5,colframe=green!50!black]
		\textbf{Descripción:} Ignora restricciones internas de \LaTeX{}
		
		\textbf{Ejemplo:}
		\begin{lstlisting}[language=TeX]
			\begin{table}[!h]
				...
			\end{table}
		\end{lstlisting}
		
		\tcblower
		\faLightbulb\ \textbf{Nota:} El ! relaja las restricciones de \LaTeX{}
	\end{tcolorbox}
	
	\subsection{Comando \textbackslash centering}
	
	\subsubsection*{\texttt{\textbackslash centering}}
	\begin{tcolorbox}[colback=green!5,colframe=green!50!black]
		\textbf{Descripción:} Centra el contenido de la tabla
		
		\textbf{Ejemplo:}
		\begin{lstlisting}[language=TeX]
			\begin{table}[h]
				\centering
				\begin{tabular}{cc}
					...
				\end{tabular}
			\end{table}
		\end{lstlisting}
		
		\tcblower
		\faLightbulb\ \textbf{Nota:} Preferible a \texttt{\textbackslash begin\{center\}...\textbackslash end\{center\}}
	\end{tcolorbox}
	
	\subsubsection*{\texttt{\textbackslash raggedright}}
	\begin{tcolorbox}[colback=green!5,colframe=green!50!black]
		\textbf{Descripción:} Alineación a la izquierda
		
		\textbf{Ejemplo:}
		\begin{lstlisting}[language=TeX]
			\raggedright
			\begin{tabular}{cc}...
			\end{lstlisting}
		\end{tcolorbox}
		
		\subsubsection*{\texttt{\textbackslash raggedleft}}
		\begin{tcolorbox}[colback=green!5,colframe=green!50!black]
			\textbf{Descripción:} Alineación a la derecha
			
			\textbf{Ejemplo:}
			\begin{lstlisting}[language=TeX]
				\raggedleft
				\begin{tabular}{cc}...
				\end{lstlisting}
			\end{tcolorbox}
			
			\subsection{Comando \textbackslash caption}
			
			\subsubsection*{\texttt{\textbackslash caption\{texto\}}}
			\begin{tcolorbox}[colback=green!5,colframe=green!50!black]
				\textbf{Descripción:} Título de tabla básico
				
				\textbf{Ejemplo:}
				\begin{lstlisting}[language=TeX]
					\caption{Resultados experimentales}
				\end{lstlisting}
			\end{tcolorbox}
			
			\subsubsection*{\texttt{\textbackslash caption[corto]\{largo\}}}
			\begin{tcolorbox}[colback=green!5,colframe=green!50!black]
				\textbf{Descripción:} Versión corta para índice de tablas
				
				\textbf{Ejemplo:}
				\begin{lstlisting}[language=TeX]
					\caption[Resultados]{Resultados completos del experimento realizado}
				\end{lstlisting}
				
				\tcblower
				\faLightbulb\ \textbf{Nota:} Corto aparece en \texttt{\textbackslash listoftables}
			\end{tcolorbox}
			
			\subsubsection*{\texttt{\textbackslash caption*\{texto\}}}
			\begin{tcolorbox}[colback=green!5,colframe=green!50!black]
				\textbf{Descripción:} Caption sin numeración (requiere paquete caption)
				
				\textbf{Ejemplo:}
				\begin{lstlisting}[language=TeX]
					\usepackage{caption}
					\caption*{Nota: Sin número}
				\end{lstlisting}
				
				\tcblower
				\faLightbulb\ \textbf{Nota:} Necesita \texttt{\textbackslash usepackage\{caption\}}
			\end{tcolorbox}
			
			\subsection{Comando \textbackslash label}
			
			\subsubsection*{\texttt{\textbackslash label\{nombre\}}}
			\begin{tcolorbox}[colback=green!5,colframe=green!50!black]
				\textbf{Descripción:} Crea etiqueta identificadora
				
				\textbf{Ejemplo:}
				\begin{lstlisting}[language=TeX]
					\label{tab:resultados}
				\end{lstlisting}
				
				\tcblower
				\faLightbulb\ \textbf{Nota:} Usar prefijo \texttt{tab:} es convención común
			\end{tcolorbox}
			
			\subsubsection*{\texttt{\textbackslash ref\{nombre\}}}
			\begin{tcolorbox}[colback=green!5,colframe=green!50!black]
				\textbf{Descripción:} Referencia al número de tabla
				
				\textbf{Ejemplo:}
				\begin{lstlisting}[language=TeX]
					Ver Tabla \ref{tab:resultados}
				\end{lstlisting}
				
				\tcblower
				\faLightbulb\ \textbf{Nota:} Muestra solo el número
			\end{tcolorbox}
			
			\subsubsection*{\texttt{\textbackslash pageref\{nombre\}}}
			\begin{tcolorbox}[colback=green!5,colframe=green!50!black]
				\textbf{Descripción:} Referencia al número de página
				
				\textbf{Ejemplo:}
				\begin{lstlisting}[language=TeX]
					En la página \pageref{tab:resultados}
				\end{lstlisting}
			\end{tcolorbox}
			
			%\newpage
			
			\section{Entorno tabular}
			
			\subsection*{Sintaxis básica}
			\begin{lstlisting}[language=TeX]
				\begin{tabular}{especificación}
					... contenido ...
				\end{tabular}
			\end{lstlisting}
			
			\subsection{Especificación de Columnas}
			
			Define alineación y número de columnas.
			
			\subsubsection*{\texttt{\{l\}} -- Left}
			\begin{tcolorbox}[colback=green!5,colframe=green!50!black]
				\textbf{Descripción:} Columna alineada a la izquierda
				
				\textbf{Ejemplo:}
				\begin{lstlisting}[language=TeX]
					\begin{tabular}{l}
						Texto izquierda \\
					\end{tabular}
				\end{lstlisting}
			\end{tcolorbox}
			
			\subsubsection*{\texttt{\{c\}} -- Center}
			\begin{tcolorbox}[colback=green!5,colframe=green!50!black]
				\textbf{Descripción:} Columna centrada
				
				\textbf{Ejemplo:}
				\begin{lstlisting}[language=TeX]
					\begin{tabular}{c}
						Texto centrado \\
					\end{tabular}
				\end{lstlisting}
			\end{tcolorbox}
			
			\subsubsection*{\texttt{\{r\}} -- Right}
			\begin{tcolorbox}[colback=green!5,colframe=green!50!black]
				\textbf{Descripción:} Columna alineada a la derecha
				
				\textbf{Ejemplo:}
				\begin{lstlisting}[language=TeX]
					\begin{tabular}{r}
						Texto derecha \\
					\end{tabular}
				\end{lstlisting}
			\end{tcolorbox}
			
			\subsubsection*{\texttt{\{p\{ancho\}\}} -- Paragraph}
			\begin{tcolorbox}[colback=green!5,colframe=green!50!black]
				\textbf{Descripción:} Columna con ancho fijo y texto justificado
				
				\textbf{Ejemplo:}
				\begin{lstlisting}[language=TeX]
					\begin{tabular}{p{3cm}}
						Texto largo que se ajusta \\
					\end{tabular}
				\end{lstlisting}
			\end{tcolorbox}
			
			\subsubsection*{\texttt{\{m\{ancho\}\}} -- Middle}
			\begin{tcolorbox}[colback=green!5,colframe=green!50!black]
				\textbf{Descripción:} Como p pero verticalmente centrado (requiere array)
				
				\textbf{Ejemplo:}
				\begin{lstlisting}[language=TeX]
					\usepackage{array}
					\begin{tabular}{m{3cm}}
					\end{lstlisting}
					
					\tcblower
					\faLightbulb\ \textbf{Nota:} Necesita \texttt{\textbackslash usepackage\{array\}}
				\end{tcolorbox}
				
				\subsubsection*{\texttt{\{b\{ancho\}\}} -- Bottom}
				\begin{tcolorbox}[colback=green!5,colframe=green!50!black]
					\textbf{Descripción:} Como p pero alineado abajo (requiere array)
					
					\textbf{Ejemplo:}
					\begin{lstlisting}[language=TeX]
						\usepackage{array}
						\begin{tabular}{b{3cm}}
						\end{lstlisting}
						
						\tcblower
						\faLightbulb\ \textbf{Nota:} Necesita \texttt{\textbackslash usepackage\{array\}}
					\end{tcolorbox}
					
					\subsubsection*{\texttt{\{|c|\}} -- Líneas verticales}
					\begin{tcolorbox}[colback=green!5,colframe=green!50!black]
						\textbf{Descripción:} | añade línea vertical
						
						\textbf{Ejemplo:}
						\begin{lstlisting}[language=TeX]
							\begin{tabular}{|c|c|c|}
								A & B & C \\
							\end{tabular}
						\end{lstlisting}
					\end{tcolorbox}
					
					\subsubsection*{\texttt{\{lll\}} -- Múltiples columnas}
					\begin{tcolorbox}[colback=green!5,colframe=green!50!black]
						\textbf{Descripción:} Ejemplo: 3 columnas izquierda
						
						\textbf{Ejemplo:}
						\begin{lstlisting}[language=TeX]
							\begin{tabular}{lll}
								A & B & C \\
							\end{tabular}
						\end{lstlisting}
					\end{tcolorbox}
					
					\subsubsection*{\texttt{\{*\{n\}\{especif\}\}} -- Repetir}
					\begin{tcolorbox}[colback=green!5,colframe=green!50!black]
						\textbf{Descripción:} Repetir especificación n veces
						
						\textbf{Ejemplo:}
						\begin{lstlisting}[language=TeX]
							\begin{tabular}{*{5}{c}}
								% 5 columnas centradas
							\end{tabular}
						\end{lstlisting}
						
						\tcblower
						\faLightbulb\ \textbf{Nota:} \texttt{\{*\{5\}\{c\}\}} = \texttt{\{ccccc\}}
					\end{tcolorbox}
					
					\subsection{Comandos de Contenido}
					
					\subsubsection*{\texttt{\&} -- Separador}
					\begin{tcolorbox}[colback=green!5,colframe=green!50!black]
						\textbf{Descripción:} Separador de columnas
						
						\textbf{Ejemplo:}
						\begin{lstlisting}[language=TeX]
							Columna1 & Columna2 & Columna3 \\
						\end{lstlisting}
					\end{tcolorbox}
					
					\subsubsection*{\texttt{\textbackslash\textbackslash} -- Nueva fila}
					\begin{tcolorbox}[colback=green!5,colframe=green!50!black]
						\textbf{Descripción:} Nueva fila
						
						\textbf{Ejemplo:}
						\begin{lstlisting}[language=TeX]
							Fila 1 \\
							Fila 2 \\
						\end{lstlisting}
					\end{tcolorbox}
					
					\subsubsection*{\texttt{\textbackslash\textbackslash[espacio]} -- Nueva fila con espacio}
					\begin{tcolorbox}[colback=green!5,colframe=green!50!black]
						\textbf{Descripción:} Nueva fila con espacio adicional
						
						\textbf{Ejemplo:}
						\begin{lstlisting}[language=TeX]
							Fila 1 \\[5pt]
							Fila 2 con espacio extra
						\end{lstlisting}
					\end{tcolorbox}
					
					\subsubsection*{\texttt{\textbackslash hline} -- Línea horizontal}
					\begin{tcolorbox}[colback=green!5,colframe=green!50!black]
						\textbf{Descripción:} Línea horizontal completa
						
						\textbf{Ejemplo:}
						\begin{lstlisting}[language=TeX]
							\hline
							Fila 1 \\
							\hline
						\end{lstlisting}
					\end{tcolorbox}
					
					\subsubsection*{\texttt{\textbackslash cline\{i-j\}} -- Línea parcial}
					\begin{tcolorbox}[colback=green!5,colframe=green!50!black]
						\textbf{Descripción:} Línea horizontal parcial (columnas i a j)
						
						\textbf{Ejemplo:}
						\begin{lstlisting}[language=TeX]
							Fila 1 \\
							\cline{2-3}
							Fila 2
						\end{lstlisting}
					\end{tcolorbox}
					
					\subsubsection*{\texttt{\textbackslash multicolumn}}
					\begin{tcolorbox}[colback=green!5,colframe=green!50!black]
						\textbf{Descripción:} Celda que abarca n columnas
						
						\textbf{Ejemplo:}
						\begin{lstlisting}[language=TeX]
							\multicolumn{3}{|c|}{Título centrado} \\
							\hline
						\end{lstlisting}
						
						\tcblower
						\faLightbulb\ \textbf{Nota:} n = número de columnas a fusionar
					\end{tcolorbox}
					
					\subsubsection*{\texttt{\textbackslash multirow}}
					\begin{tcolorbox}[colback=green!5,colframe=green!50!black]
						\textbf{Descripción:} Celda que abarca n filas (requiere multirow)
						
						\textbf{Ejemplo:}
						\begin{lstlisting}[language=TeX]
							\usepackage{multirow}
							\multirow{2}{*}{Texto} & Col2 \\
							& Col2 \\
						\end{lstlisting}
						
						\tcblower
						\faLightbulb\ \textbf{Nota:} Necesita \texttt{\textbackslash usepackage\{multirow\}}
					\end{tcolorbox}
					
					\subsection{Espaciado y Formato}
					
					\subsubsection*{\texttt{\textbackslash arraystretch}}
					\begin{tcolorbox}[colback=green!5,colframe=green!50!black]
						\textbf{Descripción:} Factor de espaciado vertical entre filas
						
						\textbf{Ejemplo:}
						\begin{lstlisting}[language=TeX]
							\renewcommand{\arraystretch}{1.5}
							\begin{tabular}{cc}
								...
							\end{tabular}
						\end{lstlisting}
						
						\tcblower
						\faLightbulb\ \textbf{Nota:} 1.5 = 150\% del espaciado normal
					\end{tcolorbox}
					
					\subsubsection*{\texttt{\textbackslash tabcolsep}}
					\begin{tcolorbox}[colback=green!5,colframe=green!50!black]
						\textbf{Descripción:} Espaciado horizontal entre columnas
						
						\textbf{Ejemplo:}
						\begin{lstlisting}[language=TeX]
							\setlength{\tabcolsep}{10pt}
							\begin{tabular}{cc}
								...
							\end{tabular}
						\end{lstlisting}
						
						\tcblower
						\faLightbulb\ \textbf{Nota:} Por defecto es 6pt
					\end{tcolorbox}
					
					\subsubsection*{\texttt{\textbackslash arrayrulewidth}}
					\begin{tcolorbox}[colback=green!5,colframe=green!50!black]
						\textbf{Descripción:} Grosor de las líneas
						
						\textbf{Ejemplo:}
						\begin{lstlisting}[language=TeX]
							\setlength{\arrayrulewidth}{1pt}
							\begin{tabular}{|c|c|}
								\hline
								...
							\end{tabular}
						\end{lstlisting}
					\end{tcolorbox}
					
					\subsubsection*{\texttt{@\{separador\}} -- Separador personalizado}
					\begin{tcolorbox}[colback=green!5,colframe=green!50!black]
						\textbf{Descripción:} Separador personalizado entre columnas
						
						\textbf{Ejemplo:}
						\begin{lstlisting}[language=TeX]
							\begin{tabular}{c@{:}c}
								A : B \\
							\end{tabular}
						\end{lstlisting}
						
						\tcblower
						\faLightbulb\ \textbf{Nota:} Reemplaza el espacio por defecto
					\end{tcolorbox}
					
					%\newpage
					
					\section{Paquetes para Tablas}
					
					\subsection{Paquetes Esenciales}
					
					\subsubsection*{\texttt{\textbackslash usepackage\{booktabs\}}}
					\begin{tcolorbox}[colback=green!5,colframe=green!50!black]
						\textbf{Descripción:} Líneas profesionales (\textbackslash toprule, \textbackslash midrule, \textbackslash bottomrule)
						
						\textbf{Ejemplo:}
						\begin{lstlisting}[language=TeX]
							\usepackage{booktabs}
							\begin{tabular}{ccc}
								\toprule
								A & B & C \\
								\midrule
								1 & 2 & 3 \\
								\bottomrule
							\end{tabular}
						\end{lstlisting}
						
						\tcblower
						\faLightbulb\ \textbf{Nota:} Mejor que \textbackslash hline para tablas profesionales
					\end{tcolorbox}
					
					\subsubsection*{\texttt{\textbackslash usepackage\{array\}}}
					\begin{tcolorbox}[colback=green!5,colframe=green!50!black]
						\textbf{Descripción:} Mejoras para tabular (columnas m, b, >\{\})
						
						\textbf{Ejemplo:}
						\begin{lstlisting}[language=TeX]
							\usepackage{array}
							\begin{tabular}{>{\bfseries}c}
								Negrita \\
							\end{tabular}
						\end{lstlisting}
					\end{tcolorbox}
					
					\subsubsection*{\texttt{\textbackslash usepackage\{multirow\}}}
					\begin{tcolorbox}[colback=green!5,colframe=green!50!black]
						\textbf{Descripción:} Celdas que abarcan múltiples filas
						
						\textbf{Ejemplo:}
						\begin{lstlisting}[language=TeX]
							\usepackage{multirow}
							\multirow{2}{*}{Texto}
						\end{lstlisting}
					\end{tcolorbox}
					
					\subsubsection*{\texttt{\textbackslash usepackage\{longtable\}}}
					\begin{tcolorbox}[colback=green!5,colframe=green!50!black]
						\textbf{Descripción:} Tablas que pueden ocupar múltiples páginas
						
						\textbf{Ejemplo:}
						\begin{lstlisting}[language=TeX]
							\usepackage{longtable}
							\begin{longtable}{ccc}
								\caption{Tabla larga} \\
								\hline
								...
							\end{longtable}
						\end{lstlisting}
					\end{tcolorbox}
					
					\subsubsection*{\texttt{\textbackslash usepackage\{tabularx\}}}
					\begin{tcolorbox}[colback=green!5,colframe=green!50!black]
						\textbf{Descripción:} Tablas con ancho total específico
						
						\textbf{Ejemplo:}
						\begin{lstlisting}[language=TeX]
							\usepackage{tabularx}
							\begin{tabularx}{\textwidth}{XXX}
								...
							\end{tabularx}
						\end{lstlisting}
						
						\tcblower
						\faLightbulb\ \textbf{Nota:} X ajusta automáticamente el ancho
					\end{tcolorbox}
					
					\subsubsection*{\texttt{\textbackslash usepackage\{tabulary\}}}
					\begin{tcolorbox}[colback=green!5,colframe=green!50!black]
						\textbf{Descripción:} Similar a tabularx pero mejor para párrafos
						
						\textbf{Ejemplo:}
						\begin{lstlisting}[language=TeX]
							\usepackage{tabulary}
							\begin{tabulary}{\textwidth}{LCR}
								...
							\end{tabulary}
						\end{lstlisting}
					\end{tcolorbox}
					
					\subsection{Paquetes Avanzados}
					
					\subsubsection*{\texttt{\textbackslash usepackage\{colortbl\}}}
					\begin{tcolorbox}[colback=green!5,colframe=green!50!black]
						\textbf{Descripción:} Colorear celdas, filas y columnas
						
						\textbf{Ejemplo:}
						\begin{lstlisting}[language=TeX]
							\usepackage[table]{xcolor}
							\rowcolor{gray!30} A & B \\
							\cellcolor{blue!20} C & D
						\end{lstlisting}
					\end{tcolorbox}
					
					\subsubsection*{\texttt{\textbackslash usepackage\{makecell\}}}
					\begin{tcolorbox}[colback=green!5,colframe=green!50!black]
						\textbf{Descripción:} Múltiples líneas en celdas y formato
						
						\textbf{Ejemplo:}
						\begin{lstlisting}[language=TeX]
							\usepackage{makecell}
							\makecell{Línea 1 \\ Línea 2}
						\end{lstlisting}
					\end{tcolorbox}
					
					\subsubsection*{\texttt{\textbackslash usepackage\{rotating\}}}
					\begin{tcolorbox}[colback=green!5,colframe=green!50!black]
						\textbf{Descripción:} Rotar tablas (sidewaystable)
						
						\textbf{Ejemplo:}
						\begin{lstlisting}[language=TeX]
							\usepackage{rotating}
							\begin{sidewaystable}
								...
							\end{sidewaystable}
						\end{lstlisting}
					\end{tcolorbox}
					
					\subsubsection*{\texttt{\textbackslash usepackage\{diagbox\}}}
					\begin{tcolorbox}[colback=green!5,colframe=green!50!black]
						\textbf{Descripción:} Celdas con diagonal
						
						\textbf{Ejemplo:}
						\begin{lstlisting}[language=TeX]
							\usepackage{diagbox}
							\diagbox{Fila}{Columna}
						\end{lstlisting}
					\end{tcolorbox}
					
					%\newpage
					
					\section*{\faCheckCircle\ Ejemplos Completos}
					
					\subsection*{Ejemplo Básico}
					
					\begin{tcolorbox}[colback=green!10,colframe=green!75!black,title=\faCode\ Tabla básica]
						\begin{lstlisting}[language=TeX]
							\begin{table}[htbp]
								\centering
								\caption{Resultados del experimento}
								\label{tab:resultados}
								\begin{tabular}{|l|c|r|}
									\hline
									\textbf{Nombre} & \textbf{Valor} & \textbf{Error} \\
									\hline
									Muestra A & 25.4 & ±0.2 \\
									Muestra B & 30.1 & ±0.3 \\
									\hline
								\end{tabular}
							\end{table}
						\end{lstlisting}
					\end{tcolorbox}
					
					\subsection*{Ejemplo Profesional con booktabs}
					
					\begin{tcolorbox}[colback=purple!10,colframe=purple!75!black,title=\faCode\ Tabla profesional]
						\begin{lstlisting}[language=TeX]
							\usepackage{booktabs}
							
							\begin{table}[htbp]
								\centering
								\caption{Comparación de métodos}
								\label{tab:comparacion}
								\begin{tabular}{lcc}
									\toprule
									\textbf{Método} & \textbf{Precisión} & \textbf{Tiempo (s)} \\
									\midrule
									Método A & 95.2\% & 1.23 \\
									Método B & 97.8\% & 2.45 \\
									Método C & 92.1\% & 0.89 \\
									\bottomrule
								\end{tabular}
							\end{table}
						\end{lstlisting}
					\end{tcolorbox}
					
					\section*{\faLightbulb\ Tips Importantes}
					
					\begin{tcolorbox}[colback=blue!10,colframe=blue!75!black]
						\begin{itemize}[leftmargin=*]
							\item \texttt{table} es el contenedor flotante, \texttt{tabular} crea la tabla real
							\item El \texttt{\textbackslash label} debe ir DESPUÉS del \texttt{\textbackslash caption}
							\item Usa \texttt{booktabs} para tablas profesionales (evita líneas verticales)
							\item Para tablas largas que ocupan múltiples páginas, usa \texttt{longtable}
							\item \texttt{tabularx} es útil cuando necesitas que la tabla ocupe un ancho específico
							\item \texttt{[htbp]} es la combinación más flexible para posicionamiento
							\item Evita usar demasiadas líneas verticales y horizontales (estilo minimalista es mejor)
							\item Con \texttt{booktabs}, nunca usar \texttt{\textbackslash hline} junto con \texttt{\textbackslash toprule/midrule/bottomrule}
						\end{itemize}
					\end{tcolorbox}
					
					\section*{\faTable\ Diferencias importantes}
					
					\begin{tcolorbox}[colback=yellow!10,colframe=orange!75!black,title=\faExclamationTriangle\ table vs tabular]
						\begin{itemize}[leftmargin=*]
							\item \textbf{table}: Es el entorno flotante (como \texttt{figure}). Permite usar \texttt{\textbackslash caption}, \texttt{\textbackslash label} y opciones de posición [htbp].
							\item \textbf{tabular}: Es el entorno que crea la estructura real de la tabla con filas y columnas.
							\item Puedes usar \texttt{tabular} sin \texttt{table} si no necesitas que sea flotante ni numeración automática.
							\item El \texttt{table} puede contener múltiples \texttt{tabular} si es necesario.
						\end{itemize}
					\end{tcolorbox}
					
					\section*{\faCode\ Ejemplo Avanzado con Colores}
					
					\begin{tcolorbox}[colback=orange!10,colframe=orange!75!black,title=Tabla con colores]
						\begin{lstlisting}[language=TeX]
							\usepackage[table]{xcolor}
							\usepackage{booktabs}
							
							\begin{table}[htbp]
								\centering
								\caption{Datos con colores}
								\label{tab:colores}
								\begin{tabular}{lcc}
									\toprule
									\textbf{Categoría} & \textbf{Valor 1} & \textbf{Valor 2} \\
									\midrule
									\rowcolor{gray!20}
									Categoría A & 100 & 120 \\
									Categoría B & 150 & 180 \\
									\rowcolor{gray!20}
									Categoría C & 200 & 210 \\
									\bottomrule
								\end{tabular}
							\end{table}
						\end{lstlisting}
					\end{tcolorbox}
					
					\section*{\faTable\ Ejemplo con multirow y multicolumn}
					
					\begin{tcolorbox}[colback=cyan!10,colframe=cyan!75!black,title=Celdas fusionadas]
						\begin{lstlisting}[language=TeX]
							\usepackage{multirow}
							
							\begin{table}[htbp]
								\centering
								\caption{Tabla con celdas fusionadas}
								\begin{tabular}{|c|c|c|}
									\hline
									\multicolumn{3}{|c|}{\textbf{Título General}} \\
									\hline
									\multirow{2}{*}{Grupo 1} & Dato A & 10 \\
									& Dato B & 20 \\
									\hline
									\multirow{2}{*}{Grupo 2} & Dato C & 30 \\
									& Dato D & 40 \\
									\hline
								\end{tabular}
							\end{table}
						\end{lstlisting}
					\end{tcolorbox}
					
					\vspace{1cm}
					
					\begin{center}
						\textit{Documento generado con \LaTeX{} -- \today}
					\end{center}
					
				\end{document}