\documentclass{article}
\usepackage[utf8]{inputenc}
\usepackage[T1]{fontenc}
\usepackage[spanish]{babel}
\usepackage{dirtree}

\begin{document}
	
	\dirtree{%
		.1 TikZ Libraries/.
		.2 Flechas y conectores/.
		.3 arrows (clásica, obsoleta).
		.3 arrows.meta (sistema moderno de puntas de flecha).
		.3 bending (curvar flechas en trayectorias).
		.2 Posicionamiento y coordenadas/.
		.3 positioning (nodos relativos).
		.3 calc (cálculo de coordenadas).
		.3 quotes (etiquetas rápidas en caminos).
		.3 through (circunferencias/figuras que pasan por un punto).
		.2 Formas de nodos/.
		.3 shapes (colección general).
		.3 shapes.geometric (triángulo, rombo, trapecio).
		.3 shapes.misc (nube, estrella).
		.3 shapes.symbols (símbolos decorativos).
		.3 shapes.multipart (nodos con varias partes).
		.2 Estilos de relleno y trazos/.
		.3 patterns (patrones básicos).
		.3 patterns.meta (sistema moderno de patrones).
		.3 fadings (desvanecimientos).
		.3 shadings (degradados).
		.3 shadows, shadows.blur, shadows.shadow (sombras).
		.2 Capas y fondos/.
		.3 backgrounds (capas de fondo).
		.3 fit (nodos contenedores).
		.3 matrix (matriz de nodos).
		.2 Decoraciones/.
		.3 decorations.pathmorphing (zigzag, serpenteo, pasos aleatorios).
		.3 decorations.markings (marcas en trayectorias, flechas en mitad).
		.3 decorations.text (texto a lo largo de caminos).
		.3 decorations.fractals (fractales como curva de Koch).
		.3 decorations.footprints (huellas decorativas).
		.2 Grafos y árboles/.
		.3 trees (árboles jerárquicos).
		.3 graphs (sintaxis declarativa de grafos).
		.3 graphdrawing (algoritmos automáticos, requiere LuaLaTeX).
		.2 Geometría y matemáticas/.
		.3 angles (ángulos en figuras).
		.3 intersections (cálculo de intersecciones).
		.3 lindenmayersystems (fractales tipo plantas).
		.2 Extras/.
		.3 mindmap (mapas mentales).
		.3 calendar (calendarios).
		.3 plotmarks (marcadores de gráficas).
		.3 external (externalizar figuras en PDFs).
		.3 3d (soporte 3D básico).
	}
	
\end{document}
