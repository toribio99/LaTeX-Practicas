% !TEX encoding = UTF-8 Unicode
\documentclass[10pt,a4paper]{article}

% Paquetes necesarios
\usepackage[utf8]{inputenc}
\usepackage[spanish]{babel}
\usepackage[margin=1.5cm]{geometry}
\usepackage{xcolor}
\usepackage{tcolorbox}
\usepackage{enumitem}
\usepackage{fontawesome5}
\usepackage{listings}
\usepackage{pgfplots}
\usepackage{multicol}
\pgfplotsset{compat=1.18}

% Colores personalizados
\definecolor{commandcolor}{RGB}{39,174,96}

% Configuración de listings
\lstset{
	basicstyle=\ttfamily\footnotesize,
	breaklines=true,
	columns=fullflexible,
	keepspaces=true
}

% Título
\title{\textbf{\Huge PGFPlots en \LaTeX{}} \\[3mm] \large Guía Completa de Comandos, Opciones y Librerías}
\author{}
\date{\today}
 \usepackage[
%colorlinks=true,        % Enlaces con color (en lugar de cajas)
linkcolor=blue,         % Color de enlaces internos
urlcolor=cyan,          % Color de URLs
citecolor=green,        % Color de citas bibliográficas
filecolor=magenta,      % Color de enlaces a archivos
pdfborder={0 0 0},      % Sin bordes en los enlaces
bookmarks=true,         % Crear marcadores en el PDF
bookmarksopen=true,     % Marcadores expandidos al abrir
pdftitle={Mi Título},   % Título del PDF
pdfauthor={Mi Nombre},  % Autor del PDF
pdfsubject={Tema},      % Tema del documento
pdfkeywords={palabra1, palabra2}, % Palabras clave
%hidelinks,              % Ocultar todos los bordes/colores de enlaces
unicode=true,           % Permitir caracteres Unicode en marcadores
breaklinks=true         % Permitir saltos de línea en enlaces
]{hyperref}

\begin{document}
	
	\maketitle

	\begin{tcolorbox}[colback=blue!5,colframe=blue!75!black,title=\faInfoCircle\ Introducción]
		PGFPlots es un paquete construido sobre TikZ/PGF para crear gráficas científicas de alta calidad. Soporta gráficas 2D y 3D, gráficas de barras, diagramas de dispersión, contornos, superficies, y mucho más.
	\end{tcolorbox}
	
	\tableofcontents
	
	\section{Carga del Paquete}
	
	\subsection*{\texttt{\textbackslash usepackage\{pgfplots\}}}
	\begin{tcolorbox}[colback=green!5,colframe=green!50!black]
		\textbf{Descripción:} Carga básica del paquete PGFPlots
		
		\textbf{Ejemplo:}
		\begin{lstlisting}[language=TeX]
			\usepackage{pgfplots}
			\pgfplotsset{compat=1.18}
		\end{lstlisting}
		
		\tcblower
		\faLightbulb\ \textbf{Nota:} Es recomendable especificar \texttt{compat} para usar características más recientes
	\end{tcolorbox}
	
	\subsection*{\texttt{\textbackslash usepgfplotslibrary\{library\}}}
	\begin{tcolorbox}[colback=green!5,colframe=green!50!black]
		\textbf{Descripción:} Carga librerías adicionales de PGFPlots
		
		\textbf{Ejemplo:}
		\begin{lstlisting}[language=TeX]
			\usepgfplotslibrary{polar, colorbrewer, statistics}
		\end{lstlisting}
	\end{tcolorbox}
	
	\subsection*{\texttt{\textbackslash pgfplotsset\{options\}}}
	\begin{tcolorbox}[colback=green!5,colframe=green!50!black]
		\textbf{Descripción:} Configura opciones globales para todos los gráficos
		
		\textbf{Ejemplo:}
		\begin{lstlisting}[language=TeX]
			\pgfplotsset{
				compat=1.18,
				width=10cm,
				height=6cm
			}
		\end{lstlisting}
	\end{tcolorbox}
	
	\section{Entorno axis}
	
	\subsection*{Sintaxis básica}
	\begin{lstlisting}[language=TeX]
		\begin{tikzpicture}
			\begin{axis}[opciones]
				... comandos de plot ...
			\end{axis}
		\end{tikzpicture}
	\end{lstlisting}
	
	\section{Opciones Globales del Entorno axis}
	
	\subsection{Tamaño y Dimensiones}
	
	\subsubsection*{\texttt{width=dimension, height=dimension}}
	\begin{tcolorbox}[colback=green!5,colframe=green!50!black]
		\textbf{Descripción:} Define el tamaño del gráfico
		
		\textbf{Ejemplo:}
		\begin{lstlisting}[language=TeX]
			\begin{axis}[width=10cm, height=6cm]
				\addplot {x^2};
			\end{axis}
		\end{lstlisting}
	\end{tcolorbox}
	
	\subsubsection*{\texttt{scale=factor}}
	\begin{tcolorbox}[colback=green!5,colframe=green!50!black]
		\textbf{Descripción:} Escala todo el gráfico
		
		\textbf{Ejemplo:}
		\begin{lstlisting}[language=TeX]
			\begin{axis}[scale=1.5]
				\addplot {sin(deg(x))};
			\end{axis}
		\end{lstlisting}
	\end{tcolorbox}
	
	\subsubsection*{\texttt{scale only axis}}
	\begin{tcolorbox}[colback=green!5,colframe=green!50!black]
		\textbf{Descripción:} Escala solo el área del gráfico, no las etiquetas
		
		\textbf{Ejemplo:}
		\begin{lstlisting}[language=TeX]
			\begin{axis}[width=10cm, height=6cm, scale only axis]
				\addplot {x};
			\end{axis}
		\end{lstlisting}
	\end{tcolorbox}
	
	\subsection{Límites de Ejes}
	
	\subsubsection*{\texttt{xmin=value, xmax=value, ymin=value, ymax=value}}
	\begin{tcolorbox}[colback=green!5,colframe=green!50!black]
		\textbf{Descripción:} Define los límites de los ejes
		
		\textbf{Ejemplo:}
		\begin{lstlisting}[language=TeX]
			\begin{axis}[xmin=0, xmax=10, ymin=-2, ymax=2]
				\addplot {sin(deg(x))};
			\end{axis}
		\end{lstlisting}
	\end{tcolorbox}
	
	\subsubsection*{\texttt{domain=min:max}}
	\begin{tcolorbox}[colback=green!5,colframe=green!50!black]
		\textbf{Descripción:} Define el dominio para funciones
		
		\textbf{Ejemplo:}
		\begin{lstlisting}[language=TeX]
			\begin{axis}
				\addplot[domain=-2:2] {x^3};
			\end{axis}
		\end{lstlisting}
	\end{tcolorbox}
	
	\subsubsection*{\texttt{restrict x to domain=min:max, restrict y to domain=min:max}}
	\begin{tcolorbox}[colback=green!5,colframe=green!50!black]
		\textbf{Descripción:} Restringe valores a un rango (útil para discontinuidades)
		
		\textbf{Ejemplo:}
		\begin{lstlisting}[language=TeX]
			\begin{axis}
				\addplot[restrict y to domain=-10:10] {1/x};
			\end{axis}
		\end{lstlisting}
	\end{tcolorbox}
	
	\subsection{Escalas de Ejes}
	
	\subsubsection*{\texttt{xmode=normal/log, ymode=normal/log}}
	\begin{tcolorbox}[colback=green!5,colframe=green!50!black]
		\textbf{Descripción:} Escala lineal o logarítmica
		
		\textbf{Ejemplo:}
		\begin{lstlisting}[language=TeX]
			\begin{axis}[xmode=log, ymode=log]
				\addplot {x^2};
			\end{axis}
		\end{lstlisting}
	\end{tcolorbox}
	
	\subsubsection*{\texttt{log basis x=value, log basis y=value}}
	\begin{tcolorbox}[colback=green!5,colframe=green!50!black]
		\textbf{Descripción:} Base del logaritmo (por defecto 10)
		
		\textbf{Ejemplo:}
		\begin{lstlisting}[language=TeX]
			\begin{axis}[ymode=log, log basis y=2]
				\addplot {2^x};
			\end{axis}
		\end{lstlisting}
	\end{tcolorbox}
	
	\subsubsection*{\texttt{axis equal, axis equal image}}
	\begin{tcolorbox}[colback=green!5,colframe=green!50!black]
		\textbf{Descripción:} Mantiene proporción 1:1 en los ejes
		
		\textbf{Ejemplo:}
		\begin{lstlisting}[language=TeX]
			\begin{axis}[axis equal]
				\addplot {sqrt(1-x^2)};
			\end{axis}
		\end{lstlisting}
	\end{tcolorbox}
	
	\subsection{Grillas}
	
	\subsubsection*{\texttt{grid=none/major/minor/both}}
	\begin{tcolorbox}[colback=green!5,colframe=green!50!black]
		\textbf{Descripción:} Muestra grilla en el gráfico
		
		\textbf{Ejemplo:}
		\begin{lstlisting}[language=TeX]
			\begin{axis}[grid=both]
				\addplot {x^2};
			\end{axis}
		\end{lstlisting}
	\end{tcolorbox}
	
	\subsubsection*{\texttt{grid style=\{options\}}}
	\begin{tcolorbox}[colback=green!5,colframe=green!50!black]
		\textbf{Descripción:} Estilo de la grilla
		
		\textbf{Ejemplo:}
		\begin{lstlisting}[language=TeX]
			\begin{axis}[grid=major, grid style={dashed, gray}]
				\addplot {x};
			\end{axis}
		\end{lstlisting}
	\end{tcolorbox}
	
	\subsubsection*{\texttt{major grid style=\{options\}, minor grid style=\{options\}}}
	\begin{tcolorbox}[colback=green!5,colframe=green!50!black]
		\textbf{Descripción:} Estilos separados para grilla mayor y menor
		
		\textbf{Ejemplo:}
		\begin{lstlisting}[language=TeX]
			\begin{axis}[grid=both,
				major grid style={thick, red},
				minor grid style={thin, gray!50}]
				\addplot {sin(deg(x))};
			\end{axis}
		\end{lstlisting}
	\end{tcolorbox}
	
	\subsection{Etiquetas de Ejes}
	
	\subsubsection*{\texttt{xlabel=text, ylabel=text}}
	\begin{tcolorbox}[colback=green!5,colframe=green!50!black]
		\textbf{Descripción:} Etiquetas de los ejes
		
		\textbf{Ejemplo:}
		\begin{lstlisting}[language=TeX]
			\begin{axis}[xlabel=$x$, ylabel=$f(x)$]
				\addplot {x^2};
			\end{axis}
		\end{lstlisting}
	\end{tcolorbox}
	
	\subsubsection*{\texttt{xlabel style=\{options\}, ylabel style=\{options\}}}
	\begin{tcolorbox}[colback=green!5,colframe=green!50!black]
		\textbf{Descripción:} Estilo de las etiquetas
		
		\textbf{Ejemplo:}
		\begin{lstlisting}[language=TeX]
			\begin{axis}[
				xlabel=$x$,
				xlabel style={font=\Large, color=red}]
				\addplot {x};
			\end{axis}
		\end{lstlisting}
	\end{tcolorbox}
	
	\subsection{Título}
	
	\subsubsection*{\texttt{title=text}}
	\begin{tcolorbox}[colback=green!5,colframe=green!50!black]
		\textbf{Descripción:} Título del gráfico
		
		\textbf{Ejemplo:}
		\begin{lstlisting}[language=TeX]
			\begin{axis}[title=Función Cuadrática]
				\addplot {x^2};
			\end{axis}
		\end{lstlisting}
	\end{tcolorbox}
	
	\subsubsection*{\texttt{title style=\{options\}}}
	\begin{tcolorbox}[colback=green!5,colframe=green!50!black]
		\textbf{Descripción:} Estilo del título
		
		\textbf{Ejemplo:}
		\begin{lstlisting}[language=TeX]
			\begin{axis}[
				title=Mi Gráfico,
				title style={font=\bfseries\Large, color=blue}]
				\addplot {x};
			\end{axis}
		\end{lstlisting}
	\end{tcolorbox}
	
	\subsection{Ticks (Marcas)}
	
	\subsubsection*{\texttt{xtick=\{list\}, ytick=\{list\}}}
	\begin{tcolorbox}[colback=green!5,colframe=green!50!black]
		\textbf{Descripción:} Posiciones específicas de los ticks
		
		\textbf{Ejemplo:}
		\begin{lstlisting}[language=TeX]
			\begin{axis}[xtick={0,1,2,3,4}, ytick={0,5,10,15}]
				\addplot {x^2};
			\end{axis}
		\end{lstlisting}
	\end{tcolorbox}
	
	\subsubsection*{\texttt{xticklabels=\{list\}, yticklabels=\{list\}}}
	\begin{tcolorbox}[colback=green!5,colframe=green!50!black]
		\textbf{Descripción:} Etiquetas personalizadas para los ticks
		
		\textbf{Ejemplo:}
		\begin{lstlisting}[language=TeX]
			\begin{axis}[
				xtick={1,2,3,4},
				xticklabels={Ene, Feb, Mar, Abr}]
				\addplot coordinates {(1,10) (2,15) (3,12) (4,18)};
			\end{axis}
		\end{lstlisting}
	\end{tcolorbox}
	
	\subsubsection*{\texttt{xtick distance=value, ytick distance=value}}
	\begin{tcolorbox}[colback=green!5,colframe=green!50!black]
		\textbf{Descripción:} Distancia entre ticks
		
		\textbf{Ejemplo:}
		\begin{lstlisting}[language=TeX]
			\begin{axis}[xtick distance=2, ytick distance=5]
				\addplot {x^2};
			\end{axis}
		\end{lstlisting}
	\end{tcolorbox}
	
	\subsubsection*{\texttt{minor tick num=n}}
	\begin{tcolorbox}[colback=green!5,colframe=green!50!black]
		\textbf{Descripción:} Número de ticks menores entre mayores
		
		\textbf{Ejemplo:}
		\begin{lstlisting}[language=TeX]
			\begin{axis}[minor tick num=4]
				\addplot {x};
			\end{axis}
		\end{lstlisting}
	\end{tcolorbox}
	
	\subsubsection*{\texttt{xticklabel style=\{options\}, yticklabel style=\{options\}}}
	\begin{tcolorbox}[colback=green!5,colframe=green!50!black]
		\textbf{Descripción:} Estilo de las etiquetas de ticks
		
		\textbf{Ejemplo:}
		\begin{lstlisting}[language=TeX]
			\begin{axis}[
				xticklabel style={rotate=45, anchor=east, font=\small}]
				\addplot {x^2};
			\end{axis}
		\end{lstlisting}
	\end{tcolorbox}
	
	\subsubsection*{\texttt{xtickten=\{exponents\}}}
	\begin{tcolorbox}[colback=green!5,colframe=green!50!black]
		\textbf{Descripción:} Ticks en potencias de 10 (para escala log)
		
		\textbf{Ejemplo:}
		\begin{lstlisting}[language=TeX]
			\begin{axis}[xmode=log, xtickten={-2,-1,0,1,2}]
				\addplot {10^x};
			\end{axis}
		\end{lstlisting}
	\end{tcolorbox}
	
	\subsection{Posición de Ejes}
	
	\subsubsection*{\texttt{axis lines=none/box/left/center/right/middle}}
	\begin{tcolorbox}[colback=green!5,colframe=green!50!black]
		\textbf{Descripción:} Estilo y posición de los ejes
		
		\textbf{Ejemplo:}
		\begin{lstlisting}[language=TeX]
			\begin{axis}[axis lines=center]
				\addplot {sin(deg(x))};
			\end{axis}
		\end{lstlisting}
	\end{tcolorbox}
	
	\subsubsection*{\texttt{axis x line=bottom/top/middle/center/none}}
	\begin{tcolorbox}[colback=green!5,colframe=green!50!black]
		\textbf{Descripción:} Posición del eje X
		
		\textbf{Ejemplo:}
		\begin{lstlisting}[language=TeX]
			\begin{axis}[axis x line=middle, axis y line=center]
				\addplot {x^2};
			\end{axis}
		\end{lstlisting}
	\end{tcolorbox}
	
	\subsubsection*{\texttt{hide x axis, hide y axis}}
	\begin{tcolorbox}[colback=green!5,colframe=green!50!black]
		\textbf{Descripción:} Oculta ejes específicos
		
		\textbf{Ejemplo:}
		\begin{lstlisting}[language=TeX]
			\begin{axis}[hide x axis]
				\addplot {x^2};
			\end{axis}
		\end{lstlisting}
	\end{tcolorbox}
	
	\subsection{Leyenda}
	
	\subsubsection*{\texttt{legend pos=position}}
	\begin{tcolorbox}[colback=green!5,colframe=green!50!black]
		\textbf{Descripción:} Posición de la leyenda
		
		\textbf{Posiciones:} north west, north east, south west, south east, outer north east
		
		\textbf{Ejemplo:}
		\begin{lstlisting}[language=TeX]
			\begin{axis}[legend pos=north west]
				\addplot {x}; \addlegendentry{$f(x)=x$}
				\addplot {x^2}; \addlegendentry{$f(x)=x^2$}
			\end{axis}
		\end{lstlisting}
	\end{tcolorbox}
	
	\subsubsection*{\texttt{legend style=\{options\}}}
	\begin{tcolorbox}[colback=green!5,colframe=green!50!black]
		\textbf{Descripción:} Estilo de la leyenda
		
		\textbf{Ejemplo:}
		\begin{lstlisting}[language=TeX]
			\begin{axis}[
				legend style={at={(0.5,-0.15)}, anchor=north, 
					legend columns=2}]
				\addplot {x}; \addlegendentry{Linear}
				\addplot {x^2}; \addlegendentry{Cuadrática}
			\end{axis}
		\end{lstlisting}
	\end{tcolorbox}
	
	\subsubsection*{\texttt{legend cell align=left/center/right}}
	\begin{tcolorbox}[colback=green!5,colframe=green!50!black]
		\textbf{Descripción:} Alineación del texto en la leyenda
		
		\textbf{Ejemplo:}
		\begin{lstlisting}[language=TeX]
			\begin{axis}[legend cell align=left]
				\addplot {x}; \addlegendentry{Función 1}
			\end{axis}
		\end{lstlisting}
	\end{tcolorbox}
	
	%\newpage
	
	\section{Comando \textbackslash addplot}
	
	\subsection*{Sintaxis básica}
	\begin{lstlisting}[language=TeX]
		\addplot[options] expression/coordinates;
	\end{lstlisting}
	
	\subsection{Tipos de Plot}
	
	\subsubsection*{\texttt{\textbackslash addplot \{expression\};}}
	\begin{tcolorbox}[colback=green!5,colframe=green!50!black]
		\textbf{Descripción:} Grafica una expresión matemática
		
		\textbf{Ejemplo:}
		\begin{lstlisting}[language=TeX]
			\addplot {x^2 - 2*x + 1};
			\addplot {sin(deg(x))};
			\addplot {exp(x)};
		\end{lstlisting}
	\end{tcolorbox}
	
	\subsubsection*{\texttt{\textbackslash addplot coordinates \{(x,y) ...\};}}
	\begin{tcolorbox}[colback=green!5,colframe=green!50!black]
		\textbf{Descripción:} Grafica puntos específicos
		
		\textbf{Ejemplo:}
		\begin{lstlisting}[language=TeX]
			\addplot coordinates {
				(0,0) (1,1) (2,4) (3,9) (4,16)
			};
		\end{lstlisting}
	\end{tcolorbox}
	
	\subsubsection*{\texttt{\textbackslash addplot table \{data.dat\};}}
	\begin{tcolorbox}[colback=green!5,colframe=green!50!black]
		\textbf{Descripción:} Grafica desde archivo externo
		
		\textbf{Ejemplo:}
		\begin{lstlisting}[language=TeX]
			\addplot table {datos.dat};
			\addplot table[x=columna1, y=columna2] {archivo.txt};
		\end{lstlisting}
	\end{tcolorbox}
	
	\subsubsection*{\texttt{\textbackslash addplot table \{inline data\};}}
	\begin{tcolorbox}[colback=green!5,colframe=green!50!black]
		\textbf{Descripción:} Datos tabulares inline
		
		\textbf{Ejemplo:}
		\begin{lstlisting}[language=TeX]
			\addplot table {
				x  y
				0  0
				1  1
				2  4
				3  9
			};
		\end{lstlisting}
	\end{tcolorbox}
	
	\subsubsection*{\texttt{\textbackslash addplot3 \{expression\};}}
	\begin{tcolorbox}[colback=green!5,colframe=green!50!black]
		\textbf{Descripción:} Gráficas 3D
		
		\textbf{Ejemplo:}
		\begin{lstlisting}[language=TeX]
			\begin{axis}
				\addplot3 {x^2 + y^2};
			\end{axis}
		\end{lstlisting}
	\end{tcolorbox}
	
	\subsection{Opciones de Estilo para \textbackslash addplot}
	
	\subsubsection*{\texttt{color=color, draw=color}}
	\begin{tcolorbox}[colback=green!5,colframe=green!50!black]
		\textbf{Descripción:} Color de la línea
		
		\textbf{Ejemplo:}
		\begin{lstlisting}[language=TeX]
			\addplot[red] {x^2};
			\addplot[color=blue!50!black] {x^3};
		\end{lstlisting}
	\end{tcolorbox}
	
	\subsubsection*{\texttt{mark=type}}
	\begin{tcolorbox}[colback=green!5,colframe=green!50!black]
		\textbf{Descripción:} Tipo de marcador en los puntos
		
		\textbf{Tipos:} *, o, x, +, |, -, square, square*, triangle, triangle*, diamond, pentagon, star
		
		\textbf{Ejemplo:}
		\begin{lstlisting}[language=TeX]
			\addplot[mark=*] coordinates {(0,0) (1,1) (2,4)};
			\addplot[mark=square*] {x^2};
		\end{lstlisting}
	\end{tcolorbox}
	
	\subsubsection*{\texttt{mark size=dimension}}
	\begin{tcolorbox}[colback=green!5,colframe=green!50!black]
		\textbf{Descripción:} Tamaño de los marcadores
		
		\textbf{Ejemplo:}
		\begin{lstlisting}[language=TeX]
			\addplot[mark=*, mark size=3pt] {x};
		\end{lstlisting}
	\end{tcolorbox}
	
	\subsubsection*{\texttt{mark options=\{options\}}}
	\begin{tcolorbox}[colback=green!5,colframe=green!50!black]
		\textbf{Descripción:} Opciones de estilo para marcadores
		
		\textbf{Ejemplo:}
		\begin{lstlisting}[language=TeX]
			\addplot[mark=*, mark options={fill=red, draw=blue}] {x^2};
		\end{lstlisting}
	\end{tcolorbox}
	
	\subsubsection*{\texttt{only marks}}
	\begin{tcolorbox}[colback=green!5,colframe=green!50!black]
		\textbf{Descripción:} Solo muestra marcadores, sin línea
		
		\textbf{Ejemplo:}
		\begin{lstlisting}[language=TeX]
			\addplot[only marks, mark=o] coordinates {(0,0) (1,1) (2,4)};
		\end{lstlisting}
	\end{tcolorbox}
	
	\subsubsection*{\texttt{no marks}}
	\begin{tcolorbox}[colback=green!5,colframe=green!50!black]
		\textbf{Descripción:} Solo línea, sin marcadores
		
		\textbf{Ejemplo:}
		\begin{lstlisting}[language=TeX]
			\addplot[no marks] {x^2};
		\end{lstlisting}
	\end{tcolorbox}
	
	\subsubsection*{\texttt{line width=dimension}}
	\begin{tcolorbox}[colback=green!5,colframe=green!50!black]
		\textbf{Descripción:} Grosor de la línea
		
		\textbf{Ejemplo:}
		\begin{lstlisting}[language=TeX]
			\addplot[line width=2pt] {x};
			\addplot[ultra thick] {x^2};
		\end{lstlisting}
	\end{tcolorbox}
	
	\subsubsection*{\texttt{dashed, dotted, dashdotted, densely dashed, loosely dashed}}
	\begin{tcolorbox}[colback=green!5,colframe=green!50!black]
		\textbf{Descripción:} Estilo de línea
		
		\textbf{Ejemplo:}
		\begin{lstlisting}[language=TeX]
			\addplot[dashed] {x};
			\addplot[dotted] {x^2};
			\addplot[dashdotted] {x^3};
		\end{lstlisting}
	\end{tcolorbox}
	
	\subsubsection*{\texttt{smooth}}
	\begin{tcolorbox}[colback=green!5,colframe=green!50!black]
		\textbf{Descripción:} Suaviza la curva entre puntos
		
		\textbf{Ejemplo:}
		\begin{lstlisting}[language=TeX]
			\addplot[smooth] coordinates {(0,0) (1,3) (2,1) (3,4)};
		\end{lstlisting}
	\end{tcolorbox}
	
	\subsubsection*{\texttt{const plot, jump mark left, jump mark right}}
	\begin{tcolorbox}[colback=green!5,colframe=green!50!black]
		\textbf{Descripción:} Gráfica de escalera
		
		\textbf{Ejemplo:}
		\begin{lstlisting}[language=TeX]
			\addplot[const plot] coordinates {(0,0) (1,1) (2,3) (3,2)};
			\addplot[jump mark left] coordinates {(0,0) (1,1) (2,3)};
		\end{lstlisting}
	\end{tcolorbox}
	
	\subsubsection*{\texttt{samples=n}}
	\begin{tcolorbox}[colback=green!5,colframe=green!50!black]
		\textbf{Descripción:} Número de puntos para evaluar función
		
		\textbf{Ejemplo:}
		\begin{lstlisting}[language=TeX]
			\addplot[samples=50] {sin(deg(x))};
			\addplot[samples=200] {1/x};
		\end{lstlisting}
	\end{tcolorbox}
	
	\subsubsection*{\texttt{domain=min:max}}
	\begin{tcolorbox}[colback=green!5,colframe=green!50!black]
		\textbf{Descripción:} Rango de evaluación
		
		\textbf{Ejemplo:}
		\begin{lstlisting}[language=TeX]
			\addplot[domain=-pi:pi] {sin(deg(x))};
		\end{lstlisting}
	\end{tcolorbox}
	
	\subsubsection*{\texttt{fill=color, fill opacity=value}}
	\begin{tcolorbox}[colback=green!5,colframe=green!50!black]
		\textbf{Descripción:} Rellena el área bajo la curva
		
		\textbf{Ejemplo:}
		\begin{lstlisting}[language=TeX]
			\addplot[fill=blue!20] {x^2} \closedcycle;
		\end{lstlisting}
	\end{tcolorbox}
	
	\subsection{Leyenda}
	
	\subsubsection*{\texttt{\textbackslash addlegendentry\{text\}}}
	\begin{tcolorbox}[colback=green!5,colframe=green!50!black]
		\textbf{Descripción:} Añade entrada a la leyenda
		
		\textbf{Ejemplo:}
		\begin{lstlisting}[language=TeX]
			\addplot {x^2};
			\addlegendentry{$f(x)=x^2$}
			\addplot {x^3};
			\addlegendentry{$f(x)=x^3$}
		\end{lstlisting}
	\end{tcolorbox}
	
	\subsubsection*{\texttt{\textbackslash legend\{entry1, entry2, ...\}}}
	\begin{tcolorbox}[colback=green!5,colframe=green!50!black]
		\textbf{Descripción:} Define todas las entradas de leyenda a la vez
		
		\textbf{Ejemplo:}
		\begin{lstlisting}[language=TeX]
			\addplot {x};
			\addplot {x^2};
			\legend{Linear, Cuadrática}
		\end{lstlisting}
	\end{tcolorbox}
	
	\section{Tipos Especiales de Gráficas}
	
	\subsection{Gráficas de Barras}
	
	\subsubsection*{\texttt{ybar, xbar}}
	\begin{tcolorbox}[colback=green!5,colframe=green!50!black]
		\textbf{Descripción:} Gráfica de barras verticales u horizontales
		
		\textbf{Ejemplo:}
		\begin{lstlisting}[language=TeX]
			\begin{axis}[ybar]
				\addplot coordinates {(1,10) (2,15) (3,8) (4,20)};
			\end{axis}
		\end{lstlisting}
	\end{tcolorbox}
	
	\subsubsection*{\texttt{bar width=dimension}}
	\begin{tcolorbox}[colback=green!5,colframe=green!50!black]
		\textbf{Descripción:} Ancho de las barras
		
		\textbf{Ejemplo:}
		\begin{lstlisting}[language=TeX]
			\begin{axis}[ybar, bar width=10pt]
				\addplot coordinates {(1,10) (2,15) (3,12)};
			\end{axis}
		\end{lstlisting}
	\end{tcolorbox}
	
	\subsection{Gráficas 3D}
	
	\subsubsection*{\texttt{view=\{azimuth\}\{elevation\}}}
	\begin{tcolorbox}[colback=green!5,colframe=green!50!black]
		\textbf{Descripción:} Ángulo de visión 3D
		
		\textbf{Ejemplo:}
		\begin{lstlisting}[language=TeX]
			\begin{axis}[view={60}{30}]
				\addplot3 {x^2 + y^2};
			\end{axis}
		\end{lstlisting}
	\end{tcolorbox}
	
	\subsubsection*{\texttt{surf, mesh}}
	\begin{tcolorbox}[colback=green!5,colframe=green!50!black]
		\textbf{Descripción:} Superficie o malla 3D
		
		\textbf{Ejemplo:}
		\begin{lstlisting}[language=TeX]
			\begin{axis}
				\addplot3[surf] {x^2 - y^2};
				\addplot3[mesh] {sin(deg(x)) * cos(deg(y))};
			\end{axis}
		\end{lstlisting}
	\end{tcolorbox}
	
	\section{Librerías de PGFPlots}
	
	\subsection{statistics - Estadísticas}
	
	\subsubsection*{\texttt{\textbackslash usepgfplotslibrary\{statistics\}}}
	\begin{tcolorbox}[colback=green!5,colframe=green!50!black]
		\textbf{Descripción:} Funcionalidades estadísticas (boxplot, histogram)
		
		\textbf{Ejemplo:}
		\begin{lstlisting}[language=TeX]
			\usepgfplotslibrary{statistics}
			\begin{axis}
				\addplot+[boxplot] table {data.dat};
			\end{axis}
		\end{lstlisting}
	\end{tcolorbox}
	
	\subsection{polar - Coordenadas Polares}
	
	\subsubsection*{\texttt{\textbackslash usepgfplotslibrary\{polar\}}}
	\begin{tcolorbox}[colback=green!5,colframe=green!50!black]
		\textbf{Descripción:} Gráficas en coordenadas polares
		
		\textbf{Ejemplo:}
		\begin{lstlisting}[language=TeX]
			\usepgfplotslibrary{polar}
			\begin{polaraxis}
				\addplot {sin(3*x)};
			\end{polaraxis}
		\end{lstlisting}
	\end{tcolorbox}
	
	\subsection{fillbetween - Relleno Entre Curvas}
	
	\subsubsection*{\texttt{\textbackslash usepgfplotslibrary\{fillbetween\}}}
	\begin{tcolorbox}[colback=green!5,colframe=green!50!black]
		\textbf{Descripción:} Rellena el área entre dos curvas
		
		\textbf{Ejemplo:}
		\begin{lstlisting}[language=TeX]
			\usepgfplotslibrary{fillbetween}
			\begin{axis}
				\addplot[name path=A] {x^2};
				\addplot[name path=B] {x};
				\addplot[fill=blue!20] fill between[of=A and B];
			\end{axis}
		\end{lstlisting}
	\end{tcolorbox}
	
	\subsection{groupplots - Múltiples Subgráficas}
	
	\subsubsection*{\texttt{\textbackslash usepgfplotslibrary\{groupplots\}}}
	\begin{tcolorbox}[colback=green!5,colframe=green!50!black]
		\textbf{Descripción:} Matriz de gráficas
		
		\textbf{Ejemplo:}
		\begin{lstlisting}[language=TeX]
			\usepgfplotslibrary{groupplots}
			\begin{tikzpicture}
				\begin{groupplot}[group style={group size=2 by 2}]
					\nextgroupplot
					\addplot {x};
					\nextgroupplot
					\addplot {x^2};
				\end{groupplot}
			\end{tikzpicture}
		\end{lstlisting}
	\end{tcolorbox}
	
	%\newpage
	
	\section*{\faCheckCircle\ Ejemplos Completos}
	
	\subsection*{Ejemplo 1: Gráfica Básica}
	
	\begin{tcolorbox}[colback=green!10,colframe=green!75!black,title=\faCode\ Múltiples funciones]
		\begin{lstlisting}[language=TeX]
			\begin{tikzpicture}
				\begin{axis}[
					width=12cm, height=8cm,
					xlabel=$x$, ylabel=$f(x)$,
					title=Funciones Matemáticas,
					legend pos=north west,
					grid=both,
					domain=-2:2,
					samples=100
					]
					\addplot[blue, thick] {x^2};
					\addlegendentry{$x^2$}
					
					\addplot[red, dashed, thick] {x^3};
					\addlegendentry{$x^3$}
					
					\addplot[green!70!black, dotted, ultra thick] {sin(deg(x))};
					\addlegendentry{$\sin(x)$}
				\end{axis}
			\end{tikzpicture}
		\end{lstlisting}
	\end{tcolorbox}
	
	\subsection*{Ejemplo 2: Gráfica de Barras}
	
	\begin{tcolorbox}[colback=purple!10,colframe=purple!75!black,title=\faCode\ Comparativa]
		\begin{lstlisting}[language=TeX]
			\begin{tikzpicture}
				\begin{axis}[
					ybar,
					bar width=15pt,
					ylabel=Ventas (miles),
					symbolic x coords={Ene, Feb, Mar, Abr},
					xtick=data,
					legend pos=north west,
					ymin=0
					]
					\addplot coordinates {(Ene,20) (Feb,25) (Mar,22) (Abr,30)};
					\addlegendentry{2023}
					
					\addplot coordinates {(Ene,25) (Feb,28) (Mar,27) (Abr,35)};
					\addlegendentry{2024}
				\end{axis}
			\end{tikzpicture}
		\end{lstlisting}
	\end{tcolorbox}
	
	\subsection*{Ejemplo 3: Superficie 3D}
	
	\begin{tcolorbox}[colback=cyan!10,colframe=cyan!75!black,title=\faCode\ Tridimensional]
		\begin{lstlisting}[language=TeX]
			\begin{tikzpicture}
				\begin{axis}[
					colormap/viridis,
					colorbar,
					xlabel=$x$,
					ylabel=$y$,
					zlabel=$z$,
					view={45}{30}
					]
					\addplot3[
					surf,
					samples=30,
					domain=-2:2,
					y domain=-2:2
					] {x^2 - y^2};
				\end{axis}
			\end{tikzpicture}
		\end{lstlisting}
	\end{tcolorbox}
	
	\section*{\faLightbulb\ Tips Importantes}
	
	\begin{tcolorbox}[colback=blue!10,colframe=blue!75!black]
		\begin{itemize}[leftmargin=*]
			\item Siempre especifica \texttt{compat=1.18} para usar características recientes
			\item Usa \texttt{samples=n} para controlar suavidad de curvas
			\item Para trigonométricas usa \texttt{deg(x)} para convertir radianes a grados
			\item \texttt{domain} define rango de x para funciones
			\item Usa \texttt{restrict y to domain} para evitar discontinuidades
			\item Para 3D, ajusta \texttt{view} para cambiar ángulo de visualización
			\item Usa \texttt{only marks} con \texttt{mark=*} para dispersión
			\item Para rendimiento, usa librería \texttt{external}
		\end{itemize}
	\end{tcolorbox}
	
	\section*{\faExclamationTriangle\ Errores Comunes}
	
	\begin{tcolorbox}[colback=red!10,colframe=red!75!black]
		\begin{itemize}[leftmargin=*]
			\item \textbf{Olvidar punto y coma}: \texttt{\textbackslash addplot} debe terminar con ;
			\item \textbf{Funciones trigonométricas}: Usar \texttt{sin(deg(x))} cuando x está en grados
			\item \textbf{Compilación lenta}: Demasiados samples en 3D
			\item \textbf{Leyenda incorrecta}: Número de \texttt{\textbackslash addlegendentry} debe coincidir
			\item \textbf{Discontinuidades}: No usar \texttt{restrict} causa saltos verticales
			\item \textbf{Memoria}: Gráficas 3D muy detalladas exceden memoria de TeX
		\end{itemize}
	\end{tcolorbox}
	
	\vspace{1cm}
	
	\begin{center}
		\textit{Documento generado con \LaTeX{} y PGFPlots -- \today}
		
		\textit{Para más información, consulta el manual oficial de PGFPlots.}
	\end{center}
	
\end{document}