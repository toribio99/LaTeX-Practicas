% !TEX encoding = UTF-8 Unicode
\documentclass[10pt,a4paper]{article}

% Paquetes necesarios
\usepackage[utf8]{inputenc}
\usepackage[spanish]{babel}
\usepackage[margin=1.5cm]{geometry}
\usepackage{xcolor}
\usepackage{pgfplots}
\usepackage{lipsum}
\pgfplotsset{compat=1.18}

% Librerías adicionales de PGFPlots
\usepgfplotslibrary{fillbetween}
\usepgfplotslibrary{groupplots}
\usepgfplotslibrary{polar}

% Título
\title{\textbf{\Huge Aplicaciones Prácticas de PGFPlots} \\[3mm] \large Ejemplos de Uso de Comandos y Opciones}
\author{}
\date{\today}

\usepackage[hidelinks]{hyperref}

\begin{document}

\maketitle
\tableofcontents
\newpage

\section{Configuración Básica del Paquete}

\subsection{Carga Básica con pgfplotsset}

Este ejemplo muestra la configuración básica del paquete PGFPlots con opciones globales.

\begin{verbatim}
\usepackage{pgfplots}
\pgfplotsset{compat=1.18}
\end{verbatim}

\lipsum[1][1-3]

\begin{center}
\begin{tikzpicture}
\begin{axis}
    \addplot[smooth] {x^2};
\end{axis}
\end{tikzpicture}
\end{center}

\subsection{Configuración Global con pgfplotsset}

Podemos establecer opciones globales que afectan a todos los gráficos del documento.

\begin{verbatim}
\pgfplotsset{
    width=10cm,
    height=6cm,
    grid=major
}
\end{verbatim}

\lipsum[1][1-2]

\section{Tamaño y Dimensiones}

\subsection{Opción width y height}

Control del tamaño del gráfico mediante las opciones \texttt{width} y \texttt{height}.

\lipsum[2][1-2]

\begin{center}
\begin{tikzpicture}
\begin{axis}[
    width=10cm,
    height=6cm,
    xlabel=$x$,
    ylabel=$f(x)$,
    title={Gráfico con width=10cm, height=6cm}
]
    \addplot[smooth] {x^2};
\end{axis}
\end{tikzpicture}
\end{center}

\subsection{Opción scale}

Escala todo el gráfico incluyendo etiquetas y títulos.

\lipsum[2][1-2]

\begin{center}
\begin{tikzpicture}
\begin{axis}[
    scale=1.2,
    xlabel=$x$,
    ylabel=$f(x)$,
    title={Gráfico con scale=1.2}
]
    \addplot[smooth] {sin(deg(x))};
\end{axis}
\end{tikzpicture}
\end{center}

\subsection{Opción scale only axis}

Escala solo el área del gráfico, manteniendo el tamaño de etiquetas y títulos.

\lipsum[2][1-2]

\begin{center}
\begin{tikzpicture}
\begin{axis}[
    width=10cm,
    height=6cm,
    scale only axis,
    xlabel=$x$,
    ylabel=$f(x)$,
    title={Gráfico con scale only axis}
]
    \addplot[smooth] {x};
\end{axis}
\end{tikzpicture}
\end{center}

\section{Límites de Ejes}

\subsection{Opciones xmin, xmax, ymin, ymax}

Define explícitamente los límites de los ejes.

\lipsum[3][1-2]

\begin{center}
\begin{tikzpicture}
\begin{axis}[
    xmin=0, xmax=10,
    ymin=-2, ymax=2,
    xlabel=$x$,
    ylabel=$\sin(x)$,
    title={Límites personalizados: xmin=0, xmax=10, ymin=-2, ymax=2}
]
    \addplot[blue, thick, smooth, samples=50] {sin(deg(x))};
\end{axis}
\end{tikzpicture}
\end{center}

\subsection{Opción domain}

Define el dominio para la evaluación de funciones.

\lipsum[3][1-2]

\begin{center}
\begin{tikzpicture}
\begin{axis}[
    xlabel=$x$,
    ylabel=$x^3$,
    title={Función con domain=-2:2}
]
    \addplot[red, thick, smooth, samples=50, domain=-2:2] {x^3};
\end{axis}
\end{tikzpicture}
\end{center}

\subsection{Opción restrict y to domain}

Útil para funciones con discontinuidades, restringe valores de y a un rango.

\lipsum[3][1-2]

\begin{center}
\begin{tikzpicture}
\begin{axis}[
    xlabel=$x$,
    ylabel=$1/x$,
    title={Función 1/x con restrict y to domain=-10:10},
    grid=major
]
    \addplot[green!70!black, thick, smooth, samples=50, restrict y to domain=-10:10] {1/x};
\end{axis}
\end{tikzpicture}
\end{center}

\section{Escalas de Ejes}

\subsection{Escala Logarítmica: xmode=log, ymode=log}

Cambia la escala de los ejes a logarítmica.

\lipsum[4][1-2]

\begin{center}
\begin{tikzpicture}
\begin{axis}[
    xmode=log,
    ymode=log,
    xlabel=$x$,
    ylabel=$x^2$,
    title={Gráfica logarítmica: xmode=log, ymode=log},
    grid=major
]
    \addplot[blue, thick, smooth, samples=50, domain=1:100] {x^2};
\end{axis}
\end{tikzpicture}
\end{center}

\subsection{Base del Logaritmo: log basis y}

Permite cambiar la base del logaritmo (por defecto es 10).

\lipsum[4][1-2]

\begin{center}
\begin{tikzpicture}
\begin{axis}[
    ymode=log,
    log basis y=2,
    xlabel=$x$,
    ylabel=$2^x$,
    title={Escala logarítmica base 2},
    grid=major
]
    \addplot[red, thick, smooth, samples=50, domain=0:8] {2^x};
\end{axis}
\end{tikzpicture}
\end{center}

\subsection{Opción axis equal}

Mantiene proporción 1:1 entre los ejes (útil para círculos).

\lipsum[4][1-2]

\begin{center}
\begin{tikzpicture}
\begin{axis}[
    axis equal,
    xlabel=$x$,
    ylabel=$y$,
    title={Círculo con axis equal},
    grid=major,
    domain=-1:1
]
    \addplot[blue, thick, smooth, samples=50] {sqrt(1-x^2)};
    \addplot[blue, thick, smooth, samples=50] {-sqrt(1-x^2)};
\end{axis}
\end{tikzpicture}
\end{center}

\section{Grillas}

\subsection{Opción grid=major}

Muestra una grilla principal en el gráfico.

\lipsum[5][1-2]

\begin{center}
\begin{tikzpicture}
\begin{axis}[
    grid=major,
    xlabel=$x$,
    ylabel=$x^2$,
    title={Grilla principal: grid=major}
]
    \addplot[purple, thick, smooth, samples=50] {x^2};
\end{axis}
\end{tikzpicture}
\end{center}

\subsection{Opción grid=both}

Muestra grilla principal y secundaria.

\lipsum[5][1-2]

\begin{center}
\begin{tikzpicture}
\begin{axis}[
    grid=both,
    xlabel=$x$,
    ylabel=$\sin(x)$,
    title={Grilla principal y secundaria: grid=both}
]
    \addplot[cyan!70!blue, thick, smooth, samples=50] {sin(deg(x))};
\end{axis}
\end{tikzpicture}
\end{center}

\subsection{Personalización de Estilo de Grilla}

Podemos personalizar el estilo de las grillas mayor y menor.

\lipsum[5][1-2]

\begin{center}
\begin{tikzpicture}
\begin{axis}[
    grid=both,
    major grid style={thick, red!50},
    minor grid style={thin, gray!30},
    xlabel=$x$,
    ylabel=$\cos(x)$,
    title={Grillas personalizadas}
]
    \addplot[orange, thick, smooth, samples=50] {cos(deg(x))};
\end{axis}
\end{tikzpicture}
\end{center}

\section{Etiquetas y Títulos}

\subsection{Opciones xlabel y ylabel}

Define las etiquetas de los ejes.

\lipsum[6][1-2]

\begin{center}
\begin{tikzpicture}
\begin{axis}[
    xlabel=Tiempo (segundos),
    ylabel=Distancia (metros),
    title={Movimiento Rectilíneo Uniforme}
]
    \addplot[blue, thick, smooth, samples=50] {5*x};
\end{axis}
\end{tikzpicture}
\end{center}

\subsection{Estilo de Etiquetas: xlabel style}

Personaliza el estilo de las etiquetas.

\lipsum[6][1-2]

\begin{center}
\begin{tikzpicture}
\begin{axis}[
    xlabel=$x$,
    ylabel=$f(x)$,
    xlabel style={font=\Large, color=red},
    ylabel style={font=\Large, color=blue},
    title={Etiquetas personalizadas}
]
    \addplot[green!70!black, thick, smooth, samples=50] {x^2};
\end{axis}
\end{tikzpicture}
\end{center}

\subsection{Estilo del Título: title style}

Personaliza la apariencia del título.

\lipsum[6][1-2]

\begin{center}
\begin{tikzpicture}
\begin{axis}[
    xlabel=$x$,
    ylabel=$x^3$,
    title=Función Cúbica,
    title style={font=\bfseries\Large, color=purple}
]
    \addplot[magenta, thick, smooth, samples=50] {x^3};
\end{axis}
\end{tikzpicture}
\end{center}

\section{Ticks (Marcas)}

\subsection{Posiciones Específicas: xtick y ytick}

Define manualmente las posiciones de los ticks.

\lipsum[7][1-2]

\begin{center}
\begin{tikzpicture}
\begin{axis}[
    xtick={0,1,2,3,4,5},
    ytick={0,5,10,15,20,25},
    xlabel=$x$,
    ylabel=$x^2$,
    title={Ticks personalizados}
]
    \addplot[brown, thick, smooth, samples=50] {x^2};
\end{axis}
\end{tikzpicture}
\end{center}

\subsection{Etiquetas Personalizadas: xticklabels}

Asigna etiquetas personalizadas a los ticks.

\lipsum[7][1-2]

\begin{center}
\begin{tikzpicture}
\begin{axis}[
    xtick={1,2,3,4},
    xticklabels={Ene, Feb, Mar, Abr},
    ylabel=Ventas,
    title={Ventas Mensuales},
    ymin=0
]
    \addplot coordinates {(1,10) (2,15) (3,12) (4,18)};
\end{axis}
\end{tikzpicture}
\end{center}

\subsection{Distancia entre Ticks: xtick distance}

Establece la distancia entre ticks consecutivos.

\lipsum[7][1-2]

\begin{center}
\begin{tikzpicture}
\begin{axis}[
    xtick distance=2,
    ytick distance=5,
    xlabel=$x$,
    ylabel=$2x+1$,
    title={xtick distance=2, ytick distance=5}
]
    \addplot[violet, thick, smooth, samples=50] {2*x + 1};
\end{axis}
\end{tikzpicture}
\end{center}

\subsection{Estilo de Etiquetas de Ticks}

Personaliza la apariencia de las etiquetas de ticks (ej. rotación).

\lipsum[7][1-2]

\begin{center}
\begin{tikzpicture}
\begin{axis}[
    xtick={1,2,3,4,5},
    xticklabels={Categoría A, Categoría B, Categoría C, Categoría D, Categoría E},
    xticklabel style={rotate=45, anchor=east, font=\small},
    ylabel=Valor,
    title={Etiquetas rotadas 45 grados},
    ymin=0
]
    \addplot coordinates {(1,5) (2,8) (3,6) (4,9) (5,7)};
\end{axis}
\end{tikzpicture}
\end{center}

\section{Posición de Ejes}

\subsection{Opción axis lines=center}

Coloca los ejes en el centro del gráfico (estilo cartesiano).

\lipsum[8][1-2]

\begin{center}
\begin{tikzpicture}
\begin{axis}[
    axis lines=center,
    xlabel=$x$,
    ylabel=$y$,
    title={axis lines=center}
]
    \addplot[blue, thick, smooth, samples=50] {sin(deg(x))};
\end{axis}
\end{tikzpicture}
\end{center}

\subsection{Opción axis lines=left}

Los ejes en el borde izquierdo e inferior (estilo estándar).

\lipsum[8][1-2]

\begin{center}
\begin{tikzpicture}
\begin{axis}[
    axis lines=left,
    xlabel=$x$,
    ylabel=$y$,
    title={axis lines=left}
]
    \addplot[red, thick, smooth, samples=50] {x^2};
\end{axis}
\end{tikzpicture}
\end{center}

\subsection{Opción axis lines=box}

Dibuja un recuadro completo alrededor del gráfico.

\lipsum[8][1-2]

\begin{center}
\begin{tikzpicture}
\begin{axis}[
    axis lines=box,
    xlabel=$x$,
    ylabel=$y$,
    title={axis lines=box},
    grid=major
]
    \addplot[green!70!black, thick, smooth, samples=50] {x^3};
\end{axis}
\end{tikzpicture}
\end{center}

\section{Leyendas}

\subsection{Posición de Leyenda: legend pos}

Define la posición de la leyenda en el gráfico.

\lipsum[9][1-2]

\begin{center}
\begin{tikzpicture}
\begin{axis}[
    legend pos=north west,
    xlabel=$x$,
    ylabel=$f(x)$,
    title={Leyenda en north west}
]
    \addplot[blue, thick, smooth, samples=50] {x};
    \addlegendentry{$f(x)=x$}

    \addplot[red, thick, smooth, samples=50] {x^2};
    \addlegendentry{$f(x)=x^2$}

    \addplot[green!70!black, thick, smooth, samples=50] {x^3};
    \addlegendentry{$f(x)=x^3$}
\end{axis}
\end{tikzpicture}
\end{center}

\subsection{Leyenda en Posición sur este: south east}

\lipsum[9][1-2]

\begin{center}
\begin{tikzpicture}
\begin{axis}[
    legend pos=south east,
    xlabel=$x$,
    ylabel=$f(x)$,
    title={Leyenda en south east}
]
    \addplot[cyan, thick, smooth, samples=50] {sin(deg(x))};
    \addlegendentry{$\sin(x)$}

    \addplot[magenta, thick, smooth, samples=50] {cos(deg(x))};
    \addlegendentry{$\cos(x)$}
\end{axis}
\end{tikzpicture}
\end{center}

\subsection{Estilo Personalizado de Leyenda}

Personaliza la posición y estilo de la leyenda.

\lipsum[9][1-2]

\begin{center}
\begin{tikzpicture}
\begin{axis}[
    legend style={at={(0.5,-0.15)}, anchor=north, legend columns=3},
    xlabel=$x$,
    ylabel=$f(x)$,
    title={Leyenda horizontal debajo del gráfico}
]
    \addplot[blue, thick, smooth, samples=50] {x};
    \addlegendentry{Lineal}

    \addplot[red, thick, smooth, samples=50] {x^2};
    \addlegendentry{Cuadrática}

    \addplot[green!70!black, thick, smooth, samples=50] {x^3};
    \addlegendentry{Cúbica}
\end{axis}
\end{tikzpicture}
\end{center}

\section{Tipos de Plot}

\subsection{Expresiones Matemáticas}

Grafica funciones usando expresiones matemáticas directas.

\lipsum[10][1-2]

\begin{center}
\begin{tikzpicture}
\begin{axis}[
    xlabel=$x$,
    ylabel=$f(x)$,
    title={Expresiones matemáticas},
    legend pos=north west
]
    \addplot[blue, thick, smooth, samples=50] {x^2 - 2*x + 1};
    \addlegendentry{$x^2-2x+1$}

    \addplot[red, thick, smooth, samples=50] {exp(x/5)};
    \addlegendentry{$e^{x/5}$}
\end{axis}
\end{tikzpicture}
\end{center}

\subsection{Coordenadas Específicas}

Grafica puntos específicos usando \texttt{coordinates}.

\lipsum[10][1-2]

\begin{center}
\begin{tikzpicture}
\begin{axis}[
    xlabel=Mes,
    ylabel=Temperatura (°C),
    title={Temperatura promedio mensual}
]
    \addplot[blue, thick, mark=*] coordinates {
        (1,15) (2,17) (3,20) (4,23) (5,26) (6,29)
    };
\end{axis}
\end{tikzpicture}
\end{center}

\subsection{Datos Tabulares Inline}

Usa formato de tabla inline para datos estructurados.

\lipsum[10][1-2]

\begin{center}
\begin{tikzpicture}
\begin{axis}[
    xlabel=x,
    ylabel=y,
    title={Datos en formato tabla}
]
    \addplot[red, thick, mark=square*] table {
        x  y
        0  0
        1  1
        2  4
        3  9
        4  16
        5  25
    };
\end{axis}
\end{tikzpicture}
\end{center}

\section{Estilos de Líneas y Marcadores}

\subsection{Colores}

Diferentes colores para las líneas.

\lipsum[11][1-2]

\begin{center}
\begin{tikzpicture}
\begin{axis}[
    xlabel=$x$,
    ylabel=$f(x)$,
    title={Gráficas con diferentes colores},
    legend pos=north west
]
    \addplot[red, thick, smooth, samples=50] {x};
    \addlegendentry{Rojo}

    \addplot[blue, thick, smooth, samples=50] {x+1};
    \addlegendentry{Azul}

    \addplot[green!70!black, thick, smooth, samples=50] {x+2};
    \addlegendentry{Verde}

    \addplot[purple, thick, smooth, samples=50] {x+3};
    \addlegendentry{Púrpura}
\end{axis}
\end{tikzpicture}
\end{center}

\subsection{Tipos de Marcadores}

Diferentes estilos de marcadores en los puntos.

\lipsum[11][1-2]

\begin{center}
\begin{tikzpicture}
\begin{axis}[
    xlabel=$x$,
    ylabel=$y$,
    title={Diferentes tipos de marcadores},
    legend pos=north west
]
    \addplot[mark=*, blue] coordinates {(0,0) (1,1) (2,2) (3,3)};
    \addlegendentry{mark=*}

    \addplot[mark=square*, red] coordinates {(0,0.5) (1,1.5) (2,2.5) (3,3.5)};
    \addlegendentry{mark=square*}

    \addplot[mark=triangle*, green!70!black] coordinates {(0,1) (1,2) (2,3) (3,4)};
    \addlegendentry{mark=triangle*}

    \addplot[mark=diamond*, purple] coordinates {(0,1.5) (1,2.5) (2,3.5) (3,4.5)};
    \addlegendentry{mark=diamond*}
\end{axis}
\end{tikzpicture}
\end{center}

\subsection{Opción only marks}

Muestra solo los marcadores sin líneas conectoras.

\lipsum[11][1-2]

\begin{center}
\begin{tikzpicture}
\begin{axis}[
    xlabel=$x$,
    ylabel=$y$,
    title={Gráfico de dispersión con only marks}
]
    \addplot[only marks, mark=o, blue] coordinates {
        (1,2) (2,4) (3,3) (4,5) (5,6) (6,5) (7,7)
    };
\end{axis}
\end{tikzpicture}
\end{center}

\subsection{Estilos de Línea}

Diferentes estilos: sólida, punteada, discontinua.

\lipsum[11][1-2]

\begin{center}
\begin{tikzpicture}
\begin{axis}[
    xlabel=$x$,
    ylabel=$f(x)$,
    title={Diferentes estilos de línea},
    legend pos=north west,
    domain=0:5
]
    \addplot[blue, thick, smooth, samples=50] {x};
    \addlegendentry{Sólida}

    \addplot[red, thick, dashed, smooth, samples=50] {x+1};
    \addlegendentry{Dashed}

    \addplot[green!70!black, thick, dotted, smooth, samples=50] {x+2};
    \addlegendentry{Dotted}

    \addplot[purple, thick, dashdotted, smooth, samples=50] {x+3};
    \addlegendentry{Dashdotted}
\end{axis}
\end{tikzpicture}
\end{center}

\subsection{Grosor de Línea: line width}

Control del grosor de las líneas.

\lipsum[11][1-2]

\begin{center}
\begin{tikzpicture}
\begin{axis}[
    xlabel=$x$,
    ylabel=$f(x)$,
    title={Diferentes grosores de línea},
    legend pos=north west
]
    \addplot[blue, line width=0.5pt, smooth, samples=50] {x};
    \addlegendentry{0.5pt}

    \addplot[red, line width=1pt, smooth, samples=50] {x+1};
    \addlegendentry{1pt}

    \addplot[green!70!black, line width=2pt, smooth, samples=50] {x+2};
    \addlegendentry{2pt}

    \addplot[purple, ultra thick, smooth, samples=50] {x+3};
    \addlegendentry{Ultra thick}
\end{axis}
\end{tikzpicture}
\end{center}

\subsection{Opción smooth}

Suaviza las curvas entre puntos.

\lipsum[11][1-2]

\begin{center}
\begin{tikzpicture}
\begin{axis}[
    xlabel=$x$,
    ylabel=$y$,
    title={Curva suavizada con smooth},
    legend pos=north west
]
    \addplot[blue, thick] coordinates {(0,0) (1,3) (2,1) (3,4) (4,2) (5,5)};
    \addlegendentry{Sin smooth}

    \addplot[red, thick, smooth] coordinates {(0,0.5) (1,3.5) (2,1.5) (3,4.5) (4,2.5) (5,5.5)};
    \addlegendentry{Con smooth}
\end{axis}
\end{tikzpicture}
\end{center}

\subsection{Opción samples}

Controla el número de puntos evaluados para funciones.

\lipsum[11][1-2]

\begin{center}
\begin{tikzpicture}
\begin{axis}[
    xlabel=$x$,
    ylabel=$\sin(x)$,
    title={Comparación: samples=10 vs samples=100},
    legend pos=north east,
    domain=0:10
]
    \addplot[red, thick, samples=10] {sin(deg(x))};
    \addlegendentry{samples=10}

    \addplot[blue, thick, samples=100] {sin(deg(x))};
    \addlegendentry{samples=100}
\end{axis}
\end{tikzpicture}
\end{center}

\subsection{Relleno: fill y closedcycle}

Rellena el área bajo la curva.

\lipsum[11][1-2]

\begin{center}
\begin{tikzpicture}
\begin{axis}[
    xlabel=$x$,
    ylabel=$f(x)$,
    title={Área bajo la curva con fill},
    domain=0:3
]
    \addplot[fill=blue!20, draw=blue, thick, smooth, samples=50] {x^2} \closedcycle;
\end{axis}
\end{tikzpicture}
\end{center}

\section{Gráficas de Barras}

\subsection{Barras Verticales: ybar}

Gráfico de barras verticales.

\lipsum[12][1-2]

\begin{center}
\begin{tikzpicture}
\begin{axis}[
    ybar,
    xlabel=Mes,
    ylabel=Ventas (miles),
    title={Ventas mensuales (ybar)},
    xtick=data,
    symbolic x coords={Ene, Feb, Mar, Abr, May, Jun},
    ymin=0
]
    \addplot[fill=blue!50] coordinates {
        (Ene,10) (Feb,15) (Mar,12) (Abr,18) (May,14) (Jun,20)
    };
\end{axis}
\end{tikzpicture}
\end{center}

\subsection{Barras Horizontales: xbar}

Gráfico de barras horizontales.

\lipsum[12][1-2]

\begin{center}
\begin{tikzpicture}
\begin{axis}[
    xbar,
    ylabel=Producto,
    xlabel=Cantidad vendida,
    title={Productos más vendidos (xbar)},
    ytick=data,
    symbolic y coords={Producto A, Producto B, Producto C, Producto D},
    xmin=0
]
    \addplot[fill=red!50] coordinates {
        (25,Producto A) (40,Producto B) (15,Producto C) (35,Producto D)
    };
\end{axis}
\end{tikzpicture}
\end{center}

\subsection{Ancho de Barras: bar width}

Control del ancho de las barras.

\lipsum[12][1-2]

\begin{center}
\begin{tikzpicture}
\begin{axis}[
    ybar,
    bar width=20pt,
    xlabel=Categoría,
    ylabel=Valor,
    title={Barras anchas con bar width=20pt},
    xtick=data,
    symbolic x coords={A, B, C, D},
    ymin=0
]
    \addplot[fill=green!50] coordinates {
        (A,8) (B,12) (C,10) (D,15)
    };
\end{axis}
\end{tikzpicture}
\end{center}

\subsection{Barras Múltiples Comparativas}

Comparación de múltiples conjuntos de datos con barras.

\lipsum[12][1-2]

\begin{center}
\begin{tikzpicture}
\begin{axis}[
    ybar,
    bar width=10pt,
    xlabel=Trimestre,
    ylabel=Ingresos (millones),
    title={Comparación anual de ingresos},
    legend pos=north west,
    xtick=data,
    symbolic x coords={Q1, Q2, Q3, Q4},
    ymin=0
]
    \addplot[fill=blue!50] coordinates {
        (Q1,20) (Q2,25) (Q3,22) (Q4,30)
    };
    \addlegendentry{2023}

    \addplot[fill=red!50] coordinates {
        (Q1,25) (Q2,28) (Q3,27) (Q4,35)
    };
    \addlegendentry{2024}
\end{axis}
\end{tikzpicture}
\end{center}

\section{Gráficas 3D}

\subsection{Superficie 3D: surf}

Gráfica de superficie tridimensional.

\lipsum[13][1-2]

\begin{center}
\begin{tikzpicture}
\begin{axis}[
    xlabel=$x$,
    ylabel=$y$,
    zlabel=$z$,
    title={Superficie 3D: $z = x^2 - y^2$},
    view={45}{30}
]
    \addplot3[
        surf,
        samples=20,
        domain=-2:2,
        y domain=-2:2
    ] {x^2 - y^2};
\end{axis}
\end{tikzpicture}
\end{center}

\subsection{Malla 3D: mesh}

Gráfica de malla tridimensional.

\lipsum[13][1-2]

\begin{center}
\begin{tikzpicture}
\begin{axis}[
    xlabel=$x$,
    ylabel=$y$,
    zlabel=$z$,
    title={Malla 3D: $z = \sin(x) \cdot \cos(y)$},
    view={60}{30}
]
    \addplot3[
        mesh,
        samples=15,
        domain=-pi:pi,
        y domain=-pi:pi
    ] {sin(deg(x)) * cos(deg(y))};
\end{axis}
\end{tikzpicture}
\end{center}

\subsection{Ángulo de Visión: view}

Control del ángulo de visualización con diferentes valores de azimut y elevación.

\lipsum[13][1-2]

\begin{center}
\begin{tikzpicture}
\begin{axis}[
    xlabel=$x$,
    ylabel=$y$,
    zlabel=$z$,
    title={view=\{120\}\{45\}},
    view={120}{45}
]
    \addplot3[
        surf,
        samples=15,
        domain=-1:1,
        y domain=-1:1
    ] {x^2 + y^2};
\end{axis}
\end{tikzpicture}
\end{center}

\subsection{Colormap en Superficies 3D}

Las superficies pueden usar colormaps para representar valores.

\lipsum[13][1-2]

\begin{center}
\begin{tikzpicture}
\begin{axis}[
    xlabel=$x$,
    ylabel=$y$,
    zlabel=$z$,
    title={Superficie con colormap},
    colorbar,
    colormap/cool,
    view={40}{35}
]
    \addplot3[
        surf,
        samples=20,
        domain=-2:2,
        y domain=-2:2
    ] {exp(-x^2-y^2)};
\end{axis}
\end{tikzpicture}
\end{center}

\section{Librería fillbetween}

\subsection{Relleno entre Dos Curvas}

La librería \texttt{fillbetween} permite rellenar el área entre dos funciones.

\lipsum[14][1-2]

\begin{center}
\begin{tikzpicture}
\begin{axis}[
    xlabel=$x$,
    ylabel=$y$,
    title={Área entre dos funciones},
    legend pos=north west,
    domain=0:3
]
    \addplot[name path=A, blue, thick, smooth, samples=50] {x^2};
    \addlegendentry{$x^2$}

    \addplot[name path=B, red, thick, smooth, samples=50] {2*x};
    \addlegendentry{$2x$}

    \addplot[fill=green!20, opacity=0.5] fill between[of=A and B];
\end{axis}
\end{tikzpicture}
\end{center}

\subsection{Relleno entre Curva y Eje}

También podemos rellenar entre una curva y el eje horizontal.

\lipsum[14][1-2]

\begin{center}
\begin{tikzpicture}
\begin{axis}[
    xlabel=$x$,
    ylabel=$\sin(x)$,
    title={Área bajo la curva seno},
    domain=0:2*pi,
    axis lines=center
]
    \addplot[name path=seno, blue, thick, smooth, samples=100] {sin(deg(x))};

    \path[name path=eje] (axis cs:0,0) -- (axis cs:6.28,0);

    \addplot[fill=blue!20] fill between[of=seno and eje, soft clip={domain=0:3.14}];
\end{axis}
\end{tikzpicture}
\end{center}

\section{Librería groupplots}

\subsection{Múltiples Gráficas en Matriz}

La librería \texttt{groupplots} permite crear matrices de gráficas.

\lipsum[15][1-2]

\begin{center}
\begin{tikzpicture}
\begin{groupplot}[
    group style={
        group size=2 by 2,
        horizontal sep=2cm,
        vertical sep=2cm
    },
    width=6cm,
    height=5cm
]
    \nextgroupplot[title={$f(x)=x$}, xlabel=$x$, ylabel=$y$]
    \addplot[blue, thick, smooth, samples=50] {x};

    \nextgroupplot[title={$f(x)=x^2$}, xlabel=$x$, ylabel=$y$]
    \addplot[red, thick, smooth, samples=50] {x^2};

    \nextgroupplot[title={$f(x)=\sin(x)$}, xlabel=$x$, ylabel=$y$]
    \addplot[green!70!black, thick, smooth, samples=50] {sin(deg(x))};

    \nextgroupplot[title={$f(x)=\cos(x)$}, xlabel=$x$, ylabel=$y$]
    \addplot[purple, thick, smooth, samples=50] {cos(deg(x))};
\end{groupplot}
\end{tikzpicture}
\end{center}

\subsection{Matriz de Gráficas con Ejes Compartidos}

Podemos compartir ejes entre gráficas adyacentes.

\lipsum[15][1-2]

\begin{center}
\begin{tikzpicture}
\begin{groupplot}[
    group style={
        group size=1 by 3,
        vertical sep=1.5cm,
        xlabels at=edge bottom
    },
    width=10cm,
    height=4cm,
    xlabel=$x$
]
    \nextgroupplot[title={Función Lineal}, ylabel=$x$]
    \addplot[blue, thick, smooth, samples=50] {x};

    \nextgroupplot[title={Función Cuadrática}, ylabel=$x^2$]
    \addplot[red, thick, smooth, samples=50] {x^2};

    \nextgroupplot[title={Función Cúbica}, ylabel=$x^3$]
    \addplot[green!70!black, thick, smooth, samples=50] {x^3};
\end{groupplot}
\end{tikzpicture}
\end{center}

\section{Librería polar}

\subsection{Gráficas en Coordenadas Polares}

La librería \texttt{polar} permite crear gráficas en coordenadas polares.

\lipsum[16][1-2]

\begin{center}
\begin{tikzpicture}
\begin{polaraxis}[
    title={Rosa polar: $r=\sin(3\theta)$}
]
    \addplot[blue, thick, domain=0:360, samples=100] {sin(3*x)};
\end{polaraxis}
\end{tikzpicture}
\end{center}

\subsection{Espiral en Coordenadas Polares}

Ejemplo de espiral de Arquímedes.

\lipsum[16][1-2]

\begin{center}
\begin{tikzpicture}
\begin{polaraxis}[
    title={Espiral de Arquímedes: $r=\theta$}
]
    \addplot[red, thick, domain=0:720, samples=200] {x/100};
\end{polaraxis}
\end{tikzpicture}
\end{center}

\subsection{Cardioide en Coordenadas Polares}

Forma de corazón en coordenadas polares.

\lipsum[16][1-2]

\begin{center}
\begin{tikzpicture}
\begin{polaraxis}[
    title={Cardioide: $r=1+\cos(\theta)$}
]
    \addplot[purple, thick, domain=0:360, samples=100] {1 + cos(x)};
\end{polaraxis}
\end{tikzpicture}
\end{center}

\section{Ejemplos Avanzados}

\subsection{Gráfica Científica Completa}

Ejemplo de gráfica científica con todas las características.

\lipsum[17][1-2]

\begin{center}
\begin{tikzpicture}
\begin{axis}[
    width=12cm,
    height=8cm,
    xlabel={Tiempo (s)},
    ylabel={Voltaje (V)},
    title={Señal de Voltaje vs Tiempo},
    legend pos=north east,
    grid=both,
    major grid style={gray!30},
    minor grid style={gray!10},
    domain=0:10,
    samples=200,
    axis lines=box
]
    \addplot[blue, thick, smooth, samples=200] {3*sin(deg(x)) + 0.5*sin(deg(3*x))};
    \addlegendentry{Señal 1}

    \addplot[red, thick, dashed, smooth, samples=200] {2*cos(deg(x))};
    \addlegendentry{Señal 2}

    \addplot[green!70!black, thick, dotted, smooth, samples=200] {exp(-x/5)*sin(deg(2*x))};
    \addlegendentry{Señal amortiguada}
\end{axis}
\end{tikzpicture}
\end{center}

\subsection{Gráfica Estadística con Error Bars}

Representación de datos experimentales con barras de error.

\lipsum[17][1-2]

\begin{center}
\begin{tikzpicture}
\begin{axis}[
    xlabel={Concentración (mg/L)},
    ylabel={Absorción},
    title={Curva de Calibración},
    legend pos=north west,
    grid=major
]
    \addplot[
        only marks,
        mark=*,
        mark size=3pt,
        error bars/.cd,
        y dir=both,
        y explicit
    ] coordinates {
        (0,0) +- (0,0.02)
        (1,0.15) +- (0,0.03)
        (2,0.28) +- (0,0.04)
        (3,0.45) +- (0,0.05)
        (4,0.58) +- (0,0.06)
        (5,0.72) +- (0,0.07)
    };
    \addlegendentry{Datos experimentales}

    \addplot[red, thick, smooth, samples=50, domain=0:5] {0.14*x + 0.01};
    \addlegendentry{Ajuste lineal}
\end{axis}
\end{tikzpicture}
\end{center}

\subsection{Gráfica Paramétrica}

Curva de Lissajous en forma paramétrica.

\lipsum[17][1-2]

\begin{center}
\begin{tikzpicture}
\begin{axis}[
    xlabel=$x$,
    ylabel=$y$,
    title={Curva de Lissajous: $x=\sin(3t)$, $y=\sin(2t)$},
    axis equal,
    axis lines=center,
    ticks=none
]
    \addplot[blue, thick, smooth, samples=200, domain=0:2*pi] ({sin(deg(3*x))}, {sin(deg(2*x))});
\end{axis}
\end{tikzpicture}
\end{center}

\section{Resumen y Mejores Prácticas}

\subsection{Consejos Importantes}

\begin{itemize}
    \item Siempre especifica \texttt{compat=1.18} para acceder a las características más recientes
    \item Usa \texttt{samples=n} apropiado: valores bajos (50) para funciones simples, valores altos (200) para funciones complejas
    \item Para funciones trigonométricas, usa \texttt{deg(x)} si $x$ está en grados
    \item Usa \texttt{restrict y to domain} para evitar discontinuidades en funciones como $1/x$
    \item Para gráficas 3D, ajusta el número de samples para balance entre calidad y rendimiento
    \item Las leyendas pueden posicionarse con \texttt{legend pos} o manualmente con \texttt{legend style}
    \item Usa \texttt{grid=both} para mayor precisión en la lectura de valores
    \item El comando \texttt{domain} define el rango de evaluación para funciones
\end{itemize}

\subsection{Errores Comunes a Evitar}

\begin{itemize}
    \item \textbf{Olvidar el punto y coma}: Todo comando \texttt{\textbackslash addplot} debe terminar con ;
    \item \textbf{Funciones trigonométricas}: Si no usas \texttt{deg(x)}, PGFPlots interpreta en radianes
    \item \textbf{Demasiados samples en 3D}: Puede causar compilación muy lenta
    \item \textbf{No usar restrict}: Funciones con discontinuidades pueden mostrar líneas verticales incorrectas
    \item \textbf{Leyendas inconsistentes}: El número de \texttt{\textbackslash addlegendentry} debe coincidir con el número de plots
\end{itemize}

\vspace{2cm}

\begin{center}
\large
\textit{Documento generado con \LaTeX{} y PGFPlots}

\textit{\today}
\end{center}

\end{document}
