% !TEX encoding = UTF-8 Unicode
\documentclass[11pt,a4paper]{article}

% Paquetes necesarios
\usepackage[utf8]{inputenc}
\usepackage[spanish]{babel}
\usepackage[margin=2cm]{geometry}
\usepackage{xcolor}
\usepackage{tcolorbox}
\usepackage{enumitem}
\usepackage{fontawesome5}
\usepackage{listings}
\usepackage{graphicx}
\usepackage{wrapfig}

% Colores personalizados
\definecolor{categorycolor}{RGB}{41,128,185}
\definecolor{commandcolor}{RGB}{39,174,96}
\definecolor{examplecolor}{RGB}{149,165,166}

% Configuración de listings
\lstset{
	basicstyle=\ttfamily\small,
	breaklines=true,
	columns=fullflexible,
	keepspaces=true
}

% Título
\title{\textbf{\Huge Entorno \texttt{figure} en \LaTeX{}}\\\large Guía Completa de Comandos y Opciones}
\author{}
\date{\today}
 \usepackage[
%colorlinks=true,        % Enlaces con color (en lugar de cajas)
linkcolor=blue,         % Color de enlaces internos
urlcolor=cyan,          % Color de URLs
citecolor=green,        % Color de citas bibliográficas
filecolor=magenta,      % Color de enlaces a archivos
pdfborder={0 0 0},      % Sin bordes en los enlaces
bookmarks=true,         % Crear marcadores en el PDF
bookmarksopen=true,     % Marcadores expandidos al abrir
pdftitle={Mi Título},   % Título del PDF
pdfauthor={Mi Nombre},  % Autor del PDF
pdfsubject={Tema},      % Tema del documento
pdfkeywords={palabra1, palabra2}, % Palabras clave
%hidelinks,              % Ocultar todos los bordes/colores de enlaces
unicode=true,           % Permitir caracteres Unicode en marcadores
breaklinks=true         % Permitir saltos de línea en enlaces
]{hyperref}

\begin{document}
	
	\maketitle
	
	\begin{tcolorbox}[colback=blue!5,colframe=blue!75!black,title=\faInfoCircle\ Introducción]
		El entorno \texttt{figure} es un contenedor flotante que permite insertar imágenes en documentos \LaTeX{} con numeración automática, títulos y referencias cruzadas.
	\end{tcolorbox}
	
	\tableofcontents
	
	\section*{\faImage\ Entorno figure}
	
	\subsection*{Sintaxis básica}
	\begin{lstlisting}[language=TeX]
		\begin{figure}[opciones]
			... contenido ...
		\end{figure}
	\end{lstlisting}
	
	\section{Opciones de Posición}
	
	Controlan dónde \LaTeX{} coloca la figura en el documento.
	
	\subsection*{\texttt{\textcolor{commandcolor}{[h]}} -- Here}
	\begin{tcolorbox}[colback=green!5,colframe=green!50!black]
		\textbf{Descripción:} Intenta colocar la figura aquí (where defined)
		
		\textbf{Ejemplo:}
		\begin{lstlisting}[language=TeX]
			\begin{figure}[h]
				\centering
				\includegraphics{imagen.png}
			\end{figure}
		\end{lstlisting}
	\end{tcolorbox}
	
	\subsection*{\texttt{\textcolor{commandcolor}{[t]}} -- Top}
	\begin{tcolorbox}[colback=green!5,colframe=green!50!black]
		\textbf{Descripción:} Parte superior de la página
		
		\textbf{Ejemplo:}
		\begin{lstlisting}[language=TeX]
			\begin{figure}[t]
				\centering
				\includegraphics{imagen.png}
			\end{figure}
		\end{lstlisting}
	\end{tcolorbox}
	
	\subsection*{\texttt{\textcolor{commandcolor}{[b]}} -- Bottom}
	\begin{tcolorbox}[colback=green!5,colframe=green!50!black]
		\textbf{Descripción:} Parte inferior de la página
		
		\textbf{Ejemplo:}
		\begin{lstlisting}[language=TeX]
			\begin{figure}[b]
				\centering
				\includegraphics{imagen.png}
			\end{figure}
		\end{lstlisting}
	\end{tcolorbox}
	
	\subsection*{\texttt{\textcolor{commandcolor}{[p]}} -- Page}
	\begin{tcolorbox}[colback=green!5,colframe=green!50!black]
		\textbf{Descripción:} Página solo para flotantes
		
		\textbf{Ejemplo:}
		\begin{lstlisting}[language=TeX]
			\begin{figure}[p]
				\centering
				\includegraphics{imagen.png}
			\end{figure}
		\end{lstlisting}
	\end{tcolorbox}
	
	\subsection*{\texttt{\textcolor{commandcolor}{[H]}} -- HERE (forzado)}
	\begin{tcolorbox}[colback=green!5,colframe=green!50!black]
		\textbf{Descripción:} Fuerza posición exacta (requiere paquete \texttt{float})
		
		\textbf{Ejemplo:}
		\begin{lstlisting}[language=TeX]
			\usepackage{float}
			\begin{figure}[H]
				\centering
				\includegraphics{imagen.png}
			\end{figure}
		\end{lstlisting}
		
		\tcblower
		\faLightbulb\ \textbf{Nota:} Necesita \texttt{\textbackslash usepackage\{float\}}
	\end{tcolorbox}
	
	\subsection*{\texttt{\textcolor{commandcolor}{[htbp]}} -- Combinación}
	\begin{tcolorbox}[colback=green!5,colframe=green!50!black]
		\textbf{Descripción:} Múltiples opciones de posición (más flexible)
		
		\textbf{Ejemplo:}
		\begin{lstlisting}[language=TeX]
			\begin{figure}[htbp]
				\centering
				\includegraphics{imagen.png}
			\end{figure}
		\end{lstlisting}
		
		\tcblower
		\faLightbulb\ \textbf{Nota:} Más flexible, \LaTeX{} elige la mejor opción
	\end{tcolorbox}
	
	\subsection*{\texttt{\textcolor{commandcolor}{[!h]}} -- Override}
	\begin{tcolorbox}[colback=green!5,colframe=green!50!black]
		\textbf{Descripción:} Ignora restricciones internas de \LaTeX{}
		
		\textbf{Ejemplo:}
		\begin{lstlisting}[language=TeX]
			\begin{figure}[!h]
				\centering
				\includegraphics{imagen.png}
			\end{figure}
		\end{lstlisting}
		
		\tcblower
		\faLightbulb\ \textbf{Nota:} El ! relaja las restricciones de \LaTeX{}
	\end{tcolorbox}
	
	%\f
	
	\section{\faImage\ Comando \textbackslash includegraphics}
	
	Inserta la imagen en la figura.
	
	\subsection*{Forma básica}
	\begin{tcolorbox}[colback=green!5,colframe=green!50!black]
		\textbf{Descripción:} Tamaño original
		
		\textbf{Ejemplo:}
		\begin{lstlisting}[language=TeX]
			\includegraphics{imagen.png}
		\end{lstlisting}
	\end{tcolorbox}
	
	\subsection*{\texttt{[width=valor]}}
	\begin{tcolorbox}[colback=green!5,colframe=green!50!black]
		\textbf{Descripción:} Ancho específico
		
		\textbf{Ejemplos:}
		\begin{lstlisting}[language=TeX]
			\includegraphics[width=5cm]{img.png}
			\includegraphics[width=0.8\textwidth]{img.png}
		\end{lstlisting}
	\end{tcolorbox}
	
	\subsection*{\texttt{[height=valor]}}
	\begin{tcolorbox}[colback=green!5,colframe=green!50!black]
		\textbf{Descripción:} Altura específica
		
		\textbf{Ejemplo:}
		\begin{lstlisting}[language=TeX]
			\includegraphics[height=4cm]{img.png}
		\end{lstlisting}
	\end{tcolorbox}
	
	\subsection*{\texttt{[scale=factor]}}
	\begin{tcolorbox}[colback=green!5,colframe=green!50!black]
		\textbf{Descripción:} Escala proporcional (1.0 = 100\%)
		
		\textbf{Ejemplo:}
		\begin{lstlisting}[language=TeX]
			\includegraphics[scale=0.5]{img.png}
		\end{lstlisting}
		
		\tcblower
		\faLightbulb\ \textbf{Nota:} 0.5 = 50\% del tamaño original
	\end{tcolorbox}
	
	\subsection*{\texttt{[angle=grados]}}
	\begin{tcolorbox}[colback=green!5,colframe=green!50!black]
		\textbf{Descripción:} Rotación en grados (sentido antihorario)
		
		\textbf{Ejemplo:}
		\begin{lstlisting}[language=TeX]
			\includegraphics[angle=90]{img.png}
		\end{lstlisting}
	\end{tcolorbox}
	
	\subsection*{\texttt{[keepaspectratio]}}
	\begin{tcolorbox}[colback=green!5,colframe=green!50!black]
		\textbf{Descripción:} Mantiene proporciones al usar width y height
		
		\textbf{Ejemplo:}
		\begin{lstlisting}[language=TeX]
			\includegraphics[width=5cm,height=3cm,keepaspectratio]{img.png}
		\end{lstlisting}
	\end{tcolorbox}
	
	\subsection*{\texttt{[trim=l b r t, clip]}}
	\begin{tcolorbox}[colback=green!5,colframe=green!50!black]
		\textbf{Descripción:} Recorta bordes (left bottom right top)
		
		\textbf{Ejemplo:}
		\begin{lstlisting}[language=TeX]
			\includegraphics[trim=1cm 2cm 1cm 2cm, clip]{img.png}
		\end{lstlisting}
	\end{tcolorbox}
	
	\subsection*{Opciones combinadas}
	\begin{tcolorbox}[colback=green!5,colframe=green!50!black]
		\textbf{Descripción:} Múltiples opciones separadas por comas
		
		\textbf{Ejemplo:}
		\begin{lstlisting}[language=TeX]
			\includegraphics[width=0.7\textwidth,angle=15,keepaspectratio]{img.png}
		\end{lstlisting}
	\end{tcolorbox}
	
	%\newpage
	
	\section{\faFont\ Comando \textbackslash caption}
	
	Añade pie de figura numerado.
	
	\subsection*{\texttt{\textbackslash caption\{texto\}}}
	\begin{tcolorbox}[colback=green!5,colframe=green!50!black]
		\textbf{Descripción:} Pie de figura básico
		
		\textbf{Ejemplo:}
		\begin{lstlisting}[language=TeX]
			\caption{Gráfica de resultados experimentales}
		\end{lstlisting}
	\end{tcolorbox}
	
	\subsection*{\texttt{\textbackslash caption[corto]\{largo\}}}
	\begin{tcolorbox}[colback=green!5,colframe=green!50!black]
		\textbf{Descripción:} Versión corta para índice de figuras
		
		\textbf{Ejemplo:}
		\begin{lstlisting}[language=TeX]
			\caption[Resultados]{Resultados experimentales completos con análisis detallado}
		\end{lstlisting}
		
		\tcblower
		\faLightbulb\ \textbf{Nota:} Corto aparece en \texttt{\textbackslash listoffigures}
	\end{tcolorbox}
	
	\subsection*{\texttt{\textbackslash caption*\{texto\}}}
	\begin{tcolorbox}[colback=green!5,colframe=green!50!black]
		\textbf{Descripción:} Caption sin numeración (requiere paquete caption)
		
		\textbf{Ejemplo:}
		\begin{lstlisting}[language=TeX]
			\usepackage{caption}
			\caption*{Nota: Sin número}
		\end{lstlisting}
		
		\tcblower
		\faLightbulb\ \textbf{Nota:} Necesita \texttt{\textbackslash usepackage\{caption\}}
	\end{tcolorbox}
	
	\section{\faTag\ Comando \textbackslash label}
	
	Etiqueta para referencias cruzadas.
	
	\subsection*{\texttt{\textbackslash label\{nombre\}}}
	\begin{tcolorbox}[colback=green!5,colframe=green!50!black]
		\textbf{Descripción:} Crea etiqueta identificadora
		
		\textbf{Ejemplo:}
		\begin{lstlisting}[language=TeX]
			\label{fig:experimento1}
		\end{lstlisting}
		
		\tcblower
		\faLightbulb\ \textbf{Nota:} Usar prefijo \texttt{fig:} es convención común
	\end{tcolorbox}
	
	\subsection*{\texttt{\textbackslash ref\{nombre\}}}
	\begin{tcolorbox}[colback=green!5,colframe=green!50!black]
		\textbf{Descripción:} Referencia al número de figura
		
		\textbf{Ejemplo:}
		\begin{lstlisting}[language=TeX]
			Ver Figura \ref{fig:experimento1}
		\end{lstlisting}
		
		\tcblower
		\faLightbulb\ \textbf{Nota:} Muestra solo el número
	\end{tcolorbox}
	
	\subsection*{\texttt{\textbackslash pageref\{nombre\}}}
	\begin{tcolorbox}[colback=green!5,colframe=green!50!black]
		\textbf{Descripción:} Referencia al número de página
		
		\textbf{Ejemplo:}
		\begin{lstlisting}[language=TeX]
			En la página \pageref{fig:experimento1}
		\end{lstlisting}
	\end{tcolorbox}
	
	%\newpage
	
	\section{\faAlignCenter\ Comando \textbackslash centering}
	
	Centra el contenido de la figura.
	
	\subsection*{\texttt{\textbackslash centering}}
	\begin{tcolorbox}[colback=green!5,colframe=green!50!black]
		\textbf{Descripción:} Comando preferido dentro de figure
		
		\textbf{Ejemplo:}
		\begin{lstlisting}[language=TeX]
			\begin{figure}[h]
				\centering
				\includegraphics{img.png}
			\end{figure}
		\end{lstlisting}
		
		\tcblower
		\faLightbulb\ \textbf{Nota:} Preferible a \texttt{\textbackslash begin\{center\}...\textbackslash end\{center\}}
	\end{tcolorbox}
	
	\subsection*{\texttt{\textbackslash raggedright}}
	\begin{tcolorbox}[colback=green!5,colframe=green!50!black]
		\textbf{Descripción:} Alineación a la izquierda
		
		\textbf{Ejemplo:}
		\begin{lstlisting}[language=TeX]
			\raggedright
			\includegraphics{img.png}
		\end{lstlisting}
	\end{tcolorbox}
	
	\subsection*{\texttt{\textbackslash raggedleft}}
	\begin{tcolorbox}[colback=green!5,colframe=green!50!black]
		\textbf{Descripción:} Alineación a la derecha
		
		\textbf{Ejemplo:}
		\begin{lstlisting}[language=TeX]
			\raggedleft
			\includegraphics{img.png}
		\end{lstlisting}
	\end{tcolorbox}
	
	\section{\faBox\ Paquetes Relacionados}
	
	Paquetes que extienden la funcionalidad del entorno figure.
	
	\subsection*{\texttt{\textbackslash usepackage\{graphicx\}}}
	\begin{tcolorbox}[colback=green!5,colframe=green!50!black]
		\textbf{Descripción:} Esencial - permite \texttt{\textbackslash includegraphics}
		
		\textbf{Ejemplo:}
		\begin{lstlisting}[language=TeX]
			\usepackage{graphicx}
		\end{lstlisting}
		
		\tcblower
		\faLightbulb\ \textbf{Nota:} Obligatorio para insertar imágenes
	\end{tcolorbox}
	
	\subsection*{\texttt{\textbackslash usepackage\{float\}}}
	\begin{tcolorbox}[colback=green!5,colframe=green!50!black]
		\textbf{Descripción:} Añade opción [H] para posición forzada
		
		\textbf{Ejemplo:}
		\begin{lstlisting}[language=TeX]
			\usepackage{float}
			\begin{figure}[H]
			\end{lstlisting}
		\end{tcolorbox}
		
		\subsection*{\texttt{\textbackslash usepackage\{caption\}}}
		\begin{tcolorbox}[colback=green!5,colframe=green!50!black]
			\textbf{Descripción:} Personalización avanzada de captions
			
			\textbf{Ejemplo:}
			\begin{lstlisting}[language=TeX]
				\usepackage{caption}
				\captionsetup{font=small,labelfont=bf}
			\end{lstlisting}
		\end{tcolorbox}
		
		\subsection*{\texttt{\textbackslash usepackage\{subcaption\}}}
		\begin{tcolorbox}[colback=green!5,colframe=green!50!black]
			\textbf{Descripción:} Subfiguras con subcaptions individuales
			
			\textbf{Ejemplo:}
			\begin{lstlisting}[language=TeX]
				\usepackage{subcaption}
				\begin{subfigure}{0.45\textwidth}
					\includegraphics{a.png}
					\caption{Parte A}
				\end{subfigure}
			\end{lstlisting}
		\end{tcolorbox}
		
		\subsection*{\texttt{\textbackslash usepackage\{wrapfig\}}}
		\begin{tcolorbox}[colback=green!5,colframe=green!50!black]
			\textbf{Descripción:} Texto envolviendo la figura
			
			\textbf{Ejemplo:}
			\begin{lstlisting}[language=TeX]
				\usepackage{wrapfig}
				\begin{wrapfigure}{r}{0.4\textwidth}
					\includegraphics{img.png}
				\end{wrapfigure}
			\end{lstlisting}

		\tcblower
		\faLightbulb\ \textbf{Opciones de posición:} \texttt{r} (derecha), \texttt{l} (izquierda), \texttt{i} (interior), \texttt{o} (exterior)
		\end{tcolorbox}

		\subsubsection*{Control de distancias en wrapfigure}

		\begin{tcolorbox}[colback=yellow!10,colframe=orange!75!black,title=\faRuler\ Parámetro \textbackslash columnsep]
			\textbf{Descripción:} Controla la distancia horizontal entre la figura y el texto que la rodea

			\textbf{Valor por defecto:} 10pt

			\textbf{Ejemplo - Ajuste global:}
			\begin{lstlisting}[language=TeX]
				% En el preámbulo
				\setlength{\columnsep}{20pt}

				\begin{wrapfigure}{r}{0.4\textwidth}
					\includegraphics{img.png}
				\end{wrapfigure}
			\end{lstlisting}

			\tcblower
			\faLightbulb\ \textbf{Nota:} Afecta a TODAS las wrapfigure del documento
		\end{tcolorbox}

		\begin{tcolorbox}[colback=yellow!10,colframe=orange!75!black,title=\faRuler\ Parámetro \textbackslash intextsep]
			\textbf{Descripción:} Controla la distancia vertical (arriba y abajo) de la figura

			\textbf{Valor por defecto:} ~12pt

			\textbf{Ejemplo - Ajuste global:}
			\begin{lstlisting}[language=TeX]
				% En el preámbulo
				\setlength{\intextsep}{15pt}

				\begin{wrapfigure}{r}{0.4\textwidth}
					\includegraphics{img.png}
				\end{wrapfigure}
			\end{lstlisting}

			\tcblower
			\faLightbulb\ \textbf{Nota:} Afecta al espacio superior e inferior
		\end{tcolorbox}

		\begin{tcolorbox}[colback=cyan!10,colframe=cyan!75!black,title=\faEdit\ Ajuste local (solo una figura)]
			\textbf{Descripción:} Cambiar distancias solo para una wrapfigure específica

			\textbf{Ejemplo:}
			\begin{lstlisting}[language=TeX]
				% Guardar valores originales y cambiar temporalmente
				\begingroup
				\setlength{\columnsep}{5pt}   % Muy cerca del texto
				\setlength{\intextsep}{20pt}  % Mucho espacio arriba/abajo

				\begin{wrapfigure}{r}{0.3\textwidth}
					\centering
					\includegraphics[width=0.28\textwidth]{imagen.png}
					\caption{Figura con ajustes especiales}
				\end{wrapfigure}

				Texto que rodea la figura...

				\endgroup  % Restaurar valores originales

				% Aquí ya se usan los valores globales de nuevo
			\end{lstlisting}

			\tcblower
			\faLightbulb\ \textbf{Nota:} Usar \texttt{\textbackslash begingroup} y \texttt{\textbackslash endgroup} para aislar cambios
		\end{tcolorbox}

		\begin{tcolorbox}[colback=purple!10,colframe=purple!75!black,title=\faTable\ Resumen de parámetros de distancia]
			\begin{center}
				\begin{tabular}{|l|l|c|}
					\hline
					\textbf{Parámetro} & \textbf{Controla} & \textbf{Default} \\
					\hline
					\texttt{\textbackslash columnsep} & Distancia horizontal (lateral) & ~10pt \\
					\hline
					\texttt{\textbackslash intextsep} & Distancia vertical (arriba/abajo) & ~12pt \\
					\hline
				\end{tabular}
			\end{center}
		\end{tcolorbox}

		\begin{tcolorbox}[colback=green!10,colframe=green!75!black,title=\faCheckCircle\ Ejemplo completo con ajustes]
			\begin{lstlisting}[language=TeX]
				\documentclass{article}
				\usepackage{graphicx}
				\usepackage{wrapfig}

				% Ajustes globales en el preámbulo
				\setlength{\columnsep}{15pt}
				\setlength{\intextsep}{10pt}

				\begin{document}

				% Figura con valores globales
				\begin{wrapfigure}{r}{0.4\textwidth}
					\centering
					\includegraphics[width=0.38\textwidth]{img1.png}
					\caption{Figura normal}
				\end{wrapfigure}

				Lorem ipsum dolor sit amet...

				% Figura con valores locales diferentes
				\begingroup
				\setlength{\columnsep}{25pt}
				\setlength{\intextsep}{5pt}

				\begin{wrapfigure}{l}{0.35\textwidth}
					\centering
					\includegraphics[width=0.33\textwidth]{img2.png}
					\caption{Figura especial}
				\end{wrapfigure}

				Texto alrededor...
				\endgroup

				\end{document}
			\end{lstlisting}
		\end{tcolorbox}

		\subsection*{\texttt{\textbackslash graphicspath\{\{carpeta/\}\}}}
		\begin{tcolorbox}[colback=green!5,colframe=green!50!black]
			\textbf{Descripción:} Define ruta para buscar imágenes
			
			\textbf{Ejemplo:}
			\begin{lstlisting}[language=TeX]
				\graphicspath{{imagenes/}{figuras/}}
			\end{lstlisting}
			
			\tcblower
			\faLightbulb\ \textbf{Nota:} Poner en el preámbulo
		\end{tcolorbox}
		
		%\newpage
		
		\section*{\faCheckCircle\ Ejemplo Completo}
		
		\begin{tcolorbox}[colback=green!10,colframe=green!75!black,title=\faCode\ Código completo de ejemplo]
			\begin{lstlisting}[language=TeX]
				\documentclass{article}
				\usepackage{graphicx}
				
				\begin{document}
					
					\begin{figure}[htbp]
						\centering
						\includegraphics[width=0.7\textwidth]{mi_grafica.png}
						\caption[Título corto]{Título largo y descriptivo de la figura}
						\label{fig:mi_grafica}
					\end{figure}
					
					Como se observa en la Figura \ref{fig:mi_grafica}...
					
				\end{document}
			\end{lstlisting}
		\end{tcolorbox}
		
		\section*{\faLightbulb\ Tips Importantes}
		
		\begin{tcolorbox}[colback=blue!10,colframe=blue!75!black]
			\begin{itemize}[leftmargin=*]
				\item El \texttt{\textbackslash label} debe ir DESPUÉS del \texttt{\textbackslash caption}
				\item Usar \texttt{\textbackslash centering} en lugar de \texttt{\textbackslash begin\{center\}} para evitar espacios extra
				\item \texttt{[htbp]} es la combinación más flexible para posicionamiento
				\item Formatos soportados: .png, .jpg, .pdf, .eps (con pdflatex usa .png, .jpg, .pdf)
				\item Es recomendable usar siempre \texttt{[htbp]} a menos que tengas una razón específica
				\item El paquete \texttt{graphicx} es obligatorio para usar \texttt{\textbackslash includegraphics}
			\end{itemize}
		\end{tcolorbox}
		
		\vspace{1cm}

		\begin{center}
			\textit{Documento generado con \LaTeX{} -- \today}
		\end{center}

	\end{document}