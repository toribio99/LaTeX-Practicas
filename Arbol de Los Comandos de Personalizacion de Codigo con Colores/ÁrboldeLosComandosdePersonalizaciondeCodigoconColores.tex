\documentclass[11pt,a4paper]{article}

% Paquetes básicos
\usepackage[utf8]{inputenc}
\usepackage[spanish,es-tabla]{babel}
\usepackage[margin=2cm]{geometry}
\usepackage{xcolor}
\usepackage{tcolorbox}
\usepackage{listings}
\tcbuselibrary{listings,skins,breakable}  % Librerías para tcblisting
\usepackage{enumitem}
\usepackage{fontawesome5}
\usepackage{multicol}

% Paquete principal para código con colores
\usepackage{listings}

% Paquetes verbatim
\usepackage{fancyvrb}
\usepackage{alltt}

% Paquetes adicionales
\usepackage{graphicx}
\usepackage{caption}
%\usepackage{hyperref}
\usepackage[hidelinks]{hyperref}

% Definir colores personalizados
\definecolor{commandcolor}{RGB}{39,174,96}
\definecolor{codebackground}{RGB}{245,245,245}
\definecolor{codegray}{RGB}{128,128,128}
\definecolor{codegreen}{RGB}{0,128,0}
\definecolor{codeblue}{RGB}{0,0,255}
\definecolor{codered}{RGB}{187,0,0}
\definecolor{codepurple}{RGB}{128,0,128}

% Colores Monokai
\definecolor{monokaibg}{RGB}{39,40,34}
\definecolor{monokaifg}{RGB}{248,248,242}
\definecolor{monokaicomment}{RGB}{117,113,94}
\definecolor{monokaistring}{RGB}{230,219,116}
\definecolor{monokaikeyword}{RGB}{249,38,114}
\definecolor{monokaifunction}{RGB}{166,226,46}
\definecolor{monokainumber}{RGB}{174,129,255}

% Colores Dracula
\definecolor{draculabg}{RGB}{40,42,54}
\definecolor{draculafg}{RGB}{248,248,242}
\definecolor{draculacomment}{RGB}{98,114,164}
\definecolor{draculacyan}{RGB}{139,233,253}
\definecolor{draculagreen}{RGB}{80,250,123}
\definecolor{draculaorange}{RGB}{255,184,108}
\definecolor{draculapink}{RGB}{255,121,198}
\definecolor{draculapurple}{RGB}{189,147,249}
\definecolor{draculared}{RGB}{255,85,85}
\definecolor{draculayellow}{RGB}{241,250,140}

% Colores Solarized Dark
\definecolor{solarizedbase03}{RGB}{0,43,54}
\definecolor{solarizedbase02}{RGB}{7,54,66}
\definecolor{solarizedbase01}{RGB}{88,110,117}
\definecolor{solarizedbase00}{RGB}{101,123,131}
\definecolor{solarizedbase0}{RGB}{131,148,150}
\definecolor{solarizedbase1}{RGB}{147,161,161}
\definecolor{solarizedyellow}{RGB}{181,137,0}
\definecolor{solarizedorange}{RGB}{203,75,22}
\definecolor{solarizedred}{RGB}{220,50,47}
\definecolor{solarizedmagenta}{RGB}{211,54,130}
\definecolor{solarizedviolet}{RGB}{108,113,196}
\definecolor{solarizedblue}{RGB}{38,139,210}
\definecolor{solarizedcyan}{RGB}{42,161,152}
\definecolor{solarizedgreen}{RGB}{133,153,0}

% Colores GitHub
\definecolor{githubbg}{RGB}{255,255,255}
\definecolor{githubfg}{RGB}{36,41,46}
\definecolor{githubcomment}{RGB}{106,115,125}
\definecolor{githubstring}{RGB}{3,47,98}
\definecolor{githubkeyword}{RGB}{215,58,73}
\definecolor{githubfunction}{RGB}{111,66,193}
\definecolor{githubnumber}{RGB}{0,92,197}

% Configuración básica de listings
\lstset{
	basicstyle=\ttfamily\footnotesize,
	breaklines=true,
	columns=fullflexible,
	keepspaces=true,
	showstringspaces=false,
	inputencoding=utf8,
	extendedchars=true,
	literate=
		{á}{{\'a}}1 {é}{{\'e}}1 {í}{{\'i}}1 {ó}{{\'o}}1 {ú}{{\'u}}1
		{Á}{{\'A}}1 {É}{{\'E}}1 {Í}{{\'I}}1 {Ó}{{\'O}}1 {Ú}{{\'U}}1
		{ñ}{{\~n}}1 {Ñ}{{\~N}}1
		{ü}{{\"u}}1 {Ü}{{\"U}}1
		{¿}{{?`}}1 {¡}{{!`}}1
}

% Definir entorno para mostrar código LaTeX en cajas
\newtcblisting{texcode}{
	listing only,
	listing options={
		language=TeX,
		basicstyle=\ttfamily\small,
		breaklines=true,
		columns=fullflexible
	},
	colback=green!5,
	colframe=green!50!black,
	arc=1mm,
	boxrule=0.5pt
}

% Título
\title{\textbf{\Huge Personalización de Código con Colores en \LaTeX{}}\\\large Guía Completa de listings, minted, verbatim y tcolorbox}
\author{}
\date{\today}
 \usepackage[hidelinks]{hyperref}

\begin{document}

\maketitle
\thispagestyle{empty}

\begin{tcolorbox}[colback=blue!5,colframe=blue!75!black,title=\faInfoCircle\ Introducción]
Esta guía exhaustiva cubre todos los paquetes, comandos y configuraciones disponibles en \LaTeX{} para mostrar código fuente con colores personalizados. Incluye ejemplos en múltiples lenguajes de programación, paletas de colores predefinidas y técnicas avanzadas de personalización.
\end{tcolorbox}

\tableofcontents
\newpage

\section{Paquete listings}

\subsection{Introducción a listings}

\begin{tcolorbox}[colback=green!5,colframe=green!50!black]
	\textbf{Descripción:} El paquete \texttt{listings} es el estándar para mostrar código fuente con resaltado de sintaxis personalizable.

	\textbf{Carga:} \verb|\usepackage{listings}|
\end{tcolorbox}

\subsection{Configuración Básica de Colores}

\subsubsection*{Opciones principales de color}

\begin{tcolorbox}[colback=green!5,colframe=green!50!black]
	\textbf{Opciones disponibles:}
	\begin{itemize}[nosep]
		\item \texttt{backgroundcolor} -- Color de fondo
		\item \texttt{commentstyle} -- Estilo de comentarios
		\item \texttt{keywordstyle} -- Estilo de palabras clave
		\item \texttt{numberstyle} -- Estilo de números de línea
		\item \texttt{stringstyle} -- Estilo de cadenas de texto
		\item \texttt{identifierstyle} -- Estilo de identificadores
		\item \texttt{basicstyle} -- Estilo base del código
	\end{itemize}
\end{tcolorbox}

\textbf{Ejemplo básico:}
\begin{texcode}
\lstset{
  backgroundcolor=\color{gray!10},
  commentstyle=\color{green},
  keywordstyle=\color{blue}\bfseries,
  numberstyle=\tiny\color{gray},
  stringstyle=\color{red},
  basicstyle=\ttfamily\footnotesize
}
\end{texcode}

\subsection{Estilos Predefinidos}

\subsubsection*{Estilo 1: Colores Básicos}

\lstdefinestyle{basico}{
	backgroundcolor=\color{gray!10},
	commentstyle=\color{codegreen},
	keywordstyle=\color{codeblue}\bfseries,
	numberstyle=\tiny\color{codegray},
	stringstyle=\color{codered},
	basicstyle=\ttfamily\footnotesize,
	breaklines=true,
	numbers=left,
	numbersep=5pt,
	frame=single,
	rulecolor=\color{gray!30}
}

\begin{tcolorbox}[colback=cyan!10,colframe=cyan!75!black,title=\faCode\ Definición]
\end{tcolorbox}

\begin{texcode}
\lstdefinestyle{basico}{
  backgroundcolor=\color{gray!10},
  commentstyle=\color{green},
  keywordstyle=\color{blue}\bfseries,
  numberstyle=\tiny\color{gray},
  stringstyle=\color{red},
  basicstyle=\ttfamily\footnotesize,
  numbers=left,
  frame=single
}
\end{texcode}

\textbf{Ejemplo en Python:}
\begin{lstlisting}[style=basico,language=Python,caption={Código Python con estilo básico}]
def factorial(n):
    """Calcula el factorial de n"""
    if n == 0:
        return 1
    else:
        return n * factorial(n-1)

# Ejemplo de uso
resultado = factorial(5)
print(f"El factorial es: {resultado}")
\end{lstlisting}

\subsubsection*{Estilo 2: Tema Monokai}

\lstdefinestyle{monokai}{
	backgroundcolor=\color{monokaibg},
	commentstyle=\color{monokaicomment}\itshape,
	keywordstyle=\color{monokaikeyword}\bfseries,
	numberstyle=\tiny\color{monokaicomment},
	stringstyle=\color{monokaistring},
	basicstyle=\ttfamily\color{monokaifg}\footnotesize,
	identifierstyle=\color{monokaifunction},
	breaklines=true,
	numbers=left,
	numbersep=8pt,
	frame=single,
	rulecolor=\color{monokaicomment},
	showstringspaces=false,
	tabsize=4
}

\begin{tcolorbox}[colback=cyan!10,colframe=cyan!75!black,title=\faCode\ Definición Monokai]
\end{tcolorbox}

\begin{texcode}
% Primero definir los colores
\definecolor{monokaibg}{RGB}{39,40,34}
\definecolor{monokaifg}{RGB}{248,248,242}
\definecolor{monokaicomment}{RGB}{117,113,94}
\definecolor{monokaistring}{RGB}{230,219,116}
\definecolor{monokaikeyword}{RGB}{249,38,114}
\definecolor{monokaifunction}{RGB}{166,226,46}

% Luego definir el estilo
\lstdefinestyle{monokai}{
  backgroundcolor=\color{monokaibg},
  commentstyle=\color{monokaicomment}\itshape,
  keywordstyle=\color{monokaikeyword}\bfseries,
  stringstyle=\color{monokaistring},
  basicstyle=\ttfamily\color{monokaifg},
  numbers=left
}
\end{texcode}

\textbf{Ejemplo en JavaScript:}
\begin{lstlisting}[style=monokai,language=Java,caption={Código JavaScript con tema Monokai}]
// Función para calcular fibonacci
function fibonacci(n) {
    if (n <= 1) return n;
    return fibonacci(n - 1) + fibonacci(n - 2);
}

// Clase de ejemplo
class Calculator {
    constructor() {
        this.result = 0;
    }

    add(a, b) {
        this.result = a + b;
        return this.result;
    }
}

const calc = new Calculator();
console.log(calc.add(5, 3));
\end{lstlisting}

\subsubsection*{Estilo 3: Tema Dracula}

\lstdefinestyle{dracula}{
	backgroundcolor=\color{draculabg},
	commentstyle=\color{draculacomment}\itshape,
	keywordstyle=\color{draculapink}\bfseries,
	numberstyle=\tiny\color{draculacomment},
	stringstyle=\color{draculayellow},
	basicstyle=\ttfamily\color{draculafg}\footnotesize,
	identifierstyle=\color{draculacyan},
	breaklines=true,
	numbers=left,
	numbersep=8pt,
	frame=lines,
	rulecolor=\color{draculapurple},
	showstringspaces=false
}

\begin{tcolorbox}[colback=cyan!10,colframe=cyan!75!black,title=\faCode\ Definición Dracula]
\end{tcolorbox}

\begin{texcode}
\definecolor{draculabg}{RGB}{40,42,54}
\definecolor{draculafg}{RGB}{248,248,242}
\definecolor{draculacomment}{RGB}{98,114,164}
\definecolor{draculapink}{RGB}{255,121,198}
\definecolor{draculayellow}{RGB}{241,250,140}
\definecolor{draculacyan}{RGB}{139,233,253}

\lstdefinestyle{dracula}{
  backgroundcolor=\color{draculabg},
  commentstyle=\color{draculacomment}\itshape,
  keywordstyle=\color{draculapink}\bfseries,
  stringstyle=\color{draculayellow},
  basicstyle=\ttfamily\color{draculafg}
}
\end{texcode}

\textbf{Ejemplo en C++:}
\begin{lstlisting}[style=dracula,language=C++,caption={Código C++ con tema Dracula}]
#include <iostream>
#include <vector>
using namespace std;

// Clase plantilla
template<typename T>
class Stack {
private:
    vector<T> elements;

public:
    void push(T const& elem) {
        elements.push_back(elem);
    }

    void pop() {
        if (elements.empty()) {
            throw out_of_range("Stack<>::pop(): stack vacía");
        }
        elements.pop_back();
    }

    T top() const {
        if (elements.empty()) {
            throw out_of_range("Stack<>::top(): stack vacía");
        }
        return elements.back();
    }
};

int main() {
    Stack<int> intStack;
    intStack.push(7);
    cout << intStack.top() << endl;
    return 0;
}
\end{lstlisting}

\subsubsection*{Estilo 4: Tema Solarized Dark}

\lstdefinestyle{solarized}{
	backgroundcolor=\color{solarizedbase03},
	commentstyle=\color{solarizedbase01}\itshape,
	keywordstyle=\color{solarizedgreen}\bfseries,
	numberstyle=\tiny\color{solarizedbase01},
	stringstyle=\color{solarizedcyan},
	basicstyle=\ttfamily\color{solarizedbase0}\footnotesize,
	identifierstyle=\color{solarizedblue},
	breaklines=true,
	numbers=left,
	numbersep=8pt,
	frame=single,
	rulecolor=\color{solarizedbase02}
}

\begin{tcolorbox}[colback=cyan!10,colframe=cyan!75!black,title=\faCode\ Definición Solarized]
\end{tcolorbox}

\begin{texcode}
\definecolor{solarizedbase03}{RGB}{0,43,54}
\definecolor{solarizedbase0}{RGB}{131,148,150}
\definecolor{solarizedgreen}{RGB}{133,153,0}
\definecolor{solarizedcyan}{RGB}{42,161,152}

\lstdefinestyle{solarized}{
  backgroundcolor=\color{solarizedbase03},
  commentstyle=\color{solarizedbase01}\itshape,
  keywordstyle=\color{solarizedgreen}\bfseries,
  stringstyle=\color{solarizedcyan},
  basicstyle=\ttfamily\color{solarizedbase0}
}
\end{texcode}

\textbf{Ejemplo en Java:}
\begin{lstlisting}[style=solarized,language=Java,caption={Código Java con tema Solarized}]
public class Persona {
    private String nombre;
    private int edad;

    // Constructor
    public Persona(String nombre, int edad) {
        this.nombre = nombre;
        this.edad = edad;
    }

    // Getters y setters
    public String getNombre() {
        return nombre;
    }

    public void setNombre(String nombre) {
        this.nombre = nombre;
    }

    // Método toString
    @Override
    public String toString() {
        return "Persona{nombre='" + nombre +
               "', edad=" + edad + '}';
    }

    public static void main(String[] args) {
        Persona persona = new Persona("Juan", 25);
        System.out.println(persona);
    }
}
\end{lstlisting}

\subsubsection*{Estilo 5: Tema GitHub}

\lstdefinestyle{github}{
	backgroundcolor=\color{githubbg},
	commentstyle=\color{githubcomment}\itshape,
	keywordstyle=\color{githubkeyword}\bfseries,
	numberstyle=\tiny\color{githubcomment},
	stringstyle=\color{githubstring},
	basicstyle=\ttfamily\color{githubfg}\footnotesize,
	identifierstyle=\color{githubfunction},
	breaklines=true,
	numbers=left,
	numbersep=8pt,
	frame=single,
	rulecolor=\color{gray!30},
	showstringspaces=false
}

\textbf{Ejemplo en SQL:}
\begin{lstlisting}[style=github,language=SQL,caption={Código SQL con tema GitHub}]
-- Crear tabla de usuarios
CREATE TABLE usuarios (
    id INT PRIMARY KEY AUTO_INCREMENT,
    nombre VARCHAR(100) NOT NULL,
    email VARCHAR(100) UNIQUE NOT NULL,
    edad INT CHECK (edad >= 18),
    fecha_registro TIMESTAMP DEFAULT CURRENT_TIMESTAMP
);

-- Insertar datos
INSERT INTO usuarios (nombre, email, edad)
VALUES
    ('Juan Pérez', 'juan@example.com', 25),
    ('María García', 'maria@example.com', 30);

-- Consulta con JOIN
SELECT u.nombre, p.titulo, p.contenido
FROM usuarios u
INNER JOIN publicaciones p ON u.id = p.usuario_id
WHERE u.edad > 20
ORDER BY p.fecha DESC
LIMIT 10;
\end{lstlisting}

\subsection{Personalización Avanzada}

\subsubsection*{Palabras Clave Adicionales}

\begin{tcolorbox}[colback=green!5,colframe=green!50!black]
	\textbf{Agregar palabras clave personalizadas:}
\end{tcolorbox}

\begin{texcode}
\lstset{
  morekeywords={async, await, const, let},
  keywordstyle=[2]\color{purple},
  keywords=[2]{self, cls},
  emphstyle=\color{orange},
  emph={importante, nota}
}
\end{texcode}

\subsubsection*{Escapado de Caracteres para LaTeX}

\begin{tcolorbox}[colback=green!5,colframe=green!50!black]
	\textbf{Permitir comandos LaTeX dentro del código:}
\end{tcolorbox}

\begin{texcode}
\lstset{
  escapechar=@,
  escapeinside={(*@}{@*)}
}

% Uso:
\begin{lstlisting}[escapechar=@]
def funcion():
    # @\textcolor{red}{Comentario resaltado}@
    pass
\end{lstlisting}
\end{texcode}

\subsubsection*{Resaltado de Líneas Específicas}

\begin{tcolorbox}[colback=green!5,colframe=green!50!black]
	\textbf{Resaltar líneas importantes:}
\end{tcolorbox}

\begin{texcode}
% En el preámbulo
\usepackage{xcolor,listings}

% Definir estilo con líneas resaltadas
\lstset{
  moredelim=**[is][\color{red}]{@}{@}
}

% O usar linebackgroundcolor
\lstset{
  linebackgroundcolor={%
    \ifodd\value{lstnumber}\color{gray!10}\fi}
}
\end{texcode}

\subsection{Lenguajes Soportados}

\begin{tcolorbox}[colback=yellow!10,colframe=orange!75!black]
	\textbf{Lenguajes principales con soporte en listings:}
	\begin{multicols}{3}
	\begin{itemize}[nosep,leftmargin=*]
		\item Python
		\item Java
		\item C/C++
		\item JavaScript
		\item PHP
		\item Ruby
		\item Go
		\item Rust
		\item Swift
		\item Kotlin
		\item SQL
		\item HTML
		\item CSS
		\item XML
		\item JSON
		\item YAML
		\item Bash
		\item PowerShell
		\item R
		\item MATLAB
		\item Perl
		\item Haskell
		\item Scala
		\item LaTeX
	\end{itemize}
	\end{multicols}
\end{tcolorbox}

\section{Paquete minted}

\subsection{Introducción a minted}

\begin{tcolorbox}[colback=green!5,colframe=green!50!black]
	\textbf{Descripción:} \texttt{minted} usa Pygments (Python) para resaltado de sintaxis muy preciso y con soporte para cientos de lenguajes.

	\textbf{Requisitos:}
	\begin{itemize}[nosep]
		\item Python instalado
		\item Pygments: \texttt{pip install Pygments}
		\item Compilar con \texttt{-shell-escape}: \texttt{pdflatex -shell-escape documento.tex}
	\end{itemize}

	\textbf{Carga:}
\end{tcolorbox}

\begin{texcode}
\usepackage{minted}
\end{texcode}

\subsection{Uso Básico de minted}

\begin{tcolorbox}[colback=green!5,colframe=green!50!black]
	\textbf{Sintaxis:}
\end{tcolorbox}

\begin{texcode}
\begin{minted}{python}
def hola():
    print("Hola Mundo")
\end{minted}

% O inline:
\mint{python}|x = 5|
\end{texcode}

\subsection{Temas de Color en minted}

\begin{tcolorbox}[colback=green!5,colframe=green!50!black]
	\textbf{Temas disponibles:}
	\begin{multicols}{3}
	\begin{itemize}[nosep,leftmargin=*]
		\item default
		\item monokai
		\item manni
		\item perldoc
		\item borland
		\item colorful
		\item murphy
		\item tango
		\item trac
		\item fruity
		\item autumn
		\item bw
		\item emacs
		\item vim
		\item pastie
		\item friendly
		\item native
		\item paraiso-dark
		\item paraiso-light
		\item solarized-dark
		\item solarized-light
		\item dracula
		\item gruvbox-dark
		\item gruvbox-light
		\item github-dark
	\end{itemize}
	\end{multicols}

	\textbf{Configuración:}
\end{tcolorbox}

\begin{texcode}
\usemintedstyle{monokai}

% O para un lenguaje específico:
\begin{minted}[style=dracula]{python}
código
\end{minted}
\end{texcode}

\subsection{Opciones de minted}

\begin{tcolorbox}[colback=green!5,colframe=green!50!black]
	\textbf{Opciones principales:}
\end{tcolorbox}

\begin{texcode}
\begin{minted}[
  bgcolor=lightgray,      % Color de fondo
  linenos,                % Números de línea
  numbersep=5pt,          % Separación de números
  frame=lines,            % Marco (none, leftline, topline, bottomline, lines, single)
  framesep=2mm,           % Separación del marco
  fontsize=\footnotesize, % Tamaño de fuente
  tabsize=4,              % Tamaño de tabulación
  breaklines,             % Romper líneas largas
  breakanywhere,          % Romper en cualquier lugar
  highlightlines={2,4-6}, % Resaltar líneas
  highlightcolor=yellow!30 % Color de resaltado
]{python}
código aquí
\end{minted}
\end{texcode}

\subsection{Comandos Útiles de minted}

\begin{tcolorbox}[colback=green!5,colframe=green!50!black]
	\textbf{Comandos disponibles:}
\end{tcolorbox}

\begin{texcode}
% Incluir archivo externo
\inputminted{python}{script.py}

% Con opciones
\inputminted[linenos,bgcolor=gray!10]{python}{script.py}

% Inline code
\mint{python}|x = 10|

% Mintinline
\mintinline{python}{def foo(): pass}

% Listar estilos disponibles
pygmentize -L styles
\end{texcode}

\section{Entornos Verbatim}

\subsection{Verbatim Básico}

\begin{tcolorbox}[colback=green!5,colframe=green!50!black]
	\textbf{Entorno básico:}
\end{tcolorbox}

\begin{texcode}
\begin{verbatim}
Texto literal sin formato
  con espacios preservados
\end{verbatim}

% Inline
\verb|código inline|
\verb+código con +
\end{texcode}

\subsection{Paquete fancyvrb con Colores}

\begin{tcolorbox}[colback=green!5,colframe=green!50!black]
	\textbf{Descripción:} \texttt{fancyvrb} permite añadir colores y formato usando la opción \texttt{formatcom}.

	\textbf{Ejemplo de uso:}

	\texttt{\textbackslash usepackage\{fancyvrb\}}

	\texttt{\textbackslash begin\{Verbatim\}[formatcom=\textbackslash color\{blue\}]}

	\texttt{Código en azul}

	\texttt{\textbackslash end\{Verbatim\}}
\end{tcolorbox}

\subsection{Paquete alltt}

\begin{tcolorbox}[colback=green!5,colframe=green!50!black]
	\textbf{Descripción:} Permite comandos LaTeX dentro de verbatim

	\textbf{Uso:}
\end{tcolorbox}

\begin{texcode}
\usepackage{alltt}

\begin{alltt}
Este es código con \textcolor{red}{color}
y \textbf{formato} de LaTeX
\end{alltt}
\end{texcode}

\section{Integración con tcolorbox}

\subsection{tcolorbox + listings}

\begin{tcolorbox}[colback=green!5,colframe=green!50!black]
	\textbf{Crear cajas de código decorativas:}
\end{tcolorbox}

\begin{texcode}
\usepackage{tcolorbox}
\tcbuselibrary{listings,skins}

% Definir estilo
\newtcblisting{micodigo}[2][]{
  listing only,
  listing options={
    language=#2,
    basicstyle=\ttfamily,
    keywordstyle=\color{blue}
  },
  colback=gray!10,
  colframe=blue!75!black,
  title=#1,
  #1
}

% Uso
\begin{micodigo}[Ejemplo Python]{python}
def suma(a, b):
    return a + b
\end{micodigo}
\end{texcode}

\subsection{tcolorbox + minted}

\begin{tcolorbox}[colback=green!5,colframe=green!50!black]
	\textbf{Combinación con minted:}
\end{tcolorbox}

\begin{texcode}
\usepackage{tcolorbox}
\tcbuselibrary{minted,skins}

% Definir entorno
\newtcblisting{pythoncode}{
  listing engine=minted,
  minted language=python,
  minted style=monokai,
  colback=monokaibg,
  colframe=monokaikeyword,
  listing only,
  left=5mm,
  enhanced,
  overlay={
    \begin{tcbclipinterior}
    \fill[monokaikeyword] (frame.south west)
      rectangle ([xshift=5mm]frame.north west);
    \end{tcbclipinterior}
  }
}
\end{texcode}

\subsection{Estilos Predefinidos con tcolorbox}

\newtcblisting{codeboxbasic}[2][]{
	listing only,
	listing options={
		language=#2,
		basicstyle=\ttfamily\small,
		keywordstyle=\color{blue}\bfseries,
		commentstyle=\color{green!60!black},
		stringstyle=\color{red}
	},
	colback=gray!5,
	colframe=blue!75!black,
	arc=2mm,
	boxrule=0.5pt,
	left=5pt,
	right=5pt,
	top=5pt,
	bottom=5pt,
	#1
}

\begin{tcolorbox}[colback=cyan!10,colframe=cyan!75!black,title=\faCode\ Definición y Ejemplo]
	\textbf{Resultado:}
\end{tcolorbox}

\begin{texcode}
\newtcblisting{codeboxbasic}[2][]{
  listing only,
  listing options={
    language=#2,
    keywordstyle=\color{blue}\bfseries
  },
  colback=gray!5,
  colframe=blue!75!black
}
\end{texcode}

\begin{codeboxbasic}[title={Python con tcolorbox}]{Python}
class Animal:
    def __init__(self, nombre):
        self.nombre = nombre

    def hablar(self):
        pass

class Perro(Animal):
    def hablar(self):
        return f"{self.nombre} dice Guau!"

perro = Perro("Firulais")
print(perro.hablar())
\end{codeboxbasic}

\section{Ejemplos en Múltiples Lenguajes}

\subsection{HTML y CSS}

\lstdefinestyle{htmlcss}{
	backgroundcolor=\color{white},
	commentstyle=\color{gray},
	keywordstyle=\color{blue!80!black},
	numberstyle=\tiny\color{gray},
	stringstyle=\color{red!80!black},
	basicstyle=\ttfamily\footnotesize,
	breaklines=true,
	numbers=left,
	frame=single,
	rulecolor=\color{gray!30},
	showstringspaces=false,
	tabsize=2
}

\textbf{HTML:}
\begin{lstlisting}[style=htmlcss,language=HTML,caption={Ejemplo HTML5}]
<!DOCTYPE html>
<html lang="es">
<head>
    <meta charset="UTF-8">
    <meta name="viewport" content="width=device-width, initial-scale=1.0">
    <title>Mi Página Web</title>
    <link rel="stylesheet" href="styles.css">
</head>
<body>
    <header>
        <h1>Bienvenido</h1>
        <nav>
            <ul>
                <li><a href="#inicio">Inicio</a></li>
                <li><a href="#sobre">Sobre</a></li>
                <li><a href="#contacto">Contacto</a></li>
            </ul>
        </nav>
    </header>

    <main>
        <section id="inicio">
            <h2>Contenido Principal</h2>
            <p>Este es un párrafo de ejemplo.</p>
        </section>
    </main>

    <footer>
        <p>&copy; 2024 Mi Sitio Web</p>
    </footer>
</body>
</html>
\end{lstlisting}

\subsection{Ruby}

\begin{lstlisting}[style=monokai,language=Ruby,caption={Código Ruby}]
# Clase de ejemplo en Ruby
class Rectangulo
  attr_accessor :ancho, :alto

  def initialize(ancho, alto)
    @ancho = ancho
    @alto = alto
  end

  def area
    @ancho * @alto
  end

  def perimetro
    2 * (@ancho + @alto)
  end

  def es_cuadrado?
    @ancho == @alto
  end
end

# Uso
rect = Rectangulo.new(5, 10)
puts "Área: #{rect.area}"
puts "Perímetro: #{rect.perimetro}"
puts "¿Es cuadrado? #{rect.es_cuadrado?}"
\end{lstlisting}

\subsection{R}

\begin{lstlisting}[style=github,language=R,caption={Código R para análisis estadístico}]
# Análisis estadístico en R
library(ggplot2)

# Crear datos de ejemplo
set.seed(123)
datos <- data.frame(
  x = rnorm(100, mean = 50, sd = 10),
  y = rnorm(100, mean = 50, sd = 10),
  grupo = sample(c("A", "B", "C"), 100, replace = TRUE)
)

# Estadísticas descriptivas
summary(datos)

# Correlación
cor(datos$x, datos$y)

# Gráfico
ggplot(datos, aes(x = x, y = y, color = grupo)) +
  geom_point(size = 3) +
  geom_smooth(method = "lm", se = FALSE) +
  theme_minimal() +
  labs(title = "Relación entre X e Y",
       x = "Variable X",
       y = "Variable Y")
\end{lstlisting}

\subsection{Bash/Shell}

\begin{lstlisting}[style=solarized,language=bash,caption={Script Bash}]
#!/bin/bash

# Script de backup automatizado

# Variables
FECHA=$(date +%Y-%m-%d)
ORIGEN="/home/usuario/documentos"
DESTINO="/backup"
LOG="/var/log/backup.log"

# Función de logging
log_mensaje() {
    echo "[$(date '+%Y-%m-%d %H:%M:%S')] $1" | tee -a $LOG
}

# Verificar si el directorio existe
if [ ! -d "$ORIGEN" ]; then
    log_mensaje "ERROR: El directorio origen no existe"
    exit 1
fi

# Crear directorio de backup
mkdir -p "$DESTINO/$FECHA"

# Realizar backup
log_mensaje "Iniciando backup..."
tar -czf "$DESTINO/$FECHA/backup.tar.gz" "$ORIGEN" 2>&1 | tee -a $LOG

if [ ${PIPESTATUS[0]} -eq 0 ]; then
    log_mensaje "Backup completado exitosamente"
else
    log_mensaje "ERROR: Falló el backup"
    exit 1
fi

# Limpiar backups antiguos (más de 7 días)
find "$DESTINO" -type f -mtime +7 -delete
log_mensaje "Backups antiguos eliminados"
\end{lstlisting}

\section{Plantillas Listas para Usar}

\subsection{Plantilla 1: Documento Técnico}

\begin{tcolorbox}[colback=purple!10,colframe=purple!75!black,title=\faCode\ Plantilla Completa,breakable]
\end{tcolorbox}

\begin{texcode}
\documentclass{article}
\usepackage[utf8]{inputenc}
\usepackage{listings}
\usepackage{xcolor}

% Colores personalizados
\definecolor{codegreen}{rgb}{0,0.6,0}
\definecolor{codegray}{rgb}{0.5,0.5,0.5}
\definecolor{codepurple}{rgb}{0.58,0,0.82}
\definecolor{backcolour}{rgb}{0.95,0.95,0.92}

% Estilo de código
\lstdefinestyle{mystyle}{
    backgroundcolor=\color{backcolour},
    commentstyle=\color{codegreen},
    keywordstyle=\color{magenta},
    numberstyle=\tiny\color{codegray},
    stringstyle=\color{codepurple},
    basicstyle=\ttfamily\footnotesize,
    breakatwhitespace=false,
    breaklines=true,
    captionpos=b,
    keepspaces=true,
    numbers=left,
    numbersep=5pt,
    showspaces=false,
    showstringspaces=false,
    showtabs=false,
    tabsize=2
}

\lstset{style=mystyle}

\begin{document}

\section{Código de Ejemplo}

\begin{lstlisting}[language=Python,caption={Mi primer script}]
def main():
    print("Hola Mundo")

if __name__ == "__main__":
    main()
\end{lstlisting}

\end{document}
\end{texcode}

\subsection{Plantilla 2: Con tcolorbox}

\begin{tcolorbox}[colback=purple!10,colframe=purple!75!black,title=\faCode\ Plantilla con tcolorbox,breakable]
\end{tcolorbox}

\begin{texcode}
\documentclass{article}
\usepackage[utf8]{inputenc}
\usepackage{tcolorbox}
\tcbuselibrary{listings,skins,breakable}
\usepackage{xcolor}

% Definir colores
\definecolor{mybg}{RGB}{245,245,245}
\definecolor{myframe}{RGB}{70,130,180}

% Crear entorno personalizado
\newtcblisting{mycode}[2][]{
  listing only,
  listing options={
    language=#2,
    basicstyle=\ttfamily\small,
    keywordstyle=\color{blue}\bfseries,
    commentstyle=\color{green!60!black}\itshape,
    stringstyle=\color{red},
    numbers=left,
    numberstyle=\tiny\color{gray}
  },
  colback=mybg,
  colframe=myframe,
  arc=3mm,
  boxrule=1pt,
  title=#1,
  fonttitle=\bfseries,
  coltitle=white,
  breakable,
  #1
}

\begin{document}

\begin{mycode}[title={Ejemplo en Python}]{Python}
def fibonacci(n):
    if n <= 1:
        return n
    return fibonacci(n-1) + fibonacci(n-2)

print(fibonacci(10))
\end{mycode}

\end{document}
\end{texcode}

\section{Tips y Mejores Prácticas}

\begin{tcolorbox}[colback=blue!10,colframe=blue!75!black,title=\faLightbulb\ Recomendaciones]
	\textbf{Para listings:}
	\begin{itemize}[leftmargin=*]
		\item Define estilos reutilizables con \texttt{\textbackslash lstdefinestyle}
		\item Usa \texttt{showstringspaces=false} para código más limpio
		\item Ajusta \texttt{tabsize} según el lenguaje (2 para JS, 4 para Python)
		\item Usa \texttt{breaklines=true} para líneas largas
		\item Define colores en el preámbulo para consistencia
	\end{itemize}

	\textbf{Para minted:}
	\begin{itemize}[leftmargin=*]
		\item Requiere compilación con \texttt{-shell-escape}
		\item Más preciso que listings pero más lento
		\item Ideal para lenguajes modernos
		\item Usa \texttt{\textbackslash usemintedstyle} para tema global
	\end{itemize}

	\textbf{Para colores:}
	\begin{itemize}[leftmargin=*]
		\item Usa RGB para colores precisos
		\item Mantén contraste adecuado (fondo claro/oscuro)
		\item Prueba temas existentes antes de crear propios
		\item Considera accesibilidad (daltonismo)
	\end{itemize}

	\textbf{Para tcolorbox:}
	\begin{itemize}[leftmargin=*]
		\item Combina con listings o minted para cajas decorativas
		\item Usa \texttt{breakable} para código largo
		\item Define entornos personalizados con \texttt{\textbackslash newtcblisting}
	\end{itemize}
\end{tcolorbox}

\section{Errores Comunes}

\begin{tcolorbox}[colback=red!10,colframe=red!75!black,title=\faExclamationTriangle\ Problemas Frecuentes]
	\textbf{Errores típicos:}
	\begin{itemize}[leftmargin=*]
		\item \textbf{minted no compila:} Falta \texttt{-shell-escape} o Pygments no instalado
		\item \textbf{Caracteres especiales:} Usar \texttt{literate} para tildes y ñ
		\item \textbf{Líneas cortadas:} Activar \texttt{breaklines=true}
		\item \textbf{Colores no aparecen:} Verificar que \texttt{xcolor} esté cargado
		\item \textbf{Números de línea desalineados:} Ajustar \texttt{numbersep}
		\item \textbf{Código fuera de página:} Usar \texttt{breakable} en tcolorbox
		\item \textbf{Espacios en strings:} Configurar \texttt{showstringspaces}
	\end{itemize}
\end{tcolorbox}

\section{Recursos Adicionales}

\begin{tcolorbox}[colback=yellow!10,colframe=orange!75!black,title=\faBook\ Documentación]
	\textbf{Comandos útiles:}
	\begin{itemize}[leftmargin=*]
		\item \texttt{texdoc listings} -- Manual de listings
		\item \texttt{texdoc minted} -- Documentación de minted
		\item \texttt{texdoc tcolorbox} -- Guía de tcolorbox
		\item \texttt{texdoc fancyvrb} -- Manual de fancyvrb
		\item \texttt{pygmentize -L styles} -- Listar temas de Pygments
		\item \texttt{pygmentize -L lexers} -- Listar lenguajes soportados
	\end{itemize}

	\textbf{Sitios web útiles:}
	\begin{itemize}[leftmargin=*]
		\item \url{https://pygments.org/styles/} -- Galería de temas Pygments
		\item \url{https://www.overleaf.com/learn/latex/Code_listing} -- Tutorial Overleaf
		\item \url{https://tex.stackexchange.com/} -- Preguntas y respuestas
	\end{itemize}
\end{tcolorbox}

\vspace{2cm}

\begin{center}
	\begin{tcolorbox}[colback=blue!5,colframe=blue!75!black,width=0.9\textwidth]
		\centering
		\large\textbf{Documento generado con \LaTeX{}}\\[0.5em]
		\normalsize\textit{Guía Completa de Personalización de Código con Colores}\\[0.5em]
		\small\today
	\end{tcolorbox}

	\vspace{1cm}

	\textit{Esta guía cubre todos los paquetes principales para mostrar}\\
	\textit{código fuente con colores en \LaTeX{}.}\\[0.5em]
	\textit{Para más información, consulta la documentación oficial.}
\end{center}

\end{document}
