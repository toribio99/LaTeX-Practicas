% !TEX encoding = UTF-8 Unicode
\documentclass[11pt,a4paper]{article}

% Paquetes necesarios
\usepackage[utf8]{inputenc}
\usepackage[T1]{fontenc}
\usepackage[spanish]{babel}
\usepackage[margin=2cm]{geometry}
\usepackage{xcolor}
\usepackage{tcolorbox}
\usepackage{enumitem}
\usepackage{booktabs}
\usepackage{listings}
\usepackage{newunicodechar}
\usepackage{amssymb}

% Definir símbolos Unicode especiales
\newunicodechar{✓}{\checkmark}
\newunicodechar{✗}{\ensuremath{\times}}
\newunicodechar{↪}{\ensuremath{\hookrightarrow}}
\newunicodechar{❌}{\ensuremath{\times}}
\newunicodechar{¡}{!`}

% Colores personalizados
\definecolor{codeblue}{RGB}{0,102,204}
\definecolor{codepurple}{RGB}{153,0,153}
\definecolor{codegreen}{RGB}{0,128,0}
\definecolor{codegray}{RGB}{128,128,128}
\definecolor{codebglight}{RGB}{245,245,245}
\definecolor{framecolor}{RGB}{102,102,102}

% Configuración de listings para LaTeX
\lstset{
    language=[LaTeX]TeX,
    basicstyle=\ttfamily\small,
    keywordstyle=\color{codeblue}\bfseries,
    commentstyle=\color{codegreen}\itshape,
    stringstyle=\color{codepurple},
    showstringspaces=false,
    breaklines=true,
    frame=single,
    rulecolor=\color{framecolor},
    backgroundcolor=\color{codebglight},
    numbers=left,
    numberstyle=\tiny\color{codegray},
    numbersep=8pt,
    tabsize=2,
    captionpos=b,
    literate=%
        {á}{{\'a}}1 {é}{{\'e}}1 {í}{{\'i}}1 {ó}{{\'o}}1 {ú}{{\'u}}1
        {Á}{{\'A}}1 {É}{{\'E}}1 {Í}{{\'I}}1 {Ó}{{\'O}}1 {Ú}{{\'U}}1
        {ñ}{{\~n}}1 {Ñ}{{\~N}}1
        {¿}{{?`}}1 {¡}{{!`}}1
}

% ============================================================
% INICIO DEL CÓDIGO DE COMPATIBILIDAD INTELIGENTE
% ============================================================

\IfFileExists{fvextra.sty}{%
    \usepackage{fvextra} % (carga fancyvrb)
    \DefineVerbatimEnvironment{codeblock}{Verbatim}{
        breaklines, breakanywhere,
        breaksymbol=\tiny\ensuremath{\hookrightarrow}\,,
        breaksymbolindentleft=0pt,
        fontsize=\small,
        frame=single, framerule=0.4pt, rulecolor=\color{black!40},
        tabsize=2
    }
    \DefineVerbatimEnvironment{codeplain}{Verbatim}{
        breaklines, breakanywhere, fontsize=\small, frame=none, tabsize=2
    }
    \DefineVerbatimEnvironment{codefine}{Verbatim}{
        breaklines, breakanywhere, fontsize=\small,
        frame=single, framerule=0.2pt, rulecolor=\color{black!20}, tabsize=2
    }
    \DefineVerbatimEnvironment{codebg}{Verbatim}{
        breaklines, breakanywhere, fontsize=\small,
        frame=single, framerule=0.4pt, rulecolor=\color{black!30},
        bgcolor=black!3, tabsize=2
    }
    \DefineVerbatimEnvironment{codelines}{Verbatim}{
        breaklines, breakanywhere, fontsize=\small, numbers=left,
        numberstyle=\tiny, numbersep=6pt,
        frame=single, framerule=0.4pt, rulecolor=\color{black!40}, tabsize=2
    }
    \DefineVerbatimEnvironment{codenosymbol}{Verbatim}{
        breaklines, breakanywhere, breaksymbol={}, fontsize=\small,
        frame=single, framerule=0.4pt, rulecolor=\color{black!40}, tabsize=2
    }
    \DefineVerbatimEnvironment{codegobbletwo}{Verbatim}{
        breaklines, breakanywhere, gobble=2,
        fontsize=\small, frame=single, framerule=0.4pt, rulecolor=\color{black!40},
        tabsize=2
    }
}{%
    % --- Fallback sin fvextra: tragarse opciones [..] y usar 'verbatim' ---
    \usepackage{verbatim}
    \newenvironment{codeblock}[1][]   {\verbatim}{\endverbatim}
    \newenvironment{codeplain}[1][]   {\verbatim}{\endverbatim}
    \newenvironment{codefine}[1][]    {\verbatim}{\endverbatim}
    \newenvironment{codebg}[1][]      {\verbatim}{\endverbatim}
    \newenvironment{codelines}[1][]   {\verbatim}{\endverbatim}
    \newenvironment{codenosymbol}[1][]{\verbatim}{\endverbatim}
    \newenvironment{codegobbletwo}[1][]{\verbatim}{\endverbatim}
}

% ============================================================
% FIN DEL CÓDIGO DE COMPATIBILIDAD INTELIGENTE
% ============================================================

% Título
\title{\textbf{\Huge Sistema de Compatibilidad Inteligente} \\[3mm]
\large Entornos Personalizados para Código en \LaTeX{}}
\author{}
\date{\today}

\usepackage[hidelinks]{hyperref}

\begin{document}

\maketitle
\tableofcontents
\newpage

\section{Introducción}

Este documento explica en detalle un sistema de \textbf{compatibilidad inteligente} para mostrar código en documentos \LaTeX. El sistema detecta automáticamente las capacidades disponibles en la instalación de \LaTeX{} y se adapta en consecuencia.

\subsection{¿Qué es la Compatibilidad Inteligente?}

La compatibilidad inteligente es un patrón de diseño que permite a un documento \LaTeX{} funcionar correctamente en diferentes entornos, independientemente de los paquetes instalados.

\begin{tcolorbox}[colback=blue!5,colframe=blue!75!black,title=Ventajas del Sistema]
\begin{itemize}[leftmargin=*]
    \item \textbf{Portabilidad}: El documento compila en cualquier instalación
    \item \textbf{Degradación elegante}: Sin paquetes avanzados, funciona con características básicas
    \item \textbf{Máxima calidad}: Con paquetes instalados, aprovecha todas las características
    \item \textbf{Transparencia}: El usuario no necesita modificar el código fuente
\end{itemize}
\end{tcolorbox}

\subsection{El Problema que Resuelve}

Cuando compartimos documentos \LaTeX, a menudo encontramos el problema de dependencias:

\begin{tcolorbox}[colback=red!5,colframe=red!75!black,title=Escenario Problemático]
\textbf{Usuario A} (con todos los paquetes): Crea un documento hermoso con \texttt{fvextra}

\textbf{Usuario B} (sin fvextra): Intenta compilar → \textcolor{red}{\textbf{ERROR}}

\textbf{Resultado}: Frustración, incompatibilidad, pérdida de tiempo
\end{tcolorbox}

\begin{tcolorbox}[colback=green!5,colframe=green!75!black,title=Solución con Compatibilidad Inteligente]
\textbf{Usuario A}: Crea documento con el sistema inteligente

\textbf{Usuario B}: Compila exitosamente (con características básicas)

\textbf{Resultado}: \textcolor{green!60!black}{\textbf{¡Funciona en ambos casos!}}
\end{tcolorbox}

\section{Estructura del Sistema}

\subsection{El Comando \textbackslash IfFileExists}

La base del sistema es el comando \texttt{\textbackslash IfFileExists}, que verifica la existencia de un archivo.

\begin{lstlisting}[caption={Sintaxis de \textbackslash IfFileExists}]
\IfFileExists{archivo}{
    % Código si el archivo EXISTE
}{
    % Código si el archivo NO existe (fallback)
}
\end{lstlisting}

\subsubsection{Parámetros Explicados}

\begin{description}[leftmargin=3cm,style=nextline]
    \item[\texttt{archivo}] Nombre del archivo a buscar (ej. \texttt{fvextra.sty})
    \item[\texttt{bloque verdadero}] Código ejecutado si el archivo existe
    \item[\texttt{bloque falso}] Código ejecutado si el archivo NO existe
\end{description}

\subsection{Aplicación en Nuestro Sistema}

Nuestro sistema verifica la existencia de \texttt{fvextra.sty}:

\begin{lstlisting}[caption={Estructura general del sistema}]
\IfFileExists{fvextra.sty}{
    % RAMA A: Con fvextra (características avanzadas)
    \usepackage{fvextra}
    \DefineVerbatimEnvironment{codeblock}{Verbatim}{...}
    % ... más definiciones avanzadas
}{
    % RAMA B: Sin fvextra (fallback básico)
    \usepackage{verbatim}
    \newenvironment{codeblock}[1][]{\verbatim}{\endverbatim}
    % ... más definiciones básicas
}
\end{lstlisting}

\section{Rama A: Con fvextra (Características Avanzadas)}

Cuando \texttt{fvextra.sty} está disponible, el sistema carga este paquete y define 7 entornos personalizados con capacidades avanzadas.

\subsection{¿Qué es fvextra?}

\texttt{fvextra} es una extensión del paquete \texttt{fancyvrb} (Fancy Verbatim) que añade:

\begin{itemize}
    \item \textbf{breaklines}: Romper líneas automáticamente
    \item \textbf{breakanywhere}: Romper en cualquier posición (no solo espacios)
    \item \textbf{breaksymbol}: Personalizar símbolo de continuación
    \item \textbf{bgcolor}: Color de fondo
    \item \textbf{frame}: Marcos alrededor del código
    \item Y muchas más opciones avanzadas
\end{itemize}

\subsection{El Comando \textbackslash DefineVerbatimEnvironment}

Este comando crea entornos personalizados basados en \texttt{Verbatim}:

\begin{lstlisting}[caption={Sintaxis de \textbackslash DefineVerbatimEnvironment}]
\DefineVerbatimEnvironment{nombre}{Verbatim}{
    opcion1=valor1,
    opcion2=valor2,
    ...
}
\end{lstlisting}

Después de definirlo, se puede usar:

\begin{lstlisting}
\begin{nombre}
    código aquí
\end{nombre}
\end{lstlisting}

\section{Los 7 Entornos Personalizados}

\subsection{Entorno 1: codeblock}

\textbf{Propósito}: Bloque de código estándar con marco y símbolo de continuación.

\subsubsection{Definición}

\begin{lstlisting}[caption={Definición de codeblock}]
\DefineVerbatimEnvironment{codeblock}{Verbatim}{
    breaklines,              % Rompe líneas largas
    breakanywhere,           % Puede romper en cualquier lugar
    breaksymbol=\tiny\ensuremath{\hookrightarrow}\,,
    breaksymbolindentleft=0pt,
    fontsize=\small,
    frame=single,
    framerule=0.4pt,
    rulecolor=\color{black!40},
    tabsize=2
}
\end{lstlisting}

\subsubsection{Opciones Explicadas}

\begin{description}[leftmargin=3.5cm,style=nextline]
    \item[\texttt{breaklines}] Activa el rompimiento automático de líneas largas
    \item[\texttt{breakanywhere}] Permite romper la línea en cualquier carácter, no solo en espacios
    \item[\texttt{breaksymbol}] Define el símbolo mostrado cuando se rompe una línea: {\tiny\ensuremath{\hookrightarrow}}
    \item[\texttt{breaksymbolindentleft}] Sangría del símbolo (0pt = sin sangría)
    \item[\texttt{fontsize}] Tamaño de la fuente (\texttt{\textbackslash small})
    \item[\texttt{frame}] Tipo de marco (\texttt{single} = línea simple)
    \item[\texttt{framerule}] Grosor del marco (0.4 puntos)
    \item[\texttt{rulecolor}] Color del marco (negro al 40\%)
    \item[\texttt{tabsize}] Espacios por tabulación (2)
\end{description}

\subsubsection{Ejemplo de Uso}

Código fuente:
\begin{lstlisting}
\begin{codeblock}
def función_muy_larga(param1, param2, param3, param4):
    return param1 + param2 + param3 + param4
\end{codeblock}
\end{lstlisting}

Resultado visual:
\begin{codeblock}
def función_muy_larga(param1, param2, param3, param4):
    return param1 + param2 + param3 + param4
\end{codeblock}

\subsection{Entorno 2: codeplain}

\textbf{Propósito}: Código simple sin marco, para uso discreto.

\subsubsection{Definición}

\begin{lstlisting}[caption={Definición de codeplain}]
\DefineVerbatimEnvironment{codeplain}{Verbatim}{
    breaklines,
    breakanywhere,
    fontsize=\small,
    frame=none,              % Sin marco
    tabsize=2
}
\end{lstlisting}

\subsubsection{Diferencia Clave}

La única diferencia importante con \texttt{codeblock} es \texttt{frame=none}, que elimina el marco alrededor del código.

\subsubsection{Ejemplo de Uso}

\begin{codeplain}
print("Este código no tiene marco alrededor")
for i in range(5):
    print(f"Línea {i}")
\end{codeplain}

\subsection{Entorno 3: codefine}

\textbf{Propósito}: Marco muy fino y sutil, para código que debe estar ligeramente delimitado.

\subsubsection{Definición}

\begin{lstlisting}[caption={Definición de codefine}]
\DefineVerbatimEnvironment{codefine}{Verbatim}{
    breaklines, breakanywhere,
    fontsize=\small,
    frame=single,
    framerule=0.2pt,         % Marco más fino (0.2 vs 0.4)
    rulecolor=\color{black!20}, % Color más claro (20% vs 40%)
    tabsize=2
}
\end{lstlisting}

\subsubsection{Comparación de Marcos}

\begin{table}[h]
\centering
\begin{tabular}{lcc}
\toprule
\textbf{Entorno} & \textbf{Grosor (pt)} & \textbf{Color (\% negro)} \\
\midrule
codeblock & 0.4 & 40\% \\
codefine & 0.2 & 20\% \\
\bottomrule
\end{tabular}
\caption{Comparación de estilos de marco}
\end{table}

\subsubsection{Ejemplo de Uso}

\begin{codefine}
SELECT nombre, edad, ciudad
FROM usuarios
WHERE edad > 18;
\end{codefine}

\subsection{Entorno 4: codebg}

\textbf{Propósito}: Código con fondo de color para máximo contraste visual.

\subsubsection{Definición}

\begin{lstlisting}[caption={Definición de codebg}]
\DefineVerbatimEnvironment{codebg}{Verbatim}{
    breaklines, breakanywhere,
    fontsize=\small,
    frame=single,
    framerule=0.4pt,
    rulecolor=\color{black!30},
    bgcolor=black!3,         % Fondo gris muy claro
    tabsize=2
}
\end{lstlisting}

\subsubsection{La Opción bgcolor}

\texttt{bgcolor=black!3} significa:
\begin{itemize}
    \item \textbf{black}: Color base (negro)
    \item \textbf{!3}: Mezcla al 3\% (97\% blanco + 3\% negro)
    \item \textbf{Resultado}: Gris muy claro, casi imperceptible
\end{itemize}

\subsubsection{Ejemplo de Uso}

\begin{codebg}
class Usuario:
    def __init__(self, nombre, edad):
        self.nombre = nombre
        self.edad = edad

    def saludar(self):
        return f"Hola, soy {self.nombre}"
\end{codebg}

\subsection{Entorno 5: codelines}

\textbf{Propósito}: Código con numeración de líneas, ideal para referencias.

\subsubsection{Definición}

\begin{lstlisting}[caption={Definición de codelines}]
\DefineVerbatimEnvironment{codelines}{Verbatim}{
    breaklines, breakanywhere,
    fontsize=\small,
    numbers=left,            % Números a la izquierda
    numberstyle=\tiny,       % Estilo de números
    numbersep=6pt,           % Separación números-código
    frame=single,
    framerule=0.4pt,
    rulecolor=\color{black!40},
    tabsize=2
}
\end{lstlisting}

\subsubsection{Opciones de Numeración}

\begin{description}[leftmargin=3cm,style=nextline]
    \item[\texttt{numbers}] Posición de números (\texttt{left}, \texttt{right}, o \texttt{none})
    \item[\texttt{numberstyle}] Estilo visual de los números (ej. \texttt{\textbackslash tiny})
    \item[\texttt{numbersep}] Distancia entre números y código (en puntos)
\end{description}

\subsubsection{Ejemplo de Uso}

\begin{codelines}
def factorial(n):
    if n == 0:
        return 1
    return n * factorial(n-1)

resultado = factorial(5)
print(f"5! = {resultado}")
\end{codelines}

\textbf{Uso práctico}: ``En la línea 4 del código anterior se ve la recursión...''

\subsection{Entorno 6: codenosymbol}

\textbf{Propósito}: Como \texttt{codeblock} pero sin símbolo de continuación.

\subsubsection{Definición}

\begin{lstlisting}[caption={Definición de codenosymbol}]
\DefineVerbatimEnvironment{codenosymbol}{Verbatim}{
    breaklines, breakanywhere,
    breaksymbol={},          % Símbolo vacío
    fontsize=\small,
    frame=single,
    framerule=0.4pt,
    rulecolor=\color{black!40},
    tabsize=2
}
\end{lstlisting}

\subsubsection{¿Cuándo usar codenosymbol?}

\begin{itemize}
    \item Cuando el símbolo ↪ puede confundir
    \item En código donde la continuación debe ser invisible
    \item Para imitar el estilo de algunos IDEs
\end{itemize}

\subsubsection{Comparación}

\textbf{Con símbolo (codeblock):}
\begin{codeblock}
Esta es una línea extremadamente larga que definitivamente necesitará romperse en múltiples líneas para caber en el ancho de página
\end{codeblock}

\textbf{Sin símbolo (codenosymbol):}
\begin{codenosymbol}
Esta es una línea extremadamente larga que definitivamente necesitará romperse en múltiples líneas para caber en el ancho de página
\end{codenosymbol}

\subsection{Entorno 7: codegobbletwo}

\textbf{Propósito}: Elimina los primeros 2 caracteres de cada línea (útil para código indentado).

\subsubsection{Definición}

\begin{lstlisting}[caption={Definición de codegobbletwo}]
\DefineVerbatimEnvironment{codegobbletwo}{Verbatim}{
    breaklines, breakanywhere,
    gobble=2,                % Elimina 2 caracteres al inicio
    fontsize=\small,
    frame=single,
    framerule=0.4pt,
    rulecolor=\color{black!40},
    tabsize=2
}
\end{lstlisting}

\subsubsection{La Opción gobble}

\texttt{gobble=n} elimina los primeros \texttt{n} caracteres de cada línea. Esto es útil cuando:

\begin{itemize}
    \item El código está indentado en el archivo .tex
    \item Quieres que se vea sin esa indentación inicial
    \item Necesitas alinear código con el texto circundante
\end{itemize}

\subsubsection{Ejemplo Sin gobble}

Si escribimos:
\begin{lstlisting}
\begin{codeblock}
  def hello():
      print("Hi")
\end{codeblock}
\end{lstlisting}

Resultado (nota los 2 espacios extra):
\begin{codeblock}
  def hello():
      print("Hi")
\end{codeblock}

\subsubsection{Ejemplo Con gobble=2}

Si escribimos:
\begin{lstlisting}
\begin{codegobbletwo}
  def hello():
      print("Hi")
\end{codegobbletwo}
\end{lstlisting}

Resultado (espacios iniciales eliminados):
\begin{codegobbletwo}
  def hello():
      print("Hi")
\end{codegobbletwo}

\section{Rama B: Sin fvextra (Fallback)}

Cuando \texttt{fvextra} NO está disponible, el sistema usa una estrategia de degradación elegante.

\subsection{Código del Fallback}

\begin{lstlisting}[caption={Implementación del fallback}]
}{%
    % --- Fallback sin fvextra ---
    \usepackage{verbatim}
    \newenvironment{codeblock}[1][]{\verbatim}{\endverbatim}
    \newenvironment{codeplain}[1][]{\verbatim}{\endverbatim}
    \newenvironment{codefine}[1][]{\verbatim}{\endverbatim}
    \newenvironment{codebg}[1][]{\verbatim}{\endverbatim}
    \newenvironment{codelines}[1][]{\verbatim}{\endverbatim}
    \newenvironment{codenosymbol}[1][]{\verbatim}{\endverbatim}
    \newenvironment{codegobbletwo}[1][]{\verbatim}{\endverbatim}
}
\end{lstlisting}

\subsection{Análisis Detallado}

\subsubsection{Carga del Paquete Básico}

\begin{lstlisting}
\usepackage{verbatim}
\end{lstlisting}

El paquete \texttt{verbatim} es parte del núcleo de \LaTeX{} y siempre está disponible. Proporciona el entorno básico \texttt{verbatim} sin características avanzadas.

\subsubsection{Definición de Entornos Simples}

\begin{lstlisting}
\newenvironment{codeblock}[1][]{\verbatim}{\endverbatim}
\end{lstlisting}

Esta línea:
\begin{enumerate}
    \item \texttt{\textbackslash newenvironment\{codeblock\}} - Crea un nuevo entorno llamado \texttt{codeblock}
    \item \texttt{[1][]} - Acepta 1 argumento opcional (que será ignorado)
    \item \texttt{\{\textbackslash verbatim\}} - Al iniciar, ejecuta \texttt{\textbackslash verbatim}
    \item \texttt{\{\textbackslash endverbatim\}} - Al terminar, ejecuta \texttt{\textbackslash endverbatim}
\end{enumerate}

\subsubsection{¿Por qué [1][]?}

El \texttt{[1][]} es crucial para la compatibilidad:

\begin{tcolorbox}[colback=yellow!10,colframe=orange!75!black,title=Sin el argumento opcional]
Si alguien escribe:
\begin{lstlisting}
\begin{codeblock}[alguna opción]
código
\end{codeblock}
\end{lstlisting}

\textcolor{red}{\textbf{ERROR}}: El entorno no acepta argumentos
\end{tcolorbox}

\begin{tcolorbox}[colback=green!10,colframe=green!75!black,title=Con [1][]]
Si alguien escribe:
\begin{lstlisting}
\begin{codeblock}[alguna opción]
código
\end{codeblock}
\end{lstlisting}

\textcolor{green!60!black}{\textbf{FUNCIONA}}: El argumento se acepta y se ignora silenciosamente
\end{tcolorbox}

\subsection{Limitaciones del Fallback}

\begin{table}[h]
\centering
\begin{tabular}{lcc}
\toprule
\textbf{Característica} & \textbf{Con fvextra} & \textbf{Sin fvextra} \\
\midrule
Romper líneas largas & ✓ & ✗ \\
Marco alrededor & ✓ & ✗ \\
Color de fondo & ✓ & ✗ \\
Numeración de líneas & ✓ & ✗ \\
Símbolo de continuación & ✓ & ✗ \\
Opción gobble & ✓ & ✗ \\
Texto monoespacio & ✓ & ✓ \\
\bottomrule
\end{tabular}
\caption{Comparación de características}
\end{table}

\section{Comparación Visual de Entornos}

\subsection{Tabla Resumen}

\begin{table}[h]
\centering
\small
\begin{tabular}{lcccc}
\toprule
\textbf{Entorno} & \textbf{Marco} & \textbf{Fondo} & \textbf{Números} & \textbf{Símbolo} \\
\midrule
codeblock & ✓ (0.4pt) & ✗ & ✗ & ✓ \\
codeplain & ✗ & ✗ & ✗ & ✓ \\
codefine & ✓ (0.2pt) & ✗ & ✗ & ✓ \\
codebg & ✓ & ✓ & ✗ & ✓ \\
codelines & ✓ & ✗ & ✓ & ✓ \\
codenosymbol & ✓ & ✗ & ✗ & ✗ \\
codegobbletwo & ✓ & ✗ & ✗ & ✓ \\
\bottomrule
\end{tabular}
\caption{Características de cada entorno}
\end{table}

\subsection{Guía de Selección}

\begin{description}[leftmargin=4cm,style=nextline]
    \item[codeblock] Uso general, código estándar
    \item[codeplain] Código inline discreto
    \item[codefine] Código con separación sutil
    \item[codebg] Código que debe destacar
    \item[codelines] Código para referencias por línea
    \item[codenosymbol] Continuación invisible
    \item[codegobbletwo] Código indentado en fuente
\end{description}

\section{Mejores Prácticas}

\subsection{Cuándo Usar Cada Entorno}

\subsubsection{Para Código de Ejemplo}
Use \texttt{codeblock} o \texttt{codelines}:
\begin{codelines}
# Ejemplo de algoritmo
def buscar(lista, objetivo):
    for i, elemento in enumerate(lista):
        if elemento == objetivo:
            return i
    return -1
\end{codelines}

\subsubsection{Para Código Inline en Párrafos}
Use \texttt{codeplain} cuando quiera código sin interrumpir el flujo visual:

El siguiente código muestra una función simple:
\begin{codeplain}
lambda x: x * 2
\end{codeplain}
que duplica su entrada.

\subsubsection{Para Código Destacado}
Use \texttt{codebg} cuando necesite máxima visibilidad:
\begin{codebg}
¡ATENCIÓN! Este código es crítico para la seguridad:
if user.is_authenticated():
    grant_access()
\end{codebg}

\subsection{Errores Comunes}

\begin{tcolorbox}[colback=red!5,colframe=red!75!black,title=❌ Error: Intentar usar comandos dentro]
\begin{lstlisting}
\begin{codeblock}
print("Hola \textbf{mundo}")  % ¡NO funciona!
\end{codeblock}
\end{lstlisting}

\textbf{Problema}: Los entornos verbatim no procesan comandos \LaTeX.
\end{tcolorbox}

\begin{tcolorbox}[colback=green!5,colframe=green!75!black,title=✓ Correcto: Usar listings para resaltado]
\begin{lstlisting}
\begin{lstlisting}[language=Python]
print("Hola mundo")  # Con resaltado de sintaxis
\end{lstlisting}
\verb|\end{lstlisting}|
\end{tcolorbox}

\subsection{Combinación con Otros Paquetes}

\subsubsection{Con tcolorbox}
\begin{lstlisting}
\begin{tcolorbox}[title=Código de Ejemplo]
\begin{codeblock}
def ejemplo():
    pass
\end{codeblock}
\end{tcolorbox}
\end{lstlisting}

\subsubsection{Con float para Posicionamiento}
\begin{lstlisting}
\begin{figure}[h]
\begin{codeblock}
código aquí
\end{codeblock}
\caption{Código del algoritmo principal}
\end{figure}
\end{lstlisting}

\section{Implementación Paso a Paso}

\subsection{Cómo Integrar en tu Documento}

\textbf{Paso 1}: Copia todo el código de compatibilidad (desde \texttt{\textbackslash IfFileExists} hasta el cierre) en tu preámbulo.

\textbf{Paso 2}: Usa los entornos libremente:

\begin{lstlisting}
\begin{document}

\section{Mi Código}

\begin{codelines}
def mi_funcion():
    return "Funciona"
\end{codelines}

\end{document}
\end{lstlisting}

\textbf{Paso 3}: Compila normalmente. El sistema detecta automáticamente las capacidades.

\subsection{Prueba de Compatibilidad}

Para verificar qué rama se está usando, puedes agregar:

\begin{lstlisting}
\IfFileExists{fvextra.sty}{
    \typeout{*** Sistema usando fvextra (avanzado) ***}
}{
    \typeout{*** Sistema usando fallback (básico) ***}
}
\end{lstlisting}

Este mensaje aparecerá en el log de compilación.

\section{Extensiones Posibles}

\subsection{Añadir Más Entornos}

Puedes extender el sistema con nuevos entornos:

\begin{lstlisting}
% En la rama con fvextra:
\DefineVerbatimEnvironment{codebigfont}{Verbatim}{
    breaklines, breakanywhere,
    fontsize=\large,  % Fuente más grande
    frame=single, framerule=1pt, rulecolor=\color{blue!60},
    tabsize=2
}

% En el fallback:
\newenvironment{codebigfont}[1][]{\verbatim}{\endverbatim}
\end{lstlisting}

\subsection{Añadir Colores Personalizados}

\begin{lstlisting}
\definecolor{myblue}{RGB}{30,144,255}

\DefineVerbatimEnvironment{codeazul}{Verbatim}{
    breaklines, breakanywhere,
    fontsize=\small,
    frame=single, rulecolor=\color{myblue},
    tabsize=2
}
\end{lstlisting}

\subsection{Verificar Múltiples Paquetes}

Puedes anidar \texttt{\textbackslash IfFileExists} para verificar múltiples dependencias:

\begin{lstlisting}
\IfFileExists{fvextra.sty}{
    \IfFileExists{minted.sty}{
        % Usar lo mejor de ambos
    }{
        % Solo fvextra
    }
}{
    % Fallback básico
}
\end{lstlisting}

\section{Casos de Uso Reales}

\subsection{Documentación de Software}

Para manuales de usuario con ejemplos de código:

\begin{tcolorbox}[colback=blue!5,colframe=blue!75!black,title=Instalación]
Para instalar el paquete, ejecute:

\begin{codeblock}
pip install mi-paquete
python setup.py install
\end{codeblock}

Luego importe en su código:

\begin{codelines}
import mi_paquete
resultado = mi_paquete.procesar(datos)
print(resultado)
\end{codelines}
\end{tcolorbox}

\subsection{Artículos Académicos}

Para papers con pseudocódigo:

\begin{codelines}
Algoritmo: Búsqueda Binaria
Entrada: array ordenado A, elemento x
Salida: índice de x en A, o -1

bajo = 0
alto = longitud(A) - 1
mientras bajo <= alto:
    medio = (bajo + alto) / 2
    si A[medio] == x:
        retornar medio
    si A[medio] < x:
        bajo = medio + 1
    sino:
        alto = medio - 1
retornar -1
\end{codelines}

\subsection{Tutoriales y Libros}

Para contenido educativo con muchos ejemplos:

\textbf{Ejemplo 1}: Bucle básico
\begin{codebg}
for i in range(10):
    print(i)
\end{codebg}

\textbf{Ejemplo 2}: Función con parámetros
\begin{codebg}
def saludar(nombre, edad):
    return f"{nombre} tiene {edad} años"
\end{codebg}

\section{Conclusiones}

\subsection{Beneficios del Sistema}

\begin{enumerate}
    \item \textbf{Robustez}: Funciona en cualquier instalación \LaTeX
    \item \textbf{Flexibilidad}: 7 estilos diferentes de código
    \item \textbf{Mantenibilidad}: Código centralizado fácil de modificar
    \item \textbf{Escalabilidad}: Fácil de extender con nuevos entornos
    \item \textbf{Transparencia}: Los usuarios finales no necesitan conocer los detalles
\end{enumerate}

\subsection{Cuándo NO Usar Este Sistema}

\begin{itemize}
    \item Si necesitas resaltado de sintaxis avanzado → usa \texttt{minted}
    \item Si todos los usuarios tienen los mismos paquetes → define directamente
    \item Si el documento es solo para ti → no necesitas compatibilidad
\end{itemize}

\subsection{Recursos Adicionales}

\begin{itemize}
    \item \textbf{fancyvrb}: \texttt{texdoc fancyvrb}
    \item \textbf{fvextra}: \texttt{texdoc fvextra}
    \item \textbf{verbatim}: \texttt{texdoc verbatim}
    \item \textbf{listings}: \texttt{texdoc listings}
\end{itemize}

\section{Apéndice: Referencia Rápida}

\subsection{Opciones de fvextra}

\begin{table}[h]
\centering
\small
\begin{tabular}{lp{8cm}}
\toprule
\textbf{Opción} & \textbf{Descripción} \\
\midrule
breaklines & Rompe líneas automáticamente \\
breakanywhere & Puede romper en cualquier carácter \\
breaksymbol & Símbolo mostrado al romper líneas \\
breaksymbolindentleft & Sangría del símbolo de ruptura \\
fontsize & Tamaño de fuente (\textbackslash small, \textbackslash large, etc.) \\
frame & Tipo de marco (single, none, leftline, etc.) \\
framerule & Grosor del marco \\
rulecolor & Color del marco \\
bgcolor & Color de fondo \\
numbers & Posición de números (left, right, none) \\
numberstyle & Estilo de los números \\
numbersep & Separación números-código \\
tabsize & Espacios por tabulación \\
gobble & Caracteres a eliminar al inicio \\
\bottomrule
\end{tabular}
\caption{Opciones principales de fvextra}
\end{table}

\subsection{Sintaxis Completa}

\begin{lstlisting}[caption={Plantilla completa del sistema}]
\IfFileExists{fvextra.sty}{%
    \usepackage{fvextra}
    \DefineVerbatimEnvironment{micodigo}{Verbatim}{
        breaklines,
        breakanywhere,
        fontsize=\small,
        frame=single,
        framerule=0.4pt,
        rulecolor=\color{black!40},
        tabsize=2
    }
}{%
    \usepackage{verbatim}
    \newenvironment{micodigo}[1][]{\verbatim}{\endverbatim}
}
\end{lstlisting}

\vspace{2cm}

\begin{center}
\large
\textit{Documento generado con \LaTeX{} usando el sistema de compatibilidad inteligente}

\textit{\today}
\end{center}

\end{document}
