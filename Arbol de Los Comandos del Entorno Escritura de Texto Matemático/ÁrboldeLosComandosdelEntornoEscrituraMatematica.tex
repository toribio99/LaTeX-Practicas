\documentclass[10pt,a4paper]{article}

% Paquetes necesarios
\usepackage[utf8]{inputenc}
\usepackage[spanish]{babel}
\usepackage[margin=1.5cm]{geometry}
\usepackage{xcolor}
\usepackage{tcolorbox}
\usepackage{listings}
\tcbuselibrary{listings,skins,breakable}  % Librerías para tcblisting
\usepackage{enumitem}
\usepackage{fontawesome5}
\usepackage{listings}
\usepackage{amsmath}
\usepackage{amssymb}
\usepackage{amsfonts}
\usepackage{mathtools}
\usepackage{amsthm}
\usepackage{bm}
\usepackage{mathrsfs}
\usepackage{dsfont}
\usepackage{multicol}


% Colores personalizados
\definecolor{commandcolor}{RGB}{39,174,96}

% Configuración de listings
\lstset{
basicstyle=\ttfamily\footnotesize,
breaklines=true,
columns=fullflexible,
keepspaces=true,
escapechar=@
}

% Título
\title{\textbf{\Huge Modo Matemático en \LaTeX{}}\\\large Guía Completa de Entornos, Comandos y Símbolos}
\author{}
\date{\today}
 \usepackage[
%colorlinks=true,        % Enlaces con color (en lugar de cajas)
linkcolor=blue,         % Color de enlaces internos
urlcolor=cyan,          % Color de URLs
citecolor=green,        % Color de citas bibliográficas
filecolor=magenta,      % Color de enlaces a archivos
pdfborder={0 0 0},      % Sin bordes en los enlaces
bookmarks=true,         % Crear marcadores en el PDF
bookmarksopen=true,     % Marcadores expandidos al abrir
pdftitle={Mi Título},   % Título del PDF
pdfauthor={Mi Nombre},  % Autor del PDF
pdfsubject={Tema},      % Tema del documento
pdfkeywords={palabra1, palabra2}, % Palabras clave
%hidelinks,              % Ocultar todos los bordes/colores de enlaces
unicode=true,           % Permitir caracteres Unicode en marcadores
breaklinks=true         % Permitir saltos de línea en enlaces
]{hyperref}

\begin{document}

\maketitle

\begin{tcolorbox}[colback=blue!5,colframe=blue!75!black,title=\faInfoCircle\ Introducción]
	\LaTeX{} proporciona un sistema completo para escribir expresiones matemáticas de alta calidad. Este documento cubre todos los entornos, comandos, símbolos y paquetes disponibles para matemáticas.
\end{tcolorbox}

\tableofcontents
%\newpage

\section{Paquetes Matemáticos}

\subsection{Paquetes Esenciales}

\subsubsection*{\texttt{\textbackslash usepackage\{amsmath\}}}
\begin{tcolorbox}[colback=green!5,colframe=green!50!black]
	\textbf{Descripción:} Paquete fundamental de AMS (American Mathematical Society)
	
	\textbf{Proporciona:} Entornos mejorados para ecuaciones, alineación, matrices, casos
	
	\textbf{Ejemplo:}
	\begin{lstlisting}[language=TeX]
		\usepackage{amsmath}
	\end{lstlisting}
\end{tcolorbox}

\subsubsection*{\texttt{\textbackslash usepackage\{amssymb\}}}
\begin{tcolorbox}[colback=green!5,colframe=green!50!black]
	\textbf{Descripción:} Símbolos matemáticos adicionales de AMS
	
	\textbf{Proporciona:} Símbolos especiales, flechas, relaciones
	
	\textbf{Ejemplo:}
	\begin{lstlisting}[language=TeX]
		\usepackage{amssymb}
	\end{lstlisting}
\end{tcolorbox}

\subsubsection*{\texttt{\textbackslash usepackage\{amsfonts\}}}
\begin{tcolorbox}[colback=green!5,colframe=green!50!black]
	\textbf{Descripción:} Fuentes matemáticas adicionales
	
	\textbf{Proporciona:} Fuentes blackboard bold, fraktur
	
	\textbf{Ejemplo:}
	\begin{lstlisting}[language=TeX]
		\usepackage{amsfonts}
	\end{lstlisting}
\end{tcolorbox}

\subsubsection*{\texttt{\textbackslash usepackage\{mathtools\}}}
\begin{tcolorbox}[colback=green!5,colframe=green!50!black]
	\textbf{Descripción:} Extensión de amsmath con correcciones y mejoras
	
	\textbf{Proporciona:} Comandos mejorados, delimitadores ajustables
	
	\textbf{Ejemplo:}
	\begin{lstlisting}[language=TeX]
		\usepackage{mathtools}
	\end{lstlisting}
\end{tcolorbox}

\subsubsection*{\texttt{\textbackslash usepackage\{amsthm\}}}
\begin{tcolorbox}[colback=green!5,colframe=green!50!black]
	\textbf{Descripción:} Entornos para teoremas, definiciones, pruebas
	
	\textbf{Ejemplo:}
	\begin{lstlisting}[language=TeX]
		\usepackage{amsthm}
		\newtheorem{theorem}{Teorema}
	\end{lstlisting}
\end{tcolorbox}

\subsection{Paquetes Adicionales}

\subsubsection*{\texttt{\textbackslash usepackage\{bm\}}}
\begin{tcolorbox}[colback=green!5,colframe=green!50!black]
	\textbf{Descripción:} Negrita matemática mejorada
	
	\textbf{Ejemplo:}
	\begin{lstlisting}[language=TeX]
		\usepackage{bm}
		$\bm{x} = \bm{\alpha}$
	\end{lstlisting}
\end{tcolorbox}

\subsubsection*{\texttt{\textbackslash usepackage\{mathrsfs\}}}
\begin{tcolorbox}[colback=green!5,colframe=green!50!black]
	\textbf{Descripción:} Fuente script (caligráfica)
	
	\textbf{Ejemplo:}
	\begin{lstlisting}[language=TeX]
		\usepackage{mathrsfs}
		$\mathscr{L}$
	\end{lstlisting}
\end{tcolorbox}

\subsubsection*{\texttt{\textbackslash usepackage\{dsfont\}}}
\begin{tcolorbox}[colback=green!5,colframe=green!50!black]
	\textbf{Descripción:} Fuente doublestroke (para indicadoras)
	
	\textbf{Ejemplo:}
	\begin{lstlisting}[language=TeX]
		\usepackage{dsfont}
		$\mathds{1}, \mathds{R}$
	\end{lstlisting}
\end{tcolorbox}

\section{Modos Matemáticos}

\subsection{Modo Matemático en Línea (Inline)}

\subsubsection*{\texttt{\$...\$}}
\begin{tcolorbox}[colback=green!5,colframe=green!50!black]
	\textbf{Descripción:} Matemáticas en línea con el texto
	
	\textbf{Ejemplo:}
	\begin{lstlisting}[language=TeX]
		La ecuación $E = mc^2$ es famosa.
	\end{lstlisting}
	
	\textbf{Resultado:} La ecuación $E = mc^2$ es famosa.
\end{tcolorbox}

\subsubsection*{\texttt{\textbackslash(...\textbackslash)}}
\begin{tcolorbox}[colback=green!5,colframe=green!50!black]
	\textbf{Descripción:} Alternativa a \$...\$ (recomendada en LaTeX)
	
	\textbf{Ejemplo:}
	\begin{lstlisting}[language=TeX]
		La fórmula \( a^2 + b^2 = c^2 \) es el teorema de Pitágoras.
	\end{lstlisting}
	
	\textbf{Resultado:} La fórmula \( a^2 + b^2 = c^2 \) es el teorema de Pitágoras.
\end{tcolorbox}

\subsection{Modo Matemático Display (Ecuaciones Destacadas)}

\subsubsection*{\texttt{\$\$...\$\$}}
\begin{tcolorbox}[colback=green!5,colframe=green!50!black]
	\textbf{Descripción:} Ecuación centrada sin número (TeX plano, no recomendado)
	
	\textbf{Ejemplo:}
	\begin{lstlisting}[language=TeX]
		$$E = mc^2$$
	\end{lstlisting}
	
	\textbf{Nota:} Preferir \texttt{\textbackslash[...\textbackslash]} en LaTeX
\end{tcolorbox}

\subsubsection*{\texttt{\textbackslash[...\textbackslash]}}
\begin{tcolorbox}[colback=green!5,colframe=green!50!black]
	\textbf{Descripción:} Ecuación centrada sin número (recomendado)
	
	\textbf{Ejemplo:}
	\begin{lstlisting}[language=TeX]
		\[ \int_0^\infty e^{-x^2} dx = \frac{\sqrt{\pi}}{2} \]
	\end{lstlisting}
	
	\textbf{Resultado:}
	\[ \int_0^\infty e^{-x^2} dx = \frac{\sqrt{\pi}}{2} \]
\end{tcolorbox}

\section{Entornos de Ecuaciones}

\subsection{Ecuación Simple}

\subsubsection*{\texttt{equation}}
\begin{tcolorbox}[colback=green!5,colframe=green!50!black]
	\textbf{Descripción:} Ecuación numerada automáticamente
	
	\textbf{Ejemplo:}
	\begin{lstlisting}[language=TeX]
		\begin{equation}
			E = mc^2
		\end{equation}
	\end{lstlisting}
	
	\textbf{Resultado:}
	\begin{equation}
		E = mc^2
	\end{equation}
\end{tcolorbox}

\subsubsection*{\texttt{equation*}}
\begin{tcolorbox}[colback=green!5,colframe=green!50!black]
	\textbf{Descripción:} Ecuación sin número (requiere amsmath)
	
	\textbf{Ejemplo:}
	\begin{lstlisting}[language=TeX]
		\begin{equation*}
			a^2 + b^2 = c^2
		\end{equation*}
	\end{lstlisting}
	
	\textbf{Resultado:}
	\begin{equation*}
		a^2 + b^2 = c^2
	\end{equation*}
\end{tcolorbox}

\subsection{Alineación de Ecuaciones}

\subsubsection*{\texttt{align}}
\begin{tcolorbox}[colback=green!5,colframe=green!50!black]
	\textbf{Descripción:} Múltiples ecuaciones alineadas (numeradas)
	
	\textbf{Ejemplo:}
	\begin{lstlisting}[language=TeX]
		\begin{align}
			x &= a + b \\
			y &= c + d \\
			z &= e + f
		\end{align}
	\end{lstlisting}
	
	\textbf{Resultado:}
	\begin{align}
		x &= a + b \\
		y &= c + d \\
		z &= e + f
	\end{align}
	
	\textbf{Nota:} \& marca el punto de alineación, \textbackslash\textbackslash{} separa líneas
\end{tcolorbox}

\subsubsection*{\texttt{align*}}
\begin{tcolorbox}[colback=green!5,colframe=green!50!black]
	\textbf{Descripción:} Alineación sin numeración
	
	\textbf{Ejemplo:}
	\begin{lstlisting}[language=TeX]
		\begin{align*}
			f(x) &= x^2 + 2x + 1 \\
			&= (x+1)^2
		\end{align*}
	\end{lstlisting}
	
	\textbf{Resultado:}
	\begin{align*}
		f(x) &= x^2 + 2x + 1 \\
		&= (x+1)^2
	\end{align*}
\end{tcolorbox}

\subsubsection*{\texttt{aligned}}
\begin{tcolorbox}[colback=green!5,colframe=green!50!black]
	\textbf{Descripción:} Alineación dentro de otro entorno
	
	\textbf{Ejemplo:}
	\begin{lstlisting}[language=TeX]
		\begin{equation}
			\begin{aligned}
				x &= a + b \\
				y &= c + d
			\end{aligned}
		\end{equation}
	\end{lstlisting}
	
	\textbf{Resultado:}
	\begin{equation}
		\begin{aligned}
			x &= a + b \\
			y &= c + d
		\end{aligned}
	\end{equation}
\end{tcolorbox}

\subsubsection*{\texttt{alignat}}
\begin{tcolorbox}[colback=green!5,colframe=green!50!black]
	\textbf{Descripción:} Múltiples alineaciones controladas
	
	\textbf{Ejemplo:}
	\begin{lstlisting}[language=TeX]
		\begin{alignat}{2}
			x &= a &&+ b \\
			y &= c &&+ d
		\end{alignat}
	\end{lstlisting}
	
	\textbf{Resultado:}
	\begin{alignat}{2}
		x &= a &&+ b \\
		y &= c &&+ d
	\end{alignat}
	
	\textbf{Nota:} El número indica pares de alineación
\end{tcolorbox}

\subsubsection*{\texttt{flalign}}
\begin{tcolorbox}[colback=green!5,colframe=green!50!black]
	\textbf{Descripción:} Alineación extendida al ancho completo
	
	\textbf{Ejemplo:}
	\begin{lstlisting}[language=TeX]
		\begin{flalign}
			x &= a + b & \\
			y &= c + d &
		\end{flalign}
	\end{lstlisting}
\end{tcolorbox}

\subsection{Ecuaciones Agrupadas}

\subsubsection*{\texttt{gather}}
\begin{tcolorbox}[colback=green!5,colframe=green!50!black]
	\textbf{Descripción:} Ecuaciones centradas sin alineación
	
	\textbf{Ejemplo:}
	\begin{lstlisting}[language=TeX]
		\begin{gather}
			a = b + c \\
			x = y + z \\
			p = q + r
		\end{gather}
	\end{lstlisting}
	
	\textbf{Resultado:}
	\begin{gather}
		a = b + c \\
		x = y + z \\
		p = q + r
	\end{gather}
\end{tcolorbox}

\subsubsection*{\texttt{gather*}}
\begin{tcolorbox}[colback=green!5,colframe=green!50!black]
	\textbf{Descripción:} Gather sin numeración
	
	\textbf{Ejemplo:}
	\begin{lstlisting}[language=TeX]
		\begin{gather*}
			E = mc^2 \\
			F = ma
		\end{gather*}
	\end{lstlisting}
	
	\textbf{Resultado:}
	\begin{gather*}
		E = mc^2 \\
		F = ma
	\end{gather*}
\end{tcolorbox}

\subsection{Ecuaciones Multilínea}

\subsubsection*{\texttt{multline}}
\begin{tcolorbox}[colback=green!5,colframe=green!50!black]
	\textbf{Descripción:} Ecuación larga dividida en varias líneas
	
	\textbf{Ejemplo:}
	\begin{lstlisting}[language=TeX]
		\begin{multline}
			p(x) = a_0 + a_1 x + a_2 x^2 + a_3 x^3 + \\
			a_4 x^4 + a_5 x^5 + a_6 x^6
		\end{multline}
	\end{lstlisting}
	
	\textbf{Resultado:}
	\begin{multline}
		p(x) = a_0 + a_1 x + a_2 x^2 + a_3 x^3 + \\
		a_4 x^4 + a_5 x^5 + a_6 x^6
	\end{multline}
	
	\textbf{Nota:} Primera línea izquierda, última derecha, intermedias centradas
\end{tcolorbox}

\subsubsection*{\texttt{split}}
\begin{tcolorbox}[colback=green!5,colframe=green!50!black]
	\textbf{Descripción:} Divide ecuación con alineación (dentro de equation)
	
	\textbf{Ejemplo:}
	\begin{lstlisting}[language=TeX]
		\begin{equation}
			\begin{split}
				f(x) &= (x+1)^2 \\
				&= x^2 + 2x + 1
			\end{split}
		\end{equation}
	\end{lstlisting}
	
	\textbf{Resultado:}
	\begin{equation}
		\begin{split}
			f(x) &= (x+1)^2 \\
			&= x^2 + 2x + 1
		\end{split}
	\end{equation}
\end{tcolorbox}

\section{Matrices}

\subsection{Entornos de Matrices}

\subsubsection*{\texttt{matrix}}
\begin{tcolorbox}[colback=green!5,colframe=green!50!black]
	\textbf{Descripción:} Matriz sin delimitadores
	
	\textbf{Ejemplo:}
	\begin{lstlisting}[language=TeX]
		\begin{equation}
			\begin{matrix}
				a & b \\
				c & d
			\end{matrix}
		\end{equation}
	\end{lstlisting}
	
	\textbf{Resultado:}
	\begin{equation}
		\begin{matrix}
			a & b \\
			c & d
		\end{matrix}
	\end{equation}
\end{tcolorbox}

\subsubsection*{\texttt{pmatrix}}
\begin{tcolorbox}[colback=green!5,colframe=green!50!black]
	\textbf{Descripción:} Matriz con paréntesis ( )
	
	\textbf{Ejemplo:}
	\begin{lstlisting}[language=TeX]
		\begin{equation}
			\begin{pmatrix}
				1 & 2 & 3 \\
				4 & 5 & 6 \\
				7 & 8 & 9
			\end{pmatrix}
		\end{equation}
	\end{lstlisting}
	
	\textbf{Resultado:}
	\begin{equation}
		\begin{pmatrix}
			1 & 2 & 3 \\
			4 & 5 & 6 \\
			7 & 8 & 9
		\end{pmatrix}
	\end{equation}
\end{tcolorbox}

\subsubsection*{\texttt{bmatrix}}
\begin{tcolorbox}[colback=green!5,colframe=green!50!black]
	\textbf{Descripción:} Matriz con corchetes [ ]
	
	\textbf{Ejemplo:}
	\begin{lstlisting}[language=TeX]
		\[
		\begin{bmatrix}
			a_{11} & a_{12} \\
			a_{21} & a_{22}
		\end{bmatrix}
		\]
	\end{lstlisting}
	
	\textbf{Resultado:}
	\[
	\begin{bmatrix}
		a_{11} & a_{12} \\
		a_{21} & a_{22}
	\end{bmatrix}
	\]
\end{tcolorbox}

\subsubsection*{\texttt{Bmatrix}}
\begin{tcolorbox}[colback=green!5,colframe=green!50!black]
	\textbf{Descripción:} Matriz con llaves \{ \}
	
	\textbf{Ejemplo:}
	\begin{lstlisting}[language=TeX]
		\[
		\begin{Bmatrix}
			1 & 0 \\
			0 & 1
		\end{Bmatrix}
		\]
	\end{lstlisting}
	
	\textbf{Resultado:}
	\[
	\begin{Bmatrix}
		1 & 0 \\
		0 & 1
	\end{Bmatrix}
	\]
\end{tcolorbox}

\subsubsection*{\texttt{vmatrix}}
\begin{tcolorbox}[colback=green!5,colframe=green!50!black]
	\textbf{Descripción:} Matriz con barras | | (determinante)
	
	\textbf{Ejemplo:}
	\begin{lstlisting}[language=TeX]
		\[
		\begin{vmatrix}
			a & b \\
			c & d
		\end{vmatrix} = ad - bc
		\]
	\end{lstlisting}
	
	\textbf{Resultado:}
	\[
	\begin{vmatrix}
		a & b \\
		c & d
	\end{vmatrix} = ad - bc
	\]
\end{tcolorbox}

\subsubsection*{\texttt{Vmatrix}}
\begin{tcolorbox}[colback=green!5,colframe=green!50!black]
	\textbf{Descripción:} Matriz con barras dobles || || (norma)
	
	\textbf{Ejemplo:}
	\begin{lstlisting}[language=TeX]
		\[
		\begin{Vmatrix}
			x \\
			y \\
			z
		\end{Vmatrix}
		\]
	\end{lstlisting}
	
	\textbf{Resultado:}
	\[
	\begin{Vmatrix}
		x \\
		y \\
		z
	\end{Vmatrix}
	\]
\end{tcolorbox}

\subsubsection*{\texttt{smallmatrix}}
\begin{tcolorbox}[colback=green!5,colframe=green!50!black]
	\textbf{Descripción:} Matriz pequeña para inline
	
	\textbf{Ejemplo:}
	\begin{lstlisting}[language=TeX]
		La matriz $\bigl(\begin{smallmatrix} a&b\\ c&d \end{smallmatrix}\bigr)$
	\end{lstlisting}
	
	\textbf{Resultado:} La matriz $\bigl(\begin{smallmatrix} a&b\\ c&d \end{smallmatrix}\bigr)$
\end{tcolorbox}

\subsection{Matrices Aumentadas}

\subsubsection*{Matriz aumentada con línea vertical}
\begin{tcolorbox}[colback=green!5,colframe=green!50!black]
	\textbf{Ejemplo:}
	\begin{lstlisting}[language=TeX]
		\[
		\left[\begin{array}{cc|c}
			1 & 2 & 3 \\
			4 & 5 & 6
		\end{array}\right]
		\]
	\end{lstlisting}
	
	\textbf{Resultado:}
	\[
	\left[\begin{array}{cc|c}
		1 & 2 & 3 \\
		4 & 5 & 6
	\end{array}\right]
	\]
\end{tcolorbox}

\section{Fracciones y Binomios}

\subsection{Fracciones}

\subsubsection*{\texttt{\textbackslash frac\{numerador\}\{denominador\}}}
\begin{tcolorbox}[colback=green!5,colframe=green!50!black]
	\textbf{Descripción:} Fracción estándar
	
	\textbf{Ejemplo:}
	\begin{lstlisting}[language=TeX]
		$\frac{a}{b}$, $\frac{x^2 + y^2}{x - y}$
	\end{lstlisting}
	
	\textbf{Resultado:} $\frac{a}{b}$, $\frac{x^2 + y^2}{x - y}$
\end{tcolorbox}

\subsubsection*{\texttt{\textbackslash dfrac\{numerador\}\{denominador\}}}
\begin{tcolorbox}[colback=green!5,colframe=green!50!black]
	\textbf{Descripción:} Fracción display style (tamaño completo)
	
	\textbf{Ejemplo:}
	\begin{lstlisting}[language=TeX]
		$\dfrac{a}{b}$ en línea
	\end{lstlisting}
	
	\textbf{Resultado:} $\dfrac{a}{b}$ en línea
\end{tcolorbox}

\subsubsection*{\texttt{\textbackslash tfrac\{numerador\}\{denominador\}}}
\begin{tcolorbox}[colback=green!5,colframe=green!50!black]
	\textbf{Descripción:} Fracción text style (compacta)
	
	\textbf{Ejemplo:}
	\begin{lstlisting}[language=TeX]
		\[ \tfrac{a}{b} \text{ es más pequeña} \]
	\end{lstlisting}
	
	\textbf{Resultado:} \[ \tfrac{a}{b} \text{ es más pequeña} \]
\end{tcolorbox}

\subsubsection*{\texttt{\textbackslash cfrac\{numerador\}\{denominador\}}}
\begin{tcolorbox}[colback=green!5,colframe=green!50!black]
	\textbf{Descripción:} Fracción continua (requiere amsmath)
	
	\textbf{Ejemplo:}
	\begin{lstlisting}[language=TeX]
		\[
		x = a_0 + \cfrac{1}{a_1 + \cfrac{1}{a_2 + \cfrac{1}{a_3}}}
		\]
	\end{lstlisting}
	
	\textbf{Resultado:}
	\[
	x = a_0 + \cfrac{1}{a_1 + \cfrac{1}{a_2 + \cfrac{1}{a_3}}}
	\]
\end{tcolorbox}

\subsection{Binomios}

\subsubsection*{\texttt{\textbackslash binom\{n\}\{k\}}}
\begin{tcolorbox}[colback=green!5,colframe=green!50!black]
	\textbf{Descripción:} Coeficiente binomial
	
	\textbf{Ejemplo:}
	\begin{lstlisting}[language=TeX]
		$\binom{n}{k} = \frac{n!}{k!(n-k)!}$
	\end{lstlisting}
	
	\textbf{Resultado:} $\binom{n}{k} = \frac{n!}{k!(n-k)!}$
\end{tcolorbox}

\subsubsection*{\texttt{\textbackslash dbinom, \textbackslash tbinom}}
\begin{tcolorbox}[colback=green!5,colframe=green!50!black]
	\textbf{Descripción:} Binomios en display/text style
	
	\textbf{Ejemplo:}
	\begin{lstlisting}[language=TeX]
		$\dbinom{n}{k}$ display, $\tbinom{n}{k}$ text
	\end{lstlisting}
	
	\textbf{Resultado:} $\dbinom{n}{k}$ display, $\tbinom{n}{k}$ text
\end{tcolorbox}

\section{Raíces}

\subsubsection*{\texttt{\textbackslash sqrt\{x\}}}
\begin{tcolorbox}[colback=green!5,colframe=green!50!black]
	\textbf{Descripción:} Raíz cuadrada
	
	\textbf{Ejemplo:}
	\begin{lstlisting}[language=TeX]
		$\sqrt{x^2 + y^2}$
	\end{lstlisting}
	
	\textbf{Resultado:} $\sqrt{x^2 + y^2}$
\end{tcolorbox}

\subsubsection*{\texttt{\textbackslash sqrt[n]\{x\}}}
\begin{tcolorbox}[colback=green!5,colframe=green!50!black]
	\textbf{Descripción:} Raíz n-ésima
	
	\textbf{Ejemplo:}
	\begin{lstlisting}[language=TeX]
		$\sqrt[3]{27} = 3$, $\sqrt[n]{x}$
	\end{lstlisting}
	
	\textbf{Resultado:} $\sqrt[3]{27} = 3$, $\sqrt[n]{x}$
\end{tcolorbox}

\section{Subíndices y Superíndices}

\subsection{Básicos}

\subsubsection*{\texttt{\^{}\{superíndice\}}}
\begin{tcolorbox}[colback=green!5,colframe=green!50!black]
	\textbf{Descripción:} Superíndice (exponente)
	
	\textbf{Ejemplo:}
	\begin{lstlisting}[language=TeX]
		$x^2$, $e^{i\pi}$, $a^{b^c}$
	\end{lstlisting}
	
	\textbf{Resultado:} $x^2$, $e^{i\pi}$, $a^{b^c}$
\end{tcolorbox}

\subsubsection*{\texttt{\_\{subíndice\}}}
\begin{tcolorbox}[colback=green!5,colframe=green!50!black]
	\textbf{Descripción:} Subíndice

	\textbf{Ejemplo:}
	\begin{lstlisting}[language=TeX]
		$x_i$, $a_{ij}$, $x_{i_j}$
	\end{lstlisting}

	\textbf{Resultado:} $x_i$, $a_{ij}$, $x_{i_j}$
\end{tcolorbox}

\subsubsection*{Combinación de sub y superíndices}
\begin{tcolorbox}[colback=green!5,colframe=green!50!black]
	\textbf{Ejemplo:}
	\begin{lstlisting}[language=TeX]
		$x_i^2$, $a_1^2 + a_2^2$, ${}_nC_r$
	\end{lstlisting}
	
	\textbf{Resultado:} $x_i^2$, $a_1^2 + a_2^2$, ${}_nC_r$
\end{tcolorbox}

\subsection{Límites, Sumas e Integrales}

\subsubsection*{\texttt{\textbackslash lim\_\{x \textbackslash to a\}}}
\begin{tcolorbox}[colback=green!5,colframe=green!50!black]
	\textbf{Descripción:} Límite
	
	\textbf{Ejemplo:}
	\begin{lstlisting}[language=TeX]
		$\lim_{x \to 0} \frac{\sin x}{x} = 1$
		\[ \lim_{n \to \infty} \left(1 + \frac{1}{n}\right)^n = e \]
	\end{lstlisting}
	
	\textbf{Resultado:} $\lim_{x \to 0} \frac{\sin x}{x} = 1$
	\[ \lim_{n \to \infty} \left(1 + \frac{1}{n}\right)^n = e \]
\end{tcolorbox}

\subsubsection*{\texttt{\textbackslash sum\_\{i=1\}\^{}\{n\}}}
\begin{tcolorbox}[colback=green!5,colframe=green!50!black]
	\textbf{Descripción:} Sumatoria
	
	\textbf{Ejemplo:}
	\begin{lstlisting}[language=TeX]
		$\sum_{i=1}^{n} i = \frac{n(n+1)}{2}$
		\[ \sum_{k=0}^{\infty} \frac{1}{2^k} = 2 \]
	\end{lstlisting}
	
	\textbf{Resultado:} $\sum_{i=1}^{n} i = \frac{n(n+1)}{2}$
	\[ \sum_{k=0}^{\infty} \frac{1}{2^k} = 2 \]
\end{tcolorbox}

\subsubsection*{\texttt{\textbackslash prod\_\{i=1\}\^{}\{n\}}}
\begin{tcolorbox}[colback=green!5,colframe=green!50!black]
	\textbf{Descripción:} Productoria
	
	\textbf{Ejemplo:}
	\begin{lstlisting}[language=TeX]
		$n! = \prod_{i=1}^{n} i$
	\end{lstlisting}
	
	\textbf{Resultado:} $n! = \prod_{i=1}^{n} i$
\end{tcolorbox}

\subsubsection*{\texttt{\textbackslash int\_a\^{}b}}
\begin{tcolorbox}[colback=green!5,colframe=green!50!black]
	\textbf{Descripción:} Integral definida
	
	\textbf{Ejemplo:}
	\begin{lstlisting}[language=TeX]
		$\int_0^1 x^2 dx = \frac{1}{3}$
		\[ \int_{-\infty}^{\infty} e^{-x^2} dx = \sqrt{\pi} \]
	\end{lstlisting}
	
	\textbf{Resultado:} $\int_0^1 x^2 dx = \frac{1}{3}$
	\[ \int_{-\infty}^{\infty} e^{-x^2} dx = \sqrt{\pi} \]
\end{tcolorbox}

\subsubsection*{\texttt{\textbackslash iint, \textbackslash iiint, \textbackslash iiiint}}
\begin{tcolorbox}[colback=green!5,colframe=green!50!black]
	\textbf{Descripción:} Integrales múltiples
	
	\textbf{Ejemplo:}
	\begin{lstlisting}[language=TeX]
		$\iint_D f(x,y) \, dA$, $\iiint_V g(x,y,z) \, dV$
	\end{lstlisting}
	
	\textbf{Resultado:} $\iint_D f(x,y) \, dA$, $\iiint_V g(x,y,z) \, dV$
\end{tcolorbox}

\subsubsection*{\texttt{\textbackslash oint}}
\begin{tcolorbox}[colback=green!5,colframe=green!50!black]
	\textbf{Descripción:} Integral de contorno
	
	\textbf{Ejemplo:}
	\begin{lstlisting}[language=TeX]
		$\oint_C \vec{F} \cdot d\vec{r}$
	\end{lstlisting}
	
	\textbf{Resultado:} $\oint_C \vec{F} \cdot d\vec{r}$
\end{tcolorbox}

\section{Delimitadores}

\subsection{Paréntesis y Corchetes}

\subsubsection*{Delimitadores básicos}
\begin{tcolorbox}[colback=green!5,colframe=green!50!black]
	\textbf{Ejemplo:}
	\begin{lstlisting}[language=TeX]
		$(x)$, $[x]$, $\{x\}$, $|x|$, $\|x\|$, $\langle x \rangle$
	\end{lstlisting}
	
	\textbf{Resultado:} $(x)$, $[x]$, $\{x\}$, $|x|$, $\|x\|$, $\langle x \rangle$
\end{tcolorbox}

\subsubsection*{\texttt{\textbackslash left y \textbackslash right}}
\begin{tcolorbox}[colback=green!5,colframe=green!50!black]
	\textbf{Descripción:} Delimitadores que se ajustan automáticamente
	
	\textbf{Ejemplo:}
	\begin{lstlisting}[language=TeX]
		$\left( \frac{a}{b} \right)$, $\left[ \sum_{i=1}^{n} x_i \right]$
		$\left\{ x \in \mathbb{R} : x > 0 \right\}$
	\end{lstlisting}
	
	\textbf{Resultado:} $\left( \frac{a}{b} \right)$, $\left[ \sum_{i=1}^{n} x_i \right]$, $\left\{ x \in \mathbb{R} : x > 0 \right\}$
\end{tcolorbox}

\subsubsection*{\texttt{\textbackslash left. y \textbackslash right.}}
\begin{tcolorbox}[colback=green!5,colframe=green!50!black]
	\textbf{Descripción:} Delimitador invisible (para equilibrar)
	
	\textbf{Ejemplo:}
	\begin{lstlisting}[language=TeX]
		$\left. \frac{dy}{dx} \right|_{x=0}$
	\end{lstlisting}
	
	\textbf{Resultado:} $\left. \frac{dy}{dx} \right|_{x=0}$
\end{tcolorbox}

\subsubsection*{\texttt{\textbackslash big, \textbackslash Big, \textbackslash bigg, \textbackslash Bigg}}
\begin{tcolorbox}[colback=green!5,colframe=green!50!black]
	\textbf{Descripción:} Tamaños manuales de delimitadores
	
	\textbf{Ejemplo:}
	\begin{lstlisting}[language=TeX]
		$\big( \Big( \bigg( \Bigg( x$
	\end{lstlisting}
	
	\textbf{Resultado:} $\big( \Big( \bigg( \Bigg( x$
\end{tcolorbox}

\subsection{Delimitadores Especiales}

\subsubsection*{Más delimitadores}
\begin{tcolorbox}[colback=green!5,colframe=green!50!black]
	\textbf{Ejemplo:}
	\begin{lstlisting}[language=TeX]
		$\lfloor x \rfloor$, $\lceil x \rceil$, $\langle x, y \rangle$
		$\ulcorner x \urcorner$, $\llcorner x \lrcorner$
	\end{lstlisting}
	
	\textbf{Resultado:} $\lfloor x \rfloor$, $\lceil x \rceil$, $\langle x, y \rangle$, $\ulcorner x \urcorner$, $\llcorner x \lrcorner$
\end{tcolorbox}

\section{Acentos y Decoraciones}

\subsection{Acentos Matemáticos}

\subsubsection*{Acentos sobre símbolos}
\begin{tcolorbox}[colback=green!5,colframe=green!50!black]
	\textbf{Ejemplo:}
	\begin{lstlisting}[language=TeX]
		$\hat{x}$, $\bar{x}$, $\tilde{x}$, $\vec{x}$, $\dot{x}$, $\ddot{x}$
		$\acute{x}$, $\grave{x}$, $\breve{x}$, $\check{x}$
	\end{lstlisting}
	
	\textbf{Resultado:} $\hat{x}$, $\bar{x}$, $\tilde{x}$, $\vec{x}$, $\dot{x}$, $\ddot{x}$, $\acute{x}$, $\grave{x}$, $\breve{x}$, $\check{x}$
\end{tcolorbox}

\subsubsection*{Acentos anchos}
\begin{tcolorbox}[colback=green!5,colframe=green!50!black]
	\textbf{Ejemplo:}
	\begin{lstlisting}[language=TeX]
		$\widehat{xyz}$, $\widetilde{abc}$, $\overline{AB}$, $\underline{text}$
	\end{lstlisting}
	
	\textbf{Resultado:} $\widehat{xyz}$, $\widetilde{abc}$, $\overline{AB}$, $\underline{text}$
\end{tcolorbox}

\subsection{Flechas sobre Símbolos}

\subsubsection*{Vectores y flechas}
\begin{tcolorbox}[colback=green!5,colframe=green!50!black]
	\textbf{Ejemplo:}
	\begin{lstlisting}[language=TeX]
		$\vec{v}$, $\overrightarrow{AB}$, $\overleftarrow{BA}$
		$\overline{AB}$, $\underline{text}$
	\end{lstlisting}
	
	\textbf{Resultado:} $\vec{v}$, $\overrightarrow{AB}$, $\overleftarrow{BA}$, $\overline{AB}$, $\underline{text}$
\end{tcolorbox}

\subsection{Llaves sobre/bajo Expresiones}

\subsubsection*{\texttt{\textbackslash overbrace, \textbackslash underbrace}}
\begin{tcolorbox}[colback=green!5,colframe=green!50!black]
	\textbf{Descripción:} Llave sobre o bajo expresión
	
	\textbf{Ejemplo:}
	\begin{lstlisting}[language=TeX]
		$\overbrace{a + b + c}^{\text{suma}}$
		$\underbrace{x + y + z}_{\text{variables}}$
	\end{lstlisting}
	
	\textbf{Resultado:} $\overbrace{a + b + c}^{\text{suma}}$, $\underbrace{x + y + z}_{\text{variables}}$
\end{tcolorbox}

\subsubsection*{\texttt{\textbackslash stackrel}}
\begin{tcolorbox}[colback=green!5,colframe=green!50!black]
	\textbf{Descripción:} Coloca símbolo sobre otro
	
	\textbf{Ejemplo:}
	\begin{lstlisting}[language=TeX]
		$\stackrel{def}{=}$, $\stackrel{?}{=}$
	\end{lstlisting}
	
	\textbf{Resultado:} $\stackrel{def}{=}$, $\stackrel{?}{=}$
\end{tcolorbox}

\section{Espaciado}

\subsection{Espacios en Modo Matemático}

\subsubsection*{Comandos de espacio}
\begin{tcolorbox}[colback=green!5,colframe=green!50!black]
	\textbf{Espacios disponibles:}
	\begin{itemize}[nosep]
		\item \texttt{\textbackslash,} -- Espacio fino (3/18 quad)
		\item \texttt{\textbackslash:} -- Espacio medio (4/18 quad)
		\item \texttt{\textbackslash;} -- Espacio grueso (5/18 quad)
		\item \texttt{\textbackslash quad} -- Espacio de 1em
		\item \texttt{\textbackslash qquad} -- Espacio de 2em
		\item \texttt{\textbackslash !} -- Espacio negativo fino
	\end{itemize}
	
	\textbf{Ejemplo:}
	\begin{lstlisting}[language=TeX]
		$a\,b$ $a\:b$ $a\;b$ $a\quad b$ $a\qquad b$
	\end{lstlisting}
	
	\textbf{Resultado:} $a\,b$ $a\:b$ $a\;b$ $a\quad b$ $a\qquad b$
\end{tcolorbox}

\section{Texto en Modo Matemático}

\subsubsection*{\texttt{\textbackslash text\{...\}}}
\begin{tcolorbox}[colback=green!5,colframe=green!50!black]
	\textbf{Descripción:} Inserta texto normal en modo matemático
	
	\textbf{Ejemplo:}
	\begin{lstlisting}[language=TeX]
		$x = 5 \text{ donde } x \text{ es un entero}$
	\end{lstlisting}
	
	\textbf{Resultado:} $x = 5 \text{ donde } x \text{ es un entero}$
\end{tcolorbox}

\subsubsection*{\texttt{\textbackslash textrm, \textbackslash textit, \textbackslash textbf}}
\begin{tcolorbox}[colback=green!5,colframe=green!50!black]
	\textbf{Descripción:} Texto con formato específico
	
	\textbf{Ejemplo:}
	\begin{lstlisting}[language=TeX]
		$\textrm{roman}$, $\textit{italic}$, $\textbf{bold}$
	\end{lstlisting}
	
	\textbf{Resultado:} $\textrm{roman}$, $\textit{italic}$, $\textbf{bold}$
\end{tcolorbox}

\section{Fuentes Matemáticas}

\subsection{Estilos de Fuentes}

\subsubsection*{\texttt{\textbackslash mathrm\{...\}}}
\begin{tcolorbox}[colback=green!5,colframe=green!50!black]
	\textbf{Descripción:} Romano (para texto en fórmulas)
	
	\textbf{Ejemplo:}
	\begin{lstlisting}[language=TeX]
		$\mathrm{d}x$, $\sin x$ vs $\mathrm{sin} x$
	\end{lstlisting}
	
	\textbf{Resultado:} $\mathrm{d}x$, $\sin x$ vs $\mathrm{sin} x$
\end{tcolorbox}

\subsubsection*{\texttt{\textbackslash mathit\{...\}}}
\begin{tcolorbox}[colback=green!5,colframe=green!50!black]
	\textbf{Descripción:} Itálica matemática
	
	\textbf{Ejemplo:}
	\begin{lstlisting}[language=TeX]
		$\mathit{variable}$
	\end{lstlisting}
	
	\textbf{Resultado:} $\mathit{variable}$
\end{tcolorbox}

\subsubsection*{\texttt{\textbackslash mathbf\{...\}}}
\begin{tcolorbox}[colback=green!5,colframe=green!50!black]
	\textbf{Descripción:} Negrita (para vectores)
	
	\textbf{Ejemplo:}
	\begin{lstlisting}[language=TeX]
		$\mathbf{v}$, $\mathbf{A}$
	\end{lstlisting}
	
	\textbf{Resultado:} $\mathbf{v}$, $\mathbf{A}$
\end{tcolorbox}

\subsubsection*{\texttt{\textbackslash mathsf\{...\}}}
\begin{tcolorbox}[colback=green!5,colframe=green!50!black]
	\textbf{Descripción:} Sans serif
	
	\textbf{Ejemplo:}
	\begin{lstlisting}[language=TeX]
		$\mathsf{T}$, $\mathsf{ABC}$
	\end{lstlisting}
	
	\textbf{Resultado:} $\mathsf{T}$, $\mathsf{ABC}$
\end{tcolorbox}

\subsubsection*{\texttt{\textbackslash mathtt\{...\}}}
\begin{tcolorbox}[colback=green!5,colframe=green!50!black]
	\textbf{Descripción:} Typewriter (monoespaciada)
	
	\textbf{Ejemplo:}
	\begin{lstlisting}[language=TeX]
		$\mathtt{code}$
	\end{lstlisting}
	
	\textbf{Resultado:} $\mathtt{code}$
\end{tcolorbox}

\subsubsection*{\texttt{\textbackslash mathcal\{...\}}}
\begin{tcolorbox}[colback=green!5,colframe=green!50!black]
	\textbf{Descripción:} Caligráfica (solo mayúsculas)
	
	\textbf{Ejemplo:}
	\begin{lstlisting}[language=TeX]
		$\mathcal{L}$, $\mathcal{F}$, $\mathcal{ABCD}$
	\end{lstlisting}
	
	\textbf{Resultado:} $\mathcal{L}$, $\mathcal{F}$, $\mathcal{ABCD}$
\end{tcolorbox}

\subsubsection*{\texttt{\textbackslash mathscr\{...\}}}
\begin{tcolorbox}[colback=green!5,colframe=green!50!black]
	\textbf{Descripción:} Script (requiere mathrsfs)
	
	\textbf{Ejemplo:}
	\begin{lstlisting}[language=TeX]
		\usepackage{mathrsfs}
		$\mathscr{L}$, $\mathscr{H}$
	\end{lstlisting}
	
	\textbf{Resultado:} $\mathscr{L}$, $\mathscr{H}$
\end{tcolorbox}

\subsubsection*{\texttt{\textbackslash mathbb\{...\}}}
\begin{tcolorbox}[colback=green!5,colframe=green!50!black]
	\textbf{Descripción:} Blackboard bold (conjuntos numéricos)
	
	\textbf{Ejemplo:}
	\begin{lstlisting}[language=TeX]
		$\mathbb{R}$, $\mathbb{N}$, $\mathbb{Z}$, $\mathbb{Q}$, $\mathbb{C}$
	\end{lstlisting}
	
	\textbf{Resultado:} $\mathbb{R}$, $\mathbb{N}$, $\mathbb{Z}$, $\mathbb{Q}$, $\mathbb{C}$
\end{tcolorbox}

\subsubsection*{\texttt{\textbackslash mathfrak\{...\}}}
\begin{tcolorbox}[colback=green!5,colframe=green!50!black]
	\textbf{Descripción:} Fraktur (gótica)
	
	\textbf{Ejemplo:}
	\begin{lstlisting}[language=TeX]
		$\mathfrak{g}$, $\mathfrak{A}$
	\end{lstlisting}
	
	\textbf{Resultado:} $\mathfrak{g}$, $\mathfrak{A}$
\end{tcolorbox}

\subsubsection*{\texttt{\textbackslash bm\{...\}}}
\begin{tcolorbox}[colback=green!5,colframe=green!50!black]
	\textbf{Descripción:} Negrita matemática mejorada (requiere bm)
	
	\textbf{Ejemplo:}
	\begin{lstlisting}[language=TeX]
		\usepackage{bm}
		$\bm{\alpha}$, $\bm{A}$, $\bm{\nabla}$
	\end{lstlisting}
	
	\textbf{Resultado:} $\bm{\alpha}$, $\bm{A}$, $\bm{\nabla}$
\end{tcolorbox}

\section{Operadores y Funciones}

\subsection{Funciones Estándar}

\subsubsection*{Funciones trigonométricas}
\begin{tcolorbox}[colback=green!5,colframe=green!50!black]
	\textbf{Ejemplo:}
	\begin{lstlisting}[language=TeX]
		$\sin x$, $\cos x$, $\tan x$, $\cot x$, $\sec x$, $\csc x$
		$\arcsin x$, $\arccos x$, $\arctan x$
		$\sinh x$, $\cosh x$, $\tanh x$
	\end{lstlisting}
	
	\textbf{Resultado:} $\sin x$, $\cos x$, $\tan x$, $\cot x$, $\sec x$, $\csc x$, $\arcsin x$, $\arccos x$, $\arctan x$, $\sinh x$, $\cosh x$, $\tanh x$
\end{tcolorbox}

\subsubsection*{Funciones logarítmicas y exponenciales}
\begin{tcolorbox}[colback=green!5,colframe=green!50!black]
	\textbf{Ejemplo:}
	\begin{lstlisting}[language=TeX]
		$\log x$, $\ln x$, $\lg x$, $\exp x$
	\end{lstlisting}
	
	\textbf{Resultado:} $\log x$, $\ln x$, $\lg x$, $\exp x$
\end{tcolorbox}

\subsubsection*{Otras funciones}
\begin{tcolorbox}[colback=green!5,colframe=green!50!black]
	\textbf{Ejemplo:}
	\begin{lstlisting}[language=TeX]
		$\min$, $\max$, $\sup$, $\inf$, $\lim$, $\limsup$, $\liminf$
		$\arg$, $\deg$, $\det$, $\dim$, $\ker$, $\hom$, $\gcd$
	\end{lstlisting}
	
	\textbf{Resultado:} $\min$, $\max$, $\sup$, $\inf$, $\lim$, $\limsup$, $\liminf$, $\arg$, $\deg$, $\det$, $\dim$, $\ker$, $\hom$, $\gcd$
\end{tcolorbox}

\subsection{Operadores Personalizados}

\subsubsection*{\texttt{\textbackslash operatorname\{name\}}}
\begin{tcolorbox}[colback=green!5,colframe=green!50!black]
	\textbf{Descripción:} Define operador personalizado
	
	\textbf{Ejemplo:}
	\begin{lstlisting}[language=TeX]
		$\operatorname{sen} x$, $\operatorname{tr} A$
	\end{lstlisting}
	
	\textbf{Resultado:} $\operatorname{sen} x$, $\operatorname{tr} A$
\end{tcolorbox}

\subsubsection*{\texttt{\textbackslash DeclareMathOperator}}
\begin{tcolorbox}[colback=green!5,colframe=green!50!black]
	\textbf{Descripción:} Declara operador permanente (preámbulo)
	
	\textbf{Ejemplo:}
	\begin{lstlisting}[language=TeX]
		\DeclareMathOperator{\sen}{sen}
		\DeclareMathOperator*{\argmax}{arg\,max}
		$\sen x$, $\argmax_{x \in X} f(x)$
	\end{lstlisting}
\end{tcolorbox}

\newpage

\section{Símbolos Matemáticos}

\subsection{Operadores Binarios}

\begin{tcolorbox}[colback=green!5,colframe=green!50!black]
	\begin{multicols}{3}
		\begin{itemize}[nosep,leftmargin=*]
			\item $+$ \texttt{+}
			\item $-$ \texttt{-}
			\item $\times$ \texttt{\textbackslash times}
			\item $\div$ \texttt{\textbackslash div}
			\item $\pm$ \texttt{\textbackslash pm}
			\item $\mp$ \texttt{\textbackslash mp}
			\item $\cdot$ \texttt{\textbackslash cdot}
			\item $\ast$ \texttt{\textbackslash ast}
			\item $\star$ \texttt{\textbackslash star}
			\item $\circ$ \texttt{\textbackslash circ}
			\item $\bullet$ \texttt{\textbackslash bullet}
			\item $\oplus$ \texttt{\textbackslash oplus}
			\item $\ominus$ \texttt{\textbackslash ominus}
			\item $\otimes$ \texttt{\textbackslash otimes}
			\item $\oslash$ \texttt{\textbackslash oslash}
			\item $\odot$ \texttt{\textbackslash odot}
			\item $\cup$ \texttt{\textbackslash cup}
			\item $\cap$ \texttt{\textbackslash cap}
			\item $\sqcup$ \texttt{\textbackslash sqcup}
			\item $\sqcap$ \texttt{\textbackslash sqcap}
			\item $\vee$ \texttt{\textbackslash vee}
			\item $\wedge$ \texttt{\textbackslash wedge}
			\item $\setminus$ \texttt{\textbackslash setminus}
			\item $\wr$ \texttt{\textbackslash wr}
			\item $\diamond$ \texttt{\textbackslash diamond}
			\item $\bigtriangleup$ \texttt{\textbackslash bigtriangleup}
			\item $\bigtriangledown$ \texttt{\textbackslash bigtriangledown}
			\item $\triangleleft$ \texttt{\textbackslash triangleleft}
			\item $\triangleright$ \texttt{\textbackslash triangleright}
			\item $\bigcirc$ \texttt{\textbackslash bigcirc}
			\item $\amalg$ \texttt{\textbackslash amalg}
			\item $\dagger$ \texttt{\textbackslash dagger}
			\item $\ddagger$ \texttt{\textbackslash ddagger}
		\end{itemize}
	\end{multicols}
\end{tcolorbox}

\subsection{Relaciones}

\begin{tcolorbox}[colback=green!5,colframe=green!50!black]
	\begin{multicols}{3}
		\begin{itemize}[nosep,leftmargin=*]
			\item $=$ \texttt{=}
			\item $\neq$ \texttt{\textbackslash neq}
			\item $<$ \texttt{<}
			\item $>$ \texttt{>}
			\item $\leq$ \texttt{\textbackslash leq}
			\item $\geq$ \texttt{\textbackslash geq}
			\item $\ll$ \texttt{\textbackslash ll}
			\item $\gg$ \texttt{\textbackslash gg}
			\item $\equiv$ \texttt{\textbackslash equiv}
			\item $\sim$ \texttt{\textbackslash sim}
			\item $\simeq$ \texttt{\textbackslash simeq}
			\item $\approx$ \texttt{\textbackslash approx}
			\item $\cong$ \texttt{\textbackslash cong}
			\item $\propto$ \texttt{\textbackslash propto}
			\item $\perp$ \texttt{\textbackslash perp}
			\item $\parallel$ \texttt{\textbackslash parallel}
			\item $\subset$ \texttt{\textbackslash subset}
			\item $\supset$ \texttt{\textbackslash supset}
			\item $\subseteq$ \texttt{\textbackslash subseteq}
			\item $\supseteq$ \texttt{\textbackslash supseteq}
			\item $\sqsubset$ \texttt{\textbackslash sqsubset}
			\item $\sqsupset$ \texttt{\textbackslash sqsupset}
			\item $\sqsubseteq$ \texttt{\textbackslash sqsubseteq}
			\item $\sqsupseteq$ \texttt{\textbackslash sqsupseteq}
			\item $\in$ \texttt{\textbackslash in}
			\item $\ni$ \texttt{\textbackslash ni}
			\item $\notin$ \texttt{\textbackslash notin}
			\item $\prec$ \texttt{\textbackslash prec}
			\item $\succ$ \texttt{\textbackslash succ}
			\item $\preceq$ \texttt{\textbackslash preceq}
			\item $\succeq$ \texttt{\textbackslash succeq}
			\item $\vdash$ \texttt{\textbackslash vdash}
			\item $\dashv$ \texttt{\textbackslash dashv}
			\item $\models$ \texttt{\textbackslash models}
			\item $\mid$ \texttt{\textbackslash mid}
			\item $\asymp$ \texttt{\textbackslash asymp}
			\item $\bowtie$ \texttt{\textbackslash bowtie}
			\item $\smile$ \texttt{\textbackslash smile}
			\item $\frown$ \texttt{\textbackslash frown}
		\end{itemize}
	\end{multicols}
\end{tcolorbox}

\subsection{Flechas}

\begin{tcolorbox}[colback=green!5,colframe=green!50!black]
	\begin{multicols}{3}
		\begin{itemize}[nosep,leftmargin=*]
			\item $\leftarrow$ \texttt{\textbackslash leftarrow}
			\item $\rightarrow$ \texttt{\textbackslash rightarrow}
			\item $\leftrightarrow$ \texttt{\textbackslash leftrightarrow}
			\item $\Leftarrow$ \texttt{\textbackslash Leftarrow}
			\item $\Rightarrow$ \texttt{\textbackslash Rightarrow}
			\item $\Leftrightarrow$ \texttt{\textbackslash Leftrightarrow}
			\item $\leftrightharpoons$ \texttt{\textbackslash leftrightharpoons}
			\item $\rightleftharpoons$ \texttt{\textbackslash rightleftharpoons}
			\item $\uparrow$ \texttt{\textbackslash uparrow}
			\item $\downarrow$ \texttt{\textbackslash downarrow}
			\item $\updownarrow$ \texttt{\textbackslash updownarrow}
			\item $\Uparrow$ \texttt{\textbackslash Uparrow}
			\item $\Downarrow$ \texttt{\textbackslash Downarrow}
			\item $\Updownarrow$ \texttt{\textbackslash Updownarrow}
			\item $\nearrow$ \texttt{\textbackslash nearrow}
			\item $\searrow$ \texttt{\textbackslash searrow}
			\item $\swarrow$ \texttt{\textbackslash swarrow}
			\item $\nwarrow$ \texttt{\textbackslash nwarrow}
			\item $\mapsto$ \texttt{\textbackslash mapsto}
			\item $\longmapsto$ \texttt{\textbackslash longmapsto}
			\item $\hookleftarrow$ \texttt{\textbackslash hookleftarrow}
			\item $\hookrightarrow$ \texttt{\textbackslash hookrightarrow}
			\item $\leftharpoonup$ \texttt{\textbackslash leftharpoonup}
			\item $\rightharpoonup$ \texttt{\textbackslash rightharpoonup}
			\item $\leftharpoondown$ \texttt{\textbackslash leftharpoondown}
			\item $\rightharpoondown$ \texttt{\textbackslash rightharpoondown}
			\item $\leadsto$ \texttt{\textbackslash leadsto}
			\item $\to$ \texttt{\textbackslash to}
			\item $\gets$ \texttt{\textbackslash gets}
		\end{itemize}
	\end{multicols}
\end{tcolorbox}

\subsection{Letras Griegas}

\subsubsection*{Minúsculas}
\begin{tcolorbox}[colback=green!5,colframe=green!50!black]
	\begin{multicols}{4}
		\begin{itemize}[nosep,leftmargin=*]
			\item $\alpha$ \texttt{\textbackslash alpha}
			\item $\beta$ \texttt{\textbackslash beta}
			\item $\gamma$ \texttt{\textbackslash gamma}
			\item $\delta$ \texttt{\textbackslash delta}
			\item $\epsilon$ \texttt{\textbackslash epsilon}
			\item $\varepsilon$ \texttt{\textbackslash varepsilon}
			\item $\zeta$ \texttt{\textbackslash zeta}
			\item $\eta$ \texttt{\textbackslash eta}
			\item $\theta$ \texttt{\textbackslash theta}
			\item $\vartheta$ \texttt{\textbackslash vartheta}
			\item $\iota$ \texttt{\textbackslash iota}
			\item $\kappa$ \texttt{\textbackslash kappa}
			\item $\lambda$ \texttt{\textbackslash lambda}
			\item $\mu$ \texttt{\textbackslash mu}
			\item $\nu$ \texttt{\textbackslash nu}
			\item $\xi$ \texttt{\textbackslash xi}
			\item $\pi$ \texttt{\textbackslash pi}
			\item $\varpi$ \texttt{\textbackslash varpi}
			\item $\rho$ \texttt{\textbackslash rho}
			\item $\varrho$ \texttt{\textbackslash varrho}
			\item $\sigma$ \texttt{\textbackslash sigma}
			\item $\varsigma$ \texttt{\textbackslash varsigma}
			\item $\tau$ \texttt{\textbackslash tau}
			\item $\upsilon$ \texttt{\textbackslash upsilon}
			\item $\phi$ \texttt{\textbackslash phi}
			\item $\varphi$ \texttt{\textbackslash varphi}
			\item $\chi$ \texttt{\textbackslash chi}
			\item $\psi$ \texttt{\textbackslash psi}
			\item $\omega$ \texttt{\textbackslash omega}
		\end{itemize}
	\end{multicols}
\end{tcolorbox}

\subsubsection*{Mayúsculas}
\begin{tcolorbox}[colback=green!5,colframe=green!50!black]
	\begin{multicols}{4}
		\begin{itemize}[nosep,leftmargin=*]
			\item $\Gamma$ \texttt{\textbackslash Gamma}
			\item $\Delta$ \texttt{\textbackslash Delta}
			\item $\Theta$ \texttt{\textbackslash Theta}
			\item $\Lambda$ \texttt{\textbackslash Lambda}
			\item $\Xi$ \texttt{\textbackslash Xi}
			\item $\Pi$ \texttt{\textbackslash Pi}
			\item $\Sigma$ \texttt{\textbackslash Sigma}
			\item $\Upsilon$ \texttt{\textbackslash Upsilon}
			\item $\Phi$ \texttt{\textbackslash Phi}
			\item $\Psi$ \texttt{\textbackslash Psi}
			\item $\Omega$ \texttt{\textbackslash Omega}
		\end{itemize}
	\end{multicols}
\end{tcolorbox}

\subsection{Símbolos Especiales}

\begin{tcolorbox}[colback=green!5,colframe=green!50!black]
	\begin{multicols}{3}
		\begin{itemize}[nosep,leftmargin=*]
			\item $\infty$ \texttt{\textbackslash infty}
			\item $\partial$ \texttt{\textbackslash partial}
			\item $\nabla$ \texttt{\textbackslash nabla}
			\item $\emptyset$ \texttt{\textbackslash emptyset}
			\item $\varnothing$ \texttt{\textbackslash varnothing}
			\item $\forall$ \texttt{\textbackslash forall}
			\item $\exists$ \texttt{\textbackslash exists}
			\item $\nexists$ \texttt{\textbackslash nexists}
			\item $\neg$ \texttt{\textbackslash neg}
			\item $\lnot$ \texttt{\textbackslash lnot}
			\item $\wedge$ \texttt{\textbackslash wedge}
			\item $\vee$ \texttt{\textbackslash vee}
			\item $\therefore$ \texttt{\textbackslash therefore}
			\item $\because$ \texttt{\textbackslash because}
			\item $\angle$ \texttt{\textbackslash angle}
			\item $\measuredangle$ \texttt{\textbackslash measuredangle}
			\item $\sphericalangle$ \texttt{\textbackslash sphericalangle}
			\item $\triangle$ \texttt{\textbackslash triangle}
			\item $\square$ \texttt{\textbackslash square}
			\item $\blacksquare$ \texttt{\textbackslash blacksquare}
			\item $\Box$ \texttt{\textbackslash Box}
			\item $\Diamond$ \texttt{\textbackslash Diamond}
			\item $\ell$ \texttt{\textbackslash ell}
			\item $\hbar$ \texttt{\textbackslash hbar}
			\item $\imath$ \texttt{\textbackslash imath}
			\item $\jmath$ \texttt{\textbackslash jmath}
			\item $\wp$ \texttt{\textbackslash wp}
			\item $\Re$ \texttt{\textbackslash Re}
			\item $\Im$ \texttt{\textbackslash Im}
			\item $\aleph$ \texttt{\textbackslash aleph}
			\item $\prime$ \texttt{\textbackslash prime}
			\item $\surd$ \texttt{\textbackslash surd}
			\item $\top$ \texttt{\textbackslash top}
			\item $\bot$ \texttt{\textbackslash bot}
			\item $\flat$ \texttt{\textbackslash flat}
			\item $\natural$ \texttt{\textbackslash natural}
			\item $\sharp$ \texttt{\textbackslash sharp}
			\item $\clubsuit$ \texttt{\textbackslash clubsuit}
			\item $\diamondsuit$ \texttt{\textbackslash diamondsuit}
			\item $\heartsuit$ \texttt{\textbackslash heartsuit}
			\item $\spadesuit$ \texttt{\textbackslash spadesuit}
		\end{itemize}
	\end{multicols}
\end{tcolorbox}

\subsection{Operadores Grandes}

\begin{tcolorbox}[colback=green!5,colframe=green!50!black]
	\begin{multicols}{3}
		\begin{itemize}[nosep,leftmargin=*]
			\item $\sum$ \texttt{\textbackslash sum}
			\item $\prod$ \texttt{\textbackslash prod}
			\item $\coprod$ \texttt{\textbackslash coprod}
			\item $\int$ \texttt{\textbackslash int}
			\item $\iint$ \texttt{\textbackslash iint}
			\item $\iiint$ \texttt{\textbackslash iiint}
			\item $\oint$ \texttt{\textbackslash oint}
			\item $\bigcup$ \texttt{\textbackslash bigcup}
			\item $\bigcap$ \texttt{\textbackslash bigcap}
			\item $\bigsqcup$ \texttt{\textbackslash bigsqcup}
			\item $\biguplus$ \texttt{\textbackslash biguplus}
			\item $\bigvee$ \texttt{\textbackslash bigvee}
			\item $\bigwedge$ \texttt{\textbackslash bigwedge}
			\item $\bigoplus$ \texttt{\textbackslash bigoplus}
			\item $\bigotimes$ \texttt{\textbackslash bigotimes}
			\item $\bigodot$ \texttt{\textbackslash bigodot}
		\end{itemize}
	\end{multicols}
\end{tcolorbox}

\section{Casos y Condicionales}

\subsubsection*{\texttt{cases}}
\begin{tcolorbox}[colback=green!5,colframe=green!50!black]
	\textbf{Descripción:} Definición por casos
	
	\textbf{Ejemplo:}
	\begin{lstlisting}[language=TeX]
		\[
		f(x) = \begin{cases}
			x^2 & \text{si } x \geq 0 \\
			-x^2 & \text{si } x < 0
		\end{cases}
		\]
	\end{lstlisting}
	
	\textbf{Resultado:}
	\[
	f(x) = \begin{cases}
		x^2 & \text{si } x \geq 0 \\
		-x^2 & \text{si } x < 0
	\end{cases}
	\]
\end{tcolorbox}

\subsubsection*{\texttt{dcases}}
\begin{tcolorbox}[colback=green!5,colframe=green!50!black]
	\textbf{Descripción:} Cases con fracciones en display style (mathtools)
	
	\textbf{Ejemplo:}
	\begin{lstlisting}[language=TeX]
		\[
		f(x) = \begin{dcases}
			\frac{x}{2} & x > 0 \\
			0 & x = 0
		\end{dcases}
		\]
	\end{lstlisting}
\end{tcolorbox}

\section{Entornos de Teoremas}

\subsection{Definición de Entornos}

\subsubsection*{\texttt{\textbackslash newtheorem}}
\begin{tcolorbox}[colback=green!5,colframe=green!50!black]
	\textbf{Descripción:} Define nuevo entorno de teorema (en preámbulo)
	
	\textbf{Ejemplo:}
	\begin{lstlisting}[language=TeX]
		\usepackage{amsthm}
		\newtheorem{theorem}{Teorema}[section]
		\newtheorem{lemma}[theorem]{Lema}
		\newtheorem{corollary}{Corolario}[theorem]
		\newtheorem{definition}{Definición}
		
		\theoremstyle{definition}
		\newtheorem{example}{Ejemplo}
		\newtheorem{remark}{Observación}
		
		\theoremstyle{remark}
		\newtheorem*{note}{Nota}
	\end{lstlisting}
\end{tcolorbox}

\subsection{Uso de Entornos}

\subsubsection*{Teorema}
\begin{tcolorbox}[colback=green!5,colframe=green!50!black]
	\textbf{Ejemplo:}
	\begin{lstlisting}[language=TeX]
		\begin{theorem}[Pitágoras]
			En un triángulo rectángulo, $a^2 + b^2 = c^2$.
		\end{theorem}
		
		\begin{proof}
			Demostración...
		\end{proof}
	\end{lstlisting}
\end{tcolorbox}

\section{Numeración y Referencias}

\subsection{Etiquetas y Referencias}

\subsubsection*{\texttt{\textbackslash label y \textbackslash ref}}
\begin{tcolorbox}[colback=green!5,colframe=green!50!black]
	\textbf{Ejemplo:}
	\begin{lstlisting}[language=TeX]
		\begin{equation} \label{eq:pythagoras}
			a^2 + b^2 = c^2
		\end{equation}
		
		Como vimos en la ecuación \ref{eq:pythagoras}...
		Ver ecuación~\eqref{eq:pythagoras} en la página~\pageref{eq:pythagoras}
	\end{lstlisting}
\end{tcolorbox}

\subsubsection*{\texttt{\textbackslash tag}}
\begin{tcolorbox}[colback=green!5,colframe=green!50!black]
	\textbf{Descripción:} Etiqueta personalizada para ecuación
	
	\textbf{Ejemplo:}
	\begin{lstlisting}[language=TeX]
		\begin{equation} \tag{$\star$}
			E = mc^2
		\end{equation}
		
		\begin{equation} \tag{Ec. Principal}
			F = ma
		\end{equation}
	\end{lstlisting}
\end{tcolorbox}

\subsubsection*{\texttt{\textbackslash notag o \textbackslash nonumber}}
\begin{tcolorbox}[colback=green!5,colframe=green!50!black]
	\textbf{Descripción:} Omite numeración en una línea de align
	
	\textbf{Ejemplo:}
	\begin{lstlisting}[language=TeX]
		\begin{align}
			x &= a + b \\
			y &= c + d \notag \\
			z &= e + f
		\end{align}
	\end{lstlisting}
\end{tcolorbox}

\section{Comandos Especiales}

\subsection{Cancelación}

\subsubsection*{\texttt{\textbackslash cancel}}
\begin{tcolorbox}[colback=green!5,colframe=green!50!black]
	\textbf{Descripción:} Tacha término (requiere cancel)
	
	\textbf{Ejemplo:}
	\begin{lstlisting}[language=TeX]
		\usepackage{cancel}
		$\frac{\cancel{x}(x+1)}{\cancel{x}} = x+1$
		$\cancelto{0}{x}$
	\end{lstlisting}
\end{tcolorbox}

\subsection{Extensibles}

\subsubsection*{\texttt{\textbackslash xleftarrow, \textbackslash xrightarrow}}
\begin{tcolorbox}[colback=green!5,colframe=green!50!black]
	\textbf{Descripción:} Flechas extensibles con texto
	
	\textbf{Ejemplo:}
	\begin{lstlisting}[language=TeX]
		$A \xrightarrow{f} B \xrightarrow[\text{abajo}]{\text{arriba}} C$
		$X \xleftarrow{g} Y$
	\end{lstlisting}
	
	\textbf{Resultado:} $A \xrightarrow{f} B \xrightarrow[\text{abajo}]{\text{arriba}} C$, $X \xleftarrow{g} Y$
\end{tcolorbox}

\subsection{Colores en Matemáticas}

\subsubsection*{\texttt{\textbackslash color, \textbackslash textcolor}}
\begin{tcolorbox}[colback=green!5,colframe=green!50!black]
	\textbf{Ejemplo:}
	\begin{lstlisting}[language=TeX]
		${\color{red} x^2} + {\color{blue} y^2} = 1$
		$\textcolor{red}{a} + \textcolor{green}{b}$
	\end{lstlisting}
	
	\textbf{Resultado:} ${\color{red} x^2} + {\color{blue} y^2} = 1$, $\textcolor{red}{a} + \textcolor{green}{b}$
\end{tcolorbox}

\subsection{Tamaños de Fuente}

\subsubsection*{Comandos de tamaño}
\begin{tcolorbox}[colback=green!5,colframe=green!50!black]
	\textbf{Ejemplo:}
	\begin{lstlisting}[language=TeX]
		$\displaystyle \sum_{i=1}^n$ vs $\textstyle \sum_{i=1}^n$
		$\scriptstyle \sum_{i=1}^n$ vs $\scriptscriptstyle \sum_{i=1}^n$
	\end{lstlisting}
	
	\textbf{Resultado:} $\displaystyle \sum_{i=1}^n$ vs $\textstyle \sum_{i=1}^n$ vs $\scriptstyle \sum_{i=1}^n$ vs $\scriptscriptstyle \sum_{i=1}^n$
\end{tcolorbox}

\section{Arrays Personalizados}

\subsubsection*{\texttt{array}}
\begin{tcolorbox}[colback=green!5,colframe=green!50!black]
	\textbf{Descripción:} Arreglo personalizado con alineación
	
	\textbf{Ejemplo:}
	\begin{lstlisting}[language=TeX]
		\[
		\begin{array}{c|cc}
			& x & y \\
			\hline
			A & 1 & 2 \\
			B & 3 & 4
		\end{array}
		\]
	\end{lstlisting}
	
	\textbf{Resultado:}
	\[
	\begin{array}{c|cc}
		& x & y \\
		\hline
		A & 1 & 2 \\
		B & 3 & 4
	\end{array}
	\]
\end{tcolorbox}

\newpage

\section*{\faCheckCircle\ Ejemplos Completos}

\subsection*{Ejemplo 1: Ecuaciones Complejas}

\begin{tcolorbox}[colback=green!10,colframe=green!75!black,title=\faCode\ Sistema de ecuaciones]
	\begin{lstlisting}[language=TeX]
		\begin{align}
			x + y + z &= 6 \\
			2x - y + 3z &= 14 \\
			-x + 4y - z &= 2
		\end{align}
		
		Solución: $(x,y,z) = (1,2,3)$
	\end{lstlisting}
\end{tcolorbox}

\subsection*{Ejemplo 2: Cálculo}

\begin{tcolorbox}[colback=purple!10,colframe=purple!75!black,title=\faCode\ Derivadas e integrales]
	\begin{lstlisting}[language=TeX]
		\[
		\frac{d}{dx}\left(\int_a^x f(t) \, dt\right) = f(x)
		\]
		
		\[
		\int_0^\infty e^{-x^2} dx = \frac{\sqrt{\pi}}{2}
		\]
	\end{lstlisting}
\end{tcolorbox}

\subsection*{Ejemplo 3: Álgebra Lineal}

\begin{tcolorbox}[colback=cyan!10,colframe=cyan!75!black,title=\faCode\ Matrices y determinantes]
	\begin{lstlisting}[language=TeX]
		\[
		\begin{vmatrix}
			a & b \\
			c & d
		\end{vmatrix} = ad - bc
		\]
		
		\[
		\mathbf{A} = \begin{bmatrix}
			1 & 2 & 3 \\
			4 & 5 & 6 \\
			7 & 8 & 9
		\end{bmatrix}
		\]
	\end{lstlisting}
\end{tcolorbox}

\section*{\faLightbulb\ Tips Importantes}

\begin{tcolorbox}[colback=blue!10,colframe=blue!75!black]
	\begin{itemize}[leftmargin=*]
		\item Usa \texttt{\textbackslash[...\textbackslash]} en lugar de \$\$...\$\$ para ecuaciones display
		\item Siempre carga \texttt{amsmath} para matemáticas avanzadas
		\item Usa \texttt{\textbackslash left} y \texttt{\textbackslash right} para delimitadores que se ajustan automáticamente
		\item Para vectores usa \texttt{\textbackslash mathbf} o \texttt{\textbackslash bm}
		\item Usa \texttt{\textbackslash text\{...\}} para texto dentro de ecuaciones
		\item \texttt{align} numera cada línea, \texttt{gather} también; usa versión * para sin números
		\item Para fracciones en línea considera \texttt{\textbackslash tfrac} para mantener altura
		\item \texttt{\textbackslash mathrm} para operadores no estándar
		\item Usa \texttt{\textbackslash,} para espacio fino entre diferenciales: \texttt{\$dx\textbackslash,dy\$}
		\item \texttt{\textbackslash label} debe ir después de \texttt{\textbackslash caption} o dentro de equation
	\end{itemize}
\end{tcolorbox}

\section*{\faExclamationTriangle\ Errores Comunes}

\begin{tcolorbox}[colback=red!10,colframe=red!75!black]
	\begin{itemize}[leftmargin=*]
		\item \textbf{Olvidar \textbackslash} antes de comandos: escribir \texttt{sin} en lugar de \texttt{\textbackslash sin}
		\item \textbf{Modo matemático incorrecto}: usar \$ dentro de equation
		\item \textbf{Llaves desequilibradas}: \texttt{\textbackslash left(} sin \texttt{\textbackslash right)}
		\item \textbf{Subíndices/superíndices sin llaves}: \texttt{x\_12} → \texttt{x\_\{12\}}
		\item \textbf{Espacios en nombres}: \texttt{\textbackslash operatorname\{arg max\}} → \texttt{\textbackslash operatorname\{arg\textbackslash,max\}}
		\item \textbf{Matrices sin entorno}: intentar alinear con espacios en lugar de usar matrix
		\item \textbf{Punto y coma en ecuaciones}: \LaTeX{} no necesita ; al final de ecuaciones
		\item \textbf{Cargar paquetes sin declararlos}: usar \texttt{\textbackslash mathbb} sin \texttt{amsfonts}
	\end{itemize}
\end{tcolorbox}

\section*{\faBook\ Paquetes Adicionales Útiles}

\begin{tcolorbox}[colback=yellow!10,colframe=orange!75!black]
	\textbf{Paquetes recomendados:}
	\begin{itemize}[leftmargin=*]
		\item \texttt{physics} -- Comandos para física (vectores, derivadas, brakets)
		\item \texttt{siunitx} -- Unidades del SI y formateo de números
		\item \texttt{tensor} -- Notación tensorial
		\item \texttt{mathrsfs} -- Fuente script adicional
		\item \texttt{nicefrac} -- Fracciones diagonales: 1/2
		\item \texttt{cancel} -- Cancelar términos en ecuaciones
		\item \texttt{cases} -- Más opciones para cases
		\item \texttt{empheq} -- Resaltar ecuaciones con cajas
	\end{itemize}
\end{tcolorbox}

\section*{\faKeyboard\ Atajos Comunes}

\begin{tcolorbox}[colback=teal!10,colframe=teal!75!black]
	\textbf{Convenciones útiles:}
	\begin{itemize}[leftmargin=*]
		\item Conjuntos numéricos: $\mathbb{N}, \mathbb{Z}, \mathbb{Q}, \mathbb{R}, \mathbb{C}$
		\item Vectores: $\vec{v}, \mathbf{v}, \bm{v}$
		\item Matrices: usar letras mayúsculas $A, B, X$
		\item Operadores: $\sin, \cos, \log, \lim, \max, \min$
		\item Derivadas: $\frac{d}{dx}, \frac{\partial}{\partial x}$
		\item Integrales: añadir \texttt{\textbackslash,} antes del diferencial: $\int f(x) \, dx$
	\end{itemize}
\end{tcolorbox}

\vspace{1cm}

\begin{center}
	\textit{Documento generado con \LaTeX{} -- \today}
	
	\textit{Este documento cubre las funcionalidades principales del modo matemático en \LaTeX{}.}
	
	\textit{Para más información, consulta los manuales oficiales de amsmath, mathtools y amsthm.}
\end{center}


\end{document}
